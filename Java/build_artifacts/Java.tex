%%
% Copyright (c) 2017 - 2025, Pascal Wagler;
% Copyright (c) 2014 - 2025, John MacFarlane
%
% All rights reserved.
%
% Redistribution and use in source and binary forms, with or without
% modification, are permitted provided that the following conditions
% are met:
%
% - Redistributions of source code must retain the above copyright
% notice, this list of conditions and the following disclaimer.
%
% - Redistributions in binary form must reproduce the above copyright
% notice, this list of conditions and the following disclaimer in the
% documentation and/or other materials provided with the distribution.
%
% - Neither the name of John MacFarlane nor the names of other
% contributors may be used to endorse or promote products derived
% from this software without specific prior written permission.
%
% THIS SOFTWARE IS PROVIDED BY THE COPYRIGHT HOLDERS AND CONTRIBUTORS
% "AS IS" AND ANY EXPRESS OR IMPLIED WARRANTIES, INCLUDING, BUT NOT
% LIMITED TO, THE IMPLIED WARRANTIES OF MERCHANTABILITY AND FITNESS
% FOR A PARTICULAR PURPOSE ARE DISCLAIMED. IN NO EVENT SHALL THE
% COPYRIGHT OWNER OR CONTRIBUTORS BE LIABLE FOR ANY DIRECT, INDIRECT,
% INCIDENTAL, SPECIAL, EXEMPLARY, OR CONSEQUENTIAL DAMAGES (INCLUDING,
% BUT NOT LIMITED TO, PROCUREMENT OF SUBSTITUTE GOODS OR SERVICES;
% LOSS OF USE, DATA, OR PROFITS; OR BUSINESS INTERRUPTION) HOWEVER
% CAUSED AND ON ANY THEORY OF LIABILITY, WHETHER IN CONTRACT, STRICT
% LIABILITY, OR TORT (INCLUDING NEGLIGENCE OR OTHERWISE) ARISING IN
% ANY WAY OUT OF THE USE OF THIS SOFTWARE, EVEN IF ADVISED OF THE
% POSSIBILITY OF SUCH DAMAGE.
%%

%%
% This is the Eisvogel pandoc LaTeX template.
%
% For usage information and examples visit the official GitHub page:
% https://github.com/Wandmalfarbe/pandoc-latex-template
%%
% Options for packages loaded elsewhere
\PassOptionsToPackage{unicode}{hyperref}
\PassOptionsToPackage{hyphens}{url}
\PassOptionsToPackage{dvipsnames,svgnames,x11names,table}{xcolor}
\documentclass[
  paper=a4,
  openany,
  oneside  ,captions=tableheading
]{scrbook}
\usepackage{xcolor}
\usepackage[top=1.5cm, bottom=2.5cm, left=3cm, right=3cm, includeheadfoot, heightrounded]{geometry}
\usepackage{amsmath,amssymb}

% add backlinks to footnote references, cf. https://tex.stackexchange.com/questions/302266/make-footnote-clickable-both-ways
\usepackage{footnotebackref}
\setcounter{secnumdepth}{5}
\usepackage{iftex}
\ifPDFTeX
  \usepackage[T1]{fontenc}
  \usepackage[utf8]{inputenc}
  \usepackage{textcomp} % provide euro and other symbols
\else % if luatex or xetex
  \usepackage{unicode-math} % this also loads fontspec
  \defaultfontfeatures{Scale=MatchLowercase}
  \defaultfontfeatures[\rmfamily]{Ligatures=TeX,Scale=1}
\fi
\usepackage{lmodern}
\ifPDFTeX\else
  % xetex/luatex font selection
  \setmainfont[]{LMRoman10-Regular}
  \setsansfont[]{Arial}
  \setmonofont[]{LMMono10-Regular}
\fi
% Use upquote if available, for straight quotes in verbatim environments
\IfFileExists{upquote.sty}{\usepackage{upquote}}{}
\IfFileExists{microtype.sty}{% use microtype if available
  \usepackage[]{microtype}
  \UseMicrotypeSet[protrusion]{basicmath} % disable protrusion for tt fonts
}{}

% Use setspace anyway because we change the default line spacing.
% The spacing is changed early to affect the titlepage and the TOC.
\usepackage{setspace}
\setstretch{1.2}
\makeatletter
\@ifundefined{KOMAClassName}{% if non-KOMA class
  \IfFileExists{parskip.sty}{%
    \usepackage{parskip}
  }{% else
    \setlength{\parindent}{0pt}
    \setlength{\parskip}{6pt plus 2pt minus 1pt}}
}{% if KOMA class
  \KOMAoptions{parskip=half}}
\makeatother
\usepackage{color}
\usepackage{fancyvrb}
\newcommand{\VerbBar}{|}
\newcommand{\VERB}{\Verb[commandchars=\\\{\}]}
\DefineVerbatimEnvironment{Highlighting}{Verbatim}{commandchars=\\\{\}}
% Add ',fontsize=\small' for more characters per line
\usepackage{framed}
\definecolor{shadecolor}{RGB}{248,248,248}
\newenvironment{Shaded}{\begin{snugshade}}{\end{snugshade}}
\newcommand{\AlertTok}[1]{\textcolor[rgb]{0.94,0.16,0.16}{#1}}
\newcommand{\AnnotationTok}[1]{\textcolor[rgb]{0.56,0.35,0.01}{\textbf{\textit{#1}}}}
\newcommand{\AttributeTok}[1]{\textcolor[rgb]{0.13,0.29,0.53}{#1}}
\newcommand{\BaseNTok}[1]{\textcolor[rgb]{0.00,0.00,0.81}{#1}}
\newcommand{\BuiltInTok}[1]{#1}
\newcommand{\CharTok}[1]{\textcolor[rgb]{0.31,0.60,0.02}{#1}}
\newcommand{\CommentTok}[1]{\textcolor[rgb]{0.56,0.35,0.01}{\textit{#1}}}
\newcommand{\CommentVarTok}[1]{\textcolor[rgb]{0.56,0.35,0.01}{\textbf{\textit{#1}}}}
\newcommand{\ConstantTok}[1]{\textcolor[rgb]{0.56,0.35,0.01}{#1}}
\newcommand{\ControlFlowTok}[1]{\textcolor[rgb]{0.13,0.29,0.53}{\textbf{#1}}}
\newcommand{\DataTypeTok}[1]{\textcolor[rgb]{0.13,0.29,0.53}{#1}}
\newcommand{\DecValTok}[1]{\textcolor[rgb]{0.00,0.00,0.81}{#1}}
\newcommand{\DocumentationTok}[1]{\textcolor[rgb]{0.56,0.35,0.01}{\textbf{\textit{#1}}}}
\newcommand{\ErrorTok}[1]{\textcolor[rgb]{0.64,0.00,0.00}{\textbf{#1}}}
\newcommand{\ExtensionTok}[1]{#1}
\newcommand{\FloatTok}[1]{\textcolor[rgb]{0.00,0.00,0.81}{#1}}
\newcommand{\FunctionTok}[1]{\textcolor[rgb]{0.13,0.29,0.53}{\textbf{#1}}}
\newcommand{\ImportTok}[1]{#1}
\newcommand{\InformationTok}[1]{\textcolor[rgb]{0.56,0.35,0.01}{\textbf{\textit{#1}}}}
\newcommand{\KeywordTok}[1]{\textcolor[rgb]{0.13,0.29,0.53}{\textbf{#1}}}
\newcommand{\NormalTok}[1]{#1}
\newcommand{\OperatorTok}[1]{\textcolor[rgb]{0.81,0.36,0.00}{\textbf{#1}}}
\newcommand{\OtherTok}[1]{\textcolor[rgb]{0.56,0.35,0.01}{#1}}
\newcommand{\PreprocessorTok}[1]{\textcolor[rgb]{0.56,0.35,0.01}{\textit{#1}}}
\newcommand{\RegionMarkerTok}[1]{#1}
\newcommand{\SpecialCharTok}[1]{\textcolor[rgb]{0.81,0.36,0.00}{\textbf{#1}}}
\newcommand{\SpecialStringTok}[1]{\textcolor[rgb]{0.31,0.60,0.02}{#1}}
\newcommand{\StringTok}[1]{\textcolor[rgb]{0.31,0.60,0.02}{#1}}
\newcommand{\VariableTok}[1]{\textcolor[rgb]{0.00,0.00,0.00}{#1}}
\newcommand{\VerbatimStringTok}[1]{\textcolor[rgb]{0.31,0.60,0.02}{#1}}
\newcommand{\WarningTok}[1]{\textcolor[rgb]{0.56,0.35,0.01}{\textbf{\textit{#1}}}}

% Workaround/bugfix from jannick0.
% See https://github.com/jgm/pandoc/issues/4302#issuecomment-360669013)
% or https://github.com/Wandmalfarbe/pandoc-latex-template/issues/2
%
% Redefine the verbatim environment 'Highlighting' to break long lines (with
% the help of fvextra). Redefinition is necessary because it is unlikely that
% pandoc includes fvextra in the default template.
\usepackage{fvextra}
\DefineVerbatimEnvironment{Highlighting}{Verbatim}{breaklines,fontsize=\small,commandchars=\\\{\}}

\usepackage{longtable,booktabs,array}
\newcounter{none} % for unnumbered tables
\usepackage{calc} % for calculating minipage widths
% Correct order of tables after \paragraph or \subparagraph
\usepackage{etoolbox}
\makeatletter
\patchcmd\longtable{\par}{\if@noskipsec\mbox{}\fi\par}{}{}
\makeatother
% Allow footnotes in longtable head/foot
\IfFileExists{footnotehyper.sty}{\usepackage{footnotehyper}}{\usepackage{footnote}}
\makesavenoteenv{longtable}
\usepackage{graphicx}
\makeatletter
\newsavebox\pandoc@box
\newcommand*\pandocbounded[1]{% scales image to fit in text height/width
  \sbox\pandoc@box{#1}%
  \Gscale@div\@tempa{\textheight}{\dimexpr\ht\pandoc@box+\dp\pandoc@box\relax}%
  \Gscale@div\@tempb{\linewidth}{\wd\pandoc@box}%
  \ifdim\@tempb\p@<\@tempa\p@\let\@tempa\@tempb\fi% select the smaller of both
  \ifdim\@tempa\p@<\p@\scalebox{\@tempa}{\usebox\pandoc@box}%
  \else\usebox{\pandoc@box}%
  \fi%
}
% Set default figure placement to htbp
% Make use of float-package and set default placement for figures to H.
% The option H means 'PUT IT HERE' (as  opposed to the standard h option which means 'You may put it here if you like').
\usepackage{float}
\floatplacement{figure}{H}
\makeatother
\ifLuaTeX
  \usepackage{luacolor}
  \usepackage[soul]{lua-ul}
\else
  \usepackage{soul}
\fi
\setlength{\emergencystretch}{3em} % prevent overfull lines
\providecommand{\tightlist}{%
  \setlength{\itemsep}{0pt}\setlength{\parskip}{0pt}}

\usepackage{caption}
\captionsetup[figure]{labelsep=none, justification=centering}
\usepackage{etoc} % Replaces minitoc
\usepackage[version=4]{mhchem}
\usepackage{amsmath}
\usepackage{amssymb}
\usepackage{mathtools}
% \mtcselectlanguage{english} - Removed
\definecolor{mylinkcolor}{HTML}{07455c}
\definecolor{myurlcolor}{HTML}{07455c}
% --- Chapter Title Styling (KOMA-Script) ---
\renewcommand*\chapterformat{\thechapter.\enskip}
\addtokomafont{chapter}{\centering}
\RedeclareSectionCommand[beforeskip=0pt,afterskip=20pt]{chapter}
% --- Table Styling ---
\rowcolors{2}{RoyalBlue!20}{white}
\renewcommand{\arraystretch}{1.2}
% -------------------------------------------
\usepackage[most]{tcolorbox}
\usepackage{fontawesome5}
\usepackage{xcolor}

% Define custom colors
\definecolor{notecolor}{RGB}{97, 175, 239}   % Blue
\definecolor{tipcolor}{RGB}{20, 196, 255} % Purple
\definecolor{warningcolor}{RGB}{229, 192, 123} % Orange
\definecolor{attentioncolor}{RGB}{224, 108, 117} % Red
\definecolor{analogycolor}{RGB}{152, 195, 121}    % Green

% Generic Pretty Box Style
% #1 = Color name
% #2 = Icon command
% #3 = Title text
\newtcolorbox{prettybox}[3]{
    enhanced,
    colback=#1!5!white,      % Very light background
    colframe=#1,             % Border color
    coltitle=#1!50!black,    % Title text color (darker version of base)
    title={#2\ \ \textbf{#3}},
    fonttitle=\bfseries\large,
    attach boxed title to top left={xshift=5mm, yshift=-3.5mm}, % Adjust for half-in/out
    boxed title style={
        colback=white,
        colframe=white,
        boxrule=0pt,
        top=0pt,
        bottom=0pt,
        left=2pt,
        right=2pt
    },
    top=1.5em, % Space for the floating title
    bottom=1em,
    left=1em,
    right=1em,
    arc=3pt,
    boxrule=1pt,
    drop fuzzy shadow,   
    parbox=false,        
    breakable            
}

% Define specific environments
% usage: \begin{noteblock}{Title} ... \end{noteblock}

\newenvironment{noteblock}[1]
  {\begin{prettybox}{notecolor}{\faInfoCircle}{#1}}
  {\end{prettybox}}

\newenvironment{tipblock}[1]
  {\begin{prettybox}{tipcolor}{\faLightbulb}{#1}}
  {\end{prettybox}}

\newenvironment{warningblock}[1]
  {\begin{prettybox}{warningcolor}{\faExclamationTriangle}{#1}}
  {\end{prettybox}}

\newenvironment{attentionblock}[1]
  {\begin{prettybox}{attentioncolor}{\faExclamationCircle}{#1}}
  {\end{prettybox}}

\newenvironment{cautionblock}[1]
  {\begin{prettybox}{attentioncolor}{\faRadiation}{#1}}
  {\end{prettybox}}

\newenvironment{importantblock}[1]
  {\begin{prettybox}{attentioncolor}{\faStar}{#1}}
  {\end{prettybox}}

% Analogy Block
\newenvironment{analogyblock}[1]
  {\begin{prettybox}{analogycolor}{\faShapes}{#1}}
  {\end{prettybox}}
\usepackage{bookmark}
\IfFileExists{xurl.sty}{\usepackage{xurl}}{} % add URL line breaks if available
\urlstyle{same}
\definecolor{default-linkcolor}{HTML}{A50000}
\definecolor{default-filecolor}{HTML}{A50000}
\definecolor{default-citecolor}{HTML}{4077C0}
\definecolor{default-urlcolor}{HTML}{4077C0}

\hypersetup{
  pdftitle={Java},
  pdfauthor={Milind},
  colorlinks=true,
  linkcolor={mylinkcolor},
  filecolor={default-filecolor},
  citecolor={default-citecolor},
  urlcolor={myurlcolor},
  breaklinks=true,
  pdfcreator={LaTeX via pandoc with the Eisvogel template}}

\title{Java}
\usepackage{etoolbox}
\makeatletter
\providecommand{\subtitle}[1]{% add subtitle to \maketitle
  \apptocmd{\@title}{\par {\large #1 \par}}{}{}
}
\makeatother
\subtitle{Personal Notes \& References}
\author{Milind}
\date{}


%
% for the background color of the title page
%

%
% break urls
%
\PassOptionsToPackage{hyphens}{url}

%
% When using babel or polyglossia with biblatex, loading csquotes is recommended
% to ensure that quoted texts are typeset according to the rules of your main language.
%
\usepackage{csquotes}

%
% captions
%
\definecolor{caption-color}{HTML}{777777}
\usepackage[font={stretch=1.2}, textfont={color=caption-color}, position=top, skip=4mm, labelfont=bf, singlelinecheck=false, justification=raggedright]{caption}
\setcapindent{0em}

%
% blockquote
%
\definecolor{blockquote-border}{RGB}{221,221,221}
\definecolor{blockquote-text}{RGB}{119,119,119}
\usepackage{mdframed}
\newmdenv[rightline=false,bottomline=false,topline=false,linewidth=3pt,linecolor=blockquote-border,skipabove=\parskip]{customblockquote}
\renewenvironment{quote}{\begin{customblockquote}\list{}{\rightmargin=0em\leftmargin=0em}%
\item\relax\color{blockquote-text}\ignorespaces}{\unskip\unskip\endlist\end{customblockquote}}

%
% Source Sans Pro as the default font family
% Source Code Pro for monospace text
%
% 'default' option sets the default
% font family to Source Sans Pro, not \sfdefault.
%
% Note that the font has been officially renamed to `Source Sans 3`, and
% the version provided by the `sourcesanspro` package is slightly outdated.
% You can install the newer version locally and use it, for example, with
% `mainfont: "Source Sans 3"` in the YAML metadata (requires XeTeX or LuaTeX).
%
\ifnum 0\ifxetex 1\fi\ifluatex 1\fi=0 % if pdftex
    \usepackage[default]{sourcesanspro}
  \usepackage{sourcecodepro}
  \else % if not pdftex
    \fi

%
% heading color
%
\definecolor{heading-color}{RGB}{40,40,40}
% By default, the KOMA-Script classes will typeset sectioning headings in
% sans-serif. Use the normal body font for headings.
\addtokomafont{disposition}{\normalfont\color{heading-color}\bfseries}

%
% variables for title, author and date
%
\usepackage{titling}
\title{Java}
\author{Milind}
\date{}

%
% tables
%

\definecolor{table-row-color}{HTML}{F5F5F5}
\definecolor{table-rule-color}{HTML}{999999}

%\arrayrulecolor{black!40}
\arrayrulecolor{table-rule-color}     % color of \toprule, \midrule, \bottomrule
\setlength\heavyrulewidth{0.3ex}      % thickness of \toprule, \bottomrule
\renewcommand{\arraystretch}{1.3}     % spacing (padding)


%
% remove paragraph indentation
%
\setlength{\parindent}{0pt}
\setlength{\parskip}{6pt plus 2pt minus 1pt}
\setlength{\emergencystretch}{3em}  % prevent overfull lines

%
%
% Listings
%
%


%
% header and footer
%
\usepackage[headsepline,footsepline]{scrlayer-scrpage}

\newpairofpagestyles{eisvogel-header-footer}{
  \clearpairofpagestyles
  \ihead*{Java}
  \chead*{}
  \ohead*{}
  \ifoot*{Milind}
  \cfoot*{}
  \ofoot*{\thepage}
  \addtokomafont{pageheadfoot}{\upshape}
}
\pagestyle{eisvogel-header-footer}

\deftripstyle{ChapterStyle}{}{}{}{}{\pagemark}{}
\renewcommand*{\chapterpagestyle}{ChapterStyle}


%
% Define watermark
%

\begin{document}


\frontmatter
% don't generate the default title
% \maketitle



\begin{titlepage}
    \newgeometry{left=2.5cm,right=2.5cm,top=2cm,bottom=2cm}
    \vspace*{1cm}
    
    
    \vspace{3cm}
    
    \centering
    {\fontsize{50}{60}\selectfont \bfseries Java \par}
    \vspace{1cm}
    {\fontsize{20}{30}\selectfont Personal Notes \& References \par}
    
    \vfill
    
    {\fontsize{18}{22}\selectfont Milind \par}
    \vspace{0.5cm}
    {\large January 10, 2026 \par}
    
    \vspace{3cm}
    \restoregeometry
\end{titlepage}

% Initialize MiniTOC (Removed, using etoc)

{
\setcounter{tocdepth}{3}
\tableofcontents
}
\mainmatter
\chapter{Java}\label{java}

\etocsettocstyle{\textbf{Chapter Contents}\par\rule{\linewidth}{0.5pt}}{\par\rule{\linewidth}{0.5pt}}
\localtableofcontents

\noindent

Java is a high-level, class-based, object-oriented programming language
that is designed to have as few implementation dependencies as possible.
It is a general-purpose programming language intended to let application
developers \emph{write once, run anywhere} (WORA).

It follows the four pillars of OOP:

\begin{enumerate}
\def\labelenumi{\arabic{enumi}.}
\tightlist
\item
  \hyperref[inheritance]{Inheritance}
\item
  \hyperref[polymorphism]{Polymorphism}
\item
  \hyperref[encapsulation]{Encapsulation}
\item
  \hyperref[abstraction]{Abstraction}
\end{enumerate}

\section{Java Environment}\label{java-environment}

The Java environment consists of three main components: JDK, JRE, and
JVM.

\subsection{JDK (Java Development Kit)}\label{jdk-java-development-kit}

It is a software development environment used for developing Java
applications. It includes the JRE, an interpreter/loader (Java), a
compiler (javac), an archiver (jar), a documentation generator
(Javadoc), and other tools needed in Java development.

\subsection{JRE (Java Runtime
Environment)}\label{jre-java-runtime-environment}

It provides the libraries, the Java Virtual Machine (JVM), and other
components to run applets and applications written in the Java
programming language. It does not contain tools for development like
compilers or debuggers.

\subsection{JVM (Java Virtual Machine)}\label{jvm-java-virtual-machine}

It is an abstract machine. It is a specification that provides a runtime
environment in which Java bytecode can be executed.

\textbf{Responsibilities of JVM:}

\begin{itemize}
\tightlist
\item
  Loads code
\item
  Verifies code
\item
  Executes code
\item
  Provides runtime environment
\end{itemize}

\begin{figure}
\centering
\includegraphics[width=\linewidth,height=12cm,keepaspectratio,alt={~}]{images/mermaid_1e798736723c4f875ac413d0452f3ad9.png}
\caption{~}
\end{figure}

\begin{tipblock}{Analogy: The Kitchen}

\begin{itemize}
\tightlist
\item
  \textbf{JDK (The Whole Kit):} The entire kitchen setup. It has the
  stove, ingredients, \textbf{AND} the recipe books/tools to
  \emph{create} new dishes.
\item
  \textbf{JRE (Runtime Env):} A kitchen that is only for \emph{serving}
  food. It has the stove and ingredients to cook (run) the meal, but no
  tools to write new recipes.
\item
  \textbf{JVM (The Chef):} The person who actually cooks. They read the
  recipe (Bytecode) and use the stove (Hardware) to make the food
  (Machine Code).
\end{itemize}

\end{tipblock}

\section{Compilation Process}\label{compilation-process}

\begin{enumerate}
\def\labelenumi{\arabic{enumi}.}
\tightlist
\item
  \textbf{Source Code}: The programmer writes code in \texttt{.java}
  files.
\item
  \textbf{Compilation}: The \texttt{javac} compiler converts the source
  code into \textbf{Bytecode} (\texttt{.class} files). Bytecode is
  platform-independent.
\item
  \textbf{Execution}: The JVM reads the bytecode and translates it into
  machine code (native machine language) for the specific operating
  system.
\end{enumerate}

\begin{figure}
\centering
\includegraphics[width=\linewidth,height=12cm,keepaspectratio,alt={~}]{images/mermaid_6e7ba5245d5d5d1714ab6c7246cfbbd0.png}
\caption{~}
\end{figure}

\section{Java Source File Structure}\label{java-source-file-structure}

A Java source file can contain multiple classes, but there are specific
rules:

\begin{enumerate}
\def\labelenumi{\arabic{enumi}.}
\tightlist
\item
  There can be only \textbf{one} \texttt{public} class per source file.
\item
  If there is a public class, the name of the file must match the name
  of the public class.
\item
  If there is no public class, the file can have any name.
\end{enumerate}

\begin{Shaded}
\begin{Highlighting}[]
\CommentTok{// File: Main.java}
\KeywordTok{public} \KeywordTok{class}\NormalTok{ Main }\OperatorTok{\{} \CommentTok{// File name must be Main.java}
    \KeywordTok{public} \DataTypeTok{static} \DataTypeTok{void} \FunctionTok{main}\OperatorTok{(}\BuiltInTok{String}\OperatorTok{[]}\NormalTok{ args}\OperatorTok{)} \OperatorTok{\{}
        \BuiltInTok{System}\OperatorTok{.}\FunctionTok{out}\OperatorTok{.}\FunctionTok{println}\OperatorTok{(}\StringTok{"Hello World"}\OperatorTok{);}
    \OperatorTok{\}}
\OperatorTok{\}}
\KeywordTok{class}\NormalTok{ Helper }\OperatorTok{\{} \CommentTok{// Non{-}public class allowed in the same file}
    \CommentTok{// ...}
\OperatorTok{\}}
\end{Highlighting}
\end{Shaded}

\section{Classes and Objects}\label{classes-and-objects}

\subsection{Class}\label{class}

A class is a blueprint or template from which objects are created. It
defines a set of properties (fields) and methods that are common to all
objects of one type.

\subsection{Object}\label{object}

An object is an instance of a class. It has:

\begin{itemize}
\tightlist
\item
  \textbf{State}: Represented by attributes (variables).
\item
  \textbf{Behavior}: Represented by methods.
\item
  \textbf{Identity}: A unique name (reference) to interact with it.
\end{itemize}

\begin{tipblock}{Analogy: Blueprint vs.~House}

\begin{itemize}
\tightlist
\item
  \textbf{Class} = \textbf{The Blueprint}. It's just a drawing on paper.
  You can't live in it. It describes what the house \emph{will} look
  like.
\item
  \textbf{Object} = \textbf{The House}. It's the actual physical
  building created from the blueprint. You can build 100 identical
  houses (Objects) from 1 blueprint (Class).
\end{itemize}

\end{tipblock}

\begin{Shaded}
\begin{Highlighting}[]
\KeywordTok{class}\NormalTok{ Car }\OperatorTok{\{}
    \CommentTok{// Fields (State)}
    \BuiltInTok{String}\NormalTok{ color}\OperatorTok{;}
    \BuiltInTok{String}\NormalTok{ model}\OperatorTok{;}

    \CommentTok{// Method (Behavior)}
    \DataTypeTok{void} \FunctionTok{drive}\OperatorTok{()} \OperatorTok{\{}
        \BuiltInTok{System}\OperatorTok{.}\FunctionTok{out}\OperatorTok{.}\FunctionTok{println}\OperatorTok{(}\StringTok{"Driving "} \OperatorTok{+}\NormalTok{ model}\OperatorTok{);}
    \OperatorTok{\}}
\OperatorTok{\}}

\KeywordTok{public} \KeywordTok{class}\NormalTok{ Test }\OperatorTok{\{}
    \KeywordTok{public} \DataTypeTok{static} \DataTypeTok{void} \FunctionTok{main}\OperatorTok{(}\BuiltInTok{String}\OperatorTok{[]}\NormalTok{ args}\OperatorTok{)} \OperatorTok{\{}
\NormalTok{        Car myCar }\OperatorTok{=} \KeywordTok{new} \FunctionTok{Car}\OperatorTok{();} \CommentTok{// Creating an object}
\NormalTok{        myCar}\OperatorTok{.}\FunctionTok{model} \OperatorTok{=} \StringTok{"Tesla"}\OperatorTok{;}
\NormalTok{        myCar}\OperatorTok{.}\FunctionTok{drive}\OperatorTok{();}
    \OperatorTok{\}}
\OperatorTok{\}}
\end{Highlighting}
\end{Shaded}

\section{Access Specifiers}\label{access-specifiers}

Access specifiers determine the visibility of classes, methods, and
variables.

{\def\LTcaptype{none} % do not increment counter
\begin{longtable}[]{@{}
  >{\raggedright\arraybackslash}p{(\linewidth - 8\tabcolsep) * \real{0.2000}}
  >{\centering\arraybackslash}p{(\linewidth - 8\tabcolsep) * \real{0.2000}}
  >{\centering\arraybackslash}p{(\linewidth - 8\tabcolsep) * \real{0.2000}}
  >{\centering\arraybackslash}p{(\linewidth - 8\tabcolsep) * \real{0.2000}}
  >{\centering\arraybackslash}p{(\linewidth - 8\tabcolsep) * \real{0.2000}}@{}}
\toprule\noalign{}
\begin{minipage}[b]{\linewidth}\raggedright
Specifier
\end{minipage} & \begin{minipage}[b]{\linewidth}\centering
Class
\end{minipage} & \begin{minipage}[b]{\linewidth}\centering
Package
\end{minipage} & \begin{minipage}[b]{\linewidth}\centering
Subclass
\end{minipage} & \begin{minipage}[b]{\linewidth}\centering
World
\end{minipage} \\
\midrule\noalign{}
\endhead
\bottomrule\noalign{}
\endlastfoot
\textbf{public} & Yes & Yes & Yes & Yes \\
\textbf{protected} & Yes & Yes & Yes & No \\
\textbf{default} (no modifier) & Yes & Yes & No & No \\
\textbf{private} & Yes & No & No & No \\
\end{longtable}
}

\begin{enumerate}
\def\labelenumi{\arabic{enumi}.}
\tightlist
\item
  \textbf{private}: Accessible only within the class.
\item
  \textbf{default}: Accessible only within the package.
\item
  \textbf{protected}: Accessible within the package and by subclasses in
  other packages.
\item
  \textbf{public}: Accessible everywhere.
\end{enumerate}

\section{Static Members}\label{static-members}

The \texttt{static} keyword indicates that a member belongs to the
\textbf{class} rather than to any specific instance (object).

\subsection{Static Variable}\label{static-variable}

\begin{itemize}
\tightlist
\item
  Shared among all instances of the class.
\item
  Memory is allocated only once when the class is loaded.
\end{itemize}

\subsection{Static Method}\label{static-method}

\begin{itemize}
\tightlist
\item
  Can be called without creating an object of the class.
\item
  Can only access static data members and other static methods.
\item
  Cannot use \texttt{this} or \texttt{super} keywords.
\end{itemize}

\begin{Shaded}
\begin{Highlighting}[]
\KeywordTok{class}\NormalTok{ Counter }\OperatorTok{\{}
    \DataTypeTok{static} \DataTypeTok{int}\NormalTok{ count }\OperatorTok{=} \DecValTok{0}\OperatorTok{;} \CommentTok{// Static variable}

    \FunctionTok{Counter}\OperatorTok{()} \OperatorTok{\{}
\NormalTok{        count}\OperatorTok{++;} \CommentTok{// Increments the shared counter}
    \OperatorTok{\}}

    \DataTypeTok{static} \DataTypeTok{void} \FunctionTok{showCount}\OperatorTok{()} \OperatorTok{\{} \CommentTok{// Static method}
        \BuiltInTok{System}\OperatorTok{.}\FunctionTok{out}\OperatorTok{.}\FunctionTok{println}\OperatorTok{(}\StringTok{"Count: "} \OperatorTok{+}\NormalTok{ count}\OperatorTok{);}
    \OperatorTok{\}}
\OperatorTok{\}}
\end{Highlighting}
\end{Shaded}

\section{Final Members}\label{final-members}

The \texttt{final} keyword is used to restrict the user.

\begin{enumerate}
\def\labelenumi{\arabic{enumi}.}
\tightlist
\item
  \textbf{Final Variable}: Value cannot be changed (constant).
\item
  \textbf{Final Method}: Cannot be overridden by subclasses.
\item
  \textbf{Final Class}: Cannot be inherited.
\end{enumerate}

\begin{Shaded}
\begin{Highlighting}[]
\DataTypeTok{final} \DataTypeTok{int}\NormalTok{ MAX\_SPEED }\OperatorTok{=} \DecValTok{100}\OperatorTok{;}
\CommentTok{// MAX\_SPEED = 120; // Compilation Error}
\end{Highlighting}
\end{Shaded}

\section{Constructors}\label{constructors}

A constructor is a block of code similar to a method that is called when
an instance of an object is created.

\subsection{Rules}\label{rules}

\begin{itemize}
\tightlist
\item
  Name must be the same as the class name.
\item
  No return type (not even \texttt{void}).
\item
  Called automatically when \texttt{new} is used.
\end{itemize}

\subsection{Types}\label{types}

\begin{enumerate}
\def\labelenumi{\arabic{enumi}.}
\tightlist
\item
  \textbf{Default Constructor}: Provided by the compiler if no
  constructor is defined.
\item
  \textbf{No-Args Constructor}: A constructor with no parameters defined
  by the user.
\item
  \textbf{Parameterized Constructor}: A constructor with parameters to
  initialize fields.
\item
  \textbf{Copy Constructor}: Used to create an object by copying
  variables from another object.
\end{enumerate}

\begin{Shaded}
\begin{Highlighting}[]
\KeywordTok{class}\NormalTok{ Student }\OperatorTok{\{}
    \BuiltInTok{String}\NormalTok{ name}\OperatorTok{;}
    \DataTypeTok{int}\NormalTok{ age}\OperatorTok{;}

    \CommentTok{// No{-}Args Constructor}
    \FunctionTok{Student}\OperatorTok{()} \OperatorTok{\{}
        \KeywordTok{this}\OperatorTok{.}\FunctionTok{name} \OperatorTok{=} \StringTok{"Unknown"}\OperatorTok{;}
        \KeywordTok{this}\OperatorTok{.}\FunctionTok{age} \OperatorTok{=} \DecValTok{0}\OperatorTok{;}
    \OperatorTok{\}}

    \CommentTok{// Parameterized Constructor}
    \FunctionTok{Student}\OperatorTok{(}\BuiltInTok{String}\NormalTok{ name}\OperatorTok{,} \DataTypeTok{int}\NormalTok{ age}\OperatorTok{)} \OperatorTok{\{}
        \KeywordTok{this}\OperatorTok{.}\FunctionTok{name} \OperatorTok{=}\NormalTok{ name}\OperatorTok{;}
        \KeywordTok{this}\OperatorTok{.}\FunctionTok{age} \OperatorTok{=}\NormalTok{ age}\OperatorTok{;}
    \OperatorTok{\}}

    \CommentTok{// Copy Constructor}
    \FunctionTok{Student}\OperatorTok{(}\NormalTok{Student s}\OperatorTok{)} \OperatorTok{\{}
        \KeywordTok{this}\OperatorTok{.}\FunctionTok{name} \OperatorTok{=}\NormalTok{ s}\OperatorTok{.}\FunctionTok{name}\OperatorTok{;}
        \KeywordTok{this}\OperatorTok{.}\FunctionTok{age} \OperatorTok{=}\NormalTok{ s}\OperatorTok{.}\FunctionTok{age}\OperatorTok{;}
    \OperatorTok{\}}
\OperatorTok{\}}
\end{Highlighting}
\end{Shaded}

\section{\texorpdfstring{\texttt{this}
Keyword}{this Keyword}}\label{this-keyword}

\texttt{this} is a reference variable that refers to the \textbf{current
object}.

\textbf{Usages:}

\begin{enumerate}
\def\labelenumi{\arabic{enumi}.}
\tightlist
\item
  Refer to current class instance variable (to resolve shadowing).
\item
  Invoke current class constructor (\texttt{this()}).
\item
  Return the current class instance.
\end{enumerate}

\begin{Shaded}
\begin{Highlighting}[]
\KeywordTok{class}\NormalTok{ A }\OperatorTok{\{}
    \DataTypeTok{int}\NormalTok{ x}\OperatorTok{;}

    \FunctionTok{A}\OperatorTok{(}\DataTypeTok{int}\NormalTok{ x}\OperatorTok{)} \OperatorTok{\{}
        \KeywordTok{this}\OperatorTok{.}\FunctionTok{x} \OperatorTok{=}\NormalTok{ x}\OperatorTok{;} \CommentTok{// Distinguishes instance variable from parameter}
    \OperatorTok{\}}

    \FunctionTok{A}\OperatorTok{()} \OperatorTok{\{}
        \KeywordTok{this}\OperatorTok{(}\DecValTok{10}\OperatorTok{);} \CommentTok{// Calls the parameterized constructor}
    \OperatorTok{\}}
\OperatorTok{\}}
\end{Highlighting}
\end{Shaded}

\newpage

\chapter{Inheritance}\label{inheritance}

\etocsettocstyle{\textbf{Chapter Contents}\par\rule{\linewidth}{0.5pt}}{\par\rule{\linewidth}{0.5pt}}
\localtableofcontents

\noindent

Inheritance is a mechanism where one class acquires the properties and
behaviors (methods) of another class. This creates an ``IS-A''
relationship (e.g., a \texttt{Dog} IS-A \texttt{Animal}).

We need at least 2 classes. We use \texttt{extends} keyword to inherit
properties from one class to another.

The class which is inherited is called Parent or Super class. The class
which inherits is called Child, Derived or Sub Class.

\begin{tipblock}{Analogy: DNA \& Genetics}

\begin{itemize}
\tightlist
\item
  \textbf{Parent Class:} Your Father. He has blue eyes (Properties) and
  walks fast (Behavior).
\item
  \textbf{Child Class:} You. You \emph{automatically} get the blue eyes
  and fast walk (Inheritance).
\item
  \textbf{Extending:} You can also learn to play guitar (New Method),
  which your father couldn't do. You are an ``extension'' of him.
\end{itemize}

\end{tipblock}

Private properties and methods (\texttt{private}) are \emph{not directly
accessible} by the child class, but they are still part of the object's
structure. Protected (\texttt{protected}) and \texttt{default}
(package-private) members are inherited and accessible within the same
package. \texttt{protected} members are also accessible to subclasses in
different packages.

\begin{Shaded}
\begin{Highlighting}[]
\KeywordTok{class}\NormalTok{ A }\OperatorTok{\{} \CommentTok{// Parent / Super class}
    \DataTypeTok{int}\NormalTok{ a }\OperatorTok{=} \DecValTok{10}\OperatorTok{;}

    \DataTypeTok{int} \FunctionTok{func}\OperatorTok{()\{}
        \ControlFlowTok{return} \DecValTok{10}\OperatorTok{;}
    \OperatorTok{\}}

    \KeywordTok{private} \DataTypeTok{int} \FunctionTok{privateFunc}\OperatorTok{()\{}
        \ControlFlowTok{return} \DecValTok{55}\OperatorTok{;}
    \OperatorTok{\}}
\OperatorTok{\}}

\KeywordTok{class}\NormalTok{ B }\KeywordTok{extends}\NormalTok{ A }\OperatorTok{\{} \CommentTok{// Child / Sub class}
    \DataTypeTok{void} \FunctionTok{test}\OperatorTok{()} \OperatorTok{\{}
        \BuiltInTok{System}\OperatorTok{.}\FunctionTok{out}\OperatorTok{.}\FunctionTok{println}\OperatorTok{(}\NormalTok{a}\OperatorTok{);} \CommentTok{// Works}
        \BuiltInTok{System}\OperatorTok{.}\FunctionTok{out}\OperatorTok{.}\FunctionTok{println}\OperatorTok{(}\FunctionTok{func}\OperatorTok{());} \CommentTok{// Works}
        \CommentTok{// System.out.println(privateFunc()); // Error: privateFunc() has private access in A}
    \OperatorTok{\}}
\OperatorTok{\}}
\end{Highlighting}
\end{Shaded}

Here, B is child class and A is parent class.

\subsection{Types of Inheritance}\label{types-of-inheritance}

\subsubsection{Single Level}\label{single-level}

When one class inherits another. (A -\textgreater{} B)

\begin{Shaded}
\begin{Highlighting}[]
\KeywordTok{class}\NormalTok{ A }\OperatorTok{\{\}}

\KeywordTok{class}\NormalTok{ B }\KeywordTok{extends}\NormalTok{ A }\OperatorTok{\{\}}
\end{Highlighting}
\end{Shaded}

\subsubsection{Multi Level}\label{multi-level}

When one class inherits another, which in turn inherits from another
class. (A -\textgreater{} B -\textgreater{} C)

\begin{Shaded}
\begin{Highlighting}[]
\KeywordTok{class}\NormalTok{ A }\OperatorTok{\{\}}

\KeywordTok{class}\NormalTok{ B }\KeywordTok{extends}\NormalTok{ A }\OperatorTok{\{\}}

\KeywordTok{class}\NormalTok{ C }\KeywordTok{extends}\NormalTok{ B }\OperatorTok{\{\}}
\end{Highlighting}
\end{Shaded}

\subsubsection{Hierarchical}\label{hierarchical}

When many classes inherit one single class. (A -\textgreater{} B, A
-\textgreater{} C)

\begin{Shaded}
\begin{Highlighting}[]
\KeywordTok{class}\NormalTok{ A }\OperatorTok{\{\}}

\KeywordTok{class}\NormalTok{ B }\KeywordTok{extends}\NormalTok{ A }\OperatorTok{\{\}}

\KeywordTok{class}\NormalTok{ C }\KeywordTok{extends}\NormalTok{ A }\OperatorTok{\{\}}
\end{Highlighting}
\end{Shaded}

\subsubsection{Multiple}\label{multiple}

When one class inherits from many classes.

Java does not allow for multiple inheritance using classes. This is to
avoid the ``Diamond Problem'' (ambiguity if two parent classes have a
method with the same name).

We must use interface to achieve this.

\begin{Shaded}
\begin{Highlighting}[]
\KeywordTok{class}\NormalTok{ A }\OperatorTok{\{\}}

\KeywordTok{class}\NormalTok{ B }\OperatorTok{\{\}}

\CommentTok{// class C extends A,B \{\} // This does not work. Compilation Error.}
\end{Highlighting}
\end{Shaded}

\subsubsection{Hybrid}\label{hybrid}

When we use more than one kind of inheritance at the same time.

Java's class inheritance is hybrid, but without Multiple inheritance. A
common example is Hierarchical + Multi Level.

\begin{Shaded}
\begin{Highlighting}[]
\KeywordTok{class}\NormalTok{ A }\OperatorTok{\{\}} \CommentTok{// Grandparent}

\KeywordTok{class}\NormalTok{ B }\KeywordTok{extends}\NormalTok{ A }\OperatorTok{\{\}} \CommentTok{// Parent 1}

\KeywordTok{class}\NormalTok{ C }\KeywordTok{extends}\NormalTok{ B }\OperatorTok{\{\}} \CommentTok{// Child (Multi{-}level)}

\KeywordTok{class}\NormalTok{ D }\KeywordTok{extends}\NormalTok{ A }\OperatorTok{\{\}} \CommentTok{// Parent 2 (Hierarchical)}
\end{Highlighting}
\end{Shaded}

\subsection{Method Overriding}\label{method-overriding}

When a child class provides a specific implementation for a method that
is already defined in its parent class.

The method signature (name, parameters) must be the same.

The @Override annotation is used to tell the compiler we intend to
override a method.

\begin{Shaded}
\begin{Highlighting}[]
\KeywordTok{class}\NormalTok{ Animal }\OperatorTok{\{}
    \DataTypeTok{void} \FunctionTok{makeSound}\OperatorTok{()} \OperatorTok{\{}
        \BuiltInTok{System}\OperatorTok{.}\FunctionTok{out}\OperatorTok{.}\FunctionTok{println}\OperatorTok{(}\StringTok{"Generic animal sound"}\OperatorTok{);}
    \OperatorTok{\}}
\OperatorTok{\}}

\KeywordTok{class}\NormalTok{ Dog }\KeywordTok{extends}\NormalTok{ Animal }\OperatorTok{\{}
    \AttributeTok{@Override}
    \DataTypeTok{void} \FunctionTok{makeSound}\OperatorTok{()} \OperatorTok{\{}
        \BuiltInTok{System}\OperatorTok{.}\FunctionTok{out}\OperatorTok{.}\FunctionTok{println}\OperatorTok{(}\StringTok{"Woof"}\OperatorTok{);}
    \OperatorTok{\}}
\OperatorTok{\}}
\end{Highlighting}
\end{Shaded}

\subsection{\texorpdfstring{\texttt{super}
Keyword}{super Keyword}}\label{super-keyword}

\texttt{super} is a reference variable used within a child class to
refer to the immediate parent class object.

It is used in two ways:

\begin{enumerate}
\def\labelenumi{\arabic{enumi}.}
\tightlist
\item
  To call the super class's constructor.
\item
  To access the super class's members (variables or methods).
\end{enumerate}

\begin{Shaded}
\begin{Highlighting}[]
\KeywordTok{super}\OperatorTok{.}\FunctionTok{member} \CommentTok{// e.g., super.variable, super.method()}
\KeywordTok{super}\OperatorTok{()}      \CommentTok{// e.g., super(), super(parameter)}
\end{Highlighting}
\end{Shaded}

The first one is used for calling the super class's members. This is
most useful when a child class has overridden a method and needs to call
the parent's version.

The second one is used for calling the super class's constructor.

\begin{itemize}
\tightlist
\item
  \texttt{super()} \emph{must} be the first statement in a child class's
  constructor.
\item
  If you do not explicitly call \texttt{super()}, the compiler
  automatically inserts a call to the parent's non-parameterized
  \texttt{super()} constructor.
\end{itemize}

\begin{Shaded}
\begin{Highlighting}[]
\KeywordTok{class}\NormalTok{ Parent }\OperatorTok{\{}
    \BuiltInTok{String}\NormalTok{ name}\OperatorTok{;}

    \FunctionTok{Parent}\OperatorTok{(}\BuiltInTok{String}\NormalTok{ name}\OperatorTok{)} \OperatorTok{\{}
        \KeywordTok{this}\OperatorTok{.}\FunctionTok{name} \OperatorTok{=}\NormalTok{ name}\OperatorTok{;}
    \OperatorTok{\}}

    \DataTypeTok{void} \FunctionTok{print}\OperatorTok{()} \OperatorTok{\{}
        \BuiltInTok{System}\OperatorTok{.}\FunctionTok{out}\OperatorTok{.}\FunctionTok{println}\OperatorTok{(}\StringTok{"Parent print"}\OperatorTok{);}
    \OperatorTok{\}}
\OperatorTok{\}}

\KeywordTok{class}\NormalTok{ Child }\KeywordTok{extends}\NormalTok{ Parent }\OperatorTok{\{}
    \FunctionTok{Child}\OperatorTok{(}\BuiltInTok{String}\NormalTok{ name}\OperatorTok{)} \OperatorTok{\{}
        \CommentTok{// Must call super() because Parent has no default constructor}
        \KeywordTok{super}\OperatorTok{(}\NormalTok{name}\OperatorTok{);}
    \OperatorTok{\}}

    \AttributeTok{@Override}
    \DataTypeTok{void} \FunctionTok{print}\OperatorTok{()} \OperatorTok{\{}
        \KeywordTok{super}\OperatorTok{.}\FunctionTok{print}\OperatorTok{();} \CommentTok{// Calls Parent\textquotesingle{}s print() method}
        \BuiltInTok{System}\OperatorTok{.}\FunctionTok{out}\OperatorTok{.}\FunctionTok{println}\OperatorTok{(}\StringTok{"Child print"}\OperatorTok{);}
        \BuiltInTok{System}\OperatorTok{.}\FunctionTok{out}\OperatorTok{.}\FunctionTok{println}\OperatorTok{(}\KeywordTok{super}\OperatorTok{.}\FunctionTok{name}\OperatorTok{);} \CommentTok{// Accesses Parent\textquotesingle{}s variable}
    \OperatorTok{\}}
\OperatorTok{\}}
\end{Highlighting}
\end{Shaded}

\newpage

\chapter{Polymorphism}\label{polymorphism}

\etocsettocstyle{\textbf{Chapter Contents}\par\rule{\linewidth}{0.5pt}}{\par\rule{\linewidth}{0.5pt}}
\localtableofcontents

\noindent

``Many Forms''

Polymorphism is an OOP mechanism where an object (or method) can take on
many different forms. Its main advantage is code reusability and
flexibility.

\begin{tipblock}{Analogy: One Person, Many Roles}

Consider a man named \textbf{John}.

\begin{itemize}
\tightlist
\item
  At home, he behaves as a \textbf{Father}.
\item
  At work, he behaves as an \textbf{Employee}.
\item
  At the store, he behaves as a \textbf{Customer}.
\end{itemize}

\textbf{John} is the same object, but his behavior changes based on the
context (who is interacting with him). This is Polymorphism.

\end{tipblock}

It is achieved by two ways:

\begin{itemize}
\tightlist
\item
  \textbf{Compile-Time Polymorphism (Static Binding):} Achieved via
  \textbf{Method Overloading}.
\item
  \textbf{Run-Time Polymorphism (Dynamic Binding):} Achieved via
  \textbf{Method Overriding}.
\end{itemize}

\subsection{Compile-Time Polymorphism (Method
Overloading)}\label{compile-time-polymorphism-method-overloading}

When a class has multiple methods with the same name but different
parameters, it is known as method overloading. The correct method to
call is decided at \textbf{compile-time} based on the arguments
provided.

Method overloading is distinguished by:

\begin{itemize}
\tightlist
\item
  The \textbf{no. of parameters}
\item
  The \textbf{datatypes} of parameters
\item
  The \textbf{sequence} of parameters of different types
\end{itemize}

\textbf{Note:} The return type alone is \textbf{not} enough to overload
a method.

\begin{Shaded}
\begin{Highlighting}[]
\KeywordTok{class}\NormalTok{ Calculator }\OperatorTok{\{}
    \DataTypeTok{void} \FunctionTok{add}\OperatorTok{(}\DataTypeTok{int}\NormalTok{ a}\OperatorTok{,} \DataTypeTok{int}\NormalTok{ b}\OperatorTok{)} \OperatorTok{\{}
        \BuiltInTok{System}\OperatorTok{.}\FunctionTok{out}\OperatorTok{.}\FunctionTok{println}\OperatorTok{(}\NormalTok{a }\OperatorTok{+}\NormalTok{ b}\OperatorTok{);}
    \OperatorTok{\}}

    \DataTypeTok{void} \FunctionTok{add}\OperatorTok{(}\DataTypeTok{int}\NormalTok{ a}\OperatorTok{,} \DataTypeTok{int}\NormalTok{ b}\OperatorTok{,} \DataTypeTok{int}\NormalTok{ c}\OperatorTok{)} \OperatorTok{\{} \CommentTok{// Different no. of parameters}
        \BuiltInTok{System}\OperatorTok{.}\FunctionTok{out}\OperatorTok{.}\FunctionTok{println}\OperatorTok{(}\NormalTok{a }\OperatorTok{+}\NormalTok{ b }\OperatorTok{+}\NormalTok{ c}\OperatorTok{);}
    \OperatorTok{\}}

    \DataTypeTok{void} \FunctionTok{add}\OperatorTok{(}\DataTypeTok{double}\NormalTok{ a}\OperatorTok{,} \DataTypeTok{double}\NormalTok{ b}\OperatorTok{)} \OperatorTok{\{} \CommentTok{// Different datatypes}
        \BuiltInTok{System}\OperatorTok{.}\FunctionTok{out}\OperatorTok{.}\FunctionTok{println}\OperatorTok{(}\NormalTok{a }\OperatorTok{+}\NormalTok{ b}\OperatorTok{);}
    \OperatorTok{\}}
\OperatorTok{\}}
\end{Highlighting}
\end{Shaded}

\subsubsection{Constructor Overloading}\label{constructor-overloading}

We can also overload constructors in the same way to provide different
ways of initializing an object.

\begin{Shaded}
\begin{Highlighting}[]
\KeywordTok{class}\NormalTok{ A }\OperatorTok{\{}
    \FunctionTok{A}\OperatorTok{()\{}
        \CommentTok{// non{-}parameterized constructor}
    \OperatorTok{\}}
    \FunctionTok{A}\OperatorTok{(}\DataTypeTok{int}\NormalTok{ a}\OperatorTok{)\{}
        \CommentTok{// parameterized constructor}
    \OperatorTok{\}}

    \CommentTok{// To call a constructor from another constructor in the same class:}
    \FunctionTok{A}\OperatorTok{(}\DataTypeTok{int}\NormalTok{ a}\OperatorTok{,} \DataTypeTok{int}\NormalTok{ b}\OperatorTok{)} \OperatorTok{\{}
        \KeywordTok{this}\OperatorTok{(}\NormalTok{a}\OperatorTok{);} \CommentTok{// \textquotesingle{}this()\textquotesingle{} calls another constructor}
        \CommentTok{// ...}
    \OperatorTok{\}}
\OperatorTok{\}}
\end{Highlighting}
\end{Shaded}

\subsection{Run-Time Polymorphism (Method
Overriding)}\label{run-time-polymorphism-method-overriding}

When a method in a \textbf{subclass} has the same name, parameters, and
return type as a method in its \textbf{superclass}, it is known as
method overriding.

This is also known as dynamic polymorphism because the actual method
call is resolved at \textbf{run-time}, based on the type of the
\emph{object}, not the type of the \emph{reference}.

This requires an ``IS-A'' relationship
(\hyperref[inheritance]{Inheritance}).

\subsubsection{Rules for Method
Overriding}\label{rules-for-method-overriding}

\begin{itemize}
\tightlist
\item
  The method signature (name and parameters) must be identical.
\item
  The return type must be the same or a \emph{covariant} type (a
  subclass of the original return type).
\item
  The access modifier in the child class must be the same or \emph{less
  restrictive} (e.g., \texttt{protected} in parent can be
  \texttt{public} in child, but not \texttt{private}).
\item
  \texttt{final} or \texttt{static} methods cannot be overridden.
\end{itemize}

\subsubsection{Dynamic Method Dispatch}\label{dynamic-method-dispatch}

This is the mechanism that makes run-time polymorphism work. We can use
a parent class reference to hold a child class object.

\begin{Shaded}
\begin{Highlighting}[]
\KeywordTok{class}\NormalTok{ Animal }\OperatorTok{\{} \CommentTok{// Parent class}
    \DataTypeTok{void} \FunctionTok{sound}\OperatorTok{()} \OperatorTok{\{}
        \BuiltInTok{System}\OperatorTok{.}\FunctionTok{out}\OperatorTok{.}\FunctionTok{println}\OperatorTok{(}\StringTok{"Animal makes a sound"}\OperatorTok{);}
    \OperatorTok{\}}
\OperatorTok{\}}

\KeywordTok{class}\NormalTok{ Dog }\KeywordTok{extends}\NormalTok{ Animal }\OperatorTok{\{} \CommentTok{// Child class}
    \AttributeTok{@Override}
    \DataTypeTok{void} \FunctionTok{sound}\OperatorTok{()} \OperatorTok{\{}
        \CommentTok{// \textquotesingle{}super\textquotesingle{} keyword calls the parent\textquotesingle{}s method}
        \CommentTok{// super.sound();}
        \BuiltInTok{System}\OperatorTok{.}\FunctionTok{out}\OperatorTok{.}\FunctionTok{println}\OperatorTok{(}\StringTok{"Woof! Woof!"}\OperatorTok{);}
    \OperatorTok{\}}
\OperatorTok{\}}

\KeywordTok{class}\NormalTok{ Cat }\KeywordTok{extends}\NormalTok{ Animal }\OperatorTok{\{} \CommentTok{// Child class}
    \AttributeTok{@Override}
    \DataTypeTok{void} \FunctionTok{sound}\OperatorTok{()} \OperatorTok{\{}
        \BuiltInTok{System}\OperatorTok{.}\FunctionTok{out}\OperatorTok{.}\FunctionTok{println}\OperatorTok{(}\StringTok{"Meow!"}\OperatorTok{);}
    \OperatorTok{\}}
\OperatorTok{\}}

\KeywordTok{public} \KeywordTok{class}\NormalTok{ Main }\OperatorTok{\{}
    \KeywordTok{public} \DataTypeTok{static} \DataTypeTok{void} \FunctionTok{main}\OperatorTok{(}\BuiltInTok{String}\OperatorTok{[]}\NormalTok{ args}\OperatorTok{)} \OperatorTok{\{}
\NormalTok{        Animal a}\OperatorTok{;} \CommentTok{// Parent class reference}

\NormalTok{        a }\OperatorTok{=} \KeywordTok{new} \FunctionTok{Dog}\OperatorTok{();} \CommentTok{// Dynamic Polymorphic Assignment}
\NormalTok{        a}\OperatorTok{.}\FunctionTok{sound}\OperatorTok{();}     \CommentTok{// Output: Woof! Woof! (Java checks the object type at run{-}time)}

\NormalTok{        a }\OperatorTok{=} \KeywordTok{new} \FunctionTok{Cat}\OperatorTok{();} \CommentTok{// Re{-}assign to another subclass}
\NormalTok{        a}\OperatorTok{.}\FunctionTok{sound}\OperatorTok{();}     \CommentTok{// Output: Meow! (Java checks the object type again)}
    \OperatorTok{\}}
\OperatorTok{\}}
\end{Highlighting}
\end{Shaded}

Here, even though the reference \texttt{a} is of type \texttt{Animal},
the JVM calls the \texttt{sound()} method from the \emph{actual object}
(\texttt{Dog} or \texttt{Cat}) at run-time.

\newpage

\chapter{Encapsulation}\label{encapsulation}

\etocsettocstyle{\textbf{Chapter Contents}\par\rule{\linewidth}{0.5pt}}{\par\rule{\linewidth}{0.5pt}}
\localtableofcontents

\noindent

It is a fundamental OOP principle used for \textbf{data hiding}.
Encapsulation is a mechanism in which data (attributes or variables) and
the methods that operate on that data are bundled together within a
single unit (a class).

We make a ``capsule'' of a class. What's inside the capsule (the
internal state) is not available for direct access from outside the
class.

Attributes are bound to the methods. We cannot get or set the values of
the attributes directly by doing \texttt{obj.var}. We must use accessor
methods like \texttt{obj.getVar()} (a \textbf{getter}) or
\texttt{obj.setVar(value)} (a \textbf{setter}).

Encapsulation provides: - \textbf{Data Hiding (Security):} The internal
state of the object is hidden. This prevents outside code from
accidentally or maliciously corrupting the object's data. -
\textbf{Control \& Data Integrity:} By using setter methods, we can add
validation logic. Data remains authentic because we control \emph{how}
it is set. - \textbf{Flexibility \& Modularity:} The implementation
details are hidden. We can change the internal implementation (e.g.,
rename a variable, change its data type) without breaking the external
code that uses our class, as long as the public getter/setter methods
remain the same. - \textbf{Code Reusability:} Encapsulated classes are
easier to reuse as self-contained ``black boxes.''

\subsection{How to Achieve
Encapsulation}\label{how-to-achieve-encapsulation}

\begin{enumerate}
\def\labelenumi{\arabic{enumi}.}
\tightlist
\item
  Declare the class's variables (fields) as \texttt{private}.
\item
  Provide public \textbf{getter} (accessor) methods to view the
  variables.
\item
  Provide public \textbf{setter} (mutator) methods to modify the
  variables.
\end{enumerate}

Setters are where we enforce integrity constraints.

For example:

\begin{Shaded}
\begin{Highlighting}[]
\KeywordTok{class}\NormalTok{ Student }\OperatorTok{\{}
    \CommentTok{// 1. Declare variable as private}
    \KeywordTok{private} \DataTypeTok{int}\NormalTok{ age}\OperatorTok{;}
    
    \CommentTok{// 3. Public setter to initialize/modify the variable}
    \KeywordTok{public} \DataTypeTok{void} \FunctionTok{setAge}\OperatorTok{(}\DataTypeTok{int}\NormalTok{ age}\OperatorTok{)} \OperatorTok{\{}
        \CommentTok{// We add validation logic (integrity constraint)}
        \ControlFlowTok{if} \OperatorTok{(}\NormalTok{age }\OperatorTok{\textgreater{}} \DecValTok{0} \OperatorTok{\&\&}\NormalTok{ age }\OperatorTok{\textless{}} \DecValTok{150}\OperatorTok{)} \OperatorTok{\{}
            \KeywordTok{this}\OperatorTok{.}\FunctionTok{age} \OperatorTok{=}\NormalTok{ age}\OperatorTok{;}
        \OperatorTok{\}} \ControlFlowTok{else} \OperatorTok{\{}
            \BuiltInTok{System}\OperatorTok{.}\FunctionTok{out}\OperatorTok{.}\FunctionTok{println}\OperatorTok{(}\StringTok{"Invalid age provided."}\OperatorTok{);}
            \CommentTok{// Or throw new IllegalArgumentException("Invalid age");}
        \OperatorTok{\}}
    \OperatorTok{\}}
    
    \CommentTok{// 2. Public getter to access the variable}
    \KeywordTok{public} \DataTypeTok{int} \FunctionTok{getAge}\OperatorTok{()} \OperatorTok{\{}
        \ControlFlowTok{return} \KeywordTok{this}\OperatorTok{.}\FunctionTok{age}\OperatorTok{;}
    \OperatorTok{\}}
\OperatorTok{\}}

\KeywordTok{class}\NormalTok{ Main }\OperatorTok{\{}
    \KeywordTok{public} \DataTypeTok{static} \DataTypeTok{void} \FunctionTok{main}\OperatorTok{(}\BuiltInTok{String}\OperatorTok{[]}\NormalTok{ args}\OperatorTok{)} \OperatorTok{\{}
\NormalTok{        Student s1 }\OperatorTok{=} \KeywordTok{new} \FunctionTok{Student}\OperatorTok{();}
        
        \CommentTok{// s1.age = {-}10; // Error: \textquotesingle{}age\textquotesingle{} has private access}
        
\NormalTok{        s1}\OperatorTok{.}\FunctionTok{setAge}\OperatorTok{(}\DecValTok{20}\OperatorTok{);}
        \BuiltInTok{System}\OperatorTok{.}\FunctionTok{out}\OperatorTok{.}\FunctionTok{println}\OperatorTok{(}\NormalTok{s1}\OperatorTok{.}\FunctionTok{getAge}\OperatorTok{());} \CommentTok{// Output: 20}
        
\NormalTok{        s1}\OperatorTok{.}\FunctionTok{setAge}\OperatorTok{({-}}\DecValTok{10}\OperatorTok{);} \CommentTok{// Output: Invalid age provided.}
        \BuiltInTok{System}\OperatorTok{.}\FunctionTok{out}\OperatorTok{.}\FunctionTok{println}\OperatorTok{(}\NormalTok{s1}\OperatorTok{.}\FunctionTok{getAge}\OperatorTok{());} \CommentTok{// Output: 20 (The value didn\textquotesingle{}t change)}
    \OperatorTok{\}}
\OperatorTok{\}}
\end{Highlighting}
\end{Shaded}

The encapsulation is: - \textbf{Read-Only:} If we only provide a
\texttt{getter}, the variable becomes read-only from the outside. -
\textbf{Write-Only:} If we only provide a \texttt{setter}, the variable
becomes write-only.

\newpage

\chapter{Abstraction}\label{abstraction}

\etocsettocstyle{\textbf{Chapter Contents}\par\rule{\linewidth}{0.5pt}}{\par\rule{\linewidth}{0.5pt}}
\localtableofcontents

\noindent

It is a concept which is used for \textbf{hiding implementation
complexity} and showing only the essential features or functionality to
the user.

(Note: \textbf{Encapsulation} is used for \emph{data hiding} (protecting
variables). \textbf{Abstraction} is used for \emph{implementation
hiding} (hiding \emph{how} a method works)).

Abstraction focuses on \emph{what} an object does, not \emph{how} it
does it.

Abstraction can be achieved using an abstract class or by an interface.

\subsection{Abstract Class}\label{abstract-class}

We use \texttt{abstract} keyword to declare a class as abstract. An
abstract class provides a template for other classes, forcing them to
implement certain methods.

You \textbf{cannot} create an object (instantiate) of an abstract class.

\begin{itemize}
\tightlist
\item
  We can declare abstract as well as non-abstract (concrete) methods.\\
\item
  It can have constructors, static methods, and \texttt{final} methods.
\item
  It works together with
  \hyperref[polymorphismrun-time-polymorphism-method-overriding]{Polymorphism\#Run-Time Polymorphism (Method Overriding)}.
\end{itemize}

Abstract methods are declared with the \texttt{abstract} keyword. These
methods don't have a definition or body in the class in which they are
declared. They \textbf{must} be overridden by a subclass to provide the
definition.

\begin{Shaded}
\begin{Highlighting}[]
\CommentTok{// We cannot create an object of \textquotesingle{}Shape\textquotesingle{}}
\KeywordTok{abstract} \KeywordTok{class} \BuiltInTok{Shape} \OperatorTok{\{}
    \DataTypeTok{int}\NormalTok{ color}\OperatorTok{;}
    
    \CommentTok{// Concrete (non{-}abstract) method with a body}
    \DataTypeTok{void} \FunctionTok{setColor}\OperatorTok{(}\DataTypeTok{int}\NormalTok{ c}\OperatorTok{)} \OperatorTok{\{}
        \KeywordTok{this}\OperatorTok{.}\FunctionTok{color} \OperatorTok{=}\NormalTok{ c}\OperatorTok{;}
    \OperatorTok{\}}

    \CommentTok{// Abstract method {-} no body}
    \CommentTok{// Forces subclasses to provide their own version}
    \KeywordTok{abstract} \DataTypeTok{void} \FunctionTok{draw}\OperatorTok{();} 
\OperatorTok{\}}

\KeywordTok{class}\NormalTok{ Circle }\KeywordTok{extends} \BuiltInTok{Shape} \OperatorTok{\{}
    \CommentTok{// We MUST implement the abstract \textquotesingle{}draw\textquotesingle{} method}
    \AttributeTok{@Override}
    \DataTypeTok{void} \FunctionTok{draw}\OperatorTok{()} \OperatorTok{\{}
        \BuiltInTok{System}\OperatorTok{.}\FunctionTok{out}\OperatorTok{.}\FunctionTok{println}\OperatorTok{(}\StringTok{"Drawing a circle"}\OperatorTok{);}
    \OperatorTok{\}}
\OperatorTok{\}}
\end{Highlighting}
\end{Shaded}

\subsection{Interface}\label{interface}

An interface is a blueprint of a class. It specifies what a class must
do, but not how.

It is used to achieve 100\% abstraction (before Java 8).

\begin{itemize}
\tightlist
\item
  We use the \texttt{interface} keyword.
\item
  All methods in an interface are \texttt{public} and \texttt{abstract}
  by default.
\item
  All variables (fields) are \texttt{public}, \texttt{static}, and
  \texttt{final} (constants) by default.
\item
  A class uses the \texttt{implements} keyword to use an interface.
\item
  A class can implement \textbf{multiple} interfaces. This is how Java
  achieves multiple inheritance.
\end{itemize}

\begin{Shaded}
\begin{Highlighting}[]
\KeywordTok{interface}\NormalTok{ Drawable }\OperatorTok{\{}
    \DataTypeTok{void} \FunctionTok{draw}\OperatorTok{();} \CommentTok{// This is public and abstract by default}
\OperatorTok{\}}

\KeywordTok{interface}\NormalTok{ Loggable }\OperatorTok{\{}
    \DataTypeTok{void} \FunctionTok{log}\OperatorTok{(}\BuiltInTok{String}\NormalTok{ message}\OperatorTok{);}
\OperatorTok{\}}

\CommentTok{// A class can implement multiple interfaces}
\KeywordTok{class}\NormalTok{ Circle }\KeywordTok{implements}\NormalTok{ Drawable}\OperatorTok{,}\NormalTok{ Loggable }\OperatorTok{\{}
    \AttributeTok{@Override}
    \KeywordTok{public} \DataTypeTok{void} \FunctionTok{draw}\OperatorTok{()} \OperatorTok{\{}
        \BuiltInTok{System}\OperatorTok{.}\FunctionTok{out}\OperatorTok{.}\FunctionTok{println}\OperatorTok{(}\StringTok{"Drawing a circle"}\OperatorTok{);}
    \OperatorTok{\}}
    
    \AttributeTok{@Override}
    \KeywordTok{public} \DataTypeTok{void} \FunctionTok{log}\OperatorTok{(}\BuiltInTok{String}\NormalTok{ message}\OperatorTok{)} \OperatorTok{\{}
        \BuiltInTok{System}\OperatorTok{.}\FunctionTok{out}\OperatorTok{.}\FunctionTok{println}\OperatorTok{(}\StringTok{"LOG: "} \OperatorTok{+}\NormalTok{ message}\OperatorTok{);}
    \OperatorTok{\}}
\OperatorTok{\}}
\end{Highlighting}
\end{Shaded}

\subsubsection{\texorpdfstring{\texttt{default} and \texttt{static}
methods}{default and static methods}}\label{default-and-static-methods}

Java 8 allowed interfaces to have methods with implementation:

\begin{itemize}
\tightlist
\item
  \textbf{\texttt{default} methods:} A class can override them, but
  doesn't have to. It allows adding new functionality to an interface
  without breaking existing classes.
\item
  \textbf{\texttt{static} methods:} Utility methods that are part of the
  interface, not a specific object.
\end{itemize}

\paragraph{Default Methods}\label{default-methods}

Allows adding new methods to interfaces with a default implementation,
without breaking existing classes that implement the interface.

We use the default keyword.

\begin{Shaded}
\begin{Highlighting}[]
\KeywordTok{interface}\NormalTok{ MyInterface }\OperatorTok{\{}
    \DataTypeTok{void} \FunctionTok{existingMethod}\OperatorTok{();}
    
    \KeywordTok{default} \DataTypeTok{void} \FunctionTok{newDefaultMethod}\OperatorTok{()} \OperatorTok{\{}
        \BuiltInTok{System}\OperatorTok{.}\FunctionTok{out}\OperatorTok{.}\FunctionTok{println}\OperatorTok{(}\StringTok{"This is a default implementation."}\OperatorTok{);}
    \OperatorTok{\}}
\OperatorTok{\}}

\KeywordTok{class}\NormalTok{ MyClass }\KeywordTok{implements}\NormalTok{ MyInterface }\OperatorTok{\{}
    \KeywordTok{public} \DataTypeTok{void} \FunctionTok{existingMethod}\OperatorTok{()} \OperatorTok{\{}
        \CommentTok{// ...}
    \OperatorTok{\}}
    \CommentTok{// No need to implement newDefaultMethod(), it\textquotesingle{}s optional.}
\OperatorTok{\}}
\end{Highlighting}
\end{Shaded}

\paragraph{Static Method}\label{static-method-1}

Interfaces can now have static methods. These are utility methods that
are part of the interface, not the implementing class.

They are called on the interface itself, not on an instance.

\begin{Shaded}
\begin{Highlighting}[]
\KeywordTok{interface}\NormalTok{ MyInterface }\OperatorTok{\{}
    \DataTypeTok{static} \DataTypeTok{void} \FunctionTok{utilityMethod}\OperatorTok{()} \OperatorTok{\{}
        \BuiltInTok{System}\OperatorTok{.}\FunctionTok{out}\OperatorTok{.}\FunctionTok{println}\OperatorTok{(}\StringTok{"This is a static utility method."}\OperatorTok{);}
    \OperatorTok{\}}
\OperatorTok{\}}

\CommentTok{// How to call it:}
\NormalTok{MyInterface}\OperatorTok{.}\FunctionTok{utilityMethod}\OperatorTok{();}
\end{Highlighting}
\end{Shaded}

\subsection{Abstract Class
vs.~Interface}\label{abstract-class-vs.-interface}

\begin{itemize}
\item
  \textbf{Methods:} An abstract class can have both abstract and
  concrete methods. An interface (pre-Java 8) can only have abstract
  methods.
\item
  \textbf{Variables:} An abstract class can have any type of variable
  (instance, static, final). An interface can only have
  \texttt{public\ static\ final} constants.
\item
  \textbf{Inheritance:} A class can \texttt{extends} only \textbf{one}
  abstract class. A class can \texttt{implements} \textbf{many}
  interfaces.
\item
  \textbf{Constructor:} An abstract class can have a constructor (which
  is called by the subclass constructor). An interface \textbf{cannot}
  have a constructor.
\item
  \textbf{Purpose:}

  \begin{itemize}
  \tightlist
  \item
    Use an \textbf{abstract class} for an ``IS-A'' relationship, when
    you want to provide common, shared code for all subclasses (e.g.,
    \texttt{Dog} IS-A \texttt{Animal}, and all animals
    \texttt{breathe()}).
  \item
    Use an \textbf{interface} for a ``CAN-DO'' relationship, when you
    want to define a capability or contract that unrelated classes can
    share (e.g., \texttt{Circle} CAN-DO \texttt{Drawable}, \texttt{Car}
    CAN-DO \texttt{Drawable}).
  \end{itemize}
\end{itemize}

\newpage

\chapter{Exception Handling}\label{exception-handling}

\etocsettocstyle{\textbf{Chapter Contents}\par\rule{\linewidth}{0.5pt}}{\par\rule{\linewidth}{0.5pt}}
\localtableofcontents

\noindent

Java is robust i.e.~it can deal with the errors.

Exception handling is a mechanism through which we catch and manage
errors or unusual events during run time that disrupts the normal flow
of the program.

Exceptions are run-time errors.

\begin{tipblock}{Analogy: The Safety Net}

\begin{itemize}
\tightlist
\item
  \textbf{Program Execution:} Walking on a tightrope.
\item
  \textbf{Exception:} Slipping and falling.
\item
  \textbf{Default Behavior (No Handling):} You fall to the ground and
  crash (Program Terminates/Crashes).
\item
  \textbf{Exception Handling (Try-Catch):} You have a \textbf{Safety
  Net} below. If you fall, the net catches you, and you can climb back
  down safely (Graceful Termination or Recovery) instead of dying.
\end{itemize}

\end{tipblock}

To handle an exception, we can either:

\begin{itemize}
\tightlist
\item
  \textbf{Handle it:} Using a \texttt{try-catch} block.
\item
  \textbf{Duck it (Declare it):} Using the \texttt{throws} keyword,
  forcing the calling method to handle it.
\end{itemize}

\(\\\)

\begin{figure}
\centering
\includegraphics[width=\linewidth,height=12cm,keepaspectratio,alt={~}]{images/mermaid_e8324c84be579f49405a4b2f588d0f6a.png}
\caption{~}
\end{figure}

\textbf{Throwable:} The root class for everything that can be thrown.

\begin{itemize}
\item
  \textbf{Error:} Represents serious, unrecoverable problems that should
  not be caught (e.g., \texttt{OutOfMemoryError},
  \texttt{StackOverflowError}).
\item
  \textbf{Exception:} Represents conditions that a program might want to
  catch and handle.
\end{itemize}

There are two types of exceptions:

\begin{itemize}
\item
  \textbf{Checked:} Exceptions that are checked at compile-time. The
  compiler forces you to handle them (with try-catch or throws).

  E.g. \texttt{IOException}, \texttt{SQLException},
  \texttt{FileNotFoundException}, \texttt{ClassNotFoundException}.
\item
  \textbf{Unchecked (RuntimeExceptions):} Exceptions that are checked at
  run-time. They are typically programming errors and the compiler does
  not force you to handle them.

  E.g. \texttt{ArithmeticException},
  \texttt{ArrayIndexOutOfBoundsException},
  \texttt{NumberFormatException}, \texttt{NullPointerException}.
\end{itemize}

\subsection{\texorpdfstring{Using \texttt{try\ -\ catch\ -\ finally}
block}{Using try - catch - finally block}}\label{using-try---catch---finally-block}

\begin{itemize}
\tightlist
\item
  \textbf{\texttt{try}:} The code which might throw an exception is
  written inside the \texttt{try} block.
\item
  \textbf{\texttt{catch}:} The handler for the exception. If an
  exception of the specified type occurs in \texttt{try}, this block is
  executed.
\item
  \textbf{\texttt{finally}:} This block is \textbf{always} executed,
  whether an exception was thrown or not. It's used for cleanup (e.g.,
  closing a file or database connection).
\end{itemize}

\begin{Shaded}
\begin{Highlighting}[]
\ControlFlowTok{try} \OperatorTok{\{}
    \CommentTok{// Code that might cause an error}
    \DataTypeTok{int}\OperatorTok{[]}\NormalTok{ myNumbers }\OperatorTok{=} \OperatorTok{\{}\DecValTok{1}\OperatorTok{,} \DecValTok{2}\OperatorTok{,} \DecValTok{3}\OperatorTok{\};}
    \BuiltInTok{System}\OperatorTok{.}\FunctionTok{out}\OperatorTok{.}\FunctionTok{println}\OperatorTok{(}\NormalTok{myNumbers}\OperatorTok{[}\DecValTok{10}\OperatorTok{]);} \CommentTok{// This throws an exception}
\OperatorTok{\}} \ControlFlowTok{catch} \OperatorTok{(}\BuiltInTok{ArrayIndexOutOfBoundsException}\NormalTok{ e}\OperatorTok{)} \OperatorTok{\{}
    \CommentTok{// Handler for that specific error}
    \BuiltInTok{System}\OperatorTok{.}\FunctionTok{out}\OperatorTok{.}\FunctionTok{println}\OperatorTok{(}\StringTok{"Error: Index is out of bounds."}\OperatorTok{);}
\OperatorTok{\}} \ControlFlowTok{catch} \OperatorTok{(}\BuiltInTok{Exception}\NormalTok{ e}\OperatorTok{)} \OperatorTok{\{}
    \CommentTok{// A general handler for any other exception}
    \BuiltInTok{System}\OperatorTok{.}\FunctionTok{out}\OperatorTok{.}\FunctionTok{println}\OperatorTok{(}\StringTok{"Something else went wrong."}\OperatorTok{);}
\OperatorTok{\}} \ControlFlowTok{finally} \OperatorTok{\{}
    \CommentTok{// This code always runs}
    \BuiltInTok{System}\OperatorTok{.}\FunctionTok{out}\OperatorTok{.}\FunctionTok{println}\OperatorTok{(}\StringTok{"The \textquotesingle{}try{-}catch\textquotesingle{} block is finished."}\OperatorTok{);}
\OperatorTok{\}}
\end{Highlighting}
\end{Shaded}

\subsection{\texorpdfstring{\texttt{throw}
Keyword}{throw Keyword}}\label{throw-keyword}

The \texttt{throw} keyword is used to \textbf{manually} or
\textbf{explicitly} throw an exception (or any \texttt{Throwable}
object) from within a method body.

You use it when your program logic detects an error condition.

\begin{Shaded}
\begin{Highlighting}[]
\KeywordTok{public} \DataTypeTok{void} \FunctionTok{validateAge}\OperatorTok{(}\DataTypeTok{int}\NormalTok{ age}\OperatorTok{)} \OperatorTok{\{}
    \ControlFlowTok{if} \OperatorTok{(}\NormalTok{age }\OperatorTok{\textless{}} \DecValTok{18}\OperatorTok{)} \OperatorTok{\{}
        \CommentTok{// We create a new exception object and "throw" it.}
        \CommentTok{// This immediately stops the method.}
        \ControlFlowTok{throw} \KeywordTok{new} \BuiltInTok{ArithmeticException}\OperatorTok{(}\StringTok{"Access denied: You must be 18 or older."}\OperatorTok{);}
    \OperatorTok{\}} \ControlFlowTok{else} \OperatorTok{\{}
        \BuiltInTok{System}\OperatorTok{.}\FunctionTok{out}\OperatorTok{.}\FunctionTok{println}\OperatorTok{(}\StringTok{"Access granted."}\OperatorTok{);}
    \OperatorTok{\}}
\OperatorTok{\}}

\CommentTok{// How to call it:}
\KeywordTok{public} \DataTypeTok{static} \DataTypeTok{void} \FunctionTok{main}\OperatorTok{(}\BuiltInTok{String}\OperatorTok{[]}\NormalTok{ args}\OperatorTok{)} \OperatorTok{\{}
    \ControlFlowTok{try} \OperatorTok{\{}
        \FunctionTok{validateAge}\OperatorTok{(}\DecValTok{15}\OperatorTok{);}
    \OperatorTok{\}} \ControlFlowTok{catch} \OperatorTok{(}\BuiltInTok{ArithmeticException}\NormalTok{ e}\OperatorTok{)} \OperatorTok{\{}
        \BuiltInTok{System}\OperatorTok{.}\FunctionTok{out}\OperatorTok{.}\FunctionTok{println}\OperatorTok{(}\StringTok{"Caught exception: "} \OperatorTok{+}\NormalTok{ e}\OperatorTok{.}\FunctionTok{getMessage}\OperatorTok{());}
    \OperatorTok{\}}
\OperatorTok{\}}
\end{Highlighting}
\end{Shaded}

\subsection{\texorpdfstring{\texttt{throws}
Keyword}{throws Keyword}}\label{throws-keyword}

The throws keyword is used in a method signature (the method
declaration).

It tells the compiler that this method might throw a checked exception.

It does \emph{not} handle the exception. It ``ducks'' or passes the
responsibility of handling the exception to whatever method \emph{calls}
it.

\begin{Shaded}
\begin{Highlighting}[]
\CommentTok{// This method "throws" a checked exception}
\CommentTok{// It does not handle it, it just warns the caller.}
\KeywordTok{public} \DataTypeTok{void} \FunctionTok{readFile}\OperatorTok{(}\BuiltInTok{String}\NormalTok{ fileName}\OperatorTok{)} \KeywordTok{throws} \BuiltInTok{FileNotFoundException} \OperatorTok{\{}
    \BuiltInTok{File}\NormalTok{ file }\OperatorTok{=} \KeywordTok{new} \BuiltInTok{File}\OperatorTok{(}\NormalTok{fileName}\OperatorTok{);}
    \BuiltInTok{FileInputStream}\NormalTok{ fis }\OperatorTok{=} \KeywordTok{new} \BuiltInTok{FileInputStream}\OperatorTok{(}\NormalTok{file}\OperatorTok{);}
    \CommentTok{// ... logic to read file}
\OperatorTok{\}}

\CommentTok{// The calling method MUST handle it}
\KeywordTok{public} \DataTypeTok{void} \FunctionTok{openMyFile}\OperatorTok{()} \OperatorTok{\{}
    \ControlFlowTok{try} \OperatorTok{\{}
        \FunctionTok{readFile}\OperatorTok{(}\StringTok{"myFile.txt"}\OperatorTok{);}
    \OperatorTok{\}} \ControlFlowTok{catch} \OperatorTok{(}\BuiltInTok{FileNotFoundException}\NormalTok{ e}\OperatorTok{)} \OperatorTok{\{}
        \BuiltInTok{System}\OperatorTok{.}\FunctionTok{out}\OperatorTok{.}\FunctionTok{println}\OperatorTok{(}\StringTok{"Error: The file was not found."}\OperatorTok{);}
    \OperatorTok{\}}
\OperatorTok{\}}
\end{Highlighting}
\end{Shaded}

\subsection{\texorpdfstring{\texttt{throw}
vs.~\texttt{throws}}{throw vs.~throws}}\label{throw-vs.-throws}

{\def\LTcaptype{none} % do not increment counter
\begin{longtable}[]{@{}
  >{\raggedright\arraybackslash}p{(\linewidth - 4\tabcolsep) * \real{0.0873}}
  >{\raggedright\arraybackslash}p{(\linewidth - 4\tabcolsep) * \real{0.4206}}
  >{\raggedright\arraybackslash}p{(\linewidth - 4\tabcolsep) * \real{0.4921}}@{}}
\toprule\noalign{}
\begin{minipage}[b]{\linewidth}\raggedright
Feature
\end{minipage} & \begin{minipage}[b]{\linewidth}\raggedright
\texttt{throw}
\end{minipage} & \begin{minipage}[b]{\linewidth}\raggedright
\texttt{throws}
\end{minipage} \\
\midrule\noalign{}
\endhead
\bottomrule\noalign{}
\endlastfoot
\textbf{Purpose} & To \textbf{explicitly throw} an exception. & To
\textbf{declare} that a method might throw an exception. \\
\textbf{Usage} & Inside a method body. & In the method signature. \\
\textbf{Syntax} & Followed by an \textbf{instance} (object) of an
exception. & Followed by the \textbf{class names} of exceptions. \\
\textbf{Example} & \texttt{throw\ new\ IOException();} &
\texttt{void\ myMethod()\ throws\ IOException\ \{\ ...\ \}} \\
\textbf{Type} & Can throw any \texttt{Throwable}. & Used only for
\textbf{checked exceptions} (unchecked are optional). \\
\end{longtable}
}

\subsection{User Defined Exceptions}\label{user-defined-exceptions}

We can create our own exceptions by extending the \texttt{Exception}
class (for checked) or \texttt{RuntimeException} class (for unchecked).

\begin{Shaded}
\begin{Highlighting}[]
\CommentTok{// 1. Create the exception class}
\KeywordTok{class}\NormalTok{ InsufficientFundsException }\KeywordTok{extends} \BuiltInTok{Exception} \OperatorTok{\{}
    \KeywordTok{public} \FunctionTok{InsufficientFundsException}\OperatorTok{(}\BuiltInTok{String}\NormalTok{ message}\OperatorTok{)} \OperatorTok{\{}
        \KeywordTok{super}\OperatorTok{(}\NormalTok{message}\OperatorTok{);}
    \OperatorTok{\}}
\OperatorTok{\}}

\CommentTok{// 2. Use it}
\KeywordTok{class}\NormalTok{ BankAccount }\OperatorTok{\{}
    \DataTypeTok{double}\NormalTok{ balance}\OperatorTok{;}

    \DataTypeTok{void} \FunctionTok{withdraw}\OperatorTok{(}\DataTypeTok{double}\NormalTok{ amount}\OperatorTok{)} \KeywordTok{throws}\NormalTok{ InsufficientFundsException }\OperatorTok{\{}
        \ControlFlowTok{if} \OperatorTok{(}\NormalTok{amount }\OperatorTok{\textgreater{}}\NormalTok{ balance}\OperatorTok{)} \OperatorTok{\{}
            \ControlFlowTok{throw} \KeywordTok{new} \FunctionTok{InsufficientFundsException}\OperatorTok{(}\StringTok{"Not enough money!"}\OperatorTok{);}
        \OperatorTok{\}}
\NormalTok{        balance }\OperatorTok{{-}=}\NormalTok{ amount}\OperatorTok{;}
    \OperatorTok{\}}
\OperatorTok{\}}
\end{Highlighting}
\end{Shaded}

\subsection{Try with resource}\label{try-with-resource}

We can use try with resource on objects of classes which implement
\texttt{AutoClosable} interface.

\begin{Shaded}
\begin{Highlighting}[]
\CommentTok{// The resource (fis) is automatically closed at the end}
\ControlFlowTok{try} \OperatorTok{(}\BuiltInTok{FileInputStream}\NormalTok{ fis }\OperatorTok{=} \KeywordTok{new} \BuiltInTok{FileInputStream}\OperatorTok{(}\StringTok{"test.txt"}\OperatorTok{))} \OperatorTok{\{}
    \CommentTok{// read file}
\OperatorTok{\}} \ControlFlowTok{catch} \OperatorTok{(}\BuiltInTok{IOException}\NormalTok{ e}\OperatorTok{)} \OperatorTok{\{}
\NormalTok{    e}\OperatorTok{.}\FunctionTok{printStackTrace}\OperatorTok{();}
\OperatorTok{\}}
\end{Highlighting}
\end{Shaded}

We use this when we want to perform an operation on one resource and
then close it. E.g. file I/O.

\newpage

\chapter{Collections}\label{collections}

\etocsettocstyle{\textbf{Chapter Contents}\par\rule{\linewidth}{0.5pt}}{\par\rule{\linewidth}{0.5pt}}
\localtableofcontents

\noindent

Collections are predefined data structures that Java provides.
Collections are a group of objects or entities considered as a single
unit that store data.

We only need to make their objects and use them.

A package in java has either classes or interfaces. Collections have all
of their data structures defined using classes and interfaces in the
\texttt{java.util} package.

The main interfaces are \textbf{List}, \textbf{Queue}, \textbf{Set}, and
\textbf{Map}.

\begin{itemize}
\tightlist
\item
  List, Queue, and Set store single items.
\item
  Map stores data in \textbf{key-value pairs}.
\end{itemize}

\subsubsection{Synchronization}\label{synchronization}

Synchronization (in this context) means \textbf{thread-safety}.

\begin{itemize}
\tightlist
\item
  A synchronized collection can be safely accessed by multiple threads
  at the same time.
\item
  A non-synchronized collection can be corrupted if modified by multiple
  threads simultaneously.
\item
  Most modern collections (\texttt{ArrayList}, \texttt{HashMap}) are
  \textbf{non-synchronized} for better performance in single-threaded
  applications.
\item
  Legacy classes (\texttt{Vector}, \texttt{Hashtable}) are synchronized.
\end{itemize}

\subsubsection{Collection Framework}\label{collection-framework}

The Collection Framework is a group of classes and interfaces that
implements these data structures.

It is divided into two main categories:

\begin{enumerate}
\def\labelenumi{\arabic{enumi}.}
\tightlist
\item
  \texttt{java.util.Collection}: The root interface for lists, sets, and
  queues.
\item
  \texttt{java.util.Map}: A separate interface for key-value pair
  structures.
\end{enumerate}

\begin{figure}
\centering
\includegraphics[width=\linewidth,height=12cm,keepaspectratio,alt={~}]{images/mermaid_1444162a5ef9ccef97f9fa90989fe41e.png}
\caption{~}
\end{figure}

\paragraph{Collection Interface}\label{collection-interface}

It is the main interface of the collection framework that is inherited
by \texttt{List}, \texttt{Queue}, and \texttt{Set}.

\subsection{Wrapper Classes for Primitive
Datatypes}\label{wrapper-classes-for-primitive-datatypes}

Since collections can only store objects, we need wrapper classes for
primitive datatypes.

\begin{itemize}
\tightlist
\item
  \texttt{Integer} for \texttt{int}
\item
  \texttt{Double} for \texttt{double}
\item
  \texttt{Boolean} for \texttt{boolean}
\item
  \texttt{Character} for \texttt{char}
\item
  Also: \texttt{Float}, \texttt{Long}, \texttt{Short}, \texttt{Byte}
\end{itemize}

\subsection{Array}\label{array}

A basic, fixed-size data structure.

\begin{Shaded}
\begin{Highlighting}[]
\DataTypeTok{int}\OperatorTok{[]}\NormalTok{ arr1}\OperatorTok{;}
\DataTypeTok{int} \OperatorTok{[]}\NormalTok{arr2}\OperatorTok{;}
\DataTypeTok{int}\NormalTok{ arr3}\OperatorTok{[]} \OperatorTok{=} \OperatorTok{\{} \DecValTok{0}\OperatorTok{,} \DecValTok{1}\OperatorTok{,} \DecValTok{3} \OperatorTok{\};}
\end{Highlighting}
\end{Shaded}

To declare size:

\begin{Shaded}
\begin{Highlighting}[]
\BuiltInTok{String}\OperatorTok{[]}\NormalTok{ st }\OperatorTok{=} \KeywordTok{new} \BuiltInTok{String}\OperatorTok{[}\DecValTok{10}\OperatorTok{];}
\NormalTok{st}\OperatorTok{[}\DecValTok{0}\OperatorTok{]} \OperatorTok{=} \StringTok{"John"}
\NormalTok{st}\OperatorTok{[}\DecValTok{9}\OperatorTok{]} \OperatorTok{=} \StringTok{"Silksong"}
\end{Highlighting}
\end{Shaded}

\subsection{Iterator}\label{iterator}

It is an interface used to traverse (access) the data in a
\texttt{Collection}.

It traverses elements in a sequential, forward-only manner.

It allows for the safe removal of elements while traversing. You cannot
do this with a for-each loop (it would cause a
ConcurrentModificationException).

It lies in \texttt{java.util} package.

Creation of Iterator:

\begin{Shaded}
\begin{Highlighting}[]
\BuiltInTok{Iterator}\OperatorTok{\textless{}}\BuiltInTok{String}\OperatorTok{\textgreater{}}\NormalTok{ it }\OperatorTok{=}\NormalTok{ myCollection}\OperatorTok{.}\FunctionTok{iterator}\OperatorTok{();}
\end{Highlighting}
\end{Shaded}

\subsubsection{Methods of Iterator}\label{methods-of-iterator}

\begin{itemize}
\tightlist
\item
  \texttt{hasNext()\ -\textgreater{}\ bool}: Returns \texttt{true} if
  the iteration has more elements.
\item
  \texttt{next()\ -\textgreater{}\ Object}: Returns the next element in
  the iteration.
\item
  \texttt{remove()}: Removes the last element returned by
  \texttt{next()} from the collection.
\end{itemize}

\subsection{ListIterator (child of
Iterator)}\label{listiterator-child-of-iterator}

An iterator for \texttt{List}s that allows traversal in both directions.

It has these extra functions:

\begin{itemize}
\tightlist
\item
  \texttt{hasPrevious()}
\item
  \texttt{previous()}
\item
  \texttt{add(Object\ o)}
\item
  \texttt{set(Object\ o)}
\end{itemize}

It is created as:

\begin{Shaded}
\begin{Highlighting}[]
\BuiltInTok{ListIterator}\OperatorTok{\textless{}}\BuiltInTok{Integer}\OperatorTok{\textgreater{}}\NormalTok{ it }\OperatorTok{=}\NormalTok{ myList}\OperatorTok{.}\FunctionTok{listIterator}\OperatorTok{();}
\end{Highlighting}
\end{Shaded}

\section{\texorpdfstring{\hyperref[linear-collections]{Linear Collections}}{Linear Collections}}\label{linear-collections}

\section{\texorpdfstring{\hyperref[maps]{Maps}}{Maps}}\label{maps}

\section{\texorpdfstring{\hyperref[sets]{Sets}}{Sets}}\label{sets}

\newpage

\chapter{Linear Collections}\label{linear-collections}

\etocsettocstyle{\textbf{Chapter Contents}\par\rule{\linewidth}{0.5pt}}{\par\rule{\linewidth}{0.5pt}}
\localtableofcontents

\noindent

\subsection{List}\label{list}

It is an interface for an \textbf{ordered} collection of elements.

Features of this interface:

\begin{itemize}
\tightlist
\item
  Allows duplicate elements
\item
  Maintains insertion order
\item
  Supports NULL values
\item
  Index-based access
\item
  Non-synchronized (for modern implementations)
\end{itemize}

List is implemented by: \texttt{ArrayList}, \texttt{LinkedList}, and
\texttt{Vector}.

\subsubsection{Common List Methods}\label{common-list-methods}

\begin{itemize}
\tightlist
\item
  \texttt{add(Object\ o)}: Appends to the end.
\item
  \texttt{add(int\ index,\ Object\ o)}: Inserts at a specific index.
\item
  \texttt{get(int\ index)}: Returns element at the index.
\item
  \texttt{set(int\ index,\ Object\ o)}: Updates an element.
\item
  \texttt{remove(Object\ o)}: Removes the first occurrence of the
  object.
\item
  \texttt{remove(int\ index)}: Removes the element at the index.
\item
  \texttt{indexOf(Object\ o)}: Returns the index of the first
  occurrence.
\item
  \texttt{lastIndexOf(Object\ o)}
\item
  \texttt{contains(Object\ o)\ -\textgreater{}\ bool}: Checks if an
  element is present or not.
\item
  \texttt{size()}
\item
  \texttt{clear()}: Remove all the elements from the list.
\end{itemize}

\subsubsection{ArrayList (implements
List)}\label{arraylist-implements-list}

It is a class that implements a \textbf{resizable array}.

\begin{itemize}
\tightlist
\item
  It has the same features as \texttt{List}.
\item
  Fast for random access (\texttt{get()}).
\item
  Slower for insertions/deletions in the middle.
\item
  It by default has size 10 and it grows by 1.5x when full.
\end{itemize}

\subsection{LinkedList}\label{linkedlist}

Java's LinkedList is a doubly-linked list.

It also implements the Deque (Double-Ended Queue) interface.

\begin{itemize}
\tightlist
\item
  It can store duplicate as well as null values.
\item
  It follows insertion order.
\item
  It is non-synchronized.
\item
  Fast for insertions/deletions (at the beginning, middle, or end).
\item
  Slow for random access (\texttt{get()}), as it must traverse the list.
\end{itemize}

Because it implements \texttt{Deque}, it has methods for adding/removing
from both ends:

\begin{Shaded}
\begin{Highlighting}[]
\FunctionTok{addFirst}\OperatorTok{(}\NormalTok{object o}\OperatorTok{);}
\FunctionTok{addLast}\OperatorTok{(}\NormalTok{object o}\OperatorTok{);}
\FunctionTok{getFirst}\OperatorTok{();}
\FunctionTok{getLast}\OperatorTok{();}
\FunctionTok{removeFirst}\OperatorTok{();}
\FunctionTok{removeLast}\OperatorTok{();}
\end{Highlighting}
\end{Shaded}

\subsection{Vector}\label{vector}

It is a legacy class (JDK 1.0) that acts like a resizable array.

\begin{itemize}
\tightlist
\item
  It is \textbf{synchronized} (thread-safe).
\item
  It by default has size 10 and it grows by \textbf{2x} when full.
\item
  \texttt{ArrayList} is preferred in single-threaded environments.
\end{itemize}

\subsection{Stack}\label{stack}

It is a legacy class (JDK 1.0) that extends \texttt{Vector}.

\begin{itemize}
\tightlist
\item
  Implements LIFO (Last-In, First-Out) order.
\item
  It is \textbf{synchronized}.
\item
  The modern, preferred alternative is using an \texttt{ArrayDeque}.
\end{itemize}

\subsubsection{Methods}\label{methods}

\begin{Shaded}
\begin{Highlighting}[]
\FunctionTok{push}\OperatorTok{(}\NormalTok{object o}\OperatorTok{);}
\FunctionTok{pop}\OperatorTok{();}
\FunctionTok{peek}\OperatorTok{();}
\FunctionTok{search}\OperatorTok{(}\NormalTok{object o}\OperatorTok{);} \CommentTok{// returns 1{-}based index, or {-}1}
\FunctionTok{empty}\OperatorTok{();} \CommentTok{// checks if stack is empty}
\end{Highlighting}
\end{Shaded}

\subsection{Queue}\label{queue}

An interface for a collection that follows \textbf{FIFO} (First-In,
First-Out) order.

Classes used to implement queue:

\begin{itemize}
\tightlist
\item
  \texttt{LinkedList} (a general-purpose queue)
\item
  \texttt{PriorityQueue} (a priority-based queue)
\item
  \texttt{ArrayDeque} (a double-ended queue)
\end{itemize}

\subsubsection{PriorityQueue}\label{priorityqueue}

It implements the \texttt{Queue} interface, but orders elements by
priority.

\begin{itemize}
\tightlist
\item
  By default, it is a \textbf{min-heap} (the smallest element has the
  highest priority and is at the head).
\item
  We can provide a \texttt{Comparator} for custom ordering.
\item
  It is non-synchronized.
\item
  Does not follow insertion order.
\end{itemize}

Methods come in two flavors:

\begin{enumerate}
\def\labelenumi{\arabic{enumi}.}
\tightlist
\item
  \textbf{Throws Exception:} \texttt{add()}, \texttt{remove()} (removes
  head), \texttt{element()} (peeks head)
\item
  \textbf{Returns Special Value:} \texttt{offer()} (returns
  \texttt{false} if full), \texttt{poll()} (returns \texttt{null} if
  empty), \texttt{peek()} (returns \texttt{null} if empty)
\end{enumerate}

\newpage

\chapter{Maps}\label{maps}

\etocsettocstyle{\textbf{Chapter Contents}\par\rule{\linewidth}{0.5pt}}{\par\rule{\linewidth}{0.5pt}}
\localtableofcontents

\noindent

A \texttt{Map} is an object that maps keys to values. A map cannot
contain duplicate keys; each key can map to at most one value.

It does \textbf{not} inherit from the \texttt{Collection} interface.

\begin{figure}
\centering
\includegraphics[width=\linewidth,height=12cm,keepaspectratio,alt={~}]{images/mermaid_3ef2efab9f27fcab7200943365754237.png}
\caption{~}
\end{figure}

\section{Map Interface Methods}\label{map-interface-methods}

\begin{itemize}
\tightlist
\item
  \texttt{put(K\ key,\ V\ value)}: Associates the specified value with
  the specified key.
\item
  \texttt{get(Object\ key)}: Returns the value to which the specified
  key is mapped.
\item
  \texttt{remove(Object\ key)}: Removes the mapping for a key.
\item
  \texttt{containsKey(Object\ key)}: Returns \texttt{true} if this map
  contains a mapping for the specified key.
\item
  \texttt{keySet()}: Returns a \texttt{Set} view of the keys.
\item
  \texttt{entrySet()}: Returns a \texttt{Set} view of the mappings.
\end{itemize}

\section{HashMap}\label{hashmap}

\texttt{HashMap} is a hash table based implementation.

\begin{itemize}
\tightlist
\item
  \textbf{Order}: Unordered.
\item
  \textbf{Null}: Allows one null key and multiple null values.
\item
  \textbf{Synchronization}: Non-synchronized.
\item
  \textbf{Performance}: \(O(1)\) for \texttt{get} and \texttt{put}.
\end{itemize}

\begin{Shaded}
\begin{Highlighting}[]
\BuiltInTok{Map}\OperatorTok{\textless{}}\BuiltInTok{String}\OperatorTok{,} \BuiltInTok{Integer}\OperatorTok{\textgreater{}}\NormalTok{ map }\OperatorTok{=} \KeywordTok{new} \BuiltInTok{HashMap}\OperatorTok{\textless{}\textgreater{}();}
\NormalTok{map}\OperatorTok{.}\FunctionTok{put}\OperatorTok{(}\StringTok{"Apple"}\OperatorTok{,} \DecValTok{10}\OperatorTok{);}
\NormalTok{map}\OperatorTok{.}\FunctionTok{put}\OperatorTok{(}\StringTok{"Banana"}\OperatorTok{,} \DecValTok{20}\OperatorTok{);}
\NormalTok{map}\OperatorTok{.}\FunctionTok{put}\OperatorTok{(}\KeywordTok{null}\OperatorTok{,} \DecValTok{0}\OperatorTok{);} \CommentTok{// Valid}

\BuiltInTok{System}\OperatorTok{.}\FunctionTok{out}\OperatorTok{.}\FunctionTok{println}\OperatorTok{(}\NormalTok{map}\OperatorTok{.}\FunctionTok{get}\OperatorTok{(}\StringTok{"Apple"}\OperatorTok{));} \CommentTok{// 10}
\end{Highlighting}
\end{Shaded}

\section{LinkedHashMap}\label{linkedhashmap}

\texttt{LinkedHashMap} extends \texttt{HashMap} and maintains a linked
list of the entries.

\begin{itemize}
\tightlist
\item
  \textbf{Order}: Insertion order (or access order).
\item
  \textbf{Null}: Allows one null key.
\item
  \textbf{Performance}: Slightly slower than \texttt{HashMap} due to
  linked list overhead.
\end{itemize}

\begin{Shaded}
\begin{Highlighting}[]
\BuiltInTok{Map}\OperatorTok{\textless{}}\BuiltInTok{String}\OperatorTok{,} \BuiltInTok{Integer}\OperatorTok{\textgreater{}}\NormalTok{ map }\OperatorTok{=} \KeywordTok{new} \BuiltInTok{LinkedHashMap}\OperatorTok{\textless{}\textgreater{}();}
\NormalTok{map}\OperatorTok{.}\FunctionTok{put}\OperatorTok{(}\StringTok{"One"}\OperatorTok{,} \DecValTok{1}\OperatorTok{);}
\NormalTok{map}\OperatorTok{.}\FunctionTok{put}\OperatorTok{(}\StringTok{"Two"}\OperatorTok{,} \DecValTok{2}\OperatorTok{);}
\CommentTok{// Iteration will guarantee "One" then "Two"}
\end{Highlighting}
\end{Shaded}

\section{TreeMap}\label{treemap}

\texttt{TreeMap} implements \texttt{SortedMap} and uses a Red-Black
tree.

\begin{itemize}
\tightlist
\item
  \textbf{Order}: Sorted according to the natural ordering of its keys,
  or by a \texttt{Comparator}.
\item
  \textbf{Null}: \textbf{Cannot} have a null key.
\item
  \textbf{Performance}: \(O(\log n)\).
\end{itemize}

\begin{Shaded}
\begin{Highlighting}[]
\BuiltInTok{Map}\OperatorTok{\textless{}}\BuiltInTok{String}\OperatorTok{,} \BuiltInTok{Integer}\OperatorTok{\textgreater{}}\NormalTok{ map }\OperatorTok{=} \KeywordTok{new} \BuiltInTok{TreeMap}\OperatorTok{\textless{}\textgreater{}();}
\NormalTok{map}\OperatorTok{.}\FunctionTok{put}\OperatorTok{(}\StringTok{"Banana"}\OperatorTok{,} \DecValTok{2}\OperatorTok{);}
\NormalTok{map}\OperatorTok{.}\FunctionTok{put}\OperatorTok{(}\StringTok{"Apple"}\OperatorTok{,} \DecValTok{1}\OperatorTok{);}

\BuiltInTok{System}\OperatorTok{.}\FunctionTok{out}\OperatorTok{.}\FunctionTok{println}\OperatorTok{(}\NormalTok{map}\OperatorTok{);} \CommentTok{// \{Apple=1, Banana=2\} (Sorted by Key)}
\end{Highlighting}
\end{Shaded}

\section{Hashtable}\label{hashtable}

\texttt{Hashtable} is a legacy class (JDK 1.0).

\begin{itemize}
\tightlist
\item
  \textbf{Order}: Unordered.
\item
  \textbf{Null}: \textbf{Cannot} have null keys or values.
\item
  \textbf{Synchronization}: \textbf{Synchronized} (Thread-safe).
\item
  \textbf{Performance}: Slower than \texttt{HashMap} due to
  synchronization.
\end{itemize}

\section{Properties Class}\label{properties-class}

The \texttt{Properties} class extends \texttt{Hashtable}. It represents
a persistent set of properties.

\begin{itemize}
\tightlist
\item
  Keys and values are both \textbf{Strings}.
\item
  Used to read/write configuration files (\texttt{.properties}).
\end{itemize}

\textbf{Key Methods:}

\begin{itemize}
\tightlist
\item
  \texttt{setProperty(String\ key,\ String\ value)}
\item
  \texttt{getProperty(String\ key)}
\item
  \texttt{store(OutputStream\ out,\ String\ comments)}
\item
  \texttt{load(InputStream\ in)}
\end{itemize}

\begin{Shaded}
\begin{Highlighting}[]
\BuiltInTok{Properties}\NormalTok{ p }\OperatorTok{=} \KeywordTok{new} \BuiltInTok{Properties}\OperatorTok{();}
\NormalTok{p}\OperatorTok{.}\FunctionTok{setProperty}\OperatorTok{(}\StringTok{"user"}\OperatorTok{,} \StringTok{"admin"}\OperatorTok{);}
\NormalTok{p}\OperatorTok{.}\FunctionTok{setProperty}\OperatorTok{(}\StringTok{"password"}\OperatorTok{,} \StringTok{"1234"}\OperatorTok{);}

\CommentTok{// Saving to a file}
\ControlFlowTok{try} \OperatorTok{(}\BuiltInTok{FileOutputStream}\NormalTok{ fos }\OperatorTok{=} \KeywordTok{new} \BuiltInTok{FileOutputStream}\OperatorTok{(}\StringTok{"config.properties"}\OperatorTok{))} \OperatorTok{\{}
\NormalTok{    p}\OperatorTok{.}\FunctionTok{store}\OperatorTok{(}\NormalTok{fos}\OperatorTok{,} \StringTok{"User Config"}\OperatorTok{);}
\OperatorTok{\}} \ControlFlowTok{catch} \OperatorTok{(}\BuiltInTok{IOException}\NormalTok{ e}\OperatorTok{)} \OperatorTok{\{}
\NormalTok{    e}\OperatorTok{.}\FunctionTok{printStackTrace}\OperatorTok{();}
\OperatorTok{\}}
\end{Highlighting}
\end{Shaded}

\subsection{Comparison}\label{comparison}

{\def\LTcaptype{none} % do not increment counter
\begin{longtable}[]{@{}
  >{\raggedright\arraybackslash}p{(\linewidth - 8\tabcolsep) * \real{0.2581}}
  >{\raggedright\arraybackslash}p{(\linewidth - 8\tabcolsep) * \real{0.1774}}
  >{\raggedright\arraybackslash}p{(\linewidth - 8\tabcolsep) * \real{0.2097}}
  >{\raggedright\arraybackslash}p{(\linewidth - 8\tabcolsep) * \real{0.1774}}
  >{\raggedright\arraybackslash}p{(\linewidth - 8\tabcolsep) * \real{0.1774}}@{}}
\toprule\noalign{}
\begin{minipage}[b]{\linewidth}\raggedright
Feature
\end{minipage} & \begin{minipage}[b]{\linewidth}\raggedright
HashMap
\end{minipage} & \begin{minipage}[b]{\linewidth}\raggedright
LinkedHashMap
\end{minipage} & \begin{minipage}[b]{\linewidth}\raggedright
TreeMap
\end{minipage} & \begin{minipage}[b]{\linewidth}\raggedright
Hashtable
\end{minipage} \\
\midrule\noalign{}
\endhead
\bottomrule\noalign{}
\endlastfoot
\textbf{Order} & Random & Insertion & Sorted & Random \\
\textbf{Null Key} & Allowed (1) & Allowed (1) & Not Allowed & Not
Allowed \\
\textbf{Null Value} & Allowed & Allowed & Allowed & Not Allowed \\
\textbf{Synchronized} & No & No & No & Yes \\
\end{longtable}
}

\newpage

\chapter{Sets}\label{sets}

\etocsettocstyle{\textbf{Chapter Contents}\par\rule{\linewidth}{0.5pt}}{\par\rule{\linewidth}{0.5pt}}
\localtableofcontents

\noindent

A \texttt{Set} is a collection that cannot contain duplicate elements.
It models the mathematical set abstraction.

The \texttt{Set} interface contains only methods inherited from
\texttt{Collection} and adds the restriction that duplicate elements are
prohibited.

\begin{figure}
\centering
\includegraphics[width=\linewidth,height=12cm,keepaspectratio,alt={~}]{images/mermaid_9934d9007b0733cab1c3a47829717e16.png}
\caption{~}
\end{figure}

\section{HashSet}\label{hashset}

\texttt{HashSet} is the best-performing implementation. It uses a hash
table.

\begin{itemize}
\tightlist
\item
  \textbf{Order}: Unordered (no guarantee of iteration order).
\item
  \textbf{Null}: Allows one null element.
\item
  \textbf{Performance}: Constant time \(O(1)\) for basic operations
  (\texttt{add}, \texttt{remove}, \texttt{contains}), assuming the hash
  function disperses elements properly.
\item
  \textbf{Synchronization}: Not synchronized.
\end{itemize}

\begin{Shaded}
\begin{Highlighting}[]
\BuiltInTok{Set}\OperatorTok{\textless{}}\BuiltInTok{String}\OperatorTok{\textgreater{}}\NormalTok{ fruits }\OperatorTok{=} \KeywordTok{new} \BuiltInTok{HashSet}\OperatorTok{\textless{}\textgreater{}();}
\NormalTok{fruits}\OperatorTok{.}\FunctionTok{add}\OperatorTok{(}\StringTok{"Apple"}\OperatorTok{);}
\NormalTok{fruits}\OperatorTok{.}\FunctionTok{add}\OperatorTok{(}\StringTok{"Banana"}\OperatorTok{);}
\NormalTok{fruits}\OperatorTok{.}\FunctionTok{add}\OperatorTok{(}\StringTok{"Apple"}\OperatorTok{);} \CommentTok{// Duplicate, will be ignored}

\BuiltInTok{System}\OperatorTok{.}\FunctionTok{out}\OperatorTok{.}\FunctionTok{println}\OperatorTok{(}\NormalTok{fruits}\OperatorTok{);} \CommentTok{// Output order is unpredictable}
\end{Highlighting}
\end{Shaded}

\section{LinkedHashSet}\label{linkedhashset}

\texttt{LinkedHashSet} is a hash table and linked list implementation of
the \texttt{Set} interface.

\begin{itemize}
\tightlist
\item
  \textbf{Order}: Insertion order (the order in which elements were
  inserted).
\item
  \textbf{Null}: Allows one null element.
\item
  \textbf{Performance}: Slightly slower than \texttt{HashSet} due to the
  added expense of maintaining the linked list, but still \(O(1)\).
\end{itemize}

\begin{Shaded}
\begin{Highlighting}[]
\BuiltInTok{Set}\OperatorTok{\textless{}}\BuiltInTok{String}\OperatorTok{\textgreater{}}\NormalTok{ fruits }\OperatorTok{=} \KeywordTok{new} \BuiltInTok{LinkedHashSet}\OperatorTok{\textless{}\textgreater{}();}
\NormalTok{fruits}\OperatorTok{.}\FunctionTok{add}\OperatorTok{(}\StringTok{"Apple"}\OperatorTok{);}
\NormalTok{fruits}\OperatorTok{.}\FunctionTok{add}\OperatorTok{(}\StringTok{"Banana"}\OperatorTok{);}
\NormalTok{fruits}\OperatorTok{.}\FunctionTok{add}\OperatorTok{(}\StringTok{"Orange"}\OperatorTok{);}

\BuiltInTok{System}\OperatorTok{.}\FunctionTok{out}\OperatorTok{.}\FunctionTok{println}\OperatorTok{(}\NormalTok{fruits}\OperatorTok{);} \CommentTok{// Output: [Apple, Banana, Orange]}
\end{Highlighting}
\end{Shaded}

\section{SortedSet Interface}\label{sortedset-interface}

A \texttt{Set} that further provides a \emph{total ordering} on its
elements. The elements are ordered using their \textbf{natural
ordering}, or by a \texttt{Comparator} provided at sorted set creation
time.

\section{TreeSet}\label{treeset}

\texttt{TreeSet} implements \texttt{SortedSet} (and
\texttt{NavigableSet}). It uses a \textbf{Red-Black tree} structure.

\begin{itemize}
\tightlist
\item
  \textbf{Order}: Sorted (Natural ascending order or custom Comparator).
\item
  \textbf{Null}: Does \textbf{not} allow null elements (throws
  \texttt{NullPointerException}).
\item
  \textbf{Performance}: \(O(\log n)\) for basic operations.
\item
  \textbf{Use Case}: When you need a set that is always sorted.
\end{itemize}

\begin{Shaded}
\begin{Highlighting}[]
\BuiltInTok{Set}\OperatorTok{\textless{}}\BuiltInTok{Integer}\OperatorTok{\textgreater{}}\NormalTok{ numbers }\OperatorTok{=} \KeywordTok{new} \BuiltInTok{TreeSet}\OperatorTok{\textless{}\textgreater{}();}
\NormalTok{numbers}\OperatorTok{.}\FunctionTok{add}\OperatorTok{(}\DecValTok{5}\OperatorTok{);}
\NormalTok{numbers}\OperatorTok{.}\FunctionTok{add}\OperatorTok{(}\DecValTok{1}\OperatorTok{);}
\NormalTok{numbers}\OperatorTok{.}\FunctionTok{add}\OperatorTok{(}\DecValTok{10}\OperatorTok{);}

\BuiltInTok{System}\OperatorTok{.}\FunctionTok{out}\OperatorTok{.}\FunctionTok{println}\OperatorTok{(}\NormalTok{numbers}\OperatorTok{);} \CommentTok{// Output: [1, 5, 10]}
\end{Highlighting}
\end{Shaded}

\subsection{Comparison}\label{comparison-1}

{\def\LTcaptype{none} % do not increment counter
\begin{longtable}[]{@{}
  >{\raggedright\arraybackslash}p{(\linewidth - 6\tabcolsep) * \real{0.2667}}
  >{\raggedright\arraybackslash}p{(\linewidth - 6\tabcolsep) * \real{0.1467}}
  >{\raggedright\arraybackslash}p{(\linewidth - 6\tabcolsep) * \real{0.2667}}
  >{\raggedright\arraybackslash}p{(\linewidth - 6\tabcolsep) * \real{0.3200}}@{}}
\toprule\noalign{}
\begin{minipage}[b]{\linewidth}\raggedright
Feature
\end{minipage} & \begin{minipage}[b]{\linewidth}\raggedright
HashSet
\end{minipage} & \begin{minipage}[b]{\linewidth}\raggedright
LinkedHashSet
\end{minipage} & \begin{minipage}[b]{\linewidth}\raggedright
TreeSet
\end{minipage} \\
\midrule\noalign{}
\endhead
\bottomrule\noalign{}
\endlastfoot
\textbf{Internal Storage} & HashMap & HashMap + LinkedList & TreeMap
(Red-Black Tree) \\
\textbf{Order} & Unordered & Insertion Order & Sorted Order \\
\textbf{Null Elements} & Allowed (1) & Allowed (1) & Not Allowed \\
\textbf{Performance} & \(O(1)\) & \(O(1)\) & \(O(\log n)\) \\
\end{longtable}
}

\newpage

\chapter{Sorting and Comparator}\label{sorting-and-comparator}

\etocsettocstyle{\textbf{Chapter Contents}\par\rule{\linewidth}{0.5pt}}{\par\rule{\linewidth}{0.5pt}}
\localtableofcontents

\noindent

The Java sort methods use \textbf{Timm sort}, which is a hybrid of
insertion sort and merge sort.

We use the following to sort a collection:

\begin{Shaded}
\begin{Highlighting}[]
\BuiltInTok{Arrays}\OperatorTok{.}\FunctionTok{sort}\OperatorTok{();}
\BuiltInTok{Collections}\OperatorTok{.}\FunctionTok{sort}\OperatorTok{();}
\end{Highlighting}
\end{Shaded}

\texttt{Arrays.sort()} can sort static arrays like \texttt{int\ {[}{]}}

\texttt{Collections.sort()} sorts lists (like \texttt{ArrayList}).

\subsection{Comparable Interface}\label{comparable-interface}

It is used to sort the data or object in its \textbf{natural order},
inside the class itself.

It provides one method to implement: \texttt{compareTo(T\ o)}.

We implement the \texttt{Comparable} interface in the class whose
objects are to be arranged.

\texttt{compareTo(T\ o)} compares this object with the other object o.

\begin{itemize}
\tightlist
\item
  \textbf{Negative int:} \texttt{this} object comes \emph{before}
  \texttt{o}.
\item
  \textbf{Positive int:} \texttt{this} object comes \emph{after}
  \texttt{s2}.
\item
  \textbf{Zero:} \texttt{this} object and \texttt{o} are equal in terms
  of sorting.
\end{itemize}

\begin{Shaded}
\begin{Highlighting}[]
\CommentTok{// Student class now defines its "natural order" as sorting by roll number}
\KeywordTok{class}\NormalTok{ Student }\KeywordTok{implements} \BuiltInTok{Comparable}\OperatorTok{\textless{}}\NormalTok{Student}\OperatorTok{\textgreater{}\{}
    \DataTypeTok{int}\NormalTok{ roll}\OperatorTok{;}
    \BuiltInTok{String}\NormalTok{ name}\OperatorTok{;}

    \CommentTok{// Constructor...}

    \AttributeTok{@Override}
    \KeywordTok{public} \DataTypeTok{int} \FunctionTok{compareTo}\OperatorTok{(}\NormalTok{Student other}\OperatorTok{)} \OperatorTok{\{}
        \CommentTok{// We want to sort by roll number}
        \CommentTok{// \textquotesingle{}this\textquotesingle{} is the first object, \textquotesingle{}other\textquotesingle{} is the second}
        \ControlFlowTok{return} \KeywordTok{this}\OperatorTok{.}\FunctionTok{roll} \OperatorTok{{-}}\NormalTok{ other}\OperatorTok{.}\FunctionTok{roll}\OperatorTok{;}
    \OperatorTok{\}}
\OperatorTok{\}}

\CommentTok{// Now you can just call:}
\BuiltInTok{Collections}\OperatorTok{.}\FunctionTok{sort}\OperatorTok{(}\NormalTok{listOfStudents}\OperatorTok{);}
\end{Highlighting}
\end{Shaded}

\subsection{Comparator Interface}\label{comparator-interface}

This is used when we want to define a custom sorting order,
\emph{separate} from the class.

Use a \texttt{Comparator} when:

\begin{itemize}
\tightlist
\item
  You want to sort in an order \emph{other} than the natural order.
\item
  You want to sort objects of a class you \emph{cannot modify}.
\item
  You want to define \emph{multiple different ways} to sort (e.g., by
  name, by roll, by grade).
\end{itemize}

It provides one main method: \texttt{compare(T\ o1,\ T\ o2)}.

It is implemented in a separate class.

\begin{Shaded}
\begin{Highlighting}[]
\CommentTok{// A separate class to define sorting by name}
\KeywordTok{class}\NormalTok{ SortStudentByName }\KeywordTok{implements} \BuiltInTok{Comparator}\OperatorTok{\textless{}}\NormalTok{Student}\OperatorTok{\textgreater{}} \OperatorTok{\{}

    \AttributeTok{@Override}
    \KeywordTok{public} \DataTypeTok{int} \FunctionTok{compare}\OperatorTok{(}\NormalTok{Student s1}\OperatorTok{,}\NormalTok{ Student s2}\OperatorTok{)} \OperatorTok{\{}
        \CommentTok{// Use String\textquotesingle{}s built{-}in compareTo for alphabetical sorting}
        \ControlFlowTok{return}\NormalTok{ s1}\OperatorTok{.}\FunctionTok{name}\OperatorTok{.}\FunctionTok{compareTo}\OperatorTok{(}\NormalTok{s2}\OperatorTok{.}\FunctionTok{name}\OperatorTok{);}
    \OperatorTok{\}}
\OperatorTok{\}}

\CommentTok{// Now you can sort using this new logic:}
\BuiltInTok{Collections}\OperatorTok{.}\FunctionTok{sort}\OperatorTok{(}\NormalTok{listOfStudents}\OperatorTok{,} \KeywordTok{new} \FunctionTok{SortStudentByName}\OperatorTok{());}
\end{Highlighting}
\end{Shaded}

\subsection{Sorting with Lambda
Expressions}\label{sorting-with-lambda-expressions}

Since \texttt{Comparator} is a
\textbf{\hyperref[functional-interface]{Functional Interface}} (it has
only one abstract method), we can use a lambda expression instead of
writing a whole new class.

\begin{Shaded}
\begin{Highlighting}[]
\CommentTok{// Sort by name using a lambda}
\BuiltInTok{Comparator}\OperatorTok{\textless{}}\NormalTok{Student}\OperatorTok{\textgreater{}}\NormalTok{ byName }\OperatorTok{=} \OperatorTok{(}\NormalTok{s1}\OperatorTok{,}\NormalTok{ s2}\OperatorTok{)} \OperatorTok{{-}\textgreater{}}\NormalTok{ s1}\OperatorTok{.}\FunctionTok{name}\OperatorTok{.}\FunctionTok{compareTo}\OperatorTok{(}\NormalTok{s2}\OperatorTok{.}\FunctionTok{name}\OperatorTok{);}

\CommentTok{// Sort by roll using a lambda}
\BuiltInTok{Comparator}\OperatorTok{\textless{}}\NormalTok{Student}\OperatorTok{\textgreater{}}\NormalTok{ byRoll }\OperatorTok{=} \OperatorTok{(}\NormalTok{s1}\OperatorTok{,}\NormalTok{ s2}\OperatorTok{)} \OperatorTok{{-}\textgreater{}}\NormalTok{ s1}\OperatorTok{.}\FunctionTok{roll} \OperatorTok{{-}}\NormalTok{ s2}\OperatorTok{.}\FunctionTok{roll}\OperatorTok{;}

\CommentTok{// You can pass the lambda directly into the sort method:}
\BuiltInTok{Collections}\OperatorTok{.}\FunctionTok{sort}\OperatorTok{(}\NormalTok{listOfStudents}\OperatorTok{,} \OperatorTok{(}\NormalTok{s1}\OperatorTok{,}\NormalTok{ s2}\OperatorTok{)} \OperatorTok{{-}\textgreater{}}\NormalTok{ s1}\OperatorTok{.}\FunctionTok{name}\OperatorTok{.}\FunctionTok{compareTo}\OperatorTok{(}\NormalTok{s2}\OperatorTok{.}\FunctionTok{name}\OperatorTok{));}

\CommentTok{// We can even "chain" comparators}
\CommentTok{// Sort by name, and if names are the same, sort by roll}
\BuiltInTok{Comparator}\OperatorTok{\textless{}}\NormalTok{Student}\OperatorTok{\textgreater{}}\NormalTok{ byNameThenRoll }\OperatorTok{=}\NormalTok{ ExampleStudent}\OperatorTok{.}\FunctionTok{byName}
        \OperatorTok{.}\FunctionTok{thenComparing}\OperatorTok{(}\NormalTok{ExampleStudent}\OperatorTok{.}\FunctionTok{byRoll}\OperatorTok{);}
\BuiltInTok{Collections}\OperatorTok{.}\FunctionTok{sort}\OperatorTok{(}\NormalTok{listOfStudents}\OperatorTok{,}\NormalTok{ byNameThenRoll}\OperatorTok{);}
\end{Highlighting}
\end{Shaded}

\newpage

\chapter{Multi Threading}\label{multi-threading}

\etocsettocstyle{\textbf{Chapter Contents}\par\rule{\linewidth}{0.5pt}}{\par\rule{\linewidth}{0.5pt}}
\localtableofcontents

\noindent

A \textbf{thread} is a single, lightweight execution path within a
process. A \textbf{process} is a self-contained execution environment
(an instance of a program).

Multithreading is a mechanism in Java where multiple threads of
execution run concurrently (or appear to run at the same time) within a
single process. They share the same memory space.

This is used to:

\begin{itemize}
\tightlist
\item
  Perform multiple operations at the same time.
\item
  Maximize CPU utilization.
\item
  Improve application responsiveness (e.g., keeping a UI responsive
  while a background task runs).
\end{itemize}

\begin{tipblock}{Analogy: Kitchen Staff}

\begin{itemize}
\tightlist
\item
  \textbf{Single-Threaded:} One chef doing everything. He chops veggies,
  then cooks, then plates. If chopping takes 10 mins, the stove sits
  idle.
\item
  \textbf{Multi-Threaded:} A full team.

  \begin{itemize}
  \tightlist
  \item
    \textbf{Thread 1 (Chef):} Cooking on the stove.
  \item
    \textbf{Thread 2 (Sous Chef):} Chopping vegetables.
  \item
    \textbf{Thread 3 (Dishwasher):} Cleaning plates.
  \end{itemize}
\item
  \textbf{Result:} Everything happens at once (Concurrency), and the
  meal is ready much faster.
\end{itemize}

\end{tipblock}

\subsection{Thread Life Cycle}\label{thread-life-cycle}

A thread goes through several states:

\begin{figure}
\centering
\includegraphics[width=\linewidth,height=12cm,keepaspectratio,alt={~}]{images/mermaid_4c48113ca2c68b7954222978162e91db.png}
\caption{~}
\end{figure}

\begin{enumerate}
\def\labelenumi{\arabic{enumi}.}
\tightlist
\item
  \textbf{New:} The thread object has been created, but \texttt{start()}
  has not been called.
\item
  \textbf{Runnable:} The thread is ready to run. \texttt{start()} has
  been called, and it's waiting for the thread scheduler to allocate CPU
  time.
\item
  \textbf{Running:} The thread is actively executing its code.
\item
  \textbf{Waiting / Blocked:} The thread is temporarily inactive. It
  might be waiting for a lock (to enter a \texttt{synchronized} block),
  or it has been told to wait (\texttt{wait()}), sleep
  (\texttt{sleep()}), or join (\texttt{join()}).
\item
  \textbf{Terminated (Dead):} The thread has finished its \texttt{run()}
  method or has otherwise stopped.
\end{enumerate}

\subsection{Thread Priorities}\label{thread-priorities}

Every thread has a priority, which is an integer between 1 and 10. The
thread scheduler uses this to decide which thread to run.

\begin{itemize}
\tightlist
\item
  \textbf{MIN\_PRIORITY (1)}
\item
  \textbf{NORM\_PRIORITY (5)} - Default
\item
  \textbf{MAX\_PRIORITY (10)}
\end{itemize}

\begin{Shaded}
\begin{Highlighting}[]
\BuiltInTok{Thread}\NormalTok{ t1 }\OperatorTok{=} \KeywordTok{new} \BuiltInTok{Thread}\OperatorTok{();}
\NormalTok{t1}\OperatorTok{.}\FunctionTok{setPriority}\OperatorTok{(}\BuiltInTok{Thread}\OperatorTok{.}\FunctionTok{MAX\_PRIORITY}\OperatorTok{);} \CommentTok{// 10}
\BuiltInTok{System}\OperatorTok{.}\FunctionTok{out}\OperatorTok{.}\FunctionTok{println}\OperatorTok{(}\NormalTok{t1}\OperatorTok{.}\FunctionTok{getPriority}\OperatorTok{());}
\end{Highlighting}
\end{Shaded}

\subsection{How to Create a Thread}\label{how-to-create-a-thread}

There are two ways:

\subsubsection{\texorpdfstring{By extending the \texttt{Thread}
class}{By extending the Thread class}}\label{by-extending-the-thread-class}

You override the \texttt{run()} method with the code you want the thread
to execute.

\begin{Shaded}
\begin{Highlighting}[]
\KeywordTok{class}\NormalTok{ MyThread }\KeywordTok{extends} \BuiltInTok{Thread} \OperatorTok{\{}
    \AttributeTok{@Override}
    \KeywordTok{public} \DataTypeTok{void} \FunctionTok{run}\OperatorTok{()} \OperatorTok{\{}
        \CommentTok{// This is the job the thread will do}
        \BuiltInTok{System}\OperatorTok{.}\FunctionTok{out}\OperatorTok{.}\FunctionTok{println}\OperatorTok{(}\StringTok{"Thread is running by extending Thread class..."}\OperatorTok{);}
    \OperatorTok{\}}
\OperatorTok{\}}

\CommentTok{// To use it:}
\NormalTok{MyThread t1 }\OperatorTok{=} \KeywordTok{new} \FunctionTok{MyThread}\OperatorTok{();}
\NormalTok{t1}\OperatorTok{.}\FunctionTok{start}\OperatorTok{();} \CommentTok{// This calls the run() method in a new thread}
\end{Highlighting}
\end{Shaded}

\subsubsection{\texorpdfstring{By implementing the \texttt{Runnable}
interface}{By implementing the Runnable interface}}\label{by-implementing-the-runnable-interface}

This is the \textbf{preferred} way because Java does not support
multiple inheritance. By implementing \texttt{Runnable}, your class is
free to extend another class.

You implement the \texttt{run()} method, and then pass an instance of
your class to the \texttt{Thread} constructor.

\begin{Shaded}
\begin{Highlighting}[]
\KeywordTok{class}\NormalTok{ MyRunnable }\KeywordTok{implements} \BuiltInTok{Runnable} \OperatorTok{\{}
    \AttributeTok{@Override}
    \KeywordTok{public} \DataTypeTok{void} \FunctionTok{run}\OperatorTok{()} \OperatorTok{\{}
        \CommentTok{// This is the job the thread will do}
        \BuiltInTok{System}\OperatorTok{.}\FunctionTok{out}\OperatorTok{.}\FunctionTok{println}\OperatorTok{(}\StringTok{"Thread is running by implementing Runnable..."}\OperatorTok{);}
    \OperatorTok{\}}
\OperatorTok{\}}

\CommentTok{// To use it:}
\NormalTok{MyRunnable myRunnable }\OperatorTok{=} \KeywordTok{new} \FunctionTok{MyRunnable}\OperatorTok{();}
\BuiltInTok{Thread}\NormalTok{ t2 }\OperatorTok{=} \KeywordTok{new} \BuiltInTok{Thread}\OperatorTok{(}\NormalTok{myRunnable}\OperatorTok{);}
\NormalTok{t2}\OperatorTok{.}\FunctionTok{start}\OperatorTok{();} \CommentTok{// This calls myRunnable.run() in a new thread}

\CommentTok{// With a Lambda (since Runnable is a Functional Interface):}
\BuiltInTok{Runnable}\NormalTok{ r }\OperatorTok{=} \OperatorTok{()} \OperatorTok{{-}\textgreater{}} \BuiltInTok{System}\OperatorTok{.}\FunctionTok{out}\OperatorTok{.}\FunctionTok{println}\OperatorTok{(}\StringTok{"Thread running from lambda..."}\OperatorTok{);}
\BuiltInTok{Thread}\NormalTok{ t3 }\OperatorTok{=} \KeywordTok{new} \BuiltInTok{Thread}\OperatorTok{(}\NormalTok{r}\OperatorTok{);}
\NormalTok{t3}\OperatorTok{.}\FunctionTok{start}\OperatorTok{();}
\end{Highlighting}
\end{Shaded}

\subsection{Key Thread Methods}\label{key-thread-methods}

\begin{itemize}
\item
  \textbf{\texttt{start()}}: This is the method you call to begin the
  thread. It puts the thread in the \textbf{Runnable} state. The JVM
  then calls the \texttt{run()} method. \textbf{You never call
  \texttt{run()} directly.}
\item
  \textbf{\texttt{run()}}: This is where you put the logic that the
  thread will execute.
\item
  \textbf{\texttt{sleep(long\ millis)}}: (Static method) Pauses the
  \textbf{current} thread for a specified time. It does \emph{not}
  release any locks it holds.
\item
  \textbf{\texttt{join()}}: This method makes the \emph{current} thread
  wait until the thread it's called on (\texttt{t.join()}) is
  \textbf{Terminated}.
\item
  \textbf{\texttt{yield()}}: (Static method) A hint to the scheduler
  that the current thread is willing to give up its time slice.
\item
  \textbf{\texttt{isAlive()}}: Checks if the thread is in the New or
  Terminated state.
\end{itemize}

\subsection{Synchronization}\label{synchronization-1}

When multiple threads access shared resources (like a variable), you can
get problems like data inconsistency or race conditions.

Synchronization is the mechanism to control this access, ensuring only
one thread can access a shared resource at a time.

This is achieved using the synchronized keyword (on methods or blocks)
or other concurrency tools.

\newpage

\chapter{Synchronization}\label{synchronization}

\etocsettocstyle{\textbf{Chapter Contents}\par\rule{\linewidth}{0.5pt}}{\par\rule{\linewidth}{0.5pt}}
\localtableofcontents

\noindent

It is a mechanism to control access of multiple threads to shared
resources.

If we don't do synchronization, it leads to:

\begin{itemize}
\tightlist
\item
  \textbf{Race condition:} A situation where the outcome of an operation
  depends on the unpredictable sequence or timing of other threads'
  operations.
\item
  \textbf{Data inconsistency:} The shared data is left in an incorrect
  or invalid state. We can achieve synchronization using several
  mechanisms:
\end{itemize}

\subsection{Mutual Exclusion (Mutex)}\label{mutual-exclusion-mutex}

A \textbf{Mutex} (or lock) is a simple synchronization primitive. It
ensures that only one thread can be in a \textbf{critical section} (the
part of the code accessing the shared resource) at a time.

\begin{itemize}
\tightlist
\item
  A thread \textbf{acquires} the lock before entering the critical
  section.
\item
  If the lock is already held, the thread blocks (waits) until the lock
  is \textbf{released}.
\item
  The thread \textbf{releases} the lock after exiting the critical
  section.
\end{itemize}

\subsection{Semaphores}\label{semaphores}

A semaphore is a more general synchronization tool. It manages a counter
(a number of ``permits'').

\begin{itemize}
\tightlist
\item
  \textbf{Counting Semaphore:} Allows up to \(N\) threads to access a
  resource. A thread must \texttt{acquire()} a permit to proceed. If the
  counter is zero, the thread blocks. When done, it \texttt{release()}
  the permit.
\item
  \textbf{Binary Semaphore:} A semaphore with \(N=1\). It acts just like
  a Mutex.
\end{itemize}

\subsection{\texorpdfstring{Monitors (Java's
\texttt{synchronized})}{Monitors (Java's synchronized)}}\label{monitors-javas-synchronized}

A \textbf{Monitor} is a high-level construct that combines a Mutex with
condition variables, making synchronization easier. Java's
\texttt{synchronized} keyword is a built-in implementation of a monitor.

\begin{itemize}
\item
  \textbf{Synchronized Method:}\\
  \texttt{java\ \ \ public\ synchronized\ void\ safeMethod()\ \{\ \ \ \ \ \ \ //\ This\ entire\ method\ is\ a\ critical\ section.\ \ \ \ \ \ \ //\ The\ lock\ is\ on\ the\ \textquotesingle{}this\textquotesingle{}\ object\ instance.\ \ \ \}}
\item
  \textbf{Synchronized Block:} ```java public void myMethod() \{ //
  \ldots{} non-critical code \ldots{}

\begin{verbatim}
  // The lock can be 'this' or any other shared object
  synchronized(this) { 
      // Critical section: only one thread at a time
      // can execute this block on the *same object*.
  }

  // ... other non-critical code ...
\end{verbatim}

  \} ```
\item
  \textbf{Static Synchronized Method:}
  \texttt{java\ \ \ public\ static\ synchronized\ void\ safeStaticMethod()\ \{\ \ \ \ \ \ \ //\ The\ lock\ is\ on\ the\ Class\ object\ (e.g.,\ MyClass.class),\ \ \ \ \ \ \ //\ not\ an\ instance.\ \ \ \}}
\end{itemize}

\subsection{Key Thread Methods}\label{key-thread-methods-1}

\begin{itemize}
\tightlist
\item
  \textbf{\texttt{sleep(long\ millis)}}: (Static method) Pauses the
  \textbf{current} thread for a specified time. \textbf{It does not
  release any locks it holds.}
\item
  \textbf{\texttt{join()}}: A thread (\texttt{t1}) calls
  \texttt{t2.join()}. This makes \texttt{t1} (the \emph{calling} thread)
  wait until \texttt{t2} (the thread the method is called on) completes
  its execution. Join kills the thread it is used on.
\item
  \textbf{\texttt{yield()}}: (Static method) A hint to the thread
  scheduler that the current thread is willing to give up its current
  time slice. The scheduler may ignore this. It transfers thread from
  running to runnable. It does not kill the thread.
\item
  \textbf{\texttt{isAlive()}}: Checks if a thread has been started and
  has not yet died.
\item
  \textbf{\texttt{wait()}, \texttt{notify()}, \texttt{notifyAll()}}:
  These are methods of the \texttt{Object} class (not \texttt{Thread})
  and are used for inter-thread communication, typically within
  \texttt{synchronized} blocks.

  \begin{itemize}
  \tightlist
  \item
    \textbf{\texttt{wait()}}: Causes the current thread to release the
    lock and wait.
  \item
    \textbf{\texttt{notify()}}: Wakes up a single waiting thread.
  \item
    \textbf{\texttt{notifyAll()}}: Wakes up all waiting threads.
  \end{itemize}
\end{itemize}

\newpage

\chapter{Functional Interface}\label{functional-interface}

\etocsettocstyle{\textbf{Chapter Contents}\par\rule{\linewidth}{0.5pt}}{\par\rule{\linewidth}{0.5pt}}
\localtableofcontents

\noindent

A \textbf{Functional Interface} is an interface that contains exactly
one abstract method. It can have any number of default or static
methods.

It is also known as \textbf{SAM} (Single Abstract Method) interface.

The \texttt{Runnable} interface is a classic example: it has only one
abstract method \texttt{void\ run()}.

We use the \texttt{@FunctionalInterface} annotation to ensure
compile-time checking.

\section{Functional Programming in
Java}\label{functional-programming-in-java}

Java 8 introduced functional programming features, allowing for more
concise and declarative code.

\subsection{Key Concepts}\label{key-concepts}

\begin{enumerate}
\def\labelenumi{\arabic{enumi}.}
\tightlist
\item
  \textbf{Pure Functions}: Functions that always produce the same output
  for the same input and have no side effects.
\item
  \textbf{Immutability}: Data objects are not modified after creation.
\item
  \textbf{Higher-Order Functions}: Functions that can take other
  functions as arguments or return them as results.
\end{enumerate}

\subsection{Example: Returning a
Function}\label{example-returning-a-function}

\begin{Shaded}
\begin{Highlighting}[]
\CommentTok{// Higher{-}Order Function Example}
\KeywordTok{public}\NormalTok{ Function}\OperatorTok{\textless{}}\BuiltInTok{Integer}\OperatorTok{,} \BuiltInTok{Integer}\OperatorTok{\textgreater{}} \FunctionTok{createMultiplier}\OperatorTok{(}\DataTypeTok{int}\NormalTok{ factor}\OperatorTok{)} \OperatorTok{\{}
    \CommentTok{// Returns a function that multiplies by \textquotesingle{}factor\textquotesingle{}}
    \ControlFlowTok{return} \OperatorTok{(}\NormalTok{n}\OperatorTok{)} \OperatorTok{{-}\textgreater{}}\NormalTok{ n }\OperatorTok{*}\NormalTok{ factor}\OperatorTok{;}
\OperatorTok{\}}

\NormalTok{Function}\OperatorTok{\textless{}}\BuiltInTok{Integer}\OperatorTok{,} \BuiltInTok{Integer}\OperatorTok{\textgreater{}}\NormalTok{ doubler }\OperatorTok{=} \FunctionTok{createMultiplier}\OperatorTok{(}\DecValTok{2}\OperatorTok{);}
\BuiltInTok{System}\OperatorTok{.}\FunctionTok{out}\OperatorTok{.}\FunctionTok{println}\OperatorTok{(}\NormalTok{doubler}\OperatorTok{.}\FunctionTok{apply}\OperatorTok{(}\DecValTok{5}\OperatorTok{));} \CommentTok{// Output: 10}
\end{Highlighting}
\end{Shaded}

\section{Lambda Expressions}\label{lambda-expressions}

Lambda expressions are a concise way to represent an anonymous function
that implements a functional interface.

\textbf{Syntax}: \texttt{(parameters)\ -\textgreater{}\ \{\ body\ \}}

\subsection{Use in Threads}\label{use-in-threads}

\textbf{Old Way (Anonymous Class):}

\begin{Shaded}
\begin{Highlighting}[]
\BuiltInTok{Runnable}\NormalTok{ r1 }\OperatorTok{=} \KeywordTok{new} \BuiltInTok{Runnable}\OperatorTok{()} \OperatorTok{\{}
    \KeywordTok{public} \DataTypeTok{void} \FunctionTok{run}\OperatorTok{()} \OperatorTok{\{} \BuiltInTok{System}\OperatorTok{.}\FunctionTok{out}\OperatorTok{.}\FunctionTok{println}\OperatorTok{(}\StringTok{"Old"}\OperatorTok{);} \OperatorTok{\}}
\OperatorTok{\};}
\end{Highlighting}
\end{Shaded}

\textbf{New Way (Lambda):}

\begin{Shaded}
\begin{Highlighting}[]
\BuiltInTok{Runnable}\NormalTok{ r2 }\OperatorTok{=} \OperatorTok{()} \OperatorTok{{-}\textgreater{}} \BuiltInTok{System}\OperatorTok{.}\FunctionTok{out}\OperatorTok{.}\FunctionTok{println}\OperatorTok{(}\StringTok{"New"}\OperatorTok{);}
\BuiltInTok{Thread}\NormalTok{ t }\OperatorTok{=} \KeywordTok{new} \BuiltInTok{Thread}\OperatorTok{(}\NormalTok{r2}\OperatorTok{);}
\NormalTok{t}\OperatorTok{.}\FunctionTok{start}\OperatorTok{();}
\end{Highlighting}
\end{Shaded}

\section{Types of Functional
Interface}\label{types-of-functional-interface}

Java provides built-in functional interfaces in the
\texttt{java.util.function} package.

\subsection{Consumer}\label{consumer}

Takes an argument and returns nothing (\texttt{void}). Used for
performing actions.

\begin{itemize}
\tightlist
\item
  \textbf{Method}: \texttt{void\ accept(T\ t)}
\end{itemize}

\begin{Shaded}
\begin{Highlighting}[]
\NormalTok{Consumer}\OperatorTok{\textless{}}\BuiltInTok{String}\OperatorTok{\textgreater{}}\NormalTok{ printer }\OperatorTok{=} \OperatorTok{(}\NormalTok{s}\OperatorTok{)} \OperatorTok{{-}\textgreater{}} \BuiltInTok{System}\OperatorTok{.}\FunctionTok{out}\OperatorTok{.}\FunctionTok{println}\OperatorTok{(}\NormalTok{s}\OperatorTok{);}
\NormalTok{printer}\OperatorTok{.}\FunctionTok{accept}\OperatorTok{(}\StringTok{"Hello Consumer"}\OperatorTok{);}
\end{Highlighting}
\end{Shaded}

\subsection{Supplier}\label{supplier}

Takes no argument and returns a value. Used for lazy generation of
values.

\begin{itemize}
\tightlist
\item
  \textbf{Method}: \texttt{T\ get()}
\end{itemize}

\begin{Shaded}
\begin{Highlighting}[]
\NormalTok{Supplier}\OperatorTok{\textless{}}\BuiltInTok{Double}\OperatorTok{\textgreater{}}\NormalTok{ randomValue }\OperatorTok{=} \OperatorTok{()} \OperatorTok{{-}\textgreater{}} \BuiltInTok{Math}\OperatorTok{.}\FunctionTok{random}\OperatorTok{();}
\BuiltInTok{System}\OperatorTok{.}\FunctionTok{out}\OperatorTok{.}\FunctionTok{println}\OperatorTok{(}\NormalTok{randomValue}\OperatorTok{.}\FunctionTok{get}\OperatorTok{());}
\end{Highlighting}
\end{Shaded}

\subsection{Function}\label{function}

Takes an argument and returns a result. Used for transformation.

\begin{itemize}
\tightlist
\item
  \textbf{Method}: \texttt{R\ apply(T\ t)}
\end{itemize}

\begin{Shaded}
\begin{Highlighting}[]
\NormalTok{Function}\OperatorTok{\textless{}}\BuiltInTok{String}\OperatorTok{,} \BuiltInTok{Integer}\OperatorTok{\textgreater{}}\NormalTok{ lengthFinder }\OperatorTok{=} \OperatorTok{(}\NormalTok{s}\OperatorTok{)} \OperatorTok{{-}\textgreater{}}\NormalTok{ s}\OperatorTok{.}\FunctionTok{length}\OperatorTok{();}
\BuiltInTok{System}\OperatorTok{.}\FunctionTok{out}\OperatorTok{.}\FunctionTok{println}\OperatorTok{(}\NormalTok{lengthFinder}\OperatorTok{.}\FunctionTok{apply}\OperatorTok{(}\StringTok{"Hello"}\OperatorTok{));} \CommentTok{// 5}
\end{Highlighting}
\end{Shaded}

\subsection{Predicate}\label{predicate}

Takes an argument and returns a boolean. Used for conditional checks.

\begin{itemize}
\tightlist
\item
  \textbf{Method}: \texttt{boolean\ test(T\ t)}
\end{itemize}

\begin{Shaded}
\begin{Highlighting}[]
\BuiltInTok{Predicate}\OperatorTok{\textless{}}\BuiltInTok{Integer}\OperatorTok{\textgreater{}}\NormalTok{ isEven }\OperatorTok{=} \OperatorTok{(}\NormalTok{n}\OperatorTok{)} \OperatorTok{{-}\textgreater{}}\NormalTok{ n }\OperatorTok{\%} \DecValTok{2} \OperatorTok{==} \DecValTok{0}\OperatorTok{;}
\BuiltInTok{System}\OperatorTok{.}\FunctionTok{out}\OperatorTok{.}\FunctionTok{println}\OperatorTok{(}\NormalTok{isEven}\OperatorTok{.}\FunctionTok{test}\OperatorTok{(}\DecValTok{4}\OperatorTok{));} \CommentTok{// true}
\end{Highlighting}
\end{Shaded}

\section{Method References}\label{method-references}

A shorthand notation of a lambda expression to call a method.

\textbf{Syntax}: \texttt{ClassName::methodName}

\begin{Shaded}
\begin{Highlighting}[]
\BuiltInTok{List}\OperatorTok{\textless{}}\BuiltInTok{String}\OperatorTok{\textgreater{}}\NormalTok{ names }\OperatorTok{=} \BuiltInTok{Arrays}\OperatorTok{.}\FunctionTok{asList}\OperatorTok{(}\StringTok{"a"}\OperatorTok{,} \StringTok{"b"}\OperatorTok{,} \StringTok{"c"}\OperatorTok{);}

\CommentTok{// Lambda}
\NormalTok{names}\OperatorTok{.}\FunctionTok{forEach}\OperatorTok{(}\NormalTok{s }\OperatorTok{{-}\textgreater{}} \BuiltInTok{System}\OperatorTok{.}\FunctionTok{out}\OperatorTok{.}\FunctionTok{println}\OperatorTok{(}\NormalTok{s}\OperatorTok{));}

\CommentTok{// Method Reference}
\NormalTok{names}\OperatorTok{.}\FunctionTok{forEach}\OperatorTok{(}\BuiltInTok{System}\OperatorTok{.}\FunctionTok{out}\OperatorTok{::}\NormalTok{println}\OperatorTok{);}
\end{Highlighting}
\end{Shaded}

\section{Stream API}\label{stream-api}

The Stream API (\texttt{java.util.stream}) is used to process
collections of objects in a functional style.

\textbf{Operations:}

\begin{itemize}
\tightlist
\item
  \textbf{Intermediate}: \texttt{filter}, \texttt{map}, \texttt{sorted}
  (Lazy)
\item
  \textbf{Terminal}: \texttt{collect}, \texttt{forEach}, \texttt{reduce}
  (Eager)
\end{itemize}

\begin{Shaded}
\begin{Highlighting}[]
\BuiltInTok{List}\OperatorTok{\textless{}}\BuiltInTok{String}\OperatorTok{\textgreater{}}\NormalTok{ names }\OperatorTok{=} \BuiltInTok{Arrays}\OperatorTok{.}\FunctionTok{asList}\OperatorTok{(}\StringTok{"Alice"}\OperatorTok{,} \StringTok{"Bob"}\OperatorTok{,} \StringTok{"Charlie"}\OperatorTok{);}

\BuiltInTok{List}\OperatorTok{\textless{}}\BuiltInTok{String}\OperatorTok{\textgreater{}}\NormalTok{ result }\OperatorTok{=}\NormalTok{ names}\OperatorTok{.}\FunctionTok{stream}\OperatorTok{()}
    \OperatorTok{.}\FunctionTok{filter}\OperatorTok{(}\NormalTok{s }\OperatorTok{{-}\textgreater{}}\NormalTok{ s}\OperatorTok{.}\FunctionTok{startsWith}\OperatorTok{(}\StringTok{"A"}\OperatorTok{))}
    \OperatorTok{.}\FunctionTok{map}\OperatorTok{(}\BuiltInTok{String}\OperatorTok{::}\NormalTok{toUpperCase}\OperatorTok{)}
    \OperatorTok{.}\FunctionTok{collect}\OperatorTok{(}\NormalTok{Collectors}\OperatorTok{.}\FunctionTok{toList}\OperatorTok{());}

\BuiltInTok{System}\OperatorTok{.}\FunctionTok{out}\OperatorTok{.}\FunctionTok{println}\OperatorTok{(}\NormalTok{result}\OperatorTok{);} \CommentTok{// [ALICE]}
\end{Highlighting}
\end{Shaded}

\newpage

\chapter{Advanced Features}\label{advanced-features}

\etocsettocstyle{\textbf{Chapter Contents}\par\rule{\linewidth}{0.5pt}}{\par\rule{\linewidth}{0.5pt}}
\localtableofcontents

\noindent

This file covers modern Java language enhancements and features
introduced in recent versions (Java 7+).

\section{Try-with-resources}\label{try-with-resources}

Introduced in Java 7, it simplifies resource management by automatically
closing resources that implement the \texttt{AutoCloseable} interface.

\begin{Shaded}
\begin{Highlighting}[]
\CommentTok{// The resource (fis) is automatically closed at the end}
\ControlFlowTok{try} \OperatorTok{(}\BuiltInTok{FileInputStream}\NormalTok{ fis }\OperatorTok{=} \KeywordTok{new} \BuiltInTok{FileInputStream}\OperatorTok{(}\StringTok{"test.txt"}\OperatorTok{))} \OperatorTok{\{}
    \CommentTok{// read file}
\OperatorTok{\}} \ControlFlowTok{catch} \OperatorTok{(}\BuiltInTok{IOException}\NormalTok{ e}\OperatorTok{)} \OperatorTok{\{}
\NormalTok{    e}\OperatorTok{.}\FunctionTok{printStackTrace}\OperatorTok{();}
\OperatorTok{\}}
\end{Highlighting}
\end{Shaded}

\section{Annotations}\label{annotations}

\subsection{Type Annotations (Java 8)}\label{type-annotations-java-8}

Annotations can be placed almost anywhere a type is used, not just on
declarations. This is primarily used by plug-in type-checking tools to
find bugs (e.g., \texttt{@NonNull}).

\begin{Shaded}
\begin{Highlighting}[]
\BuiltInTok{List}\OperatorTok{\textless{}}\AttributeTok{@NonNull} \BuiltInTok{String}\OperatorTok{\textgreater{}}\NormalTok{ names }\OperatorTok{=} \KeywordTok{new} \BuiltInTok{ArrayList}\OperatorTok{\textless{}\textgreater{}();}
\end{Highlighting}
\end{Shaded}

\subsection{Repeating Annotations (Java
8)}\label{repeating-annotations-java-8}

Allows the same annotation to be applied multiple times to a single
declaration.

\begin{Shaded}
\begin{Highlighting}[]
\AttributeTok{@Schedule}\OperatorTok{(}\NormalTok{day}\OperatorTok{=}\StringTok{"Mon"}\OperatorTok{)}
\AttributeTok{@Schedule}\OperatorTok{(}\NormalTok{day}\OperatorTok{=}\StringTok{"Fri"}\OperatorTok{)}
\KeywordTok{class}\NormalTok{ MyTask }\OperatorTok{\{} \OperatorTok{\}}
\end{Highlighting}
\end{Shaded}

\section{Java Module System (Java 9)}\label{java-module-system-java-9}

Also known as \textbf{Project Jigsaw}. It is used to organize large
applications into modules.

\begin{itemize}
\tightlist
\item
  \textbf{Encapsulation}: Can hide entire packages.
\item
  \textbf{Dependencies}: Explicitly defines what a module needs.
\item
  \textbf{Descriptor}: Uses \texttt{module-info.java}.
\end{itemize}

\begin{Shaded}
\begin{Highlighting}[]
\CommentTok{// module{-}info.java}
\NormalTok{module com}\OperatorTok{.}\FunctionTok{myapp}\OperatorTok{.}\FunctionTok{main} \OperatorTok{\{}
\NormalTok{    requires com}\OperatorTok{.}\FunctionTok{myapp}\OperatorTok{.}\FunctionTok{utils}\OperatorTok{;} \CommentTok{// Depends on another module}
\NormalTok{    exports com}\OperatorTok{.}\FunctionTok{myapp}\OperatorTok{.}\FunctionTok{api}\OperatorTok{;}    \CommentTok{// Makes this package public}
\OperatorTok{\}}
\end{Highlighting}
\end{Shaded}

\section{Diamond Syntax with Anonymous Classes (Java
9)}\label{diamond-syntax-with-anonymous-classes-java-9}

The diamond operator \texttt{\textless{}\textgreater{}} can be used with
anonymous inner classes.

\begin{Shaded}
\begin{Highlighting}[]
\BuiltInTok{List}\OperatorTok{\textless{}}\BuiltInTok{String}\OperatorTok{\textgreater{}}\NormalTok{ list }\OperatorTok{=} \KeywordTok{new} \BuiltInTok{ArrayList}\OperatorTok{\textless{}\textgreater{}()} \OperatorTok{\{}
    \CommentTok{// Anonymous subclass body}
\OperatorTok{\};}
\end{Highlighting}
\end{Shaded}

\section{Local Variable Type Inference (Java
10)}\label{local-variable-type-inference-java-10}

Allows the compiler to infer the type of a local variable from its
initializer using the \texttt{var} keyword.

\begin{Shaded}
\begin{Highlighting}[]
\DataTypeTok{var}\NormalTok{ message }\OperatorTok{=} \StringTok{"Hello"}\OperatorTok{;} \CommentTok{// Inferred as String}
\DataTypeTok{var}\NormalTok{ map }\OperatorTok{=} \KeywordTok{new} \BuiltInTok{HashMap}\OperatorTok{\textless{}}\BuiltInTok{String}\OperatorTok{,} \BuiltInTok{Integer}\OperatorTok{\textgreater{}();} \CommentTok{// Inferred as HashMap}
\end{Highlighting}
\end{Shaded}

\section{Switch Expressions (Java 14)}\label{switch-expressions-java-14}

An enhanced switch statement that can be used as an expression (returns
a value).

\begin{itemize}
\tightlist
\item
  \textbf{Arrow Syntax (\texttt{-\textgreater{}})}: No fall-through, no
  \texttt{break} needed.
\item
  \textbf{Yield}: Used to return a value from a block.
\end{itemize}

\begin{Shaded}
\begin{Highlighting}[]
\BuiltInTok{String}\NormalTok{ day }\OperatorTok{=} \StringTok{"SAT"}\OperatorTok{;}
\BuiltInTok{String}\NormalTok{ type }\OperatorTok{=} \ControlFlowTok{switch} \OperatorTok{(}\NormalTok{day}\OperatorTok{)} \OperatorTok{\{}
    \ControlFlowTok{case} \StringTok{"MON"}\OperatorTok{,} \StringTok{"TUE"}\OperatorTok{,} \StringTok{"WED"}\OperatorTok{,} \StringTok{"THU"}\OperatorTok{,} \StringTok{"FRI"} \OperatorTok{{-}\textgreater{}} \StringTok{"Weekday"}\OperatorTok{;}
    \ControlFlowTok{case} \StringTok{"SAT"}\OperatorTok{,} \StringTok{"SUN"} \OperatorTok{{-}\textgreater{}} \StringTok{"Weekend"}\OperatorTok{;}
    \KeywordTok{default} \OperatorTok{{-}\textgreater{}} \StringTok{"Invalid"}\OperatorTok{;}
\OperatorTok{\};}
\end{Highlighting}
\end{Shaded}

\section{Records (Java 16)}\label{records-java-16}

A concise way to create immutable data carrier classes. The compiler
automatically generates the constructor, getters, \texttt{equals},
\texttt{hashCode}, and \texttt{toString}.

\begin{Shaded}
\begin{Highlighting}[]
\KeywordTok{record} \BuiltInTok{Point}\OperatorTok{(}\DataTypeTok{int}\NormalTok{ x}\OperatorTok{,} \DataTypeTok{int}\NormalTok{ y}\OperatorTok{)} \OperatorTok{\{\}}

\BuiltInTok{Point}\NormalTok{ p }\OperatorTok{=} \KeywordTok{new} \BuiltInTok{Point}\OperatorTok{(}\DecValTok{10}\OperatorTok{,} \DecValTok{20}\OperatorTok{);}
\BuiltInTok{System}\OperatorTok{.}\FunctionTok{out}\OperatorTok{.}\FunctionTok{println}\OperatorTok{(}\NormalTok{p}\OperatorTok{.}\FunctionTok{x}\OperatorTok{());} \CommentTok{// 10}
\end{Highlighting}
\end{Shaded}

\section{Sealed Classes (Java 17)}\label{sealed-classes-java-17}

Restricts which other classes or interfaces may extend or implement
them.

\begin{itemize}
\tightlist
\item
  \texttt{sealed}: Declares the class is sealed.
\item
  \texttt{permits}: Lists the allowed subclasses.
\end{itemize}

\begin{Shaded}
\begin{Highlighting}[]
\KeywordTok{sealed} \KeywordTok{interface} \BuiltInTok{Shape} \KeywordTok{permits}\NormalTok{ Circle}\OperatorTok{,}\NormalTok{ Square }\OperatorTok{\{\}}

\DataTypeTok{final} \KeywordTok{class}\NormalTok{ Circle }\KeywordTok{implements} \BuiltInTok{Shape} \OperatorTok{\{\}}
\DataTypeTok{final} \KeywordTok{class}\NormalTok{ Square }\KeywordTok{implements} \BuiltInTok{Shape} \OperatorTok{\{\}}
\end{Highlighting}
\end{Shaded}

\section{Text Blocks (Java 15)}\label{text-blocks-java-15}

Used to represent multi-line strings.

\begin{Shaded}
\begin{Highlighting}[]
\BuiltInTok{String}\NormalTok{ json }\OperatorTok{=} \StringTok{"""}
\StringTok{    \{}
\StringTok{        "name": "John",}
\StringTok{        "age": 30}
\StringTok{    \}}
\StringTok{    """}\OperatorTok{;}
\end{Highlighting}
\end{Shaded}

\newpage

\chapter{JDBC}\label{jdbc}

\etocsettocstyle{\textbf{Chapter Contents}\par\rule{\linewidth}{0.5pt}}{\par\rule{\linewidth}{0.5pt}}
\localtableofcontents

\noindent

JDBC is an API that helps Java applications communicate with databases.
It allows us to execute SQL statements and retrieve results.

\section{Architecture}\label{architecture}

\begin{figure}
\centering
\includegraphics[width=\linewidth,height=12cm,keepaspectratio,alt={~}]{images/mermaid_42d49cdccd14a0bec3580f23148ed8d9.png}
\caption{~}
\end{figure}

\subsection{JDBC Drivers}\label{jdbc-drivers}

\begin{enumerate}
\def\labelenumi{\arabic{enumi}.}
\tightlist
\item
  \textbf{Type 1 (JDBC-ODBC Bridge)}: Uses ODBC driver. Legacy.
\item
  \textbf{Type 2 (Native API)}: Uses client-side libraries.
\item
  \textbf{Type 3 (Network Protocol)}: Uses middleware server.
\item
  \textbf{Type 4 (Thin Driver)}: Pure Java driver. Direct connection.
  \textbf{Most common.}
\end{enumerate}

\section{Basic Steps}\label{basic-steps}

\begin{enumerate}
\def\labelenumi{\arabic{enumi}.}
\tightlist
\item
  \textbf{Import Package}: \texttt{import\ java.sql.*;}
\item
  \textbf{Load Driver}:
  \texttt{Class.forName("com.mysql.cj.jdbc.Driver");}
\item
  \textbf{Establish Connection}:
  \texttt{DriverManager.getConnection(url,\ user,\ pass);}
\item
  \textbf{Create Statement}: \texttt{con.createStatement();}
\item
  \textbf{Execute Query}: \texttt{executeQuery()} or
  \texttt{executeUpdate()}
\item
  \textbf{Process Results}: \texttt{ResultSet}
\item
  \textbf{Close Connection}: \texttt{con.close();}
\end{enumerate}

\begin{Shaded}
\begin{Highlighting}[]
\BuiltInTok{String}\NormalTok{ url }\OperatorTok{=} \StringTok{"jdbc:mysql://localhost:3306/mydb"}\OperatorTok{;}
\ControlFlowTok{try} \OperatorTok{(}\BuiltInTok{Connection}\NormalTok{ con }\OperatorTok{=} \BuiltInTok{DriverManager}\OperatorTok{.}\FunctionTok{getConnection}\OperatorTok{(}\NormalTok{url}\OperatorTok{,} \StringTok{"root"}\OperatorTok{,} \StringTok{"pass"}\OperatorTok{);}
     \BuiltInTok{Statement}\NormalTok{ st }\OperatorTok{=}\NormalTok{ con}\OperatorTok{.}\FunctionTok{createStatement}\OperatorTok{())} \OperatorTok{\{}

    \BuiltInTok{ResultSet}\NormalTok{ rs }\OperatorTok{=}\NormalTok{ st}\OperatorTok{.}\FunctionTok{executeQuery}\OperatorTok{(}\StringTok{"SELECT * FROM users"}\OperatorTok{);}
    \ControlFlowTok{while} \OperatorTok{(}\NormalTok{rs}\OperatorTok{.}\FunctionTok{next}\OperatorTok{())} \OperatorTok{\{}
        \BuiltInTok{System}\OperatorTok{.}\FunctionTok{out}\OperatorTok{.}\FunctionTok{println}\OperatorTok{(}\NormalTok{rs}\OperatorTok{.}\FunctionTok{getString}\OperatorTok{(}\StringTok{"name"}\OperatorTok{));}
    \OperatorTok{\}}
\OperatorTok{\}} \ControlFlowTok{catch} \OperatorTok{(}\BuiltInTok{SQLException}\NormalTok{ e}\OperatorTok{)} \OperatorTok{\{}
\NormalTok{    e}\OperatorTok{.}\FunctionTok{printStackTrace}\OperatorTok{();}
\OperatorTok{\}}
\end{Highlighting}
\end{Shaded}

\section{Prepared Statement}\label{prepared-statement}

Used for pre-compiled SQL statements. It is faster and prevents
\textbf{SQL Injection}.

\begin{Shaded}
\begin{Highlighting}[]
\BuiltInTok{String}\NormalTok{ sql }\OperatorTok{=} \StringTok{"INSERT INTO users (name, email) VALUES (?, ?)"}\OperatorTok{;}
\BuiltInTok{PreparedStatement}\NormalTok{ ps }\OperatorTok{=}\NormalTok{ con}\OperatorTok{.}\FunctionTok{prepareStatement}\OperatorTok{(}\NormalTok{sql}\OperatorTok{);}

\NormalTok{ps}\OperatorTok{.}\FunctionTok{setString}\OperatorTok{(}\DecValTok{1}\OperatorTok{,} \StringTok{"John"}\OperatorTok{);}
\NormalTok{ps}\OperatorTok{.}\FunctionTok{setString}\OperatorTok{(}\DecValTok{2}\OperatorTok{,} \StringTok{"john@example.com"}\OperatorTok{);}

\DataTypeTok{int}\NormalTok{ rows }\OperatorTok{=}\NormalTok{ ps}\OperatorTok{.}\FunctionTok{executeUpdate}\OperatorTok{();}
\end{Highlighting}
\end{Shaded}

\section{Transactions}\label{transactions}

A transaction is a group of operations that are treated as a single
unit. Either all succeed (Commit) or all fail (Rollback).

By default, JDBC is in \textbf{auto-commit} mode (each query is
committed immediately).

\begin{Shaded}
\begin{Highlighting}[]
\ControlFlowTok{try} \OperatorTok{\{}
\NormalTok{    con}\OperatorTok{.}\FunctionTok{setAutoCommit}\OperatorTok{(}\KeywordTok{false}\OperatorTok{);} \CommentTok{// 1. Disable auto{-}commit}

    \CommentTok{// Operation 1}
\NormalTok{    st}\OperatorTok{.}\FunctionTok{executeUpdate}\OperatorTok{(}\StringTok{"UPDATE account SET balance = balance {-} 100 WHERE id = 1"}\OperatorTok{);}

    \CommentTok{// Operation 2}
\NormalTok{    st}\OperatorTok{.}\FunctionTok{executeUpdate}\OperatorTok{(}\StringTok{"UPDATE account SET balance = balance + 100 WHERE id = 2"}\OperatorTok{);}

\NormalTok{    con}\OperatorTok{.}\FunctionTok{commit}\OperatorTok{();} \CommentTok{// 2. Commit if all good}
\OperatorTok{\}} \ControlFlowTok{catch} \OperatorTok{(}\BuiltInTok{SQLException}\NormalTok{ e}\OperatorTok{)} \OperatorTok{\{}
\NormalTok{    con}\OperatorTok{.}\FunctionTok{rollback}\OperatorTok{();} \CommentTok{// 3. Rollback if error}
\OperatorTok{\}}
\end{Highlighting}
\end{Shaded}

\section{Stored Procedures
(CallableStatement)}\label{stored-procedures-callablestatement}

Used to call stored procedures in the database.

\begin{Shaded}
\begin{Highlighting}[]
\CommentTok{// Procedure: CREATE PROCEDURE get\_user(IN id INT, OUT name VARCHAR(50))}
\BuiltInTok{CallableStatement}\NormalTok{ cs }\OperatorTok{=}\NormalTok{ con}\OperatorTok{.}\FunctionTok{prepareCall}\OperatorTok{(}\StringTok{"\{call get\_user(?, ?)\}"}\OperatorTok{);}

\NormalTok{cs}\OperatorTok{.}\FunctionTok{setInt}\OperatorTok{(}\DecValTok{1}\OperatorTok{,} \DecValTok{10}\OperatorTok{);} \CommentTok{// Input parameter}
\NormalTok{cs}\OperatorTok{.}\FunctionTok{registerOutParameter}\OperatorTok{(}\DecValTok{2}\OperatorTok{,} \BuiltInTok{Types}\OperatorTok{.}\FunctionTok{VARCHAR}\OperatorTok{);} \CommentTok{// Output parameter}

\NormalTok{cs}\OperatorTok{.}\FunctionTok{execute}\OperatorTok{();}

\BuiltInTok{String}\NormalTok{ name }\OperatorTok{=}\NormalTok{ cs}\OperatorTok{.}\FunctionTok{getString}\OperatorTok{(}\DecValTok{2}\OperatorTok{);} \CommentTok{// Get output}
\end{Highlighting}
\end{Shaded}

\section{Merging Data (Joins)}\label{merging-data-joins}

We can execute complex SQL queries involving joins using JDBC just like
normal queries.

\begin{Shaded}
\begin{Highlighting}[]
\BuiltInTok{String}\NormalTok{ sql }\OperatorTok{=} \StringTok{"SELECT u.name, o.order\_id FROM users u JOIN orders o ON u.id = o.user\_id"}\OperatorTok{;}
\BuiltInTok{ResultSet}\NormalTok{ rs }\OperatorTok{=}\NormalTok{ st}\OperatorTok{.}\FunctionTok{executeQuery}\OperatorTok{(}\NormalTok{sql}\OperatorTok{);}
\end{Highlighting}
\end{Shaded}

\newpage

\chapter{JSP}\label{jsp}

\etocsettocstyle{\textbf{Chapter Contents}\par\rule{\linewidth}{0.5pt}}{\par\rule{\linewidth}{0.5pt}}
\localtableofcontents

\noindent

JSP is a server-side technology used to create dynamic web content. It
is an extension of Servlets. JSP files are compiled into Servlets by the
container.

\section{JSP Life Cycle}\label{jsp-life-cycle}

\begin{enumerate}
\def\labelenumi{\arabic{enumi}.}
\tightlist
\item
  \textbf{Translation}: JSP -\textgreater{} Servlet (.java)
\item
  \textbf{Compilation}: Servlet -\textgreater{} Bytecode (.class)
\item
  \textbf{Loading \& Initialization}: \texttt{jspInit()}
\item
  \textbf{Execution}: \texttt{\_jspService()}
\item
  \textbf{Destruction}: \texttt{jspDestroy()}
\end{enumerate}

\section{Scripting Elements}\label{scripting-elements}

\begin{enumerate}
\def\labelenumi{\arabic{enumi}.}
\item
  \textbf{Scriptlet Tag} \texttt{\textless{}\%\ ...\ \%\textgreater{}}:
  Contains Java code.

\begin{Shaded}
\begin{Highlighting}[]
\PreprocessorTok{\textless{}\%} \DataTypeTok{int}\NormalTok{ count }\OperatorTok{=} \DecValTok{0}\OperatorTok{;}\NormalTok{ out}\OperatorTok{.}\FunctionTok{println}\OperatorTok{(}\NormalTok{count}\OperatorTok{);} \PreprocessorTok{\%\textgreater{}}
\end{Highlighting}
\end{Shaded}
\item
  \textbf{Expression Tag}
  \texttt{\textless{}\%=\ ...\ \%\textgreater{}}: Prints a value (no
  semicolon).

\begin{Shaded}
\begin{Highlighting}[]
\PreprocessorTok{\textless{}\%=} \StringTok{"Hello "} \OperatorTok{+}\NormalTok{ name }\PreprocessorTok{\%\textgreater{}}
\end{Highlighting}
\end{Shaded}
\item
  \textbf{Declaration Tag}
  \texttt{\textless{}\%!\ ...\ \%\textgreater{}}: Declares methods or
  variables.

\begin{Shaded}
\begin{Highlighting}[]
\PreprocessorTok{\textless{}\%!} \DataTypeTok{int} \FunctionTok{square}\OperatorTok{(}\DataTypeTok{int}\NormalTok{ n}\OperatorTok{)} \OperatorTok{\{} \ControlFlowTok{return}\NormalTok{ n}\OperatorTok{*}\NormalTok{n}\OperatorTok{;} \OperatorTok{\}} \PreprocessorTok{\%\textgreater{}}
\end{Highlighting}
\end{Shaded}
\end{enumerate}

\section{Implicit Objects}\label{implicit-objects}

Objects created by the container and available automatically.

\begin{enumerate}
\def\labelenumi{\arabic{enumi}.}
\tightlist
\item
  \textbf{request}: \texttt{HttpServletRequest}
\item
  \textbf{response}: \texttt{HttpServletResponse}
\item
  \textbf{out}: \texttt{JspWriter} (for sending output)
\item
  \textbf{session}: \texttt{HttpSession}
\item
  \textbf{application}: \texttt{ServletContext}
\item
  \textbf{config}: \texttt{ServletConfig}
\item
  \textbf{pageContext}: Context for the page
\item
  \textbf{page}: \texttt{this} (current servlet instance)
\item
  \textbf{exception}: \texttt{Throwable} (only in error pages)
\end{enumerate}

\section{Directives}\label{directives}

Instructions to the container. Syntax:
\texttt{\textless{}\%@\ directive\ ...\ \%\textgreater{}}

\begin{enumerate}
\def\labelenumi{\arabic{enumi}.}
\item
  \textbf{page}: Defines page settings.

\begin{Shaded}
\begin{Highlighting}[]
\BuiltInTok{\textless{}\%@ page}\OtherTok{ language}\NormalTok{=}\StringTok{"java"}\OtherTok{ contentType}\NormalTok{=}\StringTok{"text/html"}\OtherTok{ import}\NormalTok{=}\StringTok{"java.util.*"}\OtherTok{ }\BuiltInTok{\%\textgreater{}}
\end{Highlighting}
\end{Shaded}
\item
  \textbf{include}: Includes a file at translation time (static).

\begin{Shaded}
\begin{Highlighting}[]
\BuiltInTok{\textless{}\%@ include}\OtherTok{ file}\NormalTok{=}\StringTok{"header.jsp"}\OtherTok{ }\BuiltInTok{\%\textgreater{}}
\end{Highlighting}
\end{Shaded}
\item
  \textbf{taglib}: Defines tag libraries (like JSTL).

\begin{Shaded}
\begin{Highlighting}[]
\BuiltInTok{\textless{}\%@ taglib}\OtherTok{ uri}\NormalTok{=}\StringTok{"http://java.sun.com/jsp/jstl/core"}\OtherTok{ prefix}\NormalTok{=}\StringTok{"c"}\OtherTok{ }\BuiltInTok{\%\textgreater{}}
\end{Highlighting}
\end{Shaded}
\end{enumerate}

\section{Standard Actions}\label{standard-actions}

XML tags to perform common tasks.

\begin{enumerate}
\def\labelenumi{\arabic{enumi}.}
\item
  \textbf{jsp:include}: Includes a resource at request time (dynamic).

\begin{Shaded}
\begin{Highlighting}[]
\BuiltInTok{\textless{}jsp:include}\OtherTok{ page}\NormalTok{=}\StringTok{"footer.jsp"}\BuiltInTok{ /\textgreater{}}
\end{Highlighting}
\end{Shaded}
\item
  \textbf{jsp:forward}: Forwards request to another page.

\begin{Shaded}
\begin{Highlighting}[]
\BuiltInTok{\textless{}jsp:forward}\OtherTok{ page}\NormalTok{=}\StringTok{"login.jsp"}\BuiltInTok{ /\textgreater{}}
\end{Highlighting}
\end{Shaded}
\item
  \textbf{jsp:useBean}: Instantiates a JavaBean.

\begin{Shaded}
\begin{Highlighting}[]
\BuiltInTok{\textless{}jsp:useBean}\OtherTok{ id}\NormalTok{=}\StringTok{"user"}\OtherTok{ class}\NormalTok{=}\StringTok{"com.example.User"}\BuiltInTok{ /\textgreater{}}
\BuiltInTok{\textless{}jsp:setProperty}\OtherTok{ name}\NormalTok{=}\StringTok{"user"}\OtherTok{ property}\NormalTok{=}\StringTok{"name"}\OtherTok{ value}\NormalTok{=}\StringTok{"John"}\BuiltInTok{ /\textgreater{}}
\end{Highlighting}
\end{Shaded}
\end{enumerate}

\newpage

\chapter{Servlets}\label{servlets}

\etocsettocstyle{\textbf{Chapter Contents}\par\rule{\linewidth}{0.5pt}}{\par\rule{\linewidth}{0.5pt}}
\localtableofcontents

\noindent

A Servlet is a Java class that runs on a web server and responds to HTTP
requests. It extends the capabilities of a server.

\section{Architecture}\label{architecture-1}

\begin{figure}
\centering
\includegraphics[width=\linewidth,height=12cm,keepaspectratio,alt={~}]{images/mermaid_7bdc358ddbd0842bfa9bd342aeecc929.png}
\caption{~}
\end{figure}

\section{Servlet Life Cycle}\label{servlet-life-cycle}

The life cycle is managed by the Servlet Container (e.g., Tomcat).

\begin{enumerate}
\def\labelenumi{\arabic{enumi}.}
\tightlist
\item
  \textbf{Loading \& Instantiation}: The container loads the class and
  creates an instance.
\item
  \textbf{Initialization (\texttt{init()})}: Called once. Used for
  setup.
\item
  \textbf{Request Handling (\texttt{service()})}: Called for every
  request. Dispatches to \texttt{doGet()}, \texttt{doPost()}, etc.
\item
  \textbf{Destruction (\texttt{destroy()})}: Called once before removing
  the servlet. Used for cleanup.
\end{enumerate}

\begin{figure}
\centering
\includegraphics[width=\linewidth,height=12cm,keepaspectratio,alt={~}]{images/mermaid_3276dbb3b5ce7349bc9f3cb21074559f.png}
\caption{~}
\end{figure}

\section{Handling Requests}\label{handling-requests}

We typically extend \texttt{HttpServlet}.

\begin{Shaded}
\begin{Highlighting}[]
\AttributeTok{@WebServlet}\OperatorTok{(}\StringTok{"/hello"}\OperatorTok{)}
\KeywordTok{public} \KeywordTok{class}\NormalTok{ HelloServlet }\KeywordTok{extends}\NormalTok{ HttpServlet }\OperatorTok{\{}

    \CommentTok{// Handle GET requests}
    \KeywordTok{protected} \DataTypeTok{void} \FunctionTok{doGet}\OperatorTok{(}\NormalTok{HttpServletRequest req}\OperatorTok{,}\NormalTok{ HttpServletResponse res}\OperatorTok{)} \KeywordTok{throws} \BuiltInTok{IOException} \OperatorTok{\{}
\NormalTok{        res}\OperatorTok{.}\FunctionTok{setContentType}\OperatorTok{(}\StringTok{"text/html"}\OperatorTok{);}
        \BuiltInTok{PrintWriter}\NormalTok{ out }\OperatorTok{=}\NormalTok{ res}\OperatorTok{.}\FunctionTok{getWriter}\OperatorTok{();}
\NormalTok{        out}\OperatorTok{.}\FunctionTok{println}\OperatorTok{(}\StringTok{"\textless{}h1\textgreater{}Hello World\textless{}/h1\textgreater{}"}\OperatorTok{);}
    \OperatorTok{\}}

    \CommentTok{// Handle POST requests}
    \KeywordTok{protected} \DataTypeTok{void} \FunctionTok{doPost}\OperatorTok{(}\NormalTok{HttpServletRequest req}\OperatorTok{,}\NormalTok{ HttpServletResponse res}\OperatorTok{)} \KeywordTok{throws} \BuiltInTok{IOException} \OperatorTok{\{}
        \BuiltInTok{String}\NormalTok{ name }\OperatorTok{=}\NormalTok{ req}\OperatorTok{.}\FunctionTok{getParameter}\OperatorTok{(}\StringTok{"username"}\OperatorTok{);}
        \CommentTok{// Process data...}
    \OperatorTok{\}}
\OperatorTok{\}}
\end{Highlighting}
\end{Shaded}

\section{Session Tracking}\label{session-tracking}

HTTP is \textbf{stateless}. To maintain state between requests (e.g.,
login status), we use session tracking.

\subsection{Cookies}\label{cookies}

Small pieces of data stored on the client's browser.

\begin{Shaded}
\begin{Highlighting}[]
\CommentTok{// Create Cookie}
\NormalTok{Cookie c }\OperatorTok{=} \KeywordTok{new} \FunctionTok{Cookie}\OperatorTok{(}\StringTok{"user"}\OperatorTok{,} \StringTok{"John"}\OperatorTok{);}
\NormalTok{res}\OperatorTok{.}\FunctionTok{addCookie}\OperatorTok{(}\NormalTok{c}\OperatorTok{);}

\CommentTok{// Read Cookie}
\NormalTok{Cookie}\OperatorTok{[]}\NormalTok{ cookies }\OperatorTok{=}\NormalTok{ req}\OperatorTok{.}\FunctionTok{getCookies}\OperatorTok{();}
\end{Highlighting}
\end{Shaded}

\subsection{HttpSession}\label{httpsession}

Stored on the server. A unique Session ID is sent to the client (usually
via cookie).

\begin{Shaded}
\begin{Highlighting}[]
\CommentTok{// Create/Get Session}
\NormalTok{HttpSession session }\OperatorTok{=}\NormalTok{ req}\OperatorTok{.}\FunctionTok{getSession}\OperatorTok{();}
\NormalTok{session}\OperatorTok{.}\FunctionTok{setAttribute}\OperatorTok{(}\StringTok{"user"}\OperatorTok{,} \StringTok{"John"}\OperatorTok{);}

\CommentTok{// Retrieve Data}
\BuiltInTok{String}\NormalTok{ user }\OperatorTok{=} \OperatorTok{(}\BuiltInTok{String}\OperatorTok{)}\NormalTok{ session}\OperatorTok{.}\FunctionTok{getAttribute}\OperatorTok{(}\StringTok{"user"}\OperatorTok{);}

\CommentTok{// Invalidate (Logout)}
\NormalTok{session}\OperatorTok{.}\FunctionTok{invalidate}\OperatorTok{();}
\end{Highlighting}
\end{Shaded}

\subsection{URL Rewriting}\label{url-rewriting}

Appending the session ID to the URL.
\texttt{http://example.com/app?jsessionid=12345}

\subsection{Hidden Form Fields}\label{hidden-form-fields}

\texttt{\textless{}input\ type="hidden"\ name="session\_id"\ value="12345"\textgreater{}}

\newpage

\chapter{Spring Framework}\label{spring-framework}

\etocsettocstyle{\textbf{Chapter Contents}\par\rule{\linewidth}{0.5pt}}{\par\rule{\linewidth}{0.5pt}}
\localtableofcontents

\noindent

Spring is a powerful, lightweight, and open-source framework for
building enterprise Java applications. It provides comprehensive
infrastructure support, allowing developers to focus on the application
logic.

\section{Spring MVC Architecture}\label{spring-mvc-architecture}

Spring MVC (Model-View-Controller) is a framework built on the Servlet
API. It is designed around a central servlet called
\textbf{DispatcherServlet} that dispatches requests to controllers.

\subsection{Key Components}\label{key-components}

\begin{enumerate}
\def\labelenumi{\arabic{enumi}.}
\tightlist
\item
  \textbf{DispatcherServlet}: The ``Front Controller''. It receives all
  incoming HTTP requests and delegates them to other components.
\item
  \textbf{HandlerMapping}: Maps a request to a specific Controller
  method (based on URL, method, etc.).
\item
  \textbf{Controller}: Handles the request, processes business logic
  (often by calling a Service), and returns a Model and View name.
\item
  \textbf{ViewResolver}: Resolves the view name (e.g., ``home'') to an
  actual view file (e.g., \texttt{/WEB-INF/views/home.jsp}).
\item
  \textbf{View}: Renders the response (HTML, JSON, etc.).
\end{enumerate}

\subsection{Request Flow}\label{request-flow}

\begin{figure}
\centering
\includegraphics[width=\linewidth,height=12cm,keepaspectratio,alt={~}]{images/mermaid_9edb12a375215ba8c3205ada5308d411.png}
\caption{~}
\end{figure}

\begin{enumerate}
\def\labelenumi{\arabic{enumi}.}
\tightlist
\item
  \textbf{Request}: Client sends a request to the server.
\item
  \textbf{Dispatch}: \texttt{DispatcherServlet} intercepts the request.
\item
  \textbf{Mapping}: It asks \texttt{HandlerMapping} which Controller
  should handle this request.
\item
  \textbf{Execution}: It calls the \texttt{Controller}.
\item
  \textbf{Processing}: Controller processes the request and returns a
  \texttt{ModelAndView} object (data + view name).
\item
  \textbf{Resolution}: \texttt{DispatcherServlet} asks
  \texttt{ViewResolver} to find the actual View file.
\item
  \textbf{Rendering}: The View is rendered with the Model data and sent
  back to the client.
\end{enumerate}

\section{Core Concepts}\label{core-concepts}

\subsection{POJO (Plain Old Java
Object)}\label{pojo-plain-old-java-object}

A \textbf{POJO} is an ordinary Java object that is not bound by any
special restriction other than those forced by the Java Language
Specification. It does not depend on any external libraries or
frameworks.

\begin{itemize}
\tightlist
\item
  \textbf{Characteristics}:

  \begin{itemize}
  \tightlist
  \item
    Does \textbf{not} extend prespecified classes (e.g.,
    \texttt{extends\ HttpServlet}).
  \item
    Does \textbf{not} implement prespecified interfaces (e.g.,
    \texttt{implements\ EntityBean}).
  \item
    Does \textbf{not} contain prespecified annotations.
  \end{itemize}
\item
  \textbf{Benefit}: Increases readability, reusability, and testability.
  Spring empowers Pojo-based programming.
\end{itemize}

\begin{Shaded}
\begin{Highlighting}[]
\KeywordTok{public} \KeywordTok{class}\NormalTok{ StudentPojo }\OperatorTok{\{}
    \KeywordTok{private} \BuiltInTok{String}\NormalTok{ name}\OperatorTok{;}
    \KeywordTok{private} \DataTypeTok{int}\NormalTok{ id}\OperatorTok{;}

    \CommentTok{// Standard Getters and Setters}
    \KeywordTok{public} \BuiltInTok{String} \FunctionTok{getName}\OperatorTok{()} \OperatorTok{\{} \ControlFlowTok{return}\NormalTok{ name}\OperatorTok{;} \OperatorTok{\}}
    \KeywordTok{public} \DataTypeTok{void} \FunctionTok{setName}\OperatorTok{(}\BuiltInTok{String}\NormalTok{ name}\OperatorTok{)} \OperatorTok{\{} \KeywordTok{this}\OperatorTok{.}\FunctionTok{name} \OperatorTok{=}\NormalTok{ name}\OperatorTok{;} \OperatorTok{\}}
\OperatorTok{\}}
\end{Highlighting}
\end{Shaded}

\subsection{Inversion of Control (IoC)}\label{inversion-of-control-ioc}

In traditional programming, the custom code calls the library to perform
tasks. In \textbf{IoC}, the flow of control is inverted: the framework
(container) calls the custom code.

\begin{itemize}
\tightlist
\item
  \textbf{Principle}: ``Don't call us, we'll call you.'' (Hollywood
  Principle).
\item
  \textbf{Role}: The IoC Container is responsible for instantiating,
  configuring, and assembling objects (Beans).
\end{itemize}

\subsubsection{IoC Containers}\label{ioc-containers}

Spring provides two types of containers:

\begin{enumerate}
\def\labelenumi{\arabic{enumi}.}
\tightlist
\item
  \textbf{BeanFactory}: The simplest container, providing basic DI
  support. (Lazy loading).
\item
  \textbf{ApplicationContext}: Extends BeanFactory, adding
  enterprise-specific features like event propagation, declarative
  mechanisms, and integration with AOP. (Eager loading).
\end{enumerate}

\begin{Shaded}
\begin{Highlighting}[]
\KeywordTok{class}\NormalTok{ Application }\OperatorTok{\{}
    \KeywordTok{public} \DataTypeTok{static} \DataTypeTok{void} \FunctionTok{main}\OperatorTok{(}\BuiltInTok{String}\OperatorTok{[]}\NormalTok{ args}\OperatorTok{)} \OperatorTok{\{}
\NormalTok{        ApplicationContext context }\OperatorTok{=} \KeywordTok{new} \FunctionTok{ClassPathXmlApplicationContext}\OperatorTok{(}\StringTok{"applicationContext.xml"}\OperatorTok{);}
\NormalTok{        Student student }\OperatorTok{=} \OperatorTok{(}\NormalTok{Student}\OperatorTok{)}\NormalTok{ context}\OperatorTok{.}\FunctionTok{getBean}\OperatorTok{(}\StringTok{"student"}\OperatorTok{);}
\NormalTok{        student}\OperatorTok{.}\FunctionTok{getCourse}\OperatorTok{();}
    \OperatorTok{\}}
\OperatorTok{\}}
\end{Highlighting}
\end{Shaded}

\subsection{Dependency Injection (DI)}\label{dependency-injection-di}

DI is the specific design pattern used to implement IoC. It removes the
dependency creation responsibility from the class.

\begin{tipblock}{Analogy: The Restaurant}

\begin{itemize}
\tightlist
\item
  \textbf{Traditional (No IoC):} You are hungry. You go to the kitchen,
  find ingredients, cook the meal, and eat it. You control everything.
\item
  \textbf{IoC (Inversion of Control):} You sit at a table. You don't go
  to the kitchen. You just \emph{declare} what you want.
\item
  \textbf{Dependency Injection (DI):} The \textbf{Waiter} (Container)
  brings the food (Dependency) to your table. You didn't make it; it was
  \emph{injected} into your table setting by the waiter.
\end{itemize}

\end{tipblock}

\textbf{Types of DI:}

\begin{itemize}
\tightlist
\item
  \hyperref[constructor-injection-recommended]{\#Constructor Injection (Recommended)}
\item
  \hyperref[setter-injection]{\#Setter Injection}
\item
  \hyperref[field-injection]{\#Field Injection}
\end{itemize}

\subsubsection{Constructor Injection
(Recommended)}\label{constructor-injection-recommended}

Dependencies are providfed through the class constructor.

\begin{itemize}
\tightlist
\item
  \textbf{Pros}: Ensures the object is fully initialized (immutable).
  Good for mandatory dependencies.
\item
  \textbf{Cons}: Can lead to ``Constructor bloat'' if too many
  dependencies.
\end{itemize}

\begin{Shaded}
\begin{Highlighting}[]
\AttributeTok{@Service}
\KeywordTok{class}\NormalTok{ UserService }\OperatorTok{\{}
    \KeywordTok{private} \DataTypeTok{final}\NormalTok{ UserRepository repo}\OperatorTok{;}

    \AttributeTok{@Autowired}
    \KeywordTok{public} \FunctionTok{UserService}\OperatorTok{(}\NormalTok{UserRepository repo}\OperatorTok{)} \OperatorTok{\{}
        \KeywordTok{this}\OperatorTok{.}\FunctionTok{repo} \OperatorTok{=}\NormalTok{ repo}\OperatorTok{;}
    \OperatorTok{\}}
\OperatorTok{\}}
\end{Highlighting}
\end{Shaded}

\subsubsection{Setter Injection}\label{setter-injection}

Dependencies are provided through setter methods.

\begin{itemize}
\tightlist
\item
  \textbf{Pros}: Good for optional dependencies. Can re-inject
  dependencies later.
\item
  \textbf{Cons}: Object might be in a partial state if setters aren't
  called.
\end{itemize}

\begin{Shaded}
\begin{Highlighting}[]
\AttributeTok{@Service}
\KeywordTok{class}\NormalTok{ OrderService }\OperatorTok{\{}
    \KeywordTok{private}\NormalTok{ OrderRepository repo}\OperatorTok{;}

    \AttributeTok{@Autowired}
    \KeywordTok{public} \DataTypeTok{void} \FunctionTok{setRepo}\OperatorTok{(}\NormalTok{OrderRepository repo}\OperatorTok{)} \OperatorTok{\{}
        \KeywordTok{this}\OperatorTok{.}\FunctionTok{repo} \OperatorTok{=}\NormalTok{ repo}\OperatorTok{;}
    \OperatorTok{\}}
\OperatorTok{\}}
\end{Highlighting}
\end{Shaded}

\subsubsection{Field Injection}\label{field-injection}

Dependencies are injected directly into fields using reflection.

\begin{itemize}
\tightlist
\item
  \textbf{Pros}: Very concise code.
\item
  \textbf{Cons}: Hides dependencies. Difficult to test (cannot
  instantiate without reflection). \textbf{Not recommended.}
\end{itemize}

\begin{Shaded}
\begin{Highlighting}[]
\AttributeTok{@Service}
\KeywordTok{class}\NormalTok{ PaymentService }\OperatorTok{\{}
    \AttributeTok{@Autowired}
    \KeywordTok{private}\NormalTok{ PaymentRepository repo}\OperatorTok{;} \CommentTok{// Field Injection}
\OperatorTok{\}}
\end{Highlighting}
\end{Shaded}

\subsection{Aspect-Oriented Programming
(AOP)}\label{aspect-oriented-programming-aop}

AOP is a programming paradigm that aims to increase modularity by
allowing the separation of \textbf{cross-cutting concerns}
(functionality that affects multiple parts of an application, like
logging, security, or transaction management).

\textbf{Key Terminologies:}

\begin{itemize}
\tightlist
\item
  \textbf{Aspect}: A module that encapsulates a concern (e.g.,
  \texttt{LoggingAspect}).
\item
  \textbf{Advice}: The actual action to be taken (code to run).

  \begin{itemize}
  \tightlist
  \item
    \texttt{@Before}: Run before the method.
  \item
    \texttt{@After}: Run after the method (regardless of outcome).
  \item
    \texttt{@AfterReturning}: Run only if method succeeds.
  \item
    \texttt{@AfterThrowing}: Run only if method throws exception.
  \item
    \texttt{@Around}: Run before and after (can control execution).
  \end{itemize}
\item
  \textbf{Pointcut}: An expression that selects where the Advice should
  be applied (e.g., ``all methods in Service package'').
\item
  \textbf{JoinPoint}: The specific point in execution (e.g., method
  execution) where an aspect can be plugged in.
\item
  \textbf{Weaving}: The process of linking aspects with other
  application types to create an advised object.
\end{itemize}

\subsubsection{Example: Around Advice (Measuring Execution
Time)}\label{example-around-advice-measuring-execution-time}

\begin{Shaded}
\begin{Highlighting}[]
\AttributeTok{@Aspect}
\AttributeTok{@Component}
\KeywordTok{public} \KeywordTok{class}\NormalTok{ PerformanceAspect }\OperatorTok{\{}

    \AttributeTok{@Around}\OperatorTok{(}\StringTok{"execution(* com.example.service.*.*(..))"}\OperatorTok{)}
    \KeywordTok{public} \BuiltInTok{Object} \FunctionTok{measureTime}\OperatorTok{(}\NormalTok{ProceedingJoinPoint joinPoint}\OperatorTok{)} \KeywordTok{throws} \BuiltInTok{Throwable} \OperatorTok{\{}
        \DataTypeTok{long}\NormalTok{ start }\OperatorTok{=} \BuiltInTok{System}\OperatorTok{.}\FunctionTok{currentTimeMillis}\OperatorTok{();}

        \BuiltInTok{Object}\NormalTok{ result }\OperatorTok{=}\NormalTok{ joinPoint}\OperatorTok{.}\FunctionTok{proceed}\OperatorTok{();} \CommentTok{// Execute the actual method}

        \DataTypeTok{long}\NormalTok{ end }\OperatorTok{=} \BuiltInTok{System}\OperatorTok{.}\FunctionTok{currentTimeMillis}\OperatorTok{();}
        \BuiltInTok{System}\OperatorTok{.}\FunctionTok{out}\OperatorTok{.}\FunctionTok{println}\OperatorTok{(}\StringTok{"Execution time: "} \OperatorTok{+} \OperatorTok{(}\NormalTok{end }\OperatorTok{{-}}\NormalTok{ start}\OperatorTok{)} \OperatorTok{+} \StringTok{"ms"}\OperatorTok{);}

        \ControlFlowTok{return}\NormalTok{ result}\OperatorTok{;}
    \OperatorTok{\}}
\OperatorTok{\}}
\end{Highlighting}
\end{Shaded}

\section{\texorpdfstring{\hyperref[beans-and-autowiring]{Beans and Autowiring}}{Beans and Autowiring}}\label{beans-and-autowiring}

\section{Configuration Styles}\label{configuration-styles}

Spring supports three ways to define beans:

\subsection{XML Configuration (Legacy)}\label{xml-configuration-legacy}

Beans are defined in an XML file (e.g.,
\texttt{applicationContext.xml}).

\begin{Shaded}
\begin{Highlighting}[]
\NormalTok{\textless{}}\KeywordTok{bean} \OtherTok{id=}\StringTok{"Student"} \OtherTok{class=}\StringTok{"Student"}\NormalTok{\textgreater{}}
\NormalTok{    \textless{}}\KeywordTok{property} \OtherTok{name=}\StringTok{"Course"}\NormalTok{/\textgreater{}}
\NormalTok{\textless{}/}\KeywordTok{bean}\NormalTok{\textgreater{}}
\NormalTok{\textless{}}\KeywordTok{bean} \OtherTok{id=}\StringTok{"Course"} \OtherTok{class=}\StringTok{"Student"} \OtherTok{autowire=}\StringTok{"course"}\NormalTok{/\textgreater{}}
\end{Highlighting}
\end{Shaded}

\subsection{Annotation-based
Configuration}\label{annotation-based-configuration}

Beans are defined using annotations on classes.

\begin{itemize}
\tightlist
\item
  \texttt{@Component}, \texttt{@Service}, \texttt{@Repository},
  \texttt{@Controller}
\item
  \texttt{@Autowired} for injection.
\item
  Requires \texttt{\textless{}context:component-scan\textgreater{}} or
  \texttt{@ComponentScan}.
\end{itemize}

\subsection{Java-based Configuration
(Modern)}\label{java-based-configuration-modern}

Beans are defined in a Java class using \texttt{@Configuration} and
\texttt{@Bean}.

\begin{Shaded}
\begin{Highlighting}[]
\AttributeTok{@Configuration}
\KeywordTok{public} \KeywordTok{class}\NormalTok{ AppConfig }\OperatorTok{\{}
    \AttributeTok{@Bean}
    \KeywordTok{public}\NormalTok{ MyService }\FunctionTok{myService}\OperatorTok{()} \OperatorTok{\{}
        \ControlFlowTok{return} \KeywordTok{new} \FunctionTok{MyService}\OperatorTok{();}
    \OperatorTok{\}}
\OperatorTok{\}}
\end{Highlighting}
\end{Shaded}

\newpage

\chapter{Beans and Autowiring}\label{beans-and-autowiring}

\etocsettocstyle{\textbf{Chapter Contents}\par\rule{\linewidth}{0.5pt}}{\par\rule{\linewidth}{0.5pt}}
\localtableofcontents

\noindent

They are objects managed by spring IoC container. It forms the backbone
of the application.

The scope defines the lifecycle and visibility of a bean managed by the
Spring container.

{\def\LTcaptype{none} % do not increment counter
\begin{longtable}[]{@{}
  >{\raggedright\arraybackslash}p{(\linewidth - 4\tabcolsep) * \real{0.1376}}
  >{\raggedright\arraybackslash}p{(\linewidth - 4\tabcolsep) * \real{0.5596}}
  >{\raggedright\arraybackslash}p{(\linewidth - 4\tabcolsep) * \real{0.3028}}@{}}
\toprule\noalign{}
\begin{minipage}[b]{\linewidth}\raggedright
Scope
\end{minipage} & \begin{minipage}[b]{\linewidth}\raggedright
Description
\end{minipage} & \begin{minipage}[b]{\linewidth}\raggedright
Use Case
\end{minipage} \\
\midrule\noalign{}
\endhead
\bottomrule\noalign{}
\endlastfoot
\textbf{Singleton} & \textbf{(Default)} Only one instance is created per
IoC container. & Stateless beans (Services, DAOs). \\
\textbf{Prototype} & A new instance is created every time the bean is
requested. & Stateful beans (User sessions). \\
\textbf{Request} & One instance per single HTTP request. & Web apps
(Request-specific data). \\
\textbf{Session} & One instance per HTTP Session. & Web apps (User login
info). \\
\textbf{Application} & One instance per \texttt{ServletContext}. & Web
apps (Global config). \\
\textbf{WebSocket} & One instance per WebSocket lifecycle. & Real-time
apps. \\
\end{longtable}
}

\begin{Shaded}
\begin{Highlighting}[]
\AttributeTok{@Component}
\AttributeTok{@Scope}\OperatorTok{(}\StringTok{"prototype"}\OperatorTok{)}
\KeywordTok{public} \KeywordTok{class}\NormalTok{ MyPrototypeBean }\OperatorTok{\{} \KeywordTok{...} \OperatorTok{\}}
\end{Highlighting}
\end{Shaded}

Or we can use:

\begin{Shaded}
\begin{Highlighting}[]
\AttributeTok{@Component}
\AttributeTok{@prototypescope}
\KeywordTok{public} \KeywordTok{class}\NormalTok{ MyPrototypeBean }\OperatorTok{\{} \KeywordTok{...} \OperatorTok{\}}
\end{Highlighting}
\end{Shaded}

\section{Autowiring}\label{autowiring}

Autowiring is the process where Spring automatically resolves and
injects dependent beans into your bean without need for explicit
configuration.

\textbf{Modes of Autowiring:}

\begin{enumerate}
\def\labelenumi{\arabic{enumi}.}
\tightlist
\item
  \textbf{no}: (Default) No automatic wiring. You must wire explicitly.
\item
  \textbf{byName}: Spring looks for a bean with the same \textbf{name}
  as the property.
\item
  \textbf{byType}: Spring looks for a bean with the same \textbf{class
  type}. (Fails if multiple beans of same type exist).
\item
  \textbf{constructor}: Similar to byType, but applies to constructor
  arguments.
\end{enumerate}

\textbf{Handling Ambiguity (\texttt{@Qualifier} \& \texttt{@Primary}):}
If multiple beans of the same type exist (e.g., \texttt{PayPalService}
and \texttt{StripeService} both implementing \texttt{PaymentService}),
Spring throws \texttt{NoUniqueBeanDefinitionException}.

\begin{enumerate}
\def\labelenumi{\arabic{enumi}.}
\item
  \textbf{@Qualifier}: Specify the bean name to inject.

\begin{Shaded}
\begin{Highlighting}[]
\AttributeTok{@Autowired}
\AttributeTok{@Qualifier}\OperatorTok{(}\StringTok{"payPalService"}\OperatorTok{)}
\KeywordTok{private}\NormalTok{ PaymentService paymentService}\OperatorTok{;}
\end{Highlighting}
\end{Shaded}
\item
  \textbf{@Primary}: Mark one bean as the default.

\begin{Shaded}
\begin{Highlighting}[]
\AttributeTok{@Component}
\AttributeTok{@Primary}
\KeywordTok{public} \KeywordTok{class}\NormalTok{ PayPalService }\KeywordTok{implements}\NormalTok{ PaymentService }\OperatorTok{\{} \KeywordTok{...} \OperatorTok{\}}
\end{Highlighting}
\end{Shaded}
\end{enumerate}

\section{Bean Life Cycle}\label{bean-life-cycle}

The lifecycle of a Spring Bean is managed by the container, from
creation to destruction.

\begin{figure}
\centering
\includegraphics[width=\linewidth,height=12cm,keepaspectratio,alt={~}]{images/mermaid_48e14d43d712729bf3648ea55829c262.png}
\caption{~}
\end{figure}

\begin{enumerate}
\def\labelenumi{\arabic{enumi}.}
\tightlist
\item
  \textbf{Instantiation}: The container creates the bean instance (calls
  constructor).
\item
  \textbf{Populate Properties}: Dependencies are injected.
\item
  \textbf{Aware Interfaces}: If bean implements \texttt{BeanNameAware},
  etc., setters are called.
\item
  \textbf{Pre-Initialization}:
  \texttt{BeanPostProcessor.postProcessBeforeInitialization()} is
  called.
\item
  \textbf{Initialization}: Custom init method (\texttt{@PostConstruct}
  or \texttt{init-method} in XML) is executed.
\item
  \textbf{Post-Initialization}:
  \texttt{BeanPostProcessor.postProcessAfterInitialization()} is called.
  (AOP proxies are often created here).
\item
  \textbf{Ready}: Bean is ready for use.
\item
  \textbf{Destruction}: When container shuts down, \texttt{@PreDestroy}
  or \texttt{destroy-method} is called.
\end{enumerate}

\subsubsection{Implementing Life Cycle
Callbacks}\label{implementing-life-cycle-callbacks}

\textbf{Using Interfaces:}

\begin{Shaded}
\begin{Highlighting}[]
\AttributeTok{@Component}
\KeywordTok{public} \KeywordTok{class}\NormalTok{ MyBean }\KeywordTok{implements}\NormalTok{ InitializingBean}\OperatorTok{,}\NormalTok{ DisposableBean }\OperatorTok{\{}
    \AttributeTok{@Override}
    \KeywordTok{public} \DataTypeTok{void} \FunctionTok{afterPropertiesSet}\OperatorTok{()} \OperatorTok{\{}
        \CommentTok{// Initialization logic}
    \OperatorTok{\}}

    \AttributeTok{@Override}
    \KeywordTok{public} \DataTypeTok{void} \FunctionTok{destroy}\OperatorTok{()} \OperatorTok{\{}
        \CommentTok{// Cleanup logic}
    \OperatorTok{\}}
\OperatorTok{\}}
\end{Highlighting}
\end{Shaded}

\textbf{Using Annotations (Recommended):}

\begin{Shaded}
\begin{Highlighting}[]
\AttributeTok{@Component}
\KeywordTok{public} \KeywordTok{class}\NormalTok{ MyBean }\OperatorTok{\{}
    \AttributeTok{@PostConstruct}
    \KeywordTok{public} \DataTypeTok{void} \FunctionTok{init}\OperatorTok{()} \OperatorTok{\{}
        \CommentTok{// Initialization logic}
    \OperatorTok{\}}

    \AttributeTok{@PreDestroy}
    \KeywordTok{public} \DataTypeTok{void} \FunctionTok{cleanup}\OperatorTok{()} \OperatorTok{\{}
        \CommentTok{// Cleanup logic}
    \OperatorTok{\}}
\OperatorTok{\}}
\end{Highlighting}
\end{Shaded}

\newpage

\chapter{Spring Boot}\label{spring-boot}

\etocsettocstyle{\textbf{Chapter Contents}\par\rule{\linewidth}{0.5pt}}{\par\rule{\linewidth}{0.5pt}}
\localtableofcontents

\noindent

Spring Boot is an extension of the Spring Framework that simplifies the
setup and development of new Spring applications.

\textbf{Key Features:}

\begin{itemize}
\tightlist
\item
  \textbf{Auto-Configuration}: Automatically configures Spring based on
  jar dependencies.
\item
  \textbf{Standalone}: Embeds Tomcat, Jetty, or Undertow directly (no
  need to deploy WAR files).
\item
  \textbf{Starter Dependencies}: Simplified build configuration (e.g.,
  \texttt{spring-boot-starter-web}).
\item
  \textbf{Production-ready}: Metrics, health checks, externalized
  configuration.
\end{itemize}

\section{Difference between Spring and Spring
Boot}\label{difference-between-spring-and-spring-boot}

{\def\LTcaptype{none} % do not increment counter
\begin{longtable}[]{@{}
  >{\raggedright\arraybackslash}p{(\linewidth - 4\tabcolsep) * \real{0.1491}}
  >{\raggedright\arraybackslash}p{(\linewidth - 4\tabcolsep) * \real{0.3684}}
  >{\raggedright\arraybackslash}p{(\linewidth - 4\tabcolsep) * \real{0.4825}}@{}}
\toprule\noalign{}
\begin{minipage}[b]{\linewidth}\raggedright
Feature
\end{minipage} & \begin{minipage}[b]{\linewidth}\raggedright
Spring Framework
\end{minipage} & \begin{minipage}[b]{\linewidth}\raggedright
Spring Boot
\end{minipage} \\
\midrule\noalign{}
\endhead
\bottomrule\noalign{}
\endlastfoot
\textbf{Goal} & Provides infrastructure for building apps. & Simplifies
booting and development. \\
\textbf{Configuration} & Manual (XML or Java-based). &
\textbf{Auto-configuration} (Convention over Configuration). \\
\textbf{Server} & External server required (e.g., Tomcat). &
\textbf{Embedded server} (Tomcat/Jetty) included. \\
\textbf{Dependency} & Dependencies managed manually. & \textbf{Starters}
simplify dependency management. \\
\textbf{Boilerplate} & Significant boilerplate code. & Reduces
boilerplate code drastically. \\
\end{longtable}
}

\section{Build Systems}\label{build-systems}

Spring Boot projects typically use \textbf{Maven} or \textbf{Gradle}.

\begin{itemize}
\tightlist
\item
  \texttt{pom.xml} (Maven)
\item
  \texttt{build.gradle} (Gradle)
\end{itemize}

\section{Code Structure}\label{code-structure}

\begin{Shaded}
\begin{Highlighting}[]
\NormalTok{com}
 \OperatorTok{+{-}}\NormalTok{ example}
     \OperatorTok{+{-}}\NormalTok{ myapp}
         \OperatorTok{+{-}}\NormalTok{ Application}\OperatorTok{.}\FunctionTok{java} \OperatorTok{(}\NormalTok{Main }\BuiltInTok{Class}\OperatorTok{)}
         \OperatorTok{|}
         \OperatorTok{+{-}} \FunctionTok{domain} \OperatorTok{(}\NormalTok{Entities}\OperatorTok{)}
         \OperatorTok{+{-}} \FunctionTok{repository} \OperatorTok{(}\NormalTok{DAO}\OperatorTok{)}
         \OperatorTok{+{-}} \FunctionTok{service} \OperatorTok{(}\NormalTok{Business Logic}\OperatorTok{)}
         \OperatorTok{+{-}} \FunctionTok{web} \OperatorTok{(}\NormalTok{Controllers}\OperatorTok{)}
\end{Highlighting}
\end{Shaded}

\section{Spring Boot Runners}\label{spring-boot-runners}

Interfaces used to run code \emph{after} the application starts.

\begin{enumerate}
\def\labelenumi{\arabic{enumi}.}
\tightlist
\item
  \textbf{CommandLineRunner}: \texttt{run(String...\ args)}
\item
  \textbf{ApplicationRunner}: \texttt{run(ApplicationArguments\ args)}
\end{enumerate}

\section{Logging}\label{logging}

Spring Boot uses Commons Logging for all internal logging but leaves the
underlying log implementation open. Default is \textbf{Logback}.

\begin{Shaded}
\begin{Highlighting}[]
\BuiltInTok{Logger}\NormalTok{ logger }\OperatorTok{=}\NormalTok{ LoggerFactory}\OperatorTok{.}\FunctionTok{getLogger}\OperatorTok{(}\NormalTok{MyClass}\OperatorTok{.}\FunctionTok{class}\OperatorTok{);}
\NormalTok{logger}\OperatorTok{.}\FunctionTok{info}\OperatorTok{(}\StringTok{"This is an info message"}\OperatorTok{);}
\end{Highlighting}
\end{Shaded}

\section{RESTful Web Services}\label{restful-web-services}

\textbf{REST} (Representational State Transfer) is an architectural
style for web services.

\subsection{Annotations}\label{annotations-1}

\begin{itemize}
\tightlist
\item
  \texttt{@RestController}: Combines \texttt{@Controller} and
  \texttt{@ResponseBody}.
\item
  \texttt{@RequestMapping}: Maps HTTP requests to handler methods.
\item
  \texttt{@RequestBody}: Maps the HTTP request body to a Java object.
\item
  \texttt{@PathVariable}: Extracts values from the URI path.
\item
  \texttt{@RequestParam}: Extracts query parameters.
\end{itemize}

\subsection{HTTP Methods}\label{http-methods}

{\def\LTcaptype{none} % do not increment counter
\begin{longtable}[]{@{}
  >{\raggedright\arraybackslash}p{(\linewidth - 4\tabcolsep) * \real{0.1471}}
  >{\raggedright\arraybackslash}p{(\linewidth - 4\tabcolsep) * \real{0.2353}}
  >{\raggedright\arraybackslash}p{(\linewidth - 4\tabcolsep) * \real{0.6176}}@{}}
\toprule\noalign{}
\begin{minipage}[b]{\linewidth}\raggedright
Method
\end{minipage} & \begin{minipage}[b]{\linewidth}\raggedright
Annotation
\end{minipage} & \begin{minipage}[b]{\linewidth}\raggedright
Purpose
\end{minipage} \\
\midrule\noalign{}
\endhead
\bottomrule\noalign{}
\endlastfoot
\textbf{GET} & \texttt{@GetMapping} & Retrieve a resource. \\
\textbf{POST} & \texttt{@PostMapping} & Create a new resource. \\
\textbf{PUT} & \texttt{@PutMapping} & Update an existing resource (full
update). \\
\textbf{DELETE} & \texttt{@DeleteMapping} & Delete a resource. \\
\end{longtable}
}

\subsection{Example Controller}\label{example-controller}

\begin{Shaded}
\begin{Highlighting}[]
\AttributeTok{@RestController}
\AttributeTok{@RequestMapping}\OperatorTok{(}\StringTok{"/api/users"}\OperatorTok{)}
\KeywordTok{public} \KeywordTok{class}\NormalTok{ UserController }\OperatorTok{\{}

    \CommentTok{// GET /api/users/1}
    \AttributeTok{@GetMapping}\OperatorTok{(}\StringTok{"/\{id\}"}\OperatorTok{)}
    \KeywordTok{public}\NormalTok{ User }\FunctionTok{getUser}\OperatorTok{(}\AttributeTok{@PathVariable} \DataTypeTok{int}\NormalTok{ id}\OperatorTok{)} \OperatorTok{\{}
        \ControlFlowTok{return} \KeywordTok{new} \FunctionTok{User}\OperatorTok{(}\NormalTok{id}\OperatorTok{,} \StringTok{"John"}\OperatorTok{);}
    \OperatorTok{\}}

    \CommentTok{// POST /api/users}
    \AttributeTok{@PostMapping}
    \KeywordTok{public}\NormalTok{ User }\FunctionTok{createUser}\OperatorTok{(}\AttributeTok{@RequestBody}\NormalTok{ User user}\OperatorTok{)} \OperatorTok{\{}
        \CommentTok{// save user...}
        \ControlFlowTok{return}\NormalTok{ user}\OperatorTok{;}
    \OperatorTok{\}}

    \CommentTok{// GET /api/users?role=admin}
    \AttributeTok{@GetMapping}
    \KeywordTok{public} \BuiltInTok{List}\OperatorTok{\textless{}}\NormalTok{User}\OperatorTok{\textgreater{}} \FunctionTok{getUsers}\OperatorTok{(}\AttributeTok{@RequestParam}\OperatorTok{(}\NormalTok{defaultValue }\OperatorTok{=} \StringTok{"user"}\OperatorTok{)} \BuiltInTok{String}\NormalTok{ role}\OperatorTok{)} \OperatorTok{\{}
        \CommentTok{// return users by role...}
        \ControlFlowTok{return} \KeywordTok{new} \BuiltInTok{ArrayList}\OperatorTok{\textless{}\textgreater{}();}
    \OperatorTok{\}}
\OperatorTok{\}}
\end{Highlighting}
\end{Shaded}

\newpage

\backmatter
\end{document}
