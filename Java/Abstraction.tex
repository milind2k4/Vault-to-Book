% Options for packages loaded elsewhere
\PassOptionsToPackage{unicode}{hyperref}
\PassOptionsToPackage{hyphens}{url}
\documentclass[
]{article}
\usepackage{xcolor}
\usepackage{amsmath,amssymb}
\setcounter{secnumdepth}{-\maxdimen} % remove section numbering
\usepackage{iftex}
\ifPDFTeX
  \usepackage[T1]{fontenc}
  \usepackage[utf8]{inputenc}
  \usepackage{textcomp} % provide euro and other symbols
\else % if luatex or xetex
  \usepackage{unicode-math} % this also loads fontspec
  \defaultfontfeatures{Scale=MatchLowercase}
  \defaultfontfeatures[\rmfamily]{Ligatures=TeX,Scale=1}
\fi
\usepackage{lmodern}
\ifPDFTeX\else
  % xetex/luatex font selection
\fi
% Use upquote if available, for straight quotes in verbatim environments
\IfFileExists{upquote.sty}{\usepackage{upquote}}{}
\IfFileExists{microtype.sty}{% use microtype if available
  \usepackage[]{microtype}
  \UseMicrotypeSet[protrusion]{basicmath} % disable protrusion for tt fonts
}{}
\makeatletter
\@ifundefined{KOMAClassName}{% if non-KOMA class
  \IfFileExists{parskip.sty}{%
    \usepackage{parskip}
  }{% else
    \setlength{\parindent}{0pt}
    \setlength{\parskip}{6pt plus 2pt minus 1pt}}
}{% if KOMA class
  \KOMAoptions{parskip=half}}
\makeatother
\usepackage{color}
\usepackage{fancyvrb}
\newcommand{\VerbBar}{|}
\newcommand{\VERB}{\Verb[commandchars=\\\{\}]}
\DefineVerbatimEnvironment{Highlighting}{Verbatim}{commandchars=\\\{\}}
% Add ',fontsize=\small' for more characters per line
\newenvironment{Shaded}{}{}
\newcommand{\AlertTok}[1]{\textcolor[rgb]{1.00,0.00,0.00}{\textbf{#1}}}
\newcommand{\AnnotationTok}[1]{\textcolor[rgb]{0.38,0.63,0.69}{\textbf{\textit{#1}}}}
\newcommand{\AttributeTok}[1]{\textcolor[rgb]{0.49,0.56,0.16}{#1}}
\newcommand{\BaseNTok}[1]{\textcolor[rgb]{0.25,0.63,0.44}{#1}}
\newcommand{\BuiltInTok}[1]{\textcolor[rgb]{0.00,0.50,0.00}{#1}}
\newcommand{\CharTok}[1]{\textcolor[rgb]{0.25,0.44,0.63}{#1}}
\newcommand{\CommentTok}[1]{\textcolor[rgb]{0.38,0.63,0.69}{\textit{#1}}}
\newcommand{\CommentVarTok}[1]{\textcolor[rgb]{0.38,0.63,0.69}{\textbf{\textit{#1}}}}
\newcommand{\ConstantTok}[1]{\textcolor[rgb]{0.53,0.00,0.00}{#1}}
\newcommand{\ControlFlowTok}[1]{\textcolor[rgb]{0.00,0.44,0.13}{\textbf{#1}}}
\newcommand{\DataTypeTok}[1]{\textcolor[rgb]{0.56,0.13,0.00}{#1}}
\newcommand{\DecValTok}[1]{\textcolor[rgb]{0.25,0.63,0.44}{#1}}
\newcommand{\DocumentationTok}[1]{\textcolor[rgb]{0.73,0.13,0.13}{\textit{#1}}}
\newcommand{\ErrorTok}[1]{\textcolor[rgb]{1.00,0.00,0.00}{\textbf{#1}}}
\newcommand{\ExtensionTok}[1]{#1}
\newcommand{\FloatTok}[1]{\textcolor[rgb]{0.25,0.63,0.44}{#1}}
\newcommand{\FunctionTok}[1]{\textcolor[rgb]{0.02,0.16,0.49}{#1}}
\newcommand{\ImportTok}[1]{\textcolor[rgb]{0.00,0.50,0.00}{\textbf{#1}}}
\newcommand{\InformationTok}[1]{\textcolor[rgb]{0.38,0.63,0.69}{\textbf{\textit{#1}}}}
\newcommand{\KeywordTok}[1]{\textcolor[rgb]{0.00,0.44,0.13}{\textbf{#1}}}
\newcommand{\NormalTok}[1]{#1}
\newcommand{\OperatorTok}[1]{\textcolor[rgb]{0.40,0.40,0.40}{#1}}
\newcommand{\OtherTok}[1]{\textcolor[rgb]{0.00,0.44,0.13}{#1}}
\newcommand{\PreprocessorTok}[1]{\textcolor[rgb]{0.74,0.48,0.00}{#1}}
\newcommand{\RegionMarkerTok}[1]{#1}
\newcommand{\SpecialCharTok}[1]{\textcolor[rgb]{0.25,0.44,0.63}{#1}}
\newcommand{\SpecialStringTok}[1]{\textcolor[rgb]{0.73,0.40,0.53}{#1}}
\newcommand{\StringTok}[1]{\textcolor[rgb]{0.25,0.44,0.63}{#1}}
\newcommand{\VariableTok}[1]{\textcolor[rgb]{0.10,0.09,0.49}{#1}}
\newcommand{\VerbatimStringTok}[1]{\textcolor[rgb]{0.25,0.44,0.63}{#1}}
\newcommand{\WarningTok}[1]{\textcolor[rgb]{0.38,0.63,0.69}{\textbf{\textit{#1}}}}
\setlength{\emergencystretch}{3em} % prevent overfull lines
\providecommand{\tightlist}{%
  \setlength{\itemsep}{0pt}\setlength{\parskip}{0pt}}
\usepackage{bookmark}
\IfFileExists{xurl.sty}{\usepackage{xurl}}{} % add URL line breaks if available
\urlstyle{same}
\hypersetup{
  hidelinks,
  pdfcreator={LaTeX via pandoc}}

\author{}
\date{}

\begin{document}

Links: {[}{[}02 Polymorphism{]}{]}, {[}{[}01 Inheritance{]}{]} \_\_\_ \#
Abstraction It is a concept which is used for \textbf{hiding
implementation complexity} and showing only the essential features or
functionality to the user.

(Note: \textbf{Encapsulation} is used for \emph{data hiding} (protecting
variables). \textbf{Abstraction} is used for \emph{implementation
hiding} (hiding \emph{how} a method works)).

Abstraction focuses on \emph{what} an object does, not \emph{how} it
does it.

Abstraction can be achieved using an abstract class or by an interface.

\subsubsection{Abstract Class}\label{abstract-class}

We use \texttt{abstract} keyword to declare a class as abstract. An
abstract class provides a template for other classes, forcing them to
implement certain methods.

You \textbf{cannot} create an object (instantiate) of an abstract class.

\begin{itemize}
\tightlist
\item
  We can declare abstract as well as non-abstract (concrete) methods.\\
\item
  It can have constructors, static methods, and \texttt{final} methods.
\item
  It works together with {[}{[}02 Polymorphism\#Run-Time Polymorphism
  (Method Overriding){]}{]}.
\end{itemize}

Abstract methods are declared with the \texttt{abstract} keyword. These
methods don't have a definition or body in the class in which they are
declared. They \textbf{must} be overridden by a subclass to provide the
definition.

\begin{Shaded}
\begin{Highlighting}[]
\CommentTok{// We cannot create an object of \textquotesingle{}Shape\textquotesingle{}}
\KeywordTok{abstract} \KeywordTok{class} \BuiltInTok{Shape} \OperatorTok{\{}
    \DataTypeTok{int}\NormalTok{ color}\OperatorTok{;}
    
    \CommentTok{// Concrete (non{-}abstract) method with a body}
    \DataTypeTok{void} \FunctionTok{setColor}\OperatorTok{(}\DataTypeTok{int}\NormalTok{ c}\OperatorTok{)} \OperatorTok{\{}
        \KeywordTok{this}\OperatorTok{.}\FunctionTok{color} \OperatorTok{=}\NormalTok{ c}\OperatorTok{;}
    \OperatorTok{\}}

    \CommentTok{// Abstract method {-} no body}
    \CommentTok{// Forces subclasses to provide their own version}
    \KeywordTok{abstract} \DataTypeTok{void} \FunctionTok{draw}\OperatorTok{();} 
\OperatorTok{\}}

\KeywordTok{class}\NormalTok{ Circle }\KeywordTok{extends} \BuiltInTok{Shape} \OperatorTok{\{}
    \CommentTok{// We MUST implement the abstract \textquotesingle{}draw\textquotesingle{} method}
    \AttributeTok{@Override}
    \DataTypeTok{void} \FunctionTok{draw}\OperatorTok{()} \OperatorTok{\{}
        \BuiltInTok{System}\OperatorTok{.}\FunctionTok{out}\OperatorTok{.}\FunctionTok{println}\OperatorTok{(}\StringTok{"Drawing a circle"}\OperatorTok{);}
    \OperatorTok{\}}
\OperatorTok{\}}
\end{Highlighting}
\end{Shaded}

\subsubsection{Interface}\label{interface}

An interface is a blueprint of a class. It specifies what a class must
do, but not how.

It is used to achieve 100\% abstraction (before Java 8).

\begin{itemize}
\tightlist
\item
  We use the \texttt{interface} keyword.
\item
  All methods in an interface are \texttt{public} and \texttt{abstract}
  by default.
\item
  All variables (fields) are \texttt{public}, \texttt{static}, and
  \texttt{final} (constants) by default.
\item
  A class uses the \texttt{implements} keyword to use an interface.
\item
  A class can implement \textbf{multiple} interfaces. This is how Java
  achieves multiple inheritance.
\end{itemize}

\begin{Shaded}
\begin{Highlighting}[]
\KeywordTok{interface}\NormalTok{ Drawable }\OperatorTok{\{}
    \DataTypeTok{void} \FunctionTok{draw}\OperatorTok{();} \CommentTok{// This is public and abstract by default}
\OperatorTok{\}}

\KeywordTok{interface}\NormalTok{ Loggable }\OperatorTok{\{}
    \DataTypeTok{void} \FunctionTok{log}\OperatorTok{(}\BuiltInTok{String}\NormalTok{ message}\OperatorTok{);}
\OperatorTok{\}}

\CommentTok{// A class can implement multiple interfaces}
\KeywordTok{class}\NormalTok{ Circle }\KeywordTok{implements}\NormalTok{ Drawable}\OperatorTok{,}\NormalTok{ Loggable }\OperatorTok{\{}
    \AttributeTok{@Override}
    \KeywordTok{public} \DataTypeTok{void} \FunctionTok{draw}\OperatorTok{()} \OperatorTok{\{}
        \BuiltInTok{System}\OperatorTok{.}\FunctionTok{out}\OperatorTok{.}\FunctionTok{println}\OperatorTok{(}\StringTok{"Drawing a circle"}\OperatorTok{);}
    \OperatorTok{\}}
    
    \AttributeTok{@Override}
    \KeywordTok{public} \DataTypeTok{void} \FunctionTok{log}\OperatorTok{(}\BuiltInTok{String}\NormalTok{ message}\OperatorTok{)} \OperatorTok{\{}
        \BuiltInTok{System}\OperatorTok{.}\FunctionTok{out}\OperatorTok{.}\FunctionTok{println}\OperatorTok{(}\StringTok{"LOG: "} \OperatorTok{+}\NormalTok{ message}\OperatorTok{);}
    \OperatorTok{\}}
\OperatorTok{\}}
\end{Highlighting}
\end{Shaded}

\paragraph{\texorpdfstring{\texttt{default} and \texttt{static}
methods}{default and static methods}}\label{default-and-static-methods}

Java 8 allowed interfaces to have methods with implementation:

\begin{itemize}
\tightlist
\item
  \textbf{\texttt{default} methods:} A class can override them, but
  doesn't have to. It allows adding new functionality to an interface
  without breaking existing classes.
\item
  \textbf{\texttt{static} methods:} Utility methods that are part of the
  interface, not a specific object.
\end{itemize}

\subparagraph{Default Methods}\label{default-methods}

Allows adding new methods to interfaces with a default implementation,
without breaking existing classes that implement the interface.

We use the default keyword.

\begin{Shaded}
\begin{Highlighting}[]
\KeywordTok{interface}\NormalTok{ MyInterface }\OperatorTok{\{}
    \DataTypeTok{void} \FunctionTok{existingMethod}\OperatorTok{();}
    
    \KeywordTok{default} \DataTypeTok{void} \FunctionTok{newDefaultMethod}\OperatorTok{()} \OperatorTok{\{}
        \BuiltInTok{System}\OperatorTok{.}\FunctionTok{out}\OperatorTok{.}\FunctionTok{println}\OperatorTok{(}\StringTok{"This is a default implementation."}\OperatorTok{);}
    \OperatorTok{\}}
\OperatorTok{\}}

\KeywordTok{class}\NormalTok{ MyClass }\KeywordTok{implements}\NormalTok{ MyInterface }\OperatorTok{\{}
    \KeywordTok{public} \DataTypeTok{void} \FunctionTok{existingMethod}\OperatorTok{()} \OperatorTok{\{}
        \CommentTok{// ...}
    \OperatorTok{\}}
    \CommentTok{// No need to implement newDefaultMethod(), it\textquotesingle{}s optional.}
\OperatorTok{\}}
\end{Highlighting}
\end{Shaded}

\subparagraph{Static Method}\label{static-method}

Interfaces can now have static methods. These are utility methods that
are part of the interface, not the implementing class.

They are called on the interface itself, not on an instance.

\begin{Shaded}
\begin{Highlighting}[]
\KeywordTok{interface}\NormalTok{ MyInterface }\OperatorTok{\{}
    \DataTypeTok{static} \DataTypeTok{void} \FunctionTok{utilityMethod}\OperatorTok{()} \OperatorTok{\{}
        \BuiltInTok{System}\OperatorTok{.}\FunctionTok{out}\OperatorTok{.}\FunctionTok{println}\OperatorTok{(}\StringTok{"This is a static utility method."}\OperatorTok{);}
    \OperatorTok{\}}
\OperatorTok{\}}

\CommentTok{// How to call it:}
\NormalTok{MyInterface}\OperatorTok{.}\FunctionTok{utilityMethod}\OperatorTok{();}
\end{Highlighting}
\end{Shaded}

\subsubsection{Abstract Class
vs.~Interface}\label{abstract-class-vs.-interface}

\begin{itemize}
\item
  \textbf{Methods:} An abstract class can have both abstract and
  concrete methods. An interface (pre-Java 8) can only have abstract
  methods.
\item
  \textbf{Variables:} An abstract class can have any type of variable
  (instance, static, final). An interface can only have
  \texttt{public\ static\ final} constants.
\item
  \textbf{Inheritance:} A class can \texttt{extends} only \textbf{one}
  abstract class. A class can \texttt{implements} \textbf{many}
  interfaces.
\item
  \textbf{Constructor:} An abstract class can have a constructor (which
  is called by the subclass constructor). An interface \textbf{cannot}
  have a constructor.
\item
  \textbf{Purpose:}

  \begin{itemize}
  \tightlist
  \item
    Use an \textbf{abstract class} for an ``IS-A'' relationship, when
    you want to provide common, shared code for all subclasses (e.g.,
    \texttt{Dog} IS-A \texttt{Animal}, and all animals
    \texttt{breathe()}).
  \item
    Use an \textbf{interface} for a ``CAN-DO'' relationship, when you
    want to define a capability or contract that unrelated classes can
    share (e.g., \texttt{Circle} CAN-DO \texttt{Drawable}, \texttt{Car}
    CAN-DO \texttt{Drawable}).
  \end{itemize}
\end{itemize}

\end{document}
