%%
% Copyright (c) 2017 - 2025, Pascal Wagler;
% Copyright (c) 2014 - 2025, John MacFarlane
%
% All rights reserved.
%
% Redistribution and use in source and binary forms, with or without
% modification, are permitted provided that the following conditions
% are met:
%
% - Redistributions of source code must retain the above copyright
% notice, this list of conditions and the following disclaimer.
%
% - Redistributions in binary form must reproduce the above copyright
% notice, this list of conditions and the following disclaimer in the
% documentation and/or other materials provided with the distribution.
%
% - Neither the name of John MacFarlane nor the names of other
% contributors may be used to endorse or promote products derived
% from this software without specific prior written permission.
%
% THIS SOFTWARE IS PROVIDED BY THE COPYRIGHT HOLDERS AND CONTRIBUTORS
% "AS IS" AND ANY EXPRESS OR IMPLIED WARRANTIES, INCLUDING, BUT NOT
% LIMITED TO, THE IMPLIED WARRANTIES OF MERCHANTABILITY AND FITNESS
% FOR A PARTICULAR PURPOSE ARE DISCLAIMED. IN NO EVENT SHALL THE
% COPYRIGHT OWNER OR CONTRIBUTORS BE LIABLE FOR ANY DIRECT, INDIRECT,
% INCIDENTAL, SPECIAL, EXEMPLARY, OR CONSEQUENTIAL DAMAGES (INCLUDING,
% BUT NOT LIMITED TO, PROCUREMENT OF SUBSTITUTE GOODS OR SERVICES;
% LOSS OF USE, DATA, OR PROFITS; OR BUSINESS INTERRUPTION) HOWEVER
% CAUSED AND ON ANY THEORY OF LIABILITY, WHETHER IN CONTRACT, STRICT
% LIABILITY, OR TORT (INCLUDING NEGLIGENCE OR OTHERWISE) ARISING IN
% ANY WAY OUT OF THE USE OF THIS SOFTWARE, EVEN IF ADVISED OF THE
% POSSIBILITY OF SUCH DAMAGE.
%%

%%
% This is the Eisvogel pandoc LaTeX template.
%
% For usage information and examples visit the official GitHub page:
% https://github.com/Wandmalfarbe/pandoc-latex-template
%%
% Options for packages loaded elsewhere
\PassOptionsToPackage{unicode}{hyperref}
\PassOptionsToPackage{hyphens}{url}
\PassOptionsToPackage{dvipsnames,svgnames,x11names,table}{xcolor}
\documentclass[
  paper=a4,
  openany,
  oneside  ,captions=tableheading
]{scrbook}
\usepackage{xcolor}
\usepackage[top=1.5cm, bottom=2.5cm, left=3cm, right=3cm, includeheadfoot, heightrounded]{geometry}
\usepackage{amsmath,amssymb}

% add backlinks to footnote references, cf. https://tex.stackexchange.com/questions/302266/make-footnote-clickable-both-ways
\usepackage{footnotebackref}
\setcounter{secnumdepth}{5}
\usepackage{iftex}
\ifPDFTeX
  \usepackage[T1]{fontenc}
  \usepackage[utf8]{inputenc}
  \usepackage{textcomp} % provide euro and other symbols
\else % if luatex or xetex
  \usepackage{unicode-math} % this also loads fontspec
  \defaultfontfeatures{Scale=MatchLowercase}
  \defaultfontfeatures[\rmfamily]{Ligatures=TeX,Scale=1}
\fi
\usepackage{lmodern}
\ifPDFTeX\else
  % xetex/luatex font selection
  \setmainfont[]{LMRoman10-Regular}
  \setsansfont[]{Arial}
  \setmonofont[]{LMMono10-Regular}
\fi
% Use upquote if available, for straight quotes in verbatim environments
\IfFileExists{upquote.sty}{\usepackage{upquote}}{}
\IfFileExists{microtype.sty}{% use microtype if available
  \usepackage[]{microtype}
  \UseMicrotypeSet[protrusion]{basicmath} % disable protrusion for tt fonts
}{}

% Use setspace anyway because we change the default line spacing.
% The spacing is changed early to affect the titlepage and the TOC.
\usepackage{setspace}
\setstretch{1.2}
\makeatletter
\@ifundefined{KOMAClassName}{% if non-KOMA class
  \IfFileExists{parskip.sty}{%
    \usepackage{parskip}
  }{% else
    \setlength{\parindent}{0pt}
    \setlength{\parskip}{6pt plus 2pt minus 1pt}}
}{% if KOMA class
  \KOMAoptions{parskip=half}}
\makeatother
\ifLuaTeX
  \usepackage{luacolor}
  \usepackage[soul]{lua-ul}
\else
  \usepackage{soul}
\fi
\setlength{\emergencystretch}{3em} % prevent overfull lines
\providecommand{\tightlist}{%
  \setlength{\itemsep}{0pt}\setlength{\parskip}{0pt}}

\usepackage{caption}
\captionsetup[figure]{labelsep=none, justification=centering}
\usepackage{etoc} % Replaces minitoc
\usepackage[version=4]{mhchem}
\usepackage{amsmath}
\usepackage{amssymb}
\usepackage{mathtools}
\usepackage{gensymb}
\usepackage[export]{adjustbox} % For max width=... in includegraphics
\usepackage{float} % Required for [H] figure placement
\usepackage{cancel}

% --- Global Image Sizing from Config ---
\makeatletter
\let\oldincludegraphics\includegraphics
\renewcommand{\includegraphics}[2][]{
  \oldincludegraphics[max height=0.6\linewidth,keepaspectratio,#1]{#2}
}
\makeatother
% \mtcselectlanguage{english} - Removed
\definecolor{mylinkcolor}{HTML}{07455c}
\definecolor{myurlcolor}{HTML}{07455c}
% --- Chapter Title Styling (KOMA-Script) ---
\renewcommand*\chapterformat{\thechapter.\enskip}
\addtokomafont{chapter}{\centering}
\RedeclareSectionCommand[beforeskip=0pt,afterskip=20pt]{chapter}
% --- Table Styling ---
\rowcolors{2}{RoyalBlue!20}{white}
\renewcommand{\arraystretch}{1.2}
% --------------------------------------------------------
% FORCE CENTER FOR PANDOC BOUNDED IMAGES - (User Requested Fix)
% --------------------------------------------------------
\makeatletter
% Check if \pandocbounded is defined (to prevent crashes on old Pandoc versions)
\ifdef{\pandocbounded}{
  % Save the original command
  \let\oldpandocbounded\pandocbounded
  % Redefine it to include centering and a new line (\par)
  \renewcommand{\pandocbounded}[1]{
    {\centering\oldpandocbounded{#1}\par}%
  }
}{}
\makeatother
% -------------------------------------------
\usepackage[most]{tcolorbox}
\usepackage{fontawesome5}
\usepackage{xcolor}

% Define custom colors
\definecolor{notecolor}{RGB}{97, 175, 239}   % Blue
\definecolor{tipcolor}{RGB}{20, 196, 255} % Purple
\definecolor{warningcolor}{RGB}{229, 192, 123} % Orange
\definecolor{attentioncolor}{RGB}{224, 108, 117} % Red
\definecolor{analogycolor}{RGB}{152, 195, 121}    % Green

% Generic Pretty Box Style
% #1 = Color name
% #2 = Icon command
% #3 = Title text
\newtcolorbox{prettybox}[3]{
    enhanced,
    colback=#1!5!white,      % Very light background
    colframe=#1,             % Border color
    coltitle=#1!50!black,    % Title text color (darker version of base)
    title={#2\ \ \textbf{#3}},
    fonttitle=\bfseries\large,
    attach boxed title to top left={xshift=5mm, yshift=-3.5mm}, % Adjust for half-in/out
    boxed title style={
        colback=white,
        colframe=white,
        boxrule=0pt,
        top=0pt,
        bottom=0pt,
        left=2pt,
        right=2pt
    },
    top=1.5em, % Space for the floating title
    bottom=1em,
    left=1em,
    right=1em,
    arc=3pt,
    boxrule=1pt,
    drop fuzzy shadow,   
    parbox=false,        
    breakable            
}

% Define specific environments
% usage: \begin{noteblock}{Title} ... \end{noteblock}

\newenvironment{noteblock}[1]
  {\begin{prettybox}{notecolor}{\faInfoCircle}{#1}}
  {\end{prettybox}}

\newenvironment{tipblock}[1]
  {\begin{prettybox}{tipcolor}{\faLightbulb}{#1}}
  {\end{prettybox}}

\newenvironment{warningblock}[1]
  {\begin{prettybox}{warningcolor}{\faExclamationTriangle}{#1}}
  {\end{prettybox}}

\newenvironment{attentionblock}[1]
  {\begin{prettybox}{attentioncolor}{\faExclamationCircle}{#1}}
  {\end{prettybox}}

\newenvironment{cautionblock}[1]
  {\begin{prettybox}{attentioncolor}{\faRadiation}{#1}}
  {\end{prettybox}}

\newenvironment{importantblock}[1]
  {\begin{prettybox}{attentioncolor}{\faStar}{#1}}
  {\end{prettybox}}

% Analogy Block
\newenvironment{analogyblock}[1]
  {\begin{prettybox}{analogycolor}{\faShapes}{#1}}
  {\end{prettybox}}
\usepackage{bookmark}
\IfFileExists{xurl.sty}{\usepackage{xurl}}{} % add URL line breaks if available
\urlstyle{same}
\definecolor{default-linkcolor}{HTML}{A50000}
\definecolor{default-filecolor}{HTML}{A50000}
\definecolor{default-citecolor}{HTML}{4077C0}
\definecolor{default-urlcolor}{HTML}{4077C0}

\hypersetup{
  pdftitle={Alternating Current},
  pdfauthor={Anonymous},
  colorlinks=true,
  linkcolor={mylinkcolor},
  filecolor={default-filecolor},
  citecolor={default-citecolor},
  urlcolor={myurlcolor},
  breaklinks=true,
  pdfcreator={LaTeX via pandoc with the Eisvogel template}}

\title{Alternating Current}
\usepackage{etoolbox}
\makeatletter
\providecommand{\subtitle}[1]{% add subtitle to \maketitle
  \apptocmd{\@title}{\par {\large #1 \par}}{}{}
}
\makeatother
\subtitle{Personal Notes and References}
\author{Anonymous}
\date{}


%
% for the background color of the title page
%

%
% break urls
%
\PassOptionsToPackage{hyphens}{url}

%
% When using babel or polyglossia with biblatex, loading csquotes is recommended
% to ensure that quoted texts are typeset according to the rules of your main language.
%
\usepackage{csquotes}

%
% captions
%
\definecolor{caption-color}{HTML}{777777}
\usepackage[font={stretch=1.2}, textfont={color=caption-color}, position=top, skip=4mm, labelfont=bf, singlelinecheck=false, justification=raggedright]{caption}
\setcapindent{0em}

%
% blockquote
%
\definecolor{blockquote-border}{RGB}{221,221,221}
\definecolor{blockquote-text}{RGB}{119,119,119}
\usepackage{mdframed}
\newmdenv[rightline=false,bottomline=false,topline=false,linewidth=3pt,linecolor=blockquote-border,skipabove=\parskip]{customblockquote}
\renewenvironment{quote}{\begin{customblockquote}\list{}{\rightmargin=0em\leftmargin=0em}%
\item\relax\color{blockquote-text}\ignorespaces}{\unskip\unskip\endlist\end{customblockquote}}

%
% Source Sans Pro as the default font family
% Source Code Pro for monospace text
%
% 'default' option sets the default
% font family to Source Sans Pro, not \sfdefault.
%
% Note that the font has been officially renamed to `Source Sans 3`, and
% the version provided by the `sourcesanspro` package is slightly outdated.
% You can install the newer version locally and use it, for example, with
% `mainfont: "Source Sans 3"` in the YAML metadata (requires XeTeX or LuaTeX).
%
\ifnum 0\ifxetex 1\fi\ifluatex 1\fi=0 % if pdftex
    \usepackage[default]{sourcesanspro}
  \usepackage{sourcecodepro}
  \else % if not pdftex
    \fi

%
% heading color
%
\definecolor{heading-color}{RGB}{40,40,40}
% By default, the KOMA-Script classes will typeset sectioning headings in
% sans-serif. Use the normal body font for headings.
\addtokomafont{disposition}{\normalfont\color{heading-color}\bfseries}

%
% variables for title, author and date
%
\usepackage{titling}
\title{Alternating Current}
\author{Anonymous}
\date{}

%
% tables
%

%
% remove paragraph indentation
%
\setlength{\parindent}{0pt}
\setlength{\parskip}{6pt plus 2pt minus 1pt}
\setlength{\emergencystretch}{3em}  % prevent overfull lines

%
%
% Listings
%
%


%
% header and footer
%
\usepackage[headsepline,footsepline]{scrlayer-scrpage}

\newpairofpagestyles{eisvogel-header-footer}{
  \clearpairofpagestyles
  \ihead*{Alternating Current}
  \chead*{}
  \ohead*{}
  \ifoot*{Anonymous}
  \cfoot*{}
  \ofoot*{\thepage}
  \addtokomafont{pageheadfoot}{\upshape}
}
\pagestyle{eisvogel-header-footer}

\deftripstyle{ChapterStyle}{}{}{}{}{\pagemark}{}
\renewcommand*{\chapterpagestyle}{ChapterStyle}


%
% Define watermark
%

\begin{document}


\frontmatter
% don't generate the default title
% \maketitle



\begin{titlepage}
    \newgeometry{left=2.5cm,right=2.5cm,top=2cm,bottom=2cm}
    \vspace*{1cm}
    
    
    \vspace{3cm}
    
    \centering
    {\fontsize{50}{60}\selectfont \bfseries Alternating Current \par}
    \vspace{1cm}
    {\fontsize{20}{30}\selectfont Personal Notes and References \par}
    
    \vfill
    
    {\fontsize{18}{22}\selectfont Anonymous \par}
    \vspace{0.5cm}
    {\large January 12, 2026 \par}
    
    \vspace{3cm}
    \restoregeometry
\end{titlepage}

% Initialize MiniTOC (Removed, using etoc)

{
\setcounter{tocdepth}{3}
\tableofcontents
}
\mainmatter
\chapter{Alternating Current}\label{alternating-current}

\etocsettocstyle{\textbf{Chapter Contents}\par\rule{\linewidth}{0.5pt}}{\par\rule{\linewidth}{0.5pt}}
\localtableofcontents

\noindent {} When the direction of current is fixed, it is called
direct.

In Alternating current, the direction of current changes many times a
second. Thus the source also changes polarity.

\begin{figure}[H]
\centering
\includegraphics[max width=0.9\linewidth,keepaspectratio]{images/Pasted image 20240121173851.png}

\end{figure}

\subsubsection{Sinusoidal AC}\label{sinusoidal-ac}

\[i = i_{o}\sin \omega t\] \(i\) varies from \([-i_{o},i_{o}]\) and
\(i_{o}\) is called peak current or current amplitude. \(\omega\) is
angular frequency, \[\omega = 2 \pi f = \frac{2\pi}{T}\]

Anything inside of sin is called phase.

As \(i\) is varying sinusoidally \(V\) also varies sinusoidally.
\[V = V_{o} \sin \upomega t\]

\begin{figure}[H]
\centering
\includegraphics[max width=0.9\linewidth,keepaspectratio]{images/Pasted image 20240121174324.png}

\end{figure}

\paragraph{Average Current}\label{average-current}

\[i_{avg} = \frac{ \Delta q }{ \Delta t } = \frac{ \smallint i(t)dt }{ \Delta t }\]

\begin{figure}[H]
\centering
\includegraphics[max width=0.9\linewidth,keepaspectratio]{images/Pasted image 20240121175244.png}

\end{figure}

\paragraph{RMS value of AC}\label{rms-value-of-ac}

Root mean Square Value.

First square, then take mean, then take root.

It is the effective value of AC. If nothing is said, assume the value to
be rms.

\[i_{rms} = \sqrt{ \frac{ \int i^{2} \, dt  }{ \int dt  } }\]

Let the current be, \[i = i_{o}\sin (\omega t + \theta)\] Now, we find
RMS value of this current from t=0 to t=T. This value will be the same
for \(t = 0 \to 2T, 0 \to 3T\) etc.

\[i_{rms} = \sqrt{ \frac{ \int_{0}^{2\pi/\omega} i^{2} \sin ^{2}(\omega t + \theta) \, dt  }{ \int_{0}^{2\pi/\omega} dt } }\]

And finally, we get, \[i_{rms} = \frac{i_{o}}{\sqrt{ 2 }}\]

\begin{figure}[H]
\centering
\includegraphics[max width=0.9\linewidth,keepaspectratio]{images/Pasted image 20240121181835.png}

\end{figure}

The rms value is shown in the circuit. A hot wire ammeter measures rms
current.

\begin{figure}[H]
\centering
\includegraphics[max width=0.9\linewidth,keepaspectratio]{images/Pasted image 20230107180223.png}

\end{figure}

If current is given as the sum of 2 other sinusoidal out-of-phase
currents,

\[i = i_{1}\sin\upomega t + i_{2}\sin(\upomega t + \theta)\]

\[
\begin{split}
i_{rms} &= \sqrt{ \left( \frac{ i_{1} }{ \sqrt{ 2 } } \right)^{2} +  \left( \frac{ i_{2} }{ \sqrt{ 2 } } \right)^{2} + 2. \frac{ i_{1} }{ \sqrt{ 2 } }. \frac{ i_{2} }{ \sqrt{ 2 } } \cos \theta } \\
\\
i_{rms} &= \sqrt{ i_{1rms}^{2} + i_{2rms}^{2} + 2i_{1rms}i_{2rms}\cos \theta }
\end{split}
\] Which is just like addition of vectors.

\begin{figure}[H]
\centering
\includegraphics[max width=0.9\linewidth,keepaspectratio]{images/Pasted image 20240121182330.png}

\end{figure}

\begin{figure}[H]
\centering
\includegraphics[max width=0.9\linewidth,keepaspectratio]{images/Pasted image 20240121182529.png}

\end{figure}

\section{Power of Source of Load}\label{power-of-source-of-load}

\hyperref[power]{Power}

Let the current and voltage be, \[i= i_{o}\sin \omega t\]
\[V = V_{o}(\sin\omega t + \theta)\]

At any instant, \[
\begin{split}
P &= Vi \\
&= i_{o} V_{o} \sin(\omega t + \theta) \sin \omega t \\
&= \frac{ i_{o}V_{o} }{ 2 } [\cos \theta - \cos (2\omega t + \theta)] \\
&= \frac{ i_{o}V_{o} }{ 2 } \cos \theta - \frac{ i_{o}V_{o} }{ 2 } \cos (2\omega t + \theta) 
\end{split}
\]

Thus,

\begin{figure}[H]
\centering
\includegraphics[max width=0.9\linewidth,keepaspectratio]{images/Pasted image 20240121183246.png}

\end{figure}

Now, average power, \[
\begin{split}
P_{avg} &= \frac{ i_{o}V_{o} }{ 2 }\cos \theta \\
&= \frac{ i_{o} }{ \sqrt{ 2 } }. \frac{ V_{o} }{ \sqrt{ 2 } } \cos \theta \\
&= i_{rms} . V_{rms}\cos \theta 
\end{split}
\] Here, \(\theta\) is the \emph{power factor angle} or the \emph{phase
difference} between i and V. And \(\cos \theta\) is called the power
factor.

\section{AC Generator}\label{ac-generator}

It converts mechanical energy into electrical energy.

It works on the principle that when a loop is moved in a magnetic field,
an emf is generated in it.

When the coil is rotated with constant angular speed \(\upomega\), the
angle between \(\vec{B}\) and \(\vec{A}\) is, \[\theta = \upomega t\]
Now, magnetic flux, \[\phi = BAN\cos\upomega t\] From Faraday's laws,
induced emf, \[\varepsilon = -N \frac{ d\phi }{ dt }\]

Thus, \[\varepsilon = -NBA\upomega \sin\upomega t\]
\(\varepsilon = NBA\upomega\) is the max amount of emf that can be
produced.

Dividing it by \(R\), \[i = i_{o} \sin \upomega t\]

In India, \[
\begin{split}
\varepsilon_{o} &= 220\sqrt{ 2 }\ V \\
\varepsilon_{rms} &= 220\ V \\
f &= 50\ Hz \\
\omega &= 100\pi
\end{split}
\]

\begin{figure}[H]
\centering
\includegraphics[max width=0.9\linewidth,keepaspectratio]{images/Pasted image 20240121174954.png}

\end{figure}

\begin{figure}[H]
\centering
\includegraphics[max width=0.9\linewidth,keepaspectratio]{images/AC Generator.png}

\end{figure}

\newpage

\chapter{Individual Circuits}\label{individual-circuits}

\etocsettocstyle{\textbf{Chapter Contents}\par\rule{\linewidth}{0.5pt}}{\par\rule{\linewidth}{0.5pt}}
\localtableofcontents

\noindent {} \#\# Purely Resistive Circuit

\begin{figure}[H]
\centering
\includegraphics[max width=0.9\linewidth,keepaspectratio]{images/Pasted image 20240121183610.png}

\end{figure}

Applying KVL, \[
\begin{split}
V_{o}\sin\omega t - iR &= 0 \\
i &= \frac{ V_{o} }{ R } \sin\omega t \\
&= i_{o} \sin\omega t
\end{split}
\] Thus, \[i_{o} = \frac{V_{o}}{R}\]

Since voltage and current are in the same phase, we have the phasor
diagram,

\begin{figure}[H]
\centering
\includegraphics[max width=0.9\linewidth,keepaspectratio]{images/Pasted image 20240121183906.png}

\end{figure}

Here, \[\theta = 0 \implies \cos \theta = 1\] i.e.~Power Factor = 1

\[
\begin{split}
P_{avg} &= V_\text{rms}. i_\text{rms} \cos 0 \\
&= \frac{ V_{o} }{ \sqrt{ 2 } }. \frac{ i_{o} }{ \sqrt{ 2 } } . 1 \\
&= \frac{V_\text{rms}^{2}}{R} \\
&= \frac{ i_{o}^{2}R }{ 2 } 
\end{split}
\]

\section{Purely Capacitive Circuit}\label{purely-capacitive-circuit}

\begin{figure}[H]
\centering
\includegraphics[max width=0.9\linewidth,keepaspectratio]{images/Pasted image 20240121184137.png}

\end{figure}

At any time, charge on capacitor, \[q = CV_{o} \sin\omega t\]

Differentiating this, we get current, \[
\begin{split}
i &= \upomega C V_{o}\cos\omega t \\
i &= \frac{V_{o}}{\frac{1}{\omega C}} \cos\omega t \\
i &= \frac{V_{o}}{X_{C}} \sin(\omega t+90^{\circ})
\end{split}
\] here, \(X_{C}\) is called \textbf{capacitive reactance.} It is the
effective resistance provided by a capacitor in an ac circuit. Its unit
is ohm.

\[X_{C} = \frac{ 1 }{ \omega C } = \frac{ 1 }{ 2\pi fC }\]

And, \[
\begin{split}
i_{o} &= \frac{ V_{o} }{ X_{C} } \\
i_{rms} &= \frac{V_{o}}{\sqrt{ 2 }X_{C}}
\end{split}
\]

In a capacitive circuit, current is ahead of voltage by a phase angle of
\(\frac{\pi}{2}\). Thus the power factor is \(\cos 90^{\circ} = 0\), and
the average power comes out to be zero in a cycle.

In the first half, the capacitor charged (acting like a load), and in
the next half it discharges (acting like a source). Thus making the net
energy stored or dissipated in one cycle, zero.

\begin{figure}[H]
\centering
\includegraphics[max width=0.9\linewidth,keepaspectratio]{images/Pasted image 20240121184444.png}

\end{figure}

\section{Purely Inductive Circuit}\label{purely-inductive-circuit}

\begin{figure}[H]
\centering
\includegraphics[max width=0.9\linewidth,keepaspectratio]{images/Pasted image 20240121184820.png}

\end{figure}

Using KVL, \[
\begin{split}
V_{o}\sin\omega t - L \frac{ di }{ dt } &= 0 \\
i &= \frac{ -V_{o} }{ \omega L } \cos\omega t \\
&= \frac{ V_{o} }{ X_{L} } \sin (\omega t - 90^{\circ})
\end{split}
\] Here, \(X_{L}\) is called\textbf{inductive reactance.} It is the
effective resistance provided by an inductor in an ac circuit. Its unit
is ohm. \[X_{L} = \omega L = 2\pi fL\]

And, \[
\begin{split}
i_{o} &= \frac{ V_{o} }{ X_{L} } \\
i_\text{rms} &= \frac{V_{o}}{\sqrt{ 2 }X_{L}}
\end{split}
\]

In an inductive circuit, current is behind voltage by a phase angle of
\(\frac{\pi}{2}\). Thus the power factor is \(\cos 90^{\circ} = 0\), and
the average power is zero for a cycle.

For half a cycle current and magnetic energy increases (acting like a
source) and for the next half, both will decrease (acting like a
source).

\begin{figure}[H]
\centering
\includegraphics[max width=0.9\linewidth,keepaspectratio]{images/Pasted image 20240121185352.png}

\end{figure}

\newpage

\chapter{Dual Circuits}\label{dual-circuits}

\etocsettocstyle{\textbf{Chapter Contents}\par\rule{\linewidth}{0.5pt}}{\par\rule{\linewidth}{0.5pt}}
\localtableofcontents

\noindent {} Impedance is effective resistance (i.e.~reactance +
resistance) of circuit and is represented by Z.

\section{RC Circuit}\label{rc-circuit}

\begin{figure}[H]
\centering
\includegraphics[max width=0.9\linewidth,keepaspectratio]{images/Pasted image 20240121185451.png}

\end{figure}

Applying KVL, \[
\begin{split}
V - V_{R} - V_{C} &= 0 \\
V_{R} + V_{C} &= V
\end{split}
\]

We add the voltages using phasors, taking the current as base.
\[V = i\sqrt{ R^{2} + X_{C}^{2} }\]

Thus we get impedance, \[Z = \sqrt{ R^{2} + X_{C}^{2} }\] And peak
current,
\[i_{o} = \frac{ V_{o} }{ Z } = \frac{ V_{o} }{ \sqrt{ R^{2} + X_{C}^{2} } }\]

\begin{figure}[H]
\centering
\includegraphics[max width=0.9\linewidth,keepaspectratio]{images/Pasted image 20240121185746.png}

\end{figure}

From the phasor, we can see that current is ahead of voltage by a phase
of, \[\theta = \cos ^{-1} \frac{ R }{ Z }\]

Now, power factor, \[
\begin{split}
\cos \theta &= \frac{ V_{R} }{ V } \\
&= \frac{ iR }{ iZ } \\
&= \frac{ R }{ Z }
\end{split}
\]

And thus, the average power, \[
\begin{split}
P_{avg} &= i_{rms}V_{rms} \cos \theta \\
&= i_{rms}^{2}R 
\end{split}
\]

\begin{figure}[H]
\centering
\includegraphics[max width=0.9\linewidth,keepaspectratio]{images/Pasted image 20240121190615.png}

\end{figure}

\section{LR Circuit}\label{lr-circuit}

\begin{figure}[H]
\centering
\includegraphics[max width=0.9\linewidth,keepaspectratio]{images/Pasted image 20240121190634.png}

\end{figure}

Applying Kirchhoff's Loop Law, \[V = V_{R} + V_{L}\]

Adding them using phasors, with current as base,

\[V = i\sqrt{ R^{2} + X_{L}^{2} }\] Thus, impedance,
\[Z = \sqrt{ R^{2} + X_{L}^{2} }\] And peak current,
\[i_{o} = \frac{ V_{o} }{ Z }\] RMS current,
\[i_{rms} = \frac{ V_{rms} }{ Z }\]

\begin{figure}[H]
\centering
\includegraphics[max width=0.9\linewidth,keepaspectratio]{images/Pasted image 20240121190643.png}

\end{figure}

From here, we see that current is lagging voltage by angle,
\[\theta = \cos ^{-1} \frac{ R }{ Z }\]

Now, power factor, \[
\begin{split}
\cos \theta &= \frac{ V_{R} }{ V } \\
&= \frac{ iR }{ iZ } \\
&= \frac{ R }{ Z }
\end{split}
\]

Thus, we get average power, \[
\begin{split}
P_{avg} &= i_{rms} V_{rms} \cos \theta \\
&= i^{2}_{rms} R 
\end{split}
\]

\textbf{Choke Coil:} It is a wire with many loops. Thus it is kind of an
inductor.

They have resistance as well as inductance. An ideal choke coil has no
resistance.

Thus, we can consider choke coil to be inductor.

\begin{figure}[H]
\centering
\includegraphics[max width=0.9\linewidth,keepaspectratio]{images/Pasted image 20240121191221.png}

\end{figure}

\begin{figure}[H]
\centering
\includegraphics[max width=0.9\linewidth,keepaspectratio]{images/Pasted image 20240121191515.png}

\end{figure}

\section{LC Circuit}\label{lc-circuit}

\hyperref[circuit-solutions-of-lr-circuitlc-oscillations]{Circuit Solutions of LR Circuit\#LC Oscillations}

\begin{figure}[H]
\centering
\includegraphics[max width=0.9\linewidth,keepaspectratio]{images/Pasted image 20240121191527.png}

\end{figure}

Applying Kirchhoff's Loop Law, \[V = V_{L} + V_{C}\]

Adding them using phasors, with current as the base,
\[V = i|X_{L} - X_{C}|\]

Thus, impedance, \[Z = |X_{L} - X_{C}|\] And peak current,
\[i_{o} = \frac{ V_{o} }{ Z }\] RMS current,
\[i_{rms} = \frac{ V_{rms} }{ Z }\]

\begin{figure}[H]
\centering
\includegraphics[max width=0.9\linewidth,keepaspectratio]{images/Pasted image 20240121191606.png}

\end{figure}

From here we can see that the phase difference between current and
voltage in an LC circuit is always 90.

\textbf{If \(X_{L} > X_{C}\),} the circuit is called \textbf{Inductive.}

Thus, \[
\begin{split}
\omega L &> \frac{ 1 }{ \omega C } \\
\omega &> \frac{ 1 }{ \sqrt{ LC } }
\end{split}
\] And we get, \[Z = X_{L} - X_{C}\]

Current is lagging behind voltage by a phase of \(\frac{\pi}{2}\).

Since \(\cos \theta = 0\), average power comes out to be zero.

\begin{figure}[H]
\centering
\includegraphics[max width=0.9\linewidth,keepaspectratio]{images/Pasted image 20240121192029.png}

\end{figure}

\textbf{If \(X_{C} > X_{L}\),} the circuit is called Capacitive.

Thus, \[
\begin{split}
\omega L &< \frac{ 1 }{ \omega C } \\
\omega &< \frac{ 1 }{ LC }
\end{split}
\] And we get, \[Z = X_{C} - X_{L}\]

Current is ahead of the voltage by a phase of \(\frac{\pi}{2}\).

Since power factor is zero, \(\cos \theta = 0\), the average power is
zero.

\subsubsection{Graphs for LC circuit}\label{graphs-for-lc-circuit}

In Z vs \(\omega\) curve,

\[
\begin{split}
Z &= |X_{L} - X_{C}| \\
&= \left| \omega L - \frac{ 1 }{ \omega C } \right| 
\end{split}
\]

Similarly, we have \(i_{rms}\) vs \(\omega\), \[
\begin{split}
i_{rms} &= \frac{ V_{rms} }{ Z } \\
&= \frac{ V_{rms} }{ \left| \omega L - \frac{ 1 }{ \omega C }\right| }
\end{split}
\]

And thus we get, \(\omega\) resonance, \[
\begin{split}
\omega_{r} &= \frac{ 1 }{ \sqrt{ LC } } \\
f_{r} &= \frac{ 1 }{ 2\pi \sqrt{ LC } }
\end{split}
\]

\begin{figure}[H]
\centering
\includegraphics[max width=0.9\linewidth,keepaspectratio]{images/Pasted image 20240121194639.png}

\end{figure}

\newpage

\chapter{LCR Circuit}\label{lcr-circuit}

\etocsettocstyle{\textbf{Chapter Contents}\par\rule{\linewidth}{0.5pt}}{\par\rule{\linewidth}{0.5pt}}
\localtableofcontents

\noindent {} \#important

\begin{figure}[H]
\centering
\includegraphics[max width=0.9\linewidth,keepaspectratio]{images/Pasted image 20240121211020.png}

\end{figure}

Applying KVL, \[V = V_{R} + V_{L} + V_{C}\]

Adding them using phasors, taking current as base,
\[V = i\sqrt{ R^{2} + (X_{L} - X_{C})^{2} }\]
\[Z = \sqrt{ R^{2} + (X_{L} - X_{C})^{2} }\]

\begin{figure}[H]
\centering
\includegraphics[max width=0.9\linewidth,keepaspectratio]{images/Pasted image 20240121211050.png}

\end{figure}

\paragraph{Inductive Circuit}\label{inductive-circuit}

The current lags voltage.

Here, \(X_{L} > X_{C}\). That is, \[
\begin{split}
\omega L &> \frac{ 1 }{ \omega C } \\
\omega &> \frac{ 1 }{ \sqrt{ LC } }
\end{split}
\]

The resultant net voltage will be,
\[V = i\sqrt{ R^{2} + (X_{L} - X_{C})^{2} }\]

Thus, impedance is, \[Z = \sqrt{ R^{2} + (X_{L} - X_{C})^{2} }\]

The phase difference between current and voltage, \[
\begin{split}
\tan \theta &= \frac{ i(X_{L} - X_{C}) }{ iR } \\
&= \frac{ X_{L} - X_{C} }{ R } \\
\end{split}
\]

And power factor, \[
\begin{split}
\cos \theta &= \frac{ V_{R} }{ V } \\
&= \frac{ iR }{ iZ } \\
&= \frac{ R }{ Z }
\end{split}
\]

\begin{figure}[H]
\centering
\includegraphics[max width=0.9\linewidth,keepaspectratio]{images/Pasted image 20240121211725.png}

\end{figure}

\paragraph{Capacitive Circuit}\label{capacitive-circuit}

The current is ahead of voltage.

Here, \(X_{L} > X_{C}\). That is, \[
\begin{split}
\omega L &< \frac{ 1 }{ \omega C } \\
\omega &< \frac{ 1 }{ \sqrt{ LC } }
\end{split}
\]

The resultant net voltage will be,
\[V = i\sqrt{ R^{2} + (X_{C} - X_{L})^{2} }\]

Thus, impedance is, \[Z = \sqrt{ R^{2} + (X_{C} - X_{L})^{2} }\]

The phase difference between current and voltage, \[
\begin{split}
\tan \theta &= \frac{ i(X_{C} - X_{L}) }{ iR } \\
&= \frac{ X_{C} - X_{L} }{ R } \\
\end{split}
\]

And power factor, is the same.

\begin{figure}[H]
\centering
\includegraphics[max width=0.9\linewidth,keepaspectratio]{images/Pasted image 20240121211958.png}

\end{figure}

\subsection{Resonance}\label{resonance}

We have, \[i_{rms} = \frac{ V_{rms} }{ Z }\] Thus, for max. RMS current,
Z needs to be smallest. I.e.,
\[Z = \sqrt{ R^{2} + \left( \omega L - \frac{ 1 }{ \omega C } \right)^{2} }\]
must be smallest.

Z will be min, (= R), when, \[
\begin{split}
\omega L &= \frac{ 1 }{ \omega C } \\
\omega_{r} &= \frac{ 1 }{ \sqrt{ LC } }
\end{split}
\]

This angular frequency is called \emph{resonance angular frequency.}
And, resonance frequency is thus,
\[f_{r} = \frac{ \omega_{r} }{ 2\pi } = \frac{ 1 }{ 2\pi \sqrt{ LC } }\]

At resonance, \(X_{L} = X_{C}\).

At resonance, the circuit is neither capacitive nor inductive, and the
rms current is max. being, \[i_{rms} = \frac{ V_{rms} }{ R }\]

\begin{figure}[H]
\centering
\includegraphics[max width=0.9\linewidth,keepaspectratio]{images/Pasted image 20240122095303.png}

\end{figure}

\subsubsection{Voltage}\label{voltage}

Taking current as base, we add the voltages using phaser diagram.

Here, we see that \(V_{L}\) and \(V_{C}\) cancel each other out, thus
giving, \[V = V_{R} = iR\]

Thus, the circuit will behave like a purely resistive circuit at
resonance and the current and applied voltage will be in the same phase.

Power factor = 1.

\begin{figure}[H]
\centering
\includegraphics[max width=0.9\linewidth,keepaspectratio]{images/Pasted image 20240122095638.png}

\end{figure}

\subsection{Quality Factor}\label{quality-factor}

At resonance, average power, \[
\begin{split}
P_{avg} &= i_{rms}V_{rms} \cos \theta \\
&= i_{rms} (i_{rms} R) \\
&= i_{rms}^{2}R
\end{split}
\]

Now, as we vary \(\omega\), Z varies and thus \(i_{rms}\) as well as
\(V_{rms}\) varies. Thus \(P_{avg}\) also varies.

We can make a curve between P and \(\omega\).

\begin{figure}[H]
\centering
\includegraphics[max width=0.9\linewidth,keepaspectratio]{images/Pasted image 20240122100245.png}

\end{figure}

Half of average power can only occur at two angular frequencies,
\(\omega_{1}, \omega_{2}\).

They are called \emph{half power points.}

At max. power, \(i_{rms}\) is max., and thus Z is R.

Now, at half of max. power, since \(P \propto i_{rms}^{2}\), \(i_{rms}\)
has become \(1/\sqrt{ 2 }\) its max. value,
\[i_{rms} = \frac{ i_{rms(max)} }{ \sqrt{ 2 } }\] Thus, Z becomes, \[
\begin{split}
Z &= \sqrt{ 2 }R \\
\sqrt{ R^{2} + (X_{L} - X_{C})^{2} } &= \sqrt{ 2 }R \\
(X_{L} - X_{C})^{2} &= R^{2}
\end{split}
\] This will happen twice,

\begin{itemize}
\item
  \textbf{Capacitive,}

  \[
    \begin{split}
    X_{C} - X_{L} &= R \\
    \frac{ 1 }{ \omega C } - \omega L &= R \\
    1 - \omega^{2} LC &= \omega RC \\
    \omega_{1} &= \frac{ -RC + \sqrt{ R^{2} C^{2} + 4LC } }{ 2LC }
    \end{split}
    \]
\item
  \textbf{Inductive,}

  \[
    \begin{split}
    X_{L} - X_{C} &= R \\
    \omega L - \frac{ 1 }{ \omega C } &= R \\
    \omega^{2} LC - 1 &= \omega RC \\
    \omega_{2} &= \frac{ RC + \sqrt{ R^{2} C^{2} + 4LC } }{ 2LC }
    \end{split}
    \]
\end{itemize}

The gap between these \(\omega\) is \(\Delta\omega\) and is called
\textbf{Angular Frequency Bandwidth or Bandwidth.}

\[
\begin{split}
\Delta\omega &= \omega_{2} - \omega_{1} \\
&= \frac{ 2RC }{ 2LC } \\
&= \frac{ R }{ L }
\end{split}
\]

\textbf{Frequency Bandwidth,} \[
\begin{split}
\Delta f &= f_{2} - f_{1} \\
&= \frac{ \Delta\omega }{ 2\pi } \\
&= \frac{ 1 }{ 2\pi } \frac{ R }{ L }
\end{split}
\]

Now, we define, \textbf{Quality Factor Q,} \[
\begin{split}
Q &= \frac{\omega_{r}}{\Delta \omega} \\
&= \frac{ L }{ \sqrt{ LC }R } \\
&= \frac{ 1 }{ R } \sqrt{ \frac{ L }{ C } }
\end{split}
\]

Q is a unitless dimensionless quantity. It is just a ratio. Q decides
the sharpness of \(P-\omega\) graph.

Quality Factor is also define as,
\[Q = \frac{ \text{energy stored} }{ \text{energy dissipated in 1 cycle} } \times 2\pi\]
All the quantities are at resonance.

At resonance, both the capacitor and inductor will store the same
energy, thus we can write, \[
\begin{split}
Q &= \frac{ \frac{1}{2}Li_{rms}^{2} \times 2 }{ i_{rms}^{2} R \times \frac{2\pi}{\omega} } 2\pi \\
&= \frac{ L\omega }{ R } \\
&= \frac{ L }{ R \sqrt{ LC } } \\
&= \frac{ 1 }{ R } \sqrt{ \frac{ L }{ C } }
\end{split}
\]

\newpage

\chapter{Transformer}\label{transformer}

\etocsettocstyle{\textbf{Chapter Contents}\par\rule{\linewidth}{0.5pt}}{\par\rule{\linewidth}{0.5pt}}
\localtableofcontents

\noindent {} It is used to increase or decrease the potential of an
alternating current.

It is based on the principle of mutual induction.

\paragraph{Use of Transformer}\label{use-of-transformer}

Transformers are used in long distance transmission of electricity.

The voltage output of the generator is stepped up and then transmitted
through the wires. Since the current is reduced the energy lost in the
form of heat is minimum.

Near the consumer, it is again stepped down at the distributing
substations and finally a potential of 220 V reaches our homes.

\begin{figure}[H]
\centering
\includegraphics[max width=0.9\linewidth,keepaspectratio]{images/Pasted image 20240122112530.png}

\end{figure}

\subsubsection{Formation}\label{formation}

We take a soft iron core (it has least losses) to guide the magnetic
field as it is ferromagnetic.

We take the input AC source of input emf \(\varepsilon_{p}\), and wrap
primary coil of \(n_{p}\) turns on one side of the core.

As the input current is changing with time, there will be a time varying
magnetic field in the core. This magnetic field is guided by the core to
the secondary coil of \(n_{s}\) turns.

This secondary coil will have output emf \(\varepsilon_{s}\) due to the
time varying magnetic field.

Thus the coils will have a mutual inductance.

Now, applying KVL on the primary circuit, \[
\begin{split}
\varepsilon_{p} - n_{p} \frac{ d\phi }{ dt } &= 0 \\
\varepsilon_{p} &= n_{p} \frac{ d\phi }{ dt }
\end{split}
\]

emf induced in the secondary coil, \[
\begin{split}
\varepsilon_{s} &= -n_{s} \frac{ d\phi }{ dt }
\end{split}
\] The -ve here indicates that the AC of the output and input are out of
phase. I.e. \(\varepsilon_{s}\) and \(\varepsilon_{p}\) have a phase
difference of \(\pi\).

Ratio of secondary and primary emfs is thus,
\[\frac{ \varepsilon_{s} }{ \varepsilon_{p} } = \frac{ n_{s} }{ n_{p} }\]

\begin{figure}[H]
\centering
\includegraphics[max width=0.9\linewidth,keepaspectratio]{images/Pasted image 20240122130054.png}

\end{figure}

\begin{figure}[H]
\centering
\includegraphics[max width=0.9\linewidth,keepaspectratio]{images/Transformer.png}

\end{figure}

\subsubsection{Power of Ideal
Transformer}\label{power-of-ideal-transformer}

Transformer has no losses. Thus, \[P_{in} = P_{out}\]

And we define, efficiency,
\[\eta = \frac{ P_{out} }{ P_{in} } \times 100\] Thus, for an ideal
transformer, \(\eta = 100\%\)

Now, we have, \[
\begin{split}
P_{in} &= P_{out} \\
\varepsilon_{p} i_{p} &= \varepsilon_{s}i_{s} \\
\frac{ \varepsilon_{s} }{ \varepsilon_{p} } &= \frac{ i_{p} }{ i_{s} } \\
\frac{ n_{s} }{ n_{p} } &= \frac{ i_{p} }{ i_{s} }
\end{split} 
\]

Thus, current is inversely proportional to no. of turns. And, the
product \(\varepsilon i\) is constant.

\begin{figure}[H]
\centering
\includegraphics[max width=0.9\linewidth,keepaspectratio]{images/Pasted image 20240122130834.png}

\end{figure}

\subsubsection{Step-up and Step-down
transformer}\label{step-up-and-step-down-transformer}

In a \textbf{step-up transformer,} the no. of turns of secondary circuit
is more than that of primary circuit. Thus, the \emph{voltage is stepped
up} and \emph{current is stepped down.}

\[
\begin{split}
n_{s} &> n_{p} \\
\varepsilon_{s} &> \varepsilon_{p} \\
i_{s} &< i_{p} 
\end{split}
\]

\begin{figure}[H]
\centering
\includegraphics[max width=0.9\linewidth,keepaspectratio]{images/Pasted image 20240122131203.png}

\end{figure}

In a \textbf{step-down transformer,} the no. of turns of primary circuit
is more than that of secondary circuit. Thus, the \emph{voltage is
stepped down} and \emph{current is stepped up.}

\[
\begin{split}
n_{s} &< n_{p} \\
\varepsilon_{s} &< \varepsilon_{p} \\
i_{s} &> i_{p} 
\end{split}
\]

\begin{figure}[H]
\centering
\includegraphics[max width=0.9\linewidth,keepaspectratio]{images/Pasted image 20240122131329.png}

\end{figure}

\subsubsection{Energy Losses}\label{energy-losses}

There are two losses in a transformer, \textbf{copper (Cu) loss and iron
(Fe) loss.}

The \emph{heat loss} due to resistance of (Cu) wires of the windings is
called \emph{Copper Loss.}

\[H = i_{p}^{2}r_{p} + i_{s}^{2}r_{s}\]

Iron losses are the losses caused by the iron core of transformer. 1.
\textbf{Flux Leakage:} Not all the flux due to the first coil passes
through to the second due to poor design or air gaps. It can be fixed by
winding P and S one over the other. 2.
\textbf{\hyperref[electromagnetic-inductioneddy-current]{Eddy
Currents:}} Alternating magnetic flux induces eddy currents in the iron
core and causes heating, which dissipates energy. This can be reduced by
using Laminated Core. 3.
\textbf{\hyperref[ferromagnetismhysteresis]{Hysteresis:}} Due to
repeated changing of magnetic domains in the soft iron core, there will
be hysteresis and thus energy loss.

Net Power loss is, \[
\begin{split}
\text{Power Loss} &= P_{in} - P_{out} \\
&= \varepsilon_{p}i_{p} - \varepsilon_{s}i_{s} 
\end{split}
\]

Thus, iron loss can be given as,
\[\text{Iron Loss} = (\varepsilon_{p}i_{p} - \varepsilon_{s}i_{s}) - (i_{p}^{2}r_{p} + i_{s}^{2}r_{s})\]

And efficiency will be,
\[\eta = \frac{ P_{out} }{ P_{in} } \times 100 = \frac{ \varepsilon_{s}i_{s} }{ \varepsilon_{p}i_{p} } \times 100\]

\newpage

\backmatter
\end{document}
