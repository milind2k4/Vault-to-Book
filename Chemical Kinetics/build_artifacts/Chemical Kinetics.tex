%%
% Copyright (c) 2017 - 2025, Pascal Wagler;
% Copyright (c) 2014 - 2025, John MacFarlane
%
% All rights reserved.
%
% Redistribution and use in source and binary forms, with or without
% modification, are permitted provided that the following conditions
% are met:
%
% - Redistributions of source code must retain the above copyright
% notice, this list of conditions and the following disclaimer.
%
% - Redistributions in binary form must reproduce the above copyright
% notice, this list of conditions and the following disclaimer in the
% documentation and/or other materials provided with the distribution.
%
% - Neither the name of John MacFarlane nor the names of other
% contributors may be used to endorse or promote products derived
% from this software without specific prior written permission.
%
% THIS SOFTWARE IS PROVIDED BY THE COPYRIGHT HOLDERS AND CONTRIBUTORS
% "AS IS" AND ANY EXPRESS OR IMPLIED WARRANTIES, INCLUDING, BUT NOT
% LIMITED TO, THE IMPLIED WARRANTIES OF MERCHANTABILITY AND FITNESS
% FOR A PARTICULAR PURPOSE ARE DISCLAIMED. IN NO EVENT SHALL THE
% COPYRIGHT OWNER OR CONTRIBUTORS BE LIABLE FOR ANY DIRECT, INDIRECT,
% INCIDENTAL, SPECIAL, EXEMPLARY, OR CONSEQUENTIAL DAMAGES (INCLUDING,
% BUT NOT LIMITED TO, PROCUREMENT OF SUBSTITUTE GOODS OR SERVICES;
% LOSS OF USE, DATA, OR PROFITS; OR BUSINESS INTERRUPTION) HOWEVER
% CAUSED AND ON ANY THEORY OF LIABILITY, WHETHER IN CONTRACT, STRICT
% LIABILITY, OR TORT (INCLUDING NEGLIGENCE OR OTHERWISE) ARISING IN
% ANY WAY OUT OF THE USE OF THIS SOFTWARE, EVEN IF ADVISED OF THE
% POSSIBILITY OF SUCH DAMAGE.
%%

%%
% This is the Eisvogel pandoc LaTeX template.
%
% For usage information and examples visit the official GitHub page:
% https://github.com/Wandmalfarbe/pandoc-latex-template
%%
% Options for packages loaded elsewhere
\PassOptionsToPackage{unicode}{hyperref}
\PassOptionsToPackage{hyphens}{url}
\PassOptionsToPackage{dvipsnames,svgnames,x11names,table}{xcolor}
\documentclass[
  paper=a4,
  openany,
  oneside  ,captions=tableheading
]{scrbook}
\usepackage{xcolor}
\usepackage[top=1.5cm, bottom=2.5cm, left=3cm, right=3cm, includeheadfoot, heightrounded]{geometry}
\usepackage{amsmath,amssymb}

% add backlinks to footnote references, cf. https://tex.stackexchange.com/questions/302266/make-footnote-clickable-both-ways
\usepackage{footnotebackref}
\setcounter{secnumdepth}{5}
\usepackage{iftex}
\ifPDFTeX
  \usepackage[T1]{fontenc}
  \usepackage[utf8]{inputenc}
  \usepackage{textcomp} % provide euro and other symbols
\else % if luatex or xetex
  \usepackage{unicode-math} % this also loads fontspec
  \defaultfontfeatures{Scale=MatchLowercase}
  \defaultfontfeatures[\rmfamily]{Ligatures=TeX,Scale=1}
\fi
\usepackage{lmodern}
\ifPDFTeX\else
  % xetex/luatex font selection
  \setmainfont[]{LMRoman10-Regular}
  \setsansfont[]{Arial}
  \setmonofont[]{LMMono10-Regular}
\fi
% Use upquote if available, for straight quotes in verbatim environments
\IfFileExists{upquote.sty}{\usepackage{upquote}}{}
\IfFileExists{microtype.sty}{% use microtype if available
  \usepackage[]{microtype}
  \UseMicrotypeSet[protrusion]{basicmath} % disable protrusion for tt fonts
}{}

% Use setspace anyway because we change the default line spacing.
% The spacing is changed early to affect the titlepage and the TOC.
\usepackage{setspace}
\setstretch{1.2}
\makeatletter
\@ifundefined{KOMAClassName}{% if non-KOMA class
  \IfFileExists{parskip.sty}{%
    \usepackage{parskip}
  }{% else
    \setlength{\parindent}{0pt}
    \setlength{\parskip}{6pt plus 2pt minus 1pt}}
}{% if KOMA class
  \KOMAoptions{parskip=half}}
\makeatother
\usepackage{graphicx}
\makeatletter
\newsavebox\pandoc@box
\newcommand*\pandocbounded[1]{% scales image to fit in text height/width
  \sbox\pandoc@box{#1}%
  \Gscale@div\@tempa{\textheight}{\dimexpr\ht\pandoc@box+\dp\pandoc@box\relax}%
  \Gscale@div\@tempb{\linewidth}{\wd\pandoc@box}%
  \ifdim\@tempb\p@<\@tempa\p@\let\@tempa\@tempb\fi% select the smaller of both
  \ifdim\@tempa\p@<\p@\scalebox{\@tempa}{\usebox\pandoc@box}%
  \else\usebox{\pandoc@box}%
  \fi%
}
% Set default figure placement to htbp
% Make use of float-package and set default placement for figures to H.
% The option H means 'PUT IT HERE' (as  opposed to the standard h option which means 'You may put it here if you like').
\usepackage{float}
\floatplacement{figure}{H}
\makeatother
\ifLuaTeX
  \usepackage{luacolor}
  \usepackage[soul]{lua-ul}
\else
  \usepackage{soul}
\fi
\setlength{\emergencystretch}{3em} % prevent overfull lines
\providecommand{\tightlist}{%
  \setlength{\itemsep}{0pt}\setlength{\parskip}{0pt}}

\usepackage{caption}
\captionsetup[figure]{labelsep=none, justification=centering}

\usepackage{etoc} 

\usepackage[version=4]{mhchem}
\usepackage{amsmath}
\usepackage{amssymb}
\usepackage{mathtools}
\usepackage{gensymb}
\usepackage{cancel}

\usepackage[export]{adjustbox} % For max width=... in includegraphics
\usepackage{float} % Required for [H] figure placement

% --- Global Image Sizing from Config ---
\makeatletter
\let\oldincludegraphics\includegraphics
\renewcommand{\includegraphics}[2][]{
  \oldincludegraphics[max height=0.6\linewidth,keepaspectratio,#1]{#2}
}
\makeatother
% \mtcselectlanguage{english} - Removed
\definecolor{mylinkcolor}{HTML}{07455c}
\definecolor{myurlcolor}{HTML}{07455c}
% --- Chapter Title Styling (KOMA-Script) ---
\renewcommand*\chapterformat{\thechapter.\enskip}
\addtokomafont{chapter}{\centering}
\RedeclareSectionCommand[beforeskip=0pt,afterskip=20pt]{chapter}

% --- Table Styling ---
\rowcolors{2}{RoyalBlue!20}{white}
\renewcommand{\arraystretch}{1.2}

% --- Heading Styling ---
\makeatletter
\renewcommand\sectionlinesformat[4]{%
  \ifstr{#1}{section}{%
    \vspace{0.2em}
    \rule{\linewidth}{0.5pt}\par\nobreak\nointerlineskip
    \@hangfrom{\hskip #2#3}{#4}\par\nobreak\nointerlineskip
    \vspace{0.1em}
    \rule{\linewidth}{0.5pt}\par
  }{%
    \@hangfrom{\hskip #2#3}{#4}%
  }%
}
\makeatother
\usepackage[most]{tcolorbox}
\usepackage{fontawesome5}
\usepackage{xcolor}

% Define custom colors
\definecolor{notecolor}{RGB}{97, 175, 239}   % Blue
\definecolor{tipcolor}{RGB}{20, 196, 255} % Purple
\definecolor{warningcolor}{RGB}{229, 192, 123} % Orange
\definecolor{attentioncolor}{RGB}{224, 108, 117} % Red
\definecolor{analogycolor}{RGB}{152, 195, 121}    % Green

% Generic Pretty Box Style
% #1 = Color name
% #2 = Icon command
% #3 = Title text
\newtcolorbox{prettybox}[3]{
    enhanced,
    colback=#1!5!white,      % Very light background
    colframe=#1,             % Border color
    coltitle=#1!50!black,    % Title text color (darker version of base)
    title={#2\ \ \textbf{#3}},
    fonttitle=\bfseries\large,
    attach boxed title to top left={xshift=5mm, yshift=-3.5mm}, % Adjust for half-in/out
    boxed title style={
        colback=white,
        colframe=white,
        boxrule=0pt,
        top=0pt,
        bottom=0pt,
        left=2pt,
        right=2pt
    },
    top=1.5em, % Space for the floating title
    bottom=1em,
    left=1em,
    right=1em,
    arc=3pt,
    boxrule=1pt,
    drop fuzzy shadow,   
    parbox=false,        
    breakable            
}

% Define specific environments
% usage: \begin{noteblock}{Title} ... \end{noteblock}

\newenvironment{noteblock}[1]
  {\begin{prettybox}{notecolor}{\faInfoCircle}{#1}}
  {\end{prettybox}}

\newenvironment{tipblock}[1]
  {\begin{prettybox}{tipcolor}{\faLightbulb}{#1}}
  {\end{prettybox}}

\newenvironment{warningblock}[1]
  {\begin{prettybox}{warningcolor}{\faExclamationTriangle}{#1}}
  {\end{prettybox}}

\newenvironment{attentionblock}[1]
  {\begin{prettybox}{attentioncolor}{\faExclamationCircle}{#1}}
  {\end{prettybox}}

\newenvironment{cautionblock}[1]
  {\begin{prettybox}{attentioncolor}{\faRadiation}{#1}}
  {\end{prettybox}}

\newenvironment{importantblock}[1]
  {\begin{prettybox}{attentioncolor}{\faStar}{#1}}
  {\end{prettybox}}

% Analogy Block
\newenvironment{analogyblock}[1]
  {\begin{prettybox}{analogycolor}{\faShapes}{#1}}
  {\end{prettybox}}
\usepackage{bookmark}
\IfFileExists{xurl.sty}{\usepackage{xurl}}{} % add URL line breaks if available
\urlstyle{same}
\definecolor{default-linkcolor}{HTML}{A50000}
\definecolor{default-filecolor}{HTML}{A50000}
\definecolor{default-citecolor}{HTML}{4077C0}
\definecolor{default-urlcolor}{HTML}{4077C0}

\hypersetup{
  pdftitle={Chemical Kinetics},
  pdfauthor={Milind},
  colorlinks=true,
  linkcolor={mylinkcolor},
  filecolor={default-filecolor},
  citecolor={default-citecolor},
  urlcolor={myurlcolor},
  breaklinks=true,
  pdfcreator={LaTeX via pandoc with the Eisvogel template}}

\title{Chemical Kinetics}
\usepackage{etoolbox}
\makeatletter
\providecommand{\subtitle}[1]{% add subtitle to \maketitle
  \apptocmd{\@title}{\par {\large #1 \par}}{}{}
}
\makeatother
\subtitle{Personal Notes and References}
\author{Milind}
\date{}


%
% for the background color of the title page
%

%
% break urls
%
\PassOptionsToPackage{hyphens}{url}

%
% When using babel or polyglossia with biblatex, loading csquotes is recommended
% to ensure that quoted texts are typeset according to the rules of your main language.
%
\usepackage{csquotes}

%
% captions
%
\definecolor{caption-color}{HTML}{777777}
\usepackage[font={stretch=1.2}, textfont={color=caption-color}, position=top, skip=4mm, labelfont=bf, singlelinecheck=false, justification=raggedright]{caption}
\setcapindent{0em}

%
% blockquote
%
\definecolor{blockquote-border}{RGB}{221,221,221}
\definecolor{blockquote-text}{RGB}{119,119,119}
\usepackage{mdframed}
\newmdenv[rightline=false,bottomline=false,topline=false,linewidth=3pt,linecolor=blockquote-border,skipabove=\parskip]{customblockquote}
\renewenvironment{quote}{\begin{customblockquote}\list{}{\rightmargin=0em\leftmargin=0em}%
\item\relax\color{blockquote-text}\ignorespaces}{\unskip\unskip\endlist\end{customblockquote}}

%
% Source Sans Pro as the default font family
% Source Code Pro for monospace text
%
% 'default' option sets the default
% font family to Source Sans Pro, not \sfdefault.
%
% Note that the font has been officially renamed to `Source Sans 3`, and
% the version provided by the `sourcesanspro` package is slightly outdated.
% You can install the newer version locally and use it, for example, with
% `mainfont: "Source Sans 3"` in the YAML metadata (requires XeTeX or LuaTeX).
%
\ifnum 0\ifxetex 1\fi\ifluatex 1\fi=0 % if pdftex
    \usepackage[default]{sourcesanspro}
  \usepackage{sourcecodepro}
  \else % if not pdftex
    \fi

%
% heading color
%
\definecolor{heading-color}{RGB}{40,40,40}
% By default, the KOMA-Script classes will typeset sectioning headings in
% sans-serif. Use the normal body font for headings.
\addtokomafont{disposition}{\normalfont\color{heading-color}\bfseries}

%
% variables for title, author and date
%
\usepackage{titling}
\title{Chemical Kinetics}
\author{Milind}
\date{}

%
% tables
%

%
% remove paragraph indentation
%
\setlength{\parindent}{0pt}
\setlength{\parskip}{6pt plus 2pt minus 1pt}
\setlength{\emergencystretch}{3em}  % prevent overfull lines

%
%
% Listings
%
%


%
% header and footer
%
\usepackage[headsepline,footsepline]{scrlayer-scrpage}

\newpairofpagestyles{eisvogel-header-footer}{
  \clearpairofpagestyles
  \ihead*{Chemical Kinetics}
  \chead*{}
  \ohead*{}
  \ifoot*{Milind}
  \cfoot*{}
  \ofoot*{\thepage}
  \addtokomafont{pageheadfoot}{\upshape}
}
\pagestyle{eisvogel-header-footer}

\deftripstyle{ChapterStyle}{}{}{}{}{\pagemark}{}
\renewcommand*{\chapterpagestyle}{ChapterStyle}


%
% Define watermark
%

\begin{document}


\frontmatter
% don't generate the default title
% \maketitle



\begin{titlepage}
    \newgeometry{left=2.5cm,right=2.5cm,top=2cm,bottom=2cm}
    \vspace*{1cm}
    
    
    \vspace{3cm}
    
    \centering
    {\fontsize{50}{60}\selectfont \bfseries Chemical Kinetics \par}
    \vspace{1cm}
    {\fontsize{20}{30}\selectfont Personal Notes and References \par}
    
    \vfill
    
    {\fontsize{18}{22}\selectfont Milind \par}
    \vspace{0.5cm}
    {\large January 14, 2026 \par}
    
    \vspace{3cm}
    \restoregeometry
\end{titlepage}

{
\setcounter{tocdepth}{3}
\tableofcontents
}
\mainmatter
\chapter{Chemical Kinetics}\label{chemical-kinetics}

\etocsettocstyle{\textbf{Chapter Contents}\par\rule{\linewidth}{0.5pt}}{\par\rule{\linewidth}{0.5pt}}
\localtableofcontents

\noindent {}

Deals with the rate at which reactants are converted into products and
also with the effect of variation of conc., temp. etc. on on rate of
reaction.

It is a microscopic study. Thermodynamics is macroscopic study.

\subsubsection{Rate of Appearance and
Disappearance}\label{rate-of-appearance-and-disappearance}

Appearance is for products and Disappearance is for reactants.

\[\mathrm{ 2A + 3B -> C + 2D }\]

Rate of disappearance of A = no of moles of A reacted per unit time per
unit volume.
\[\mathrm{ = - \frac{ d[A]_{t} }{ dt } or - \frac{ \Delta [A] }{ \Delta t } }\]
-ve because concentration is decreasing and thus d{[}A{]} is -ve.

Similarly, we can write, rate of appearance of C,
\[\mathrm{ = \frac{ d [B]_{t} }{ dt } }\]

\paragraph{Relation between RoD and
RoA}\label{relation-between-rod-and-roa}

If all substances are in the same phase. For the reaction,
\[\mathrm{ 2A + 3B -> C + 2D }\]

We can write, \[
\begin{split}
\frac{ d[C] }{ dt } &= -\frac{ 1 }{ 2 } \frac{ d[A] }{ dt } \\
\frac{ d[D] }{ dt } &= 2 \frac{ d[C] }{ dt } \\
\end{split}
\] I.e.,
\[\mathrm{ - \frac{1}{2} \frac{ d[A] }{ dt } = - \frac{1}{3} \frac{ d[B] }{ dt } = \frac{ 1 }{ 1 } \frac{ d[C] }{ dt } = \frac{1}{2} \frac{ d[D] }{ dt } }\]

For a general reaction,
\[\mathrm{ aA_{(g)} + bB_{(g)} -> cC_{(g)} + dD_{(g)} }\] We can write,
\[\mathrm{ - \frac{ 1 }{ a }\frac{ d[A] }{ dt } = - \frac{ 1 }{ b }\frac{ d[B] }{ dt } = \frac{ 1 }{ c }\frac{ d[C] }{ dt } = \frac{ 1 }{ d }\frac{ d[D] }{ dt } }\]

\subsubsection{Rate of Reaction}\label{rate-of-reaction}

It is defined as the rate of change of conc. of any substance involved
in reaction divided by its stoichiometric coefficient in chemical given
chemical equation.

It can be the same or different than the rate of appearance or
disappearance of any substance in the reaction.

\[\mathrm{ aA + bB -> cC + dD }\]

\[\mathrm{ Rate = - \frac{ 1 }{ a }\frac{ d[A] }{ dt } = - \frac{ 1 }{ b }\frac{ d[B] }{ dt } = \frac{ 1 }{ c }\frac{ d[C] }{ dt } = \frac{ 1 }{ d }\frac{ d[D] }{ dt } }\]

\textbf{Note:} Is stoichiometry of a reaction is altered by multiplying
equation by a factor, then rate of reaction is changed. But, rate of
disappearance and rate of appearance of a substance does not change.

\textbf{Major factors affecting Rate of Reaction,} 1. Concentration or
Partial Pressure 2. Temperature 3. Catalyst 4. Surface area if solids
are involved

\section{\texorpdfstring{\hyperref[kinetics]{02 Kinetics}}{02 Kinetics}}\label{kinetics}

\subsection{\texorpdfstring{\hyperref[exp-det-of-order-and-k-of-first-order]{04 Exp. Det. of Order and k of First Order}}{04 Exp. Det. of Order and k of First Order}}\label{exp.-det.-of-order-and-k-of-first-order}

\subsection{Parallel or Competing First Order
Reaction}\label{parallel-or-competing-first-order-reaction}

A reaction where A converts to B and C simultaneously, is not parallel.

\begin{figure}[H]
\centering
\includegraphics[max width=0.9\linewidth,keepaspectratio]{images/Pasted image 20240308184024.png}

\end{figure}

A parallel reaction is one where the rate constants are different.

\begin{figure}[H]
\centering
\includegraphics[max width=0.9\linewidth,keepaspectratio]{images/Pasted image 20240308184039.png}

\end{figure}

For a general parallel reaction,

\begin{figure}[H]
\centering
\includegraphics[max width=0.9\linewidth,keepaspectratio]{images/Pasted image 20240308184308.png}

\end{figure}

We have, \[
\begin{split}
x &= \frac{ y }{ m } + \frac{ z }{ n } \\
\frac{ dx }{ dt } &= \frac{ 1 }{ m }\frac{ dy }{ dt } + \frac{ 1 }{ n }\frac{ dz }{ dt } \\
-\frac{ d[A] }{ dt } &= \frac{ 1 }{ m }mk_{1}[A] + \frac{ 1 }{ n }nk_{2}[A] \\
-\frac{ d[A] }{ dt } &= (k_{1}+k_{2})[A]
\end{split}
\] Thus the overall rate constant of parallel reaction is,
\[k = k_{1}+k_{2}\]

And the integrated rate law will be,
\[[A]_{t} = [A]_{o} e^{ -(k_{1}+k_{2})t }\]

We can also write this as,
\[\frac{ \ln 2 }{ t_{1/2}\text{(overall)} } = \frac{ \ln 2 }{ t_{1/2(1)} } + \frac{ \ln 2 }{ t_{1/2(2) } }\]

We can also relate activation energy, \[
\begin{split}
k &= k_{1} + k_{2} \\
Ae^{ -E_{a}/RT } &= A_{1}e^{ -E_{a 1}/RT } + A_{2}e^{ -E_{a 2}/RT } \\
Ae^{ -E_{a}/RT } \frac{ E_{a} }{ RT^{2} } &= A_{1}e^{ -E_{a 1}/RT } \frac{ E_{a 1} }{ RT^{2} } + A_{2}e^{ -E_{a 2}/RT } \frac{ E_{a 2} }{ RT^{2} } \\
kE_{a} &= k_{1}E_{a 1} + k_{2} E_{a 2} 
\end{split}
\] Thus we get, overall activation energy of parallel reaction,
\[E_{a} = \frac{ k_{1} E_{a 1} + k_{2} E_{a 2} }{ k_{1} + k_{2} }\]

\subsubsection{Concentration of
Products}\label{concentration-of-products}

\[
\begin{split}
\frac{ d[B] }{ dt } &= mk_{1}[A]_{t} \\
\int d[B]_{t} &= mk_{1}[A]_{o} \int e^{ -(k_{1}+k_{2})t } \, dt \\
[B]_{t} &= \frac{ mk_{1}[A]_{o} }{ k_{1} + k_{2} } (1 - e^{ -(k_{1}+k_{2})t })  
\end{split}
\]

Similarly, we get, conc. of C,
\[[C]_{t} = \frac{ nk_{2}[A]_{o} }{ k_{1} + k_{2} } (1 - e^{ -(k_{1}+k_{2})t })  \]

Thus, at any time, the ratio of conc. of B and C,
\[\frac{ [B]_{t} }{ [C]_{t} } = \frac{ mk_{1} }{ nk_{2} }\]

\subsection{First Order Reactions}\label{first-order-reactions}

\subsubsection{First Order Reversible
Reaction}\label{first-order-reversible-reaction}

\[t = \frac{ 1 }{ k_{1} + k_{2} } \ln \frac{ x_{c} }{ x_{c} - x }\]

\begin{figure}[H]
\centering
\includegraphics[max width=0.9\linewidth,keepaspectratio]{images/Pasted image 20240308201445.png}

\end{figure}

\subsubsection{First Order Sequential
Reaction}\label{first-order-sequential-reaction}

aka \textbf{Consecutive reaction}

Example is radioactive disintegration series. Note than radioactive
decay always follows first order kinetics.

\[\mathrm{ A -> I -> P }\]

\begin{figure}[H]
\centering
\includegraphics[max width=0.9\linewidth,keepaspectratio]{images/Pasted image 20240308203216.png}

\end{figure}

Then, \[
\begin{split}
[A]_{t} &= [A]_{o}e^{ -k_{1}t } \\
[I]_{t} &= \frac{ k_{1}[A]_{o} }{ k_{2}-k_{1} } (e^{ -k_{1}t } - e^{ -k_{2}t }) \\
[P]_{t} &= \frac{ [A]_{o} }{ k_{2} - k_{1} } [k_{2}(1-e^{ -k_{1}t }) - k_{1}(1-e^{ -k_{1}t })]
\end{split}
\]

The time at which conc. of intermediate is max is,
\[t_{I_{max}} = \frac{ \ln k_{1} /k_{2} }{ k_{1} - k_{2} }\]

\newpage

\chapter{Effect of Concentration}\label{effect-of-concentration}

\etocsettocstyle{\textbf{Chapter Contents}\par\rule{\linewidth}{0.5pt}}{\par\rule{\linewidth}{0.5pt}}
\localtableofcontents

\noindent {}

\hyperref[law-of-mass-action]{01 Law of Mass Action}

Experimentally, it was found that,
\[\mathrm{ Rate \propto [A]^{p}[B]^{q} }\] Thus we can write,
\[\mathrm{ Rate = k_{r}[A]^{p}[B]^{q} }\] This equation is called
\emph{Rate Law.}

p and q are called partial order wrt A and B and they may or may not be
equal to the stoichiometric coefficient of substance.

\(\mathrm{ k_{r} }\) is the rate constant of reaction and \(m = p + q\)
is the order of reaction.

\subsubsection{Order of Reaction}\label{order-of-reaction}

It is the sum of exponents of all concentration terms appearing in rate
law. It can be +ve, -ve or zero or fractional. It is determined
\emph{experimentally.} It cannot be predicted by just seeing
stoichiometry of reaction.

Order reflects the sensitivity of rate of reaction with variation in
concentration.

For example, rate of the below reaction does not depend on
concentration, \[\mathrm{ H_{2} + Cl_{2} ->[h\nu] 2HCl, Rate = k }\]

The order of below reaction is first order,
\[\mathrm{ 2N_{2}O_{5} -> 4NO_{2} + O_{2}, Rate = k[N_{2}O_{5}] }\]

The following reaction is second order,
\[\mathrm{ 2HI ->[\Delta] I_{2} + H_{2}, Rate = k[HI]^{2} }\]

And the following is zero order,
\[\mathrm{ 2HI ->[\Delta][gold surface] H_{2} + I_{2}, Rate = k }\]

\subsubsection{Rate Constant}\label{rate-constant}

Rate constant or velocity constant or specific reaction rate is equal to
rate of reaction if all concentrations of substance involved in rate law
are set equal to unity.

Its unit depends on order of reaction.

\[\mathrm{ k = \frac{ Rate }{ [A]^{p}[B]^{q} } = \frac{ mol l^{-1} s^{-1} }{ [mol l^{-1}]^{p+q} } }\]

Thus, for zero order reaction, unit of k is
\(\mathrm{ mol l^{-1} s^{-1} }\). Thus, for first order reaction, unit
of k is \(\mathrm{ s^{-1} }\). Thus, for second order reaction, unit of
k is \(\mathrm{ mol^{-1} l s^{-1} }\).

It increases with temp..

\subsubsection{Rate constant of Disappearance and
Appearance}\label{rate-constant-of-disappearance-and-appearance}

We have,
\[\mathrm{ Rate = - \frac{ 1 }{ a }\frac{ d[A] }{ dt } = - \frac{ 1 }{ b }\frac{ d[B] }{ dt } = \frac{ 1 }{ c }\frac{ d[C] }{ dt } = \frac{ 1 }{ d }\frac{ d[D] }{ dt } }\]

Also, \[\mathrm{ Rate = k[A]^{p}[B]^{q} }\]

Thus, \[\mathrm{ -\frac{ d[A] }{ dt } = ak[A]^{p}[B]^{q} }\]
\[\mathrm{ -\frac{ d[A] }{ dt } = k_{A}[A]^{p}[B]^{q} }\] Where
\(k_{A} = ak\). I.e.
\[\mathrm{ k = \frac{ k_{A} }{ a } = \frac{ k_{B} }{ b } = \frac{ k_{C} }{ c } = \frac{ k_{D} }{ d } }\]

Where \(\mathrm{ k_{A}, k_{B}, k_{C}, k_{D} }\) are rate constants for
A, B, C, D.

Thus rate constant of reaction may or may not be the same as the rate
constant of disappearance or appearance of substance.

\newpage

\chapter{Kinetics}\label{kinetics}

\etocsettocstyle{\textbf{Chapter Contents}\par\rule{\linewidth}{0.5pt}}{\par\rule{\linewidth}{0.5pt}}
\localtableofcontents

\noindent {}

\subsection{Zero Order Kinetics}\label{zero-order-kinetics}

When order of reaction is zero.

If the reaction is, \[\mathrm{ A -> P }\] Then the rate will be,
\[\mathrm{ Rate = - \frac{ d[A] }{ dt } = k[A]^{0} = k }\] This is
called \emph{differential rate law.}

Using it we can find \emph{integrated rate law,} \[
\begin{split}
- \frac{ d[A] }{ dt } &= k \\
\int_{[A]_{o}}^{[A]_{t}} d[A] &= -k \int_{0}^{t} dt \\
[A]_{o} - [A]_{t} &= kt 
\end{split}
\] Thus, we get, the integrated rate law,
\[\mathrm{ [A]_{t} = [A]_{o} - kt }\]

Now, if in t time, x mol/lit are used up from the initial a mol/lit, we
can write, \[
\begin{split}
a - x &= a - kt \\
x &= kt
\end{split}
\] Where x is the conc. of A or reactant reacted in t time. I.e.,
\[\mathrm{ x = [A]_{o} - [A]_{t} }\]

\paragraph{Half Life}\label{half-life}

Time in which half of the reactant gets reacted.

For zero order reaction, \[\mathrm{ [A]_{t} = [A]_{o} - kt }\] If
\(t = t_{1 /2}\), \(\mathrm{ [A]_{t} = [A]_{o}/2 }\), And thus,
\[\mathrm{ t_{1 /2} = \frac{ [A]_{o} }{ 2k } }\]

\textbf{Time of completion} is when reactants become zero.
\[\mathrm{ t_{c} = \frac{ [A]_{o} }{ k } = 2t_{1 /2} }\]

Only zero order reaction completes in a finite time period.

Equal concentrations of reactants react in equal time periods in zero
order reaction.

The conc. of reactant remaining after equal time periods forms AP.

\[
\begin{split}
[A]_{o} &= [A]_{o} \\
[A]_{10} &= [A]_{o} - 10k \\
[A]_{20} &= [A]_{o} - 20k \\
[A]_{30} &= [A]_{o} - 30k \\
\end{split}
\]

\subsubsection{Graph}\label{graph}

\begin{figure}[H]
\centering
\includegraphics[max width=0.9\linewidth,keepaspectratio]{images/Pasted image 20240307171302.png}

\end{figure}

\paragraph{Some PTR}\label{some-ptr}

Note that if the reaction becomes, \[\mathrm{ mA ->[k] P }\] Then rate
of reaction becomes,
\[\mathrm{ - \frac{ 1 }{ m }\frac{ d[A] }{ dt } = k }\] Thus the
integrated rate law becomes,
\[\mathrm{ [A]_{t} = [A]_{o} - mkt = [A]_{o} - k_{A}t }\] And half life
becomes,
\[\mathrm{ t_{1 /2} = \frac{ [A]_{o} }{ 2(mk) } = \frac{ [A]_{o} }{ 2k_{A} } }\]

If the reaction has 2 or more reactants, \[\mathrm{ A + 2B ->[k] P }\]
Then, \[\mathrm{ [A]_{t} = [A]_{o} - k_{A}t }\]
\[\mathrm{ [B]_{t} = [B]_{o} - k_{B}t }\]

And half life, \[\mathrm{ t_{1 /2(A)} = \frac{ [A]_{o} }{ 2k_{A} } }\]
\[\mathrm{ t_{1 /2(B)} = \frac{ [B]_{o} }{ 2k_{B} } }\]

Half life of reaction is defined when both the half lives are equal. \[
\begin{split}
t_{1 /2(A)} &= t_{1 /2(B)} \\
\frac{ [A]_{o} }{ [B]_{o} } &= \frac{ 1 }{ 2 }
\end{split}
\]

Thus for a reaction involving 2 more reactants, then half lives of all
reactants are equal when their initial conc. are taken in their
stoichiometric proportions.

In such case, \[t_{1/2(A)} = t_{1 /2(B)} = t_{1 /2(\text{reaction})}\]

\subsection{First Order Kinetics}\label{first-order-kinetics}

\[\mathrm{ A ->[k] P }\]

Differential rate law, \[\mathrm{ - \frac{ d[A] }{ dt } = k[A] }\]

Integrated rate law, \[
\begin{split}
\int_{[A]_{o}}^{[A]_{t}} \frac{ d[A] }{ [A] } &= -k \int_{0}^{t} dt \\
\ln \frac{ [A]_{t} }{ [A]_{o} } &= -kt  \\
[A]_{t} &= [A]_{o} e^{-kt}  
\end{split}
\]

And k can be written as,
\[k = \frac{ 2.303 }{ t } \log_{10} \frac{ [A]_{o} }{ [A]_{t} }\]

Half life is thus,
\[t_{1/2} = \frac{ \ln 2 }{ k } = \frac{ 0.693 }{ k }\] Which is
independent of concentration.

As we increase temp., k decreases and thus half life decreases.

Time of completion is infinite.

For first order, average life,
\[t_{avg} = \frac{ 1 }{ k } \approx 1.44 \times t_{1/2}\]

In n half lives, conc. of reactant remaining,
\[[A]_{t_{1/2}(n)} = \frac{ [A]_{o} }{ 2^{n} }\]

\subsubsection{Graph}\label{graph-1}

\begin{figure}[H]
\centering
\includegraphics[max width=0.9\linewidth,keepaspectratio]{images/Pasted image 20240307174015.png}

\end{figure}

\paragraph{Characteristics of First Order
Reaction}\label{characteristics-of-first-order-reaction}

Conc. of reactant which reacts in equal time period goes on decreasing.

Conc. of reactant remaining after equal time period forms GP. \[
\begin{split}
[A]_{o} &= [A]_{o} \\
[A]_{10} &= [A]_{o}e^{ -10k } \\
[A]_{20} &= [A]_{o}e^{ -20k } \\
[A]_{30} &= [A]_{o}e^{ -30k } \\
\end{split}
\]

Equal conc. of reactant does not react in equal time. But, equal
percentage or fraction of reactant reacts in equal time period.

\begin{figure}[H]
\centering
\includegraphics[max width=0.9\linewidth,keepaspectratio]{images/Pasted image 20240307174735.png}

\end{figure}

\pandocbounded{\includegraphics[keepaspectratio]{images/Pasted image 20240307175401.png}}
(in 2nd question it is 3/4 not 7/8)

\begin{figure}[H]
\centering
\includegraphics[max width=0.9\linewidth,keepaspectratio]{images/Pasted image 20240307175902.png}

\end{figure}

\subsection{\texorpdfstring{\hyperref[nd-order-kinetics]{03 2nd Order Kinetics}}{03 2nd Order Kinetics}}\label{nd-order-kinetics}

\subsection{nth Order Kinetics}\label{nth-order-kinetics}

\[\mathrm{ A ->[k] P }\] And differential rate law is,
\[\mathrm{ Rate = -\frac{ d[A] }{ dt } = k[A]^{n} }\]

Integrated rate law, \[
\begin{split}
-\int_{[A]_{o}}^{[A]_{t}} \frac{ d[A] }{ [A]^{n} } &= k\int_{0}^{t} dt \\
t &= \frac{ 1 }{ k(n-1) } \left( \frac{ 1 }{ [A]_{o}^{n-1} } - \frac{ 1 }{ [A]_{t}^{n-1} } \right)
\end{split}
\]

It is applicable everywhere except first order.

Half Life,
\[\mathrm{ t_{1/2} = \frac{ 1 }{ k(n-1) } \left( \frac{ 2^{n-1}-1 }{ [A]_{o}^{n-1} } \right) }\]

Thus, \[t_{1/2} \propto \frac{ 1 }{ [A]_{o}^{n-1} }\]

\newpage

\chapter{nd Order Kinetics}\label{nd-order-kinetics}

\etocsettocstyle{\textbf{Chapter Contents}\par\rule{\linewidth}{0.5pt}}{\par\rule{\linewidth}{0.5pt}}
\localtableofcontents

\noindent {}

\subsubsection{Single Reactant}\label{single-reactant}

\[\mathrm{ A ->[k] P }\]

Differential rate law,
\[\mathrm{ Rate = - \frac{ d[A] }{ dt } = k[A]^{2} }\]

Integrated rate law, \[
\begin{split}
-\int_{[A]_{o}}^{[A]_{t}} \frac{ d[A] }{ [A]_{o}^{2} } &= k \int_{0}^{t} dt \\
\frac{ 1 }{ [A]_{t} } &= \frac{ 1 }{ [A]_{o} } + kt  
\end{split}
\]

Half life, \[t_{1/2} = \frac{ 1 }{ k[A]_{o} }\]

Thus in 2nd order reaction, \[t_{1/2} \propto \frac{ 1 }{ [A]_{o} } \]

And time of completion is infinite.

Conc. after equal time period form HP.

\textbf{Graph,}

\begin{figure}[H]
\centering
\includegraphics[max width=0.9\linewidth,keepaspectratio]{images/Pasted image 20240307180826.png}

\end{figure}

\subsubsection{Two Reactants}\label{two-reactants}

\[\mathrm{ A + B ->[k] P }\]

Differential rate law,
\[\mathrm{ -\frac{ d[A] }{ dt } = -\frac{ d[B] }{ dt } = k[A][B] }\]

Integrated rate law \[
\begin{split}
- \frac{ d }{ dt }(a-x) &= k(a-x)(b-x) \\
\frac{ dx }{ dt } &= k(a-x)(b-x) \\
\int_{0}^{x} \frac{ dx }{ (a-x)(b-x) } &= k \int_{0}^{t} dt \\
k &= \frac{ 1 }{ t(a-b) } \ln \left( \frac{ a-x }{ b-x } \frac{ b }{ a } \right) 
\end{split}
\]

Thus, we get, \[t = \frac{ 1 }{ k(a-b) } \ln \frac{ b(a-x) }{ a(b-x) }\]

\paragraph{Special Case}\label{special-case}

If \(a \gg b\), then \(a-b \approx a\) and \(a-x \approx a\).

Thus we can write, \[
\begin{split}
t &= \frac{ 1 }{ k a } \ln\frac{ ab }{ a(b-x) } \\
t &= \frac{ 1 }{ k' } \ln \frac{ b }{ b-x } \\
t &= \frac{ 1 }{ k' } \ln \frac{ [B]_{o} }{ [B]_{t} } \\
\end{split}
\]

Which is the same equation as that of first order. Thus it is first
order in B. And thus is called \emph{pseudo first order reaction.}

\begin{figure}[H]
\centering
\includegraphics[max width=0.9\linewidth,keepaspectratio]{images/Pasted image 20240308105158.png}

\end{figure}

\paragraph{Pseudo Order Kinetics}\label{pseudo-order-kinetics}

It arises when, 1. Rate law involves catalyst (conc. of catalyst remains
constant) 2. Solvent taken in excess 3. One of the reactant's conc. is
taken much higher than the others.

Thus, for reaction, \[\mathrm{ A + B ->[k] P }\] If rate law is,
\[\mathrm{ Rate = k[A][B] }\] And \([A]_{o} \gg [B]_{o}\), then
\(\mathrm{ [A] }\) is almost constant. And we can write,
\[\mathrm{ Rate = k'[B] }\] where \(k'\) is called pseudo first order
rate constant and thus the reaction becomes effectively first order.

\textbf{Examples of Pseudo First Order Reaction:} 1. Hydrolysis of Alkyl
halides. \[\mathrm{ RCl + H_{2}O ->[k] ROH + HCl }\] Here,
\(\mathrm{ Rate = k[RCl][H_{2}O] }\). But since water is taken in excess
and \(\mathrm{ [H_{2}O] = 55.55 M }\), we get,
\[\mathrm{ Rate = k'[RCl], k' = k[H_{2}O] }\]

\begin{enumerate}
\def\labelenumi{\arabic{enumi}.}
\setcounter{enumi}{1}
\item
  Acid catalysed hydrolysis of sucrose.
  \[\mathrm{ C_{12}H_{22}O_{11} + H_{2}O ->[H+] \underset{ glucose }{ C_{6}H_{12}O_{6} } + \underset{ fructose }{ C_{6}H_{12}O_{6} } }\]

  Experimentally,
  \[\mathrm{ Rate = k [C_{12}H_{22}O_{11}][H_{2}O][H+] }\] However,
  \(\mathrm{ H_{2}O }\) is solvent and \(\mathrm{ H+ }\) is catalyst.
  Thus, the rate law becomes,
  \[\mathrm{ Rate = k'[C_{12}H_{22}O_{11}] }\]
\end{enumerate}

\newpage

\chapter{Exp. Det. of Order and k of First
Order}\label{exp-det-of-order-and-k-of-first-order}

\etocsettocstyle{\textbf{Chapter Contents}\par\rule{\linewidth}{0.5pt}}{\par\rule{\linewidth}{0.5pt}}
\localtableofcontents

\noindent {}

\subsection{Exp. Det. of Order of
Reaction}\label{exp.-det.-of-order-of-reaction}

Order of reaction cannot be determined by seeing the stoichiometric of
reaction.

If reaction is an elementary reaction, or single step reaction, then
order of reaction is the sum of stoichiometric coefficients of
reactants.

\begin{figure}[H]
\centering
\includegraphics[max width=0.9\linewidth,keepaspectratio]{images/Pasted image 20240308111402.png}

\end{figure}

However, we cannot find if a reaction is elementary or not just by see
the reaction.

\subsubsection{Initial Rate Method}\label{initial-rate-method}

We determine rates at various conc. of the reactants. From this we can
find the order.

\begin{figure}[H]
\centering
\includegraphics[max width=0.9\linewidth,keepaspectratio]{images/Pasted image 20240308111654.png}

\end{figure}

\subsubsection{Half Life Method}\label{half-life-method}

We know that, for an nth order reaction,
\[t_{1/2} \propto \frac{ 1 }{ a^{n-1} }\] where a is initial
concentration of reactant.

n represents the order of reaction and we cannot find partial order from
this method.

\begin{figure}[H]
\centering
\includegraphics[max width=0.9\linewidth,keepaspectratio]{images/Pasted image 20240308112032.png}

\end{figure}

\subsubsection{Hit and Trial Method}\label{hit-and-trial-method}

This is also called \textbf{Integrated Rate Law Method.}

\begin{figure}[H]
\centering
\includegraphics[max width=0.9\linewidth,keepaspectratio]{images/Pasted image 20240308113632.png}

\end{figure}

\subsection{Monitoring the Progress of 1st Order
Reaction}\label{monitoring-the-progress-of-1st-order-reaction}

i.e.~Determination of rate constant, k.

There are 3 methods, 1. Pressure Measurement 2. Titration 3. Optical
rotation measurement

\subsubsection{By Pressure Measurement}\label{by-pressure-measurement}

Applicable reaction involving at least one gas.

For constant V and T, \[p \propto c\]

Thus we can write, \[
\begin{split}
k &= \frac{ 1 }{ t } \ln \frac{ [A]_{o} }{ [A]_{t} } \\
&= \frac{ 1 }{ t } \ln \frac{ a }{ a-x } \\
&= \frac{ 1 }{ t } \ln \frac{ p_{o(A)} }{ p_{t(A)} }
\end{split}
\]

\begin{figure}[H]
\centering
\includegraphics[max width=0.9\linewidth,keepaspectratio]{images/Pasted image 20240308114608.png}

\end{figure}

A general formula is,
\[k = \frac{ 1 }{ t } \ln \frac{ p_{\infty} - p_{o} }{ p_{\infty} - p_{t} }\]

\begin{figure}[H]
\centering
\includegraphics[max width=0.9\linewidth,keepaspectratio]{images/Pasted image 20240308115920.png}

\end{figure}

\subsubsection{By Titration}\label{by-titration}

\hyperref[law-of-chemical-equivalence]{02 Equivalent Concept\#Law of Chemical Equivalence}

The rate constant will be found directly in term of volumes of titrant
used at various instances of the reaction.

\[k = \frac{ 1 }{ t } \ln \frac{ V_{o} }{ V_{t} }\]

\begin{figure}[H]
\centering
\includegraphics[max width=0.9\linewidth,keepaspectratio]{images/Pasted image 20240308120743.png}

\end{figure}

\begin{figure}[H]
\centering
\includegraphics[max width=0.9\linewidth,keepaspectratio]{images/Pasted image 20240308121319.png}

\end{figure}

\subsubsection{By Measurement of Optical
Rotation}\label{by-measurement-of-optical-rotation}

\hyperref[checking-optical-activity]{06 Optical Isomerism\#Checking Optical Activity}

This is applicable only for reaction involving optically active
compounds.

Optical rotation \(\theta\), \[
\begin{split}
\theta &\propto \text{conc.} \\
&\propto \text{thickness of polarimeter tupe (t)}
\end{split}
\] The thickness t is taken as 1 because it cancels out in the ratio
inside log in case of first order kinetics.

We can write, \[\theta = r \times c\] where, \(r \to\) \emph{specific
rotation of compound.} It is constant for an optically active compound.
\(c \to\) concentration of solution wrt that compound. \(\theta \to\)
observed rotation.

The general formula for the rate constant comes out to be,
\[k = \frac{ 1 }{ t } \ln \frac{ \theta_{\infty} - \theta_{o} }{ \theta_{\infty} - \theta_{t} }\]
It is applicable for any first order with optically active compound.

\begin{figure}[H]
\centering
\includegraphics[max width=0.9\linewidth,keepaspectratio]{images/Pasted image 20240308122421.png}

\end{figure}

\begin{figure}[H]
\centering
\includegraphics[max width=0.9\linewidth,keepaspectratio]{images/Pasted image 20240308172212.png}

\end{figure}

\newpage

\chapter{Effect of Temperature}\label{effect-of-temperature}

\etocsettocstyle{\textbf{Chapter Contents}\par\rule{\linewidth}{0.5pt}}{\par\rule{\linewidth}{0.5pt}}
\localtableofcontents

\noindent {}

It is observed that rate of reaction becomes double to triple for every
10 K rise in temp. for most chemical reactions.

\subsection{Collision Theory}\label{collision-theory}

If a reaction has to proceed in the forward direction, the reactants
must collide with each other.

Every collision is not successful.

For a collision to be successful to form products, reactants must
possess a certain min. total energy called \textbf{Threshold Energy} and
also have proper orientation.

\textbf{Activation Energy} is the additional KE possessed by reactants
relative to initial state of reactants to cross energy barrier (TS).

\begin{figure}[H]
\centering
\includegraphics[max width=0.9\linewidth,keepaspectratio]{images/Pasted image 20240308173034.png}

\end{figure}

\begin{figure}[H]
\centering
\includegraphics[max width=0.9\linewidth,keepaspectratio]{images/Pasted image 20240308173614.png}

\end{figure}

Rate of reaction is given as, Rate of Reaction = collision frequency
\(\times\) energy barrier factor \(\times\) orientation factor

\emph{Orientation factor} is also known as \emph{steric factor or
probability factor.}

\emph{Collision frequency} is no. of collisions per unit volume per unit
time.

\emph{Energy barrier factor} is \(e^{ -E_{a}/RT }\) and it represents
the fraction of reactant molecules that have KE more than or equal to
activation energy \(E_{a}\).

\paragraph{Arrhenius Equation}\label{arrhenius-equation}

Finally, the rate of reaction comes out to be,
\[\mathrm{ Rate = (Ae^{ -E_{a}/RT }) (conc.)^{order} }\]

This is written as, \[\mathrm{ Rate = k (conc.)^{order} }\] where k is
rate constant for reaction and, \[\mathrm{ k = Ae^{ -E_{a}/RT } }\] This
equation is called \textbf{Arrhenius Equation.} Here, \(E_{a} \to\)
activation energy of reaction \(A \to\) frequency factor or
\emph{Arrhenius constant.} \(T \to\) temp. in kelvin.

\begin{figure}[H]
\centering
\includegraphics[max width=0.9\linewidth,keepaspectratio]{images/Pasted image 20240308175001.png}

\end{figure}

Both \(\mathrm{ E_{a} and A }\) are taken to be independent of temp..

The max. value of k becomes A when, 1. \(T \to \infty\) 2. \(E_{a} = 0\)
i.e.~no energy barrier.

A is the product of collision frequency and orientation factor.
\[A = Z_{11}\times p\]

We have, \[
\begin{split}
k &= Ae^{ -E_{a}/RT } \\
\ln k &= \ln A - \frac{ E_{a} }{ RT }
\end{split}
\] And thus the graph will look like,

\begin{figure}[H]
\centering
\includegraphics[max width=0.9\linewidth,keepaspectratio]{images/Pasted image 20240308175144.png}

\end{figure}

\paragraph{Integral and Differential form of A.
Equation}\label{integral-and-differential-form-of-a.-equation}

Applying the log equation at two temp., we can find the integral form of
the Arrhenius equation.

At temp. \(T_{1}\), \[\ln k_{1} = \ln A - \frac{ E_{a} }{ RT_{1} }\] At
temp. \(T_{2}\), \[\ln k_{2} = \ln A - \frac{ E_{a} }{ RT_{2} }\]

Subtracting them, \[
\begin{split}
\ln \frac{ k_{1} }{ k_{2} } &= \frac{ E_{a} }{ R } \left( \frac{ 1 }{ T_{1} } - \frac{ 1 }{ T_{2} } \right) \\
\\
\log_{10} \frac{ k_{1} }{ k_{2} } &= \frac{ E_{a} }{ 2.303R } \left( \frac{ 1 }{ T_{1} } - \frac{ 1 }{ T_{2} } \right)
\end{split}
\] This is called the integral form of Arrhenius equation.

Now, differentiating the log equation wrt T, \[
\begin{split}
\ln k &= \ln A - \frac{ E_{a} }{ RT } \\
\frac{ d(\ln k) }{ dT } &= -\frac{ E_{a} }{ R } \left( -\frac{ 1 }{ T^{2} } \right) \\
\frac{ d(\ln k) }{ dT } &= \frac{ E_{a} }{ RT^{2} }
\end{split}
\]

The reaction which has greater \(E_{a}\) will be more sensitive towards
temp. variation.

Rate constant of reaction increases more sharply for a reaction at lower
temp. than at higher temp. for same temp. rise.

\begin{figure}[H]
\centering
\includegraphics[max width=0.9\linewidth,keepaspectratio]{images/Pasted image 20240308180406.png}

\end{figure}

\paragraph{Why Rate Constant becomes 2x to 3x for every 10K rise in
temp.}\label{why-rate-constant-becomes-2x-to-3x-for-every-10k-rise-in-temp.}

\hyperref[maxwell-boltzmann-distribution]{00 KTG \& Thermodynamics\#Maxwell Boltzmann Distribution}

Note that the average KE does not change much, but the fraction of
molecules having more energy than \(E_{a}\) is increased.

\begin{figure}[H]
\centering
\includegraphics[max width=0.9\linewidth,keepaspectratio]{images/Pasted image 20240308181008.png}

\end{figure}

\subsection{Reversible Reactions}\label{reversible-reactions}

\[\mathrm{ A <=>[E_{af}, k_{f}][E_{ab}, k_{b}] B }\]

At eqilibrium,
\[\mathrm{ k_{eq} = \frac{ [B]_{eq} }{ [A]_{eq} } = \frac{ k_{f} }{ k_{b} } }\]

\begin{figure}[H]
\centering
\includegraphics[max width=0.9\linewidth,keepaspectratio]{images/Pasted image 20240308181253.png}

\end{figure}

From the graph, we can see that, \[\Delta H_{r} = E_{af} - E_{ab}\]

Now, \[
\begin{split}
k_{eq} &= \frac{ A_{f}e^{ -E_{af}/RT } }{ A_{b}e^{ -E_{ab}/RT } } \\
&= \frac{ A_{f} }{ A_{b} } e^{ -(E_{af} - E_{ab})/RT } \\
&= \frac{ A_{f} }{ A_{b} } e^{ -\Delta H_{r}/RT } \\
\ln k_{eq} &= \ln \frac{ A_{f} }{ A_{b} } - \frac{ \Delta H_{r} }{ RT } \\
\frac{ d(\ln k_{eq}) }{ dT } &= \frac{ \Delta H_{r} }{ RT^{2} }
\end{split}
\] This is the differential form of
\hyperref[vant-hoff-equation]{Von\textquotesingle t Hoff Equation.}

And its integral form will be, \[
\begin{split}
\ln \frac{ k_{eq 1} }{ k_{eq 2} } &= \frac{ \Delta H_{r} }{ R } \left( \frac{ 1 }{ T_{1}^{2} } - \frac{ 1 }{ T_{2}^{2} } \right) \\
\\
\log_{10} \frac{ k_{eq 1} }{ k_{eq 2} } &= \frac{ \Delta H_{r} }{ 2.303R } \left( \frac{ 1 }{ T_{1}^{2} } - \frac{ 1 }{ T_{2}^{2} } \right) 
\end{split}
\]

As temp. increases Rate constant increases always. Irrespective of
whether reaction if endothermic or exothermic.

But, as temp. increases equilibrium constant increases for endothermic
and decreases for exothermic.

\newpage

\chapter{Effect of Catalyst and Surface
Area}\label{effect-of-catalyst-and-surface-area}

\etocsettocstyle{\textbf{Chapter Contents}\par\rule{\linewidth}{0.5pt}}{\par\rule{\linewidth}{0.5pt}}
\localtableofcontents

\noindent {}

\subsection{Effect of Catalyst}\label{effect-of-catalyst}

Catalyst is added into the reaction mixture in small amount which
without being consumed alters the mechanism of reaction and takes the
reaction through alternate path involving lower activation energy and
hence reaction becomes faster.

Catalyst does not effect, - Equilibrium constant - Thermodynamic
parameters

Catalyst decreases the value of \(E_{a}\) both forward and backward
reaction by the same amount and hence \(k_{f}\) and \(k_{b}\) are
increased by same factor.

Catalyst cannot turn a non spontaneous reaction into a spontaneous one.

\begin{figure}[H]
\centering
\includegraphics[max width=0.9\linewidth,keepaspectratio]{images/Pasted image 20240308182821.png}

\end{figure}

For catalysed path, \[
\begin{split}
k_{f}' &= A_{f}e^{ -(E_{af}-x)/RT } \\
k_{b}' &= A_{b}e^{ -(E_{ab}-x)/RT } \\
k_{eq}' &= \frac{ A_{f} }{ A_{b} } e^{ -(E_{ab} - E_{af})/RT } \\
k_{eq}' &= \frac{ A_{f} }{ A_{b} } e^{ -\Delta H_{r}/RT } \\
&= k_{eq}
\end{split}
\]

\subsection{Effect of Surface Area}\label{effect-of-surface-area}

Only applies to reactions which occur through adsorption on solid
surface.

As surface area increases, rate of reaction increases.

A solid in powdered or porous form has high surface area and hence high
rate of reaction.

\newpage

\chapter{Mechanism and Order from
Mechanism}\label{mechanism-and-order-from-mechanism}

\etocsettocstyle{\textbf{Chapter Contents}\par\rule{\linewidth}{0.5pt}}{\par\rule{\linewidth}{0.5pt}}
\localtableofcontents

\noindent {}

\subsection{Mechanism of Reaction}\label{mechanism-of-reaction}

The various steps involved in conversion of reactants into products in a
chemical reaction.

\paragraph{Molecularity}\label{molecularity}

It is the no. of reactant molecules that form TS and finally convert
into product.

It is defined for a step of a complex reaction or defined for an
elementary reaction.

It is usually 1, 2 or rarely 3 but not found to be greater than 3 as
probability of 4 or more molecules colliding is very low.

It cannot be fractional, -ve or zero.

Molecularity of 1 means \emph{unimolecular} reaction, 2 means
\emph{bimolecular} reaction and 3 means \emph{trimolecular} reaction.

For an elementary reaction, molecularity is equal to order of reaction
which is equal to sum of stoichiometric coefficients of reactants.

This indicates that zero order reaction is a complex reaction.

\subsubsection{Types of Reaction}\label{types-of-reaction}

There are two types of reaction based on mechanism,

\paragraph{Elementary}\label{elementary}

Which complete in one step. No intermediate is involved.

The given reaction or step is \textbf{Rate Determining Step (RDS).} It
is the slowest step in a reaction.

\begin{figure}[H]
\centering
\includegraphics[max width=0.9\linewidth,keepaspectratio]{images/Pasted image 20240308203736.png}

\end{figure}

\paragraph{Complex}\label{complex}

Which completes in 2 or more steps. Slowest step is RDS. Involves at
least one intermediate.

\begin{figure}[H]
\centering
\includegraphics[max width=0.9\linewidth,keepaspectratio]{images/Pasted image 20240308204922.png}

\end{figure}

\subsection{Order from given
Mechanism}\label{order-from-given-mechanism}

\paragraph{Elementary Reaction}\label{elementary-reaction}

Given step (reaction) is RDS.

\[\mathrm{ A + 2B -> P }\] \[\mathrm{ Rate = k[A][B]^{2} }\] And thus
order is 1 + 2 = 3.

\paragraph{More than one step and one of them is
Slow}\label{more-than-one-step-and-one-of-them-is-slow}

The slow step is RDS.

We write rate law of the slow step taking it to be elementary. The order
comes from this rate law.

\begin{figure}[H]
\centering
\includegraphics[max width=0.9\linewidth,keepaspectratio]{images/Pasted image 20240308205322.png}

\end{figure}

\paragraph{Intermediate in Equilibrium with
Reactants}\label{intermediate-in-equilibrium-with-reactants}

Equilibrium is always fast. Equilibrium involves 2 reactions, forwards
and backwards.

We write rate law from the slow step and if there are intermediates
involved, we remove that as in final rate law conc. of intermediate
cannot come.

The order is obtained from the final rate law.

\begin{figure}[H]
\centering
\includegraphics[max width=0.9\linewidth,keepaspectratio]{images/Pasted image 20240308205827.png}

\end{figure}

\paragraph{Info. of Fast/Slow steps are not
given}\label{info.-of-fastslow-steps-are-not-given}

We apply steady state approximation (SSA). All the steps of the reaction
are assumed to proceed with the same rate.

SSA on intermediate gives, \[\frac{ d[I]_{t} }{ dt } = 0\]

\begin{figure}[H]
\centering
\includegraphics[max width=0.9\linewidth,keepaspectratio]{images/Pasted image 20240308210556.png}

\end{figure}

In photochemical reaction, rate of reaction is proportional to intensity
of light absorbed.

\begin{figure}[H]
\centering
\includegraphics[max width=0.9\linewidth,keepaspectratio]{images/Pasted image 20240308211020.png}

\end{figure}

\newpage

\chapter{Nuclear Chemistry}\label{nuclear-chemistry}

\etocsettocstyle{\textbf{Chapter Contents}\par\rule{\linewidth}{0.5pt}}{\par\rule{\linewidth}{0.5pt}}
\localtableofcontents

\noindent {}

aka \textbf{Radioactivity}

Radioactivity is the spontaneous decay of nuclei of some isotopes of
elements to emit \(\alpha,\beta,\upgamma\) rays.

Radioactive decay follows first order kinetics but Arrhenius equation is
not applicable as rate of radioactive decay is independent of temp.,
free or combined form (e.g.~U or \(\mathrm{ UF_{6} }\))

In nuclear reaction the reacting element is changed. It does not involve
e but rather nuclei.

Rate of radioactive decay, \[\mathrm{ Rate \propto N_{t} }\] where
\(N_{t}\) is the no. of nuclei present at time t.

\[
\begin{split}
-\frac{ dN_{t} }{ dt } &\propto N_{t} \\
-\frac{ dN_{t} }{ dt } &= \lambda N_{t} \\
\end{split}
\]

\(\lambda\) is called \textbf{decay constant} and it has unit
\(s ^{-1}\), \(- dN_{t} /dt\) is called \textbf{rate of decay or
radioactivity or activity.}

Since no. of nuclei is equal to the no. of atoms, we can write,
\[- \frac{ dN_{t} }{ dt } = \lambda N_{t} = \lambda \frac{ W }{ M }N_{A}\]
where, \(W \to\) mass of radioactive substance \(M \to\) molar mass
\(N_{A} \to\) Avogadro no.

The SI unit of rate of decay or activity is \textbf{dps
i.e.~disintegration per second.} But the most commonly used unit is
\textbf{Curie.} \[\mathrm{ 1 Ci = 3.7 \times 10^{10} dps }\]

\textbf{Specific Activity} is the activity per unit mass.

No.~of nuclei after time t, \[
\begin{split}
- \frac{ dN }{ dt } &= \lambda N \\
\int \frac{ dN }{ N } &= \lambda \int dt \\
N_{t} &= N_{o} e^{ -\lambda t }  
\end{split}
\]

Decay constant,
\[\lambda = \frac{ 1 }{ t } \ln \frac{ N_{o} }{ N_{t} }\]

Half life,
\[t_{1/2} = \frac{ \ln 2 }{ \lambda } = \frac{ 0.693 }{ \lambda }\]
Average life, \[t_{avg} = \frac{ 1 }{ \Lambda }\]

No.~of nucleoids remaining after x half lives,
\[= \frac{ N_{o} }{ 2^{x} }\]

\textbf{Parallel Decay:}
\hyperref[parallel-or-competing-first-order-reaction]{Parallel First
Order Reaction}

\begin{figure}[H]
\centering
\includegraphics[max width=0.9\linewidth,keepaspectratio]{images/Pasted image 20240308213232.png}

\end{figure}

\subsubsection{Radioactive Equilibrium}\label{radioactive-equilibrium}

Observed in sequential decay.

\[\mathrm{ A ->[\lambda_{1}] B ->[\lambda_{2}] C ->[\lambda_{3}] D \dots  }\]

A is the parent element which is radioactive. B, C and D are
intermediates.

At radioactive equilibrium, rate of formation of B is equal to rate of
decay of B. This is SSA.

\[\mathrm{ \lambda_{1}N_{A} = \lambda_{2}N_{B} }\] This is the equation
for radioactive equilibrium.

\begin{figure}[H]
\centering
\includegraphics[max width=0.9\linewidth,keepaspectratio]{images/Pasted image 20240308215331.png}

\end{figure}

\subsubsection{Symbols of Some Important Nuclear
Particles}\label{symbols-of-some-important-nuclear-particles}

\[
\begin{split}
{}_{2}He^{4} &: \alpha-particle \\
\\
{}_{-1}e^{0} &: \beta-particle \\
{}_{+1}e^{0} &: Positron \\
\\
{}_{1}H^{1} &: Proton \\
{}_{1}H^{2} &: Deutron \\
{}_{1}H^{3} &: Triton \\
\\
\upgamma\ or h\nu &: \upgamma-radiation
\end{split}
\]

\textbf{\(\alpha\) decay:}
\[\mathrm{ {}_{Z}X^{A} -> {}_{Z-2}Y^{A-4} + {}_{2}He^{4} }\]

\textbf{\(\beta\) decay:}
\[\mathrm{ {}_{Z}X^{A} -> {}_{Z+1}P^{A} + {}_{-1}e^{0} }\]

\subsubsection{Applications of
Radioactivity}\label{applications-of-radioactivity}

\paragraph{Rock dating (or Helium
dating)}\label{rock-dating-or-helium-dating}

To determine age of rocks/minerals/ores.

The basis is \(\alpha\) particle decay.

He dating assumes that all \(\alpha\) particles produced in the decay
are trapped in the ore as He gas.

\begin{figure}[H]
\centering
\includegraphics[max width=0.9\linewidth,keepaspectratio]{images/Pasted image 20240308220317.png}

\end{figure}

\paragraph{Carbon dating}\label{carbon-dating}

To determine age of fossils, wood, artefacts. Mainly used to determine
age of dead trees/animals.

The N in the upper atmosphere captures neutron and forms radioactive
carbon, \(\mathrm{ {}_{6}C^{14} }\). This C has half life of its decay
of nearly 5770 years.

This radioactive C is diffused into the lower atmosphere and then goes
into the bodies of living organisms.

A radioactive equilibrium is established in the body. Rate of absorption
of \(\mathrm{ C^{14} }\) = Rate of decay of \(\mathrm{ C^{14} }\).

There is a constant conc. of \(\mathrm{ C^{14} }\) in the body of alive
body. After the death of the body, the absorption stops and the decay
continues.

\[\mathrm{ {}_{6}C^{14} -> {}_{7}N^{14} + {}_{-1}e^{0} }\]

The age comes out to be,
\[t_{age} = \frac{ 1 }{ \lambda } \ln \frac{ [C^{14}]_{o} }{ [C^{14}]_{t} }\]
where \(\mathrm{ [C^{14}]_{t} }\) is the conc. in dead animal and
\(\mathrm{ [C^{14}]_{o} }\) is the conc. in the alive animal which is
approximated on the basis of currently living animals.

\newpage

\backmatter
\end{document}
