%%
% Copyright (c) 2017 - 2025, Pascal Wagler;
% Copyright (c) 2014 - 2025, John MacFarlane
%
% All rights reserved.
%
% Redistribution and use in source and binary forms, with or without
% modification, are permitted provided that the following conditions
% are met:
%
% - Redistributions of source code must retain the above copyright
% notice, this list of conditions and the following disclaimer.
%
% - Redistributions in binary form must reproduce the above copyright
% notice, this list of conditions and the following disclaimer in the
% documentation and/or other materials provided with the distribution.
%
% - Neither the name of John MacFarlane nor the names of other
% contributors may be used to endorse or promote products derived
% from this software without specific prior written permission.
%
% THIS SOFTWARE IS PROVIDED BY THE COPYRIGHT HOLDERS AND CONTRIBUTORS
% "AS IS" AND ANY EXPRESS OR IMPLIED WARRANTIES, INCLUDING, BUT NOT
% LIMITED TO, THE IMPLIED WARRANTIES OF MERCHANTABILITY AND FITNESS
% FOR A PARTICULAR PURPOSE ARE DISCLAIMED. IN NO EVENT SHALL THE
% COPYRIGHT OWNER OR CONTRIBUTORS BE LIABLE FOR ANY DIRECT, INDIRECT,
% INCIDENTAL, SPECIAL, EXEMPLARY, OR CONSEQUENTIAL DAMAGES (INCLUDING,
% BUT NOT LIMITED TO, PROCUREMENT OF SUBSTITUTE GOODS OR SERVICES;
% LOSS OF USE, DATA, OR PROFITS; OR BUSINESS INTERRUPTION) HOWEVER
% CAUSED AND ON ANY THEORY OF LIABILITY, WHETHER IN CONTRACT, STRICT
% LIABILITY, OR TORT (INCLUDING NEGLIGENCE OR OTHERWISE) ARISING IN
% ANY WAY OUT OF THE USE OF THIS SOFTWARE, EVEN IF ADVISED OF THE
% POSSIBILITY OF SUCH DAMAGE.
%%

%%
% This is the Eisvogel pandoc LaTeX template.
%
% For usage information and examples visit the official GitHub page:
% https://github.com/Wandmalfarbe/pandoc-latex-template
%%
% Options for packages loaded elsewhere
\PassOptionsToPackage{unicode}{hyperref}
\PassOptionsToPackage{hyphens}{url}
\PassOptionsToPackage{dvipsnames,svgnames,x11names,table}{xcolor}
\documentclass[
  paper=a4,
  openany,
  oneside  ,captions=tableheading
]{scrbook}
\usepackage{xcolor}
\usepackage[top=1.5cm, bottom=2.5cm, left=3cm, right=3cm, includeheadfoot, heightrounded]{geometry}
\usepackage{amsmath,amssymb}

% add backlinks to footnote references, cf. https://tex.stackexchange.com/questions/302266/make-footnote-clickable-both-ways
\usepackage{footnotebackref}
\setcounter{secnumdepth}{5}
\usepackage{iftex}
\ifPDFTeX
  \usepackage[T1]{fontenc}
  \usepackage[utf8]{inputenc}
  \usepackage{textcomp} % provide euro and other symbols
\else % if luatex or xetex
  \usepackage{unicode-math} % this also loads fontspec
  \defaultfontfeatures{Scale=MatchLowercase}
  \defaultfontfeatures[\rmfamily]{Ligatures=TeX,Scale=1}
\fi
\usepackage{lmodern}
\ifPDFTeX\else
  % xetex/luatex font selection
  \setmainfont[]{LMRoman10-Regular}
  \setsansfont[]{Arial}
  \setmonofont[]{LMMono10-Regular}
\fi
% Use upquote if available, for straight quotes in verbatim environments
\IfFileExists{upquote.sty}{\usepackage{upquote}}{}
\IfFileExists{microtype.sty}{% use microtype if available
  \usepackage[]{microtype}
  \UseMicrotypeSet[protrusion]{basicmath} % disable protrusion for tt fonts
}{}

% Use setspace anyway because we change the default line spacing.
% The spacing is changed early to affect the titlepage and the TOC.
\usepackage{setspace}
\setstretch{1.2}
\makeatletter
\@ifundefined{KOMAClassName}{% if non-KOMA class
  \IfFileExists{parskip.sty}{%
    \usepackage{parskip}
  }{% else
    \setlength{\parindent}{0pt}
    \setlength{\parskip}{6pt plus 2pt minus 1pt}}
}{% if KOMA class
  \KOMAoptions{parskip=half}}
\makeatother
\usepackage{graphicx}
\makeatletter
\newsavebox\pandoc@box
\newcommand*\pandocbounded[1]{% scales image to fit in text height/width
  \sbox\pandoc@box{#1}%
  \Gscale@div\@tempa{\textheight}{\dimexpr\ht\pandoc@box+\dp\pandoc@box\relax}%
  \Gscale@div\@tempb{\linewidth}{\wd\pandoc@box}%
  \ifdim\@tempb\p@<\@tempa\p@\let\@tempa\@tempb\fi% select the smaller of both
  \ifdim\@tempa\p@<\p@\scalebox{\@tempa}{\usebox\pandoc@box}%
  \else\usebox{\pandoc@box}%
  \fi%
}
% Set default figure placement to htbp
% Make use of float-package and set default placement for figures to H.
% The option H means 'PUT IT HERE' (as  opposed to the standard h option which means 'You may put it here if you like').
\usepackage{float}
\floatplacement{figure}{H}
\makeatother
\ifLuaTeX
  \usepackage{luacolor}
  \usepackage[soul]{lua-ul}
\else
  \usepackage{soul}
\fi
\setlength{\emergencystretch}{3em} % prevent overfull lines
\providecommand{\tightlist}{%
  \setlength{\itemsep}{0pt}\setlength{\parskip}{0pt}}

\usepackage{caption}
\captionsetup[figure]{labelsep=none, justification=centering}
\usepackage{etoc} % Replaces minitoc
\usepackage[version=4]{mhchem}
\usepackage{amsmath}
\usepackage{amssymb}
\usepackage{mathtools}
\usepackage{gensymb}
\usepackage[export]{adjustbox} % For max width=... in includegraphics
\usepackage{cancel}
% \mtcselectlanguage{english} - Removed
\definecolor{mylinkcolor}{HTML}{07455c}
\definecolor{myurlcolor}{HTML}{07455c}
% --- Chapter Title Styling (KOMA-Script) ---
\renewcommand*\chapterformat{\thechapter.\enskip}
\addtokomafont{chapter}{\centering}
\RedeclareSectionCommand[beforeskip=0pt,afterskip=20pt]{chapter}
% --- Table Styling ---
\rowcolors{2}{RoyalBlue!20}{white}
\renewcommand{\arraystretch}{1.2}
% -------------------------------------------
\usepackage[most]{tcolorbox}
\usepackage{fontawesome5}
\usepackage{xcolor}

% Define custom colors
\definecolor{notecolor}{RGB}{97, 175, 239}   % Blue
\definecolor{tipcolor}{RGB}{20, 196, 255} % Purple
\definecolor{warningcolor}{RGB}{229, 192, 123} % Orange
\definecolor{attentioncolor}{RGB}{224, 108, 117} % Red
\definecolor{analogycolor}{RGB}{152, 195, 121}    % Green

% Generic Pretty Box Style
% #1 = Color name
% #2 = Icon command
% #3 = Title text
\newtcolorbox{prettybox}[3]{
    enhanced,
    colback=#1!5!white,      % Very light background
    colframe=#1,             % Border color
    coltitle=#1!50!black,    % Title text color (darker version of base)
    title={#2\ \ \textbf{#3}},
    fonttitle=\bfseries\large,
    attach boxed title to top left={xshift=5mm, yshift=-3.5mm}, % Adjust for half-in/out
    boxed title style={
        colback=white,
        colframe=white,
        boxrule=0pt,
        top=0pt,
        bottom=0pt,
        left=2pt,
        right=2pt
    },
    top=1.5em, % Space for the floating title
    bottom=1em,
    left=1em,
    right=1em,
    arc=3pt,
    boxrule=1pt,
    drop fuzzy shadow,   
    parbox=false,        
    breakable            
}

% Define specific environments
% usage: \begin{noteblock}{Title} ... \end{noteblock}

\newenvironment{noteblock}[1]
  {\begin{prettybox}{notecolor}{\faInfoCircle}{#1}}
  {\end{prettybox}}

\newenvironment{tipblock}[1]
  {\begin{prettybox}{tipcolor}{\faLightbulb}{#1}}
  {\end{prettybox}}

\newenvironment{warningblock}[1]
  {\begin{prettybox}{warningcolor}{\faExclamationTriangle}{#1}}
  {\end{prettybox}}

\newenvironment{attentionblock}[1]
  {\begin{prettybox}{attentioncolor}{\faExclamationCircle}{#1}}
  {\end{prettybox}}

\newenvironment{cautionblock}[1]
  {\begin{prettybox}{attentioncolor}{\faRadiation}{#1}}
  {\end{prettybox}}

\newenvironment{importantblock}[1]
  {\begin{prettybox}{attentioncolor}{\faStar}{#1}}
  {\end{prettybox}}

% Analogy Block
\newenvironment{analogyblock}[1]
  {\begin{prettybox}{analogycolor}{\faShapes}{#1}}
  {\end{prettybox}}
\usepackage{bookmark}
\IfFileExists{xurl.sty}{\usepackage{xurl}}{} % add URL line breaks if available
\urlstyle{same}
\definecolor{default-linkcolor}{HTML}{A50000}
\definecolor{default-filecolor}{HTML}{A50000}
\definecolor{default-citecolor}{HTML}{4077C0}
\definecolor{default-urlcolor}{HTML}{4077C0}

\hypersetup{
  pdftitle={Electrostatics},
  pdfauthor={Anonymous},
  colorlinks=true,
  linkcolor={mylinkcolor},
  filecolor={default-filecolor},
  citecolor={default-citecolor},
  urlcolor={myurlcolor},
  breaklinks=true,
  pdfcreator={LaTeX via pandoc with the Eisvogel template}}

\title{Electrostatics}
\usepackage{etoolbox}
\makeatletter
\providecommand{\subtitle}[1]{% add subtitle to \maketitle
  \apptocmd{\@title}{\par {\large #1 \par}}{}{}
}
\makeatother
\subtitle{Personal Notes and References}
\author{Anonymous}
\date{}


%
% for the background color of the title page
%

%
% break urls
%
\PassOptionsToPackage{hyphens}{url}

%
% When using babel or polyglossia with biblatex, loading csquotes is recommended
% to ensure that quoted texts are typeset according to the rules of your main language.
%
\usepackage{csquotes}

%
% captions
%
\definecolor{caption-color}{HTML}{777777}
\usepackage[font={stretch=1.2}, textfont={color=caption-color}, position=top, skip=4mm, labelfont=bf, singlelinecheck=false, justification=raggedright]{caption}
\setcapindent{0em}

%
% blockquote
%
\definecolor{blockquote-border}{RGB}{221,221,221}
\definecolor{blockquote-text}{RGB}{119,119,119}
\usepackage{mdframed}
\newmdenv[rightline=false,bottomline=false,topline=false,linewidth=3pt,linecolor=blockquote-border,skipabove=\parskip]{customblockquote}
\renewenvironment{quote}{\begin{customblockquote}\list{}{\rightmargin=0em\leftmargin=0em}%
\item\relax\color{blockquote-text}\ignorespaces}{\unskip\unskip\endlist\end{customblockquote}}

%
% Source Sans Pro as the default font family
% Source Code Pro for monospace text
%
% 'default' option sets the default
% font family to Source Sans Pro, not \sfdefault.
%
% Note that the font has been officially renamed to `Source Sans 3`, and
% the version provided by the `sourcesanspro` package is slightly outdated.
% You can install the newer version locally and use it, for example, with
% `mainfont: "Source Sans 3"` in the YAML metadata (requires XeTeX or LuaTeX).
%
\ifnum 0\ifxetex 1\fi\ifluatex 1\fi=0 % if pdftex
    \usepackage[default]{sourcesanspro}
  \usepackage{sourcecodepro}
  \else % if not pdftex
    \fi

%
% heading color
%
\definecolor{heading-color}{RGB}{40,40,40}
% By default, the KOMA-Script classes will typeset sectioning headings in
% sans-serif. Use the normal body font for headings.
\addtokomafont{disposition}{\normalfont\color{heading-color}\bfseries}

%
% variables for title, author and date
%
\usepackage{titling}
\title{Electrostatics}
\author{Anonymous}
\date{}

%
% tables
%

%
% remove paragraph indentation
%
\setlength{\parindent}{0pt}
\setlength{\parskip}{6pt plus 2pt minus 1pt}
\setlength{\emergencystretch}{3em}  % prevent overfull lines

%
%
% Listings
%
%


%
% header and footer
%
\usepackage[headsepline,footsepline]{scrlayer-scrpage}

\newpairofpagestyles{eisvogel-header-footer}{
  \clearpairofpagestyles
  \ihead*{Electrostatics}
  \chead*{}
  \ohead*{}
  \ifoot*{Anonymous}
  \cfoot*{}
  \ofoot*{\thepage}
  \addtokomafont{pageheadfoot}{\upshape}
}
\pagestyle{eisvogel-header-footer}

\deftripstyle{ChapterStyle}{}{}{}{}{\pagemark}{}
\renewcommand*{\chapterpagestyle}{ChapterStyle}


%
% Define watermark
%

\begin{document}


\frontmatter
% don't generate the default title
% \maketitle



\begin{titlepage}
    \newgeometry{left=2.5cm,right=2.5cm,top=2cm,bottom=2cm}
    \vspace*{1cm}
    
    
    \vspace{3cm}
    
    \centering
    {\fontsize{50}{60}\selectfont \bfseries Electrostatics \par}
    \vspace{1cm}
    {\fontsize{20}{30}\selectfont Personal Notes and References \par}
    
    \vfill
    
    {\fontsize{18}{22}\selectfont Anonymous \par}
    \vspace{0.5cm}
    {\large January 12, 2026 \par}
    
    \vspace{3cm}
    \restoregeometry
\end{titlepage}

% Initialize MiniTOC (Removed, using etoc)

{
\setcounter{tocdepth}{3}
\tableofcontents
}
\mainmatter
\chapter{Electrostatics}\label{electrostatics}

\etocsettocstyle{\textbf{Chapter Contents}\par\rule{\linewidth}{0.5pt}}{\par\rule{\linewidth}{0.5pt}}
\localtableofcontents

\noindent {} Part of physics which deals with study of charge.

\subsection{Charge}\label{charge}

Property of object due to which it can create electric and magnetic
effects around it.

It is property of an object due to which it can experience electric and
magnetic effects.

Its unit is Coulomb (C). The CGS unit is esu (electrostatic unit).
\(1\ C = 3 \times 10^{9}\ esu\). It has dimensions \(\symup{ [IT] }\).

\subsubsection{Properties of Charge}\label{properties-of-charge}

It is a scalar quantity and is of two types, +ve and -ve. -ve charge
means excess of e. +ve charge means deficiency of e.

\textbf{Charge on electron:} \(e^{-} = -1.6 \times 10^{-19}\) C
\textbf{Charge on proton:} \(p^{+} = 1.6 \times 10^{-19}\) C

Charge is quantised and not continuous. I.e., it can only exist as
integral multiples of fundamental charge, \(e^{-}\). \[Q = ne\]

Like point charges repel and unlike attract. These forces are called
\emph{electrostatic forces.}

Charge is conserved. Net charge cannot be created or destroyed.

If a charge is at rest, it can create electric effects around itself and
can experience electric forces. It cannot create or experience magnetic
forces.

If a charge is moving with constant velocity, then it will create and
experience electric and magnetic effects.

If a charge is moving with variable velocity, then it will emit
electromagnetic waves (radiations).

Charge is relativistically invariant, i.e.~value of charge doesn't
depend on its velocity.

\subsection{Materials}\label{materials}

\paragraph{Conductor}\label{conductor}

Which transfer electricity/energy with less resistance. Free e are
responsible for the transfer.

Charge on conductor stays on surface.

\pandocbounded{\includegraphics[keepaspectratio]{images/Pasted image 20231018192700.png}}

\paragraph{Non-conductors}\label{non-conductors}

aka \textbf{Insulators or Dielectric.}
\hyperref[dielectrics]{Dielectrics}

They do not have free e as to remove the outermost e a lot of E is
required.

They are bad conductors of energy and electricity. It creates a very
high resistance.

The charge in conductor stays in its volume.

\subsection{Charging}\label{charging}

\paragraph{Thermionic Effect}\label{thermionic-effect}

The KE of e is directly proportional to temp..

As we increase temp. KE increases the e get removed and the metal gets
+vely charged.

\pandocbounded{\includegraphics[keepaspectratio]{images/Pasted image 20231018193516.png}}

\paragraph{Photoelectric Effect}\label{photoelectric-effect}

A high E photon transfers its E to the outer e of the metal and removed
it thus +vely charging it.

\pandocbounded{\includegraphics[keepaspectratio]{images/Pasted image 20231018193711.png}}

\paragraph{Field Emission}\label{field-emission}

We bring a highly +vely charged object near the metal which attracts the
-vely charged e and thus charge the metal +vely.

\pandocbounded{\includegraphics[keepaspectratio]{images/Pasted image 20231018193906.png}}

\paragraph{Friction}\label{friction}

Rubbing two materials together transfers e to the material which has
more \hyperref[electron-affinity]{e affinity.}

It will charge both of the materials.

\pandocbounded{\includegraphics[keepaspectratio]{images/Pasted image 20231018194101.png}}

\paragraph{Conduction}\label{conduction}

If we connect two conductors one of which has charge, the charge gets
evenly distributed and the neutral also gets charged.

\pandocbounded{\includegraphics[keepaspectratio]{images/Pasted image 20231018194503.png}}

\paragraph{Induction}\label{induction}

Due to presence of charges near a conductor, charge on conductor may
redistribute itself. This phenomenon is known as induction.

\pandocbounded{\includegraphics[keepaspectratio]{images/Pasted image 20231018194904.png}}

To charge using induction,

\pandocbounded{\includegraphics[keepaspectratio]{images/Pasted image 20231018195305.png}}

\section{Solid Angle}\label{solid-angle}

We know that angle is defined as, \[\theta = \frac{ \text{Arc} }{ R }\]

\pandocbounded{\includegraphics[keepaspectratio]{images/Pasted image 20231212160157.png}}

Similarly, in 3D, solid angle is defined as,
\[\ohm = \frac{ \text{Area} }{ R^{2} }\] where area is the section of
spherical surface which is intersected by the cone whose apex solid
angle is \(\Omega\).

Solid angle has unit \textbf{steradian.}

\pandocbounded{\includegraphics[keepaspectratio]{images/Pasted image 20231212160254.png}}

For complete sphere,
\[\ohm = \frac{ 4\pi R^{2} }{ R^{2} } = 4\pi \ \text{steradian}\]

Like complete circle, for which angle is \(2\pi\).

\textbf{Flux per unit solid angle/steradian,}
\[\frac{ \phi }{ 4 \pi \varepsilon_{o} } = kQ\] It has unit
\(\symup{ N m^{2} steradian^{-1} }\). The dimensions will remain the
same because steradian, like radian does not have dimensions.

\subsubsection{Relation Between Solid and Plane
Angle}\label{relation-between-solid-and-plane-angle}

Relation between Radian and Steradian.

\[\ohm = \frac{ A }{ R^{2} } = 2\pi(1-\cos \theta)\] where \(\theta\) is
semi apex angle.

\pandocbounded{\includegraphics[keepaspectratio]{images/Pasted image 20231212160826.png}}

Solid angle or flux per unit solid angle is useful in calculating the
flux through a disc.

Flux through a solid angle \(\Omega\),
\[\phi = \frac{ Q \Omega }{ 4\pi \varepsilon_{o} } = kQ \Omega = kQ\ 2\pi (1 - \cos \theta)\]

\pandocbounded{\includegraphics[keepaspectratio]{images/Pasted image 20231212161457.png}}

\newpage

\chapter{Coulomb's Law}\label{coulombs-law}

\etocsettocstyle{\textbf{Chapter Contents}\par\rule{\linewidth}{0.5pt}}{\par\rule{\linewidth}{0.5pt}}
\localtableofcontents

\noindent {} It is an experimental law and it tells the electrostatic
force between two point charges. It follows inverse square law.

\emph{The force of attraction or repulsion between 2 charges is directly
proportional to the product of their magnitudes and inversely
proportional to the square of distance between them.}

\[f = \frac{ kq_{1}q_{2} }{ r^{2} } = \frac{ 1 }{ 4\pi\varepsilon_{o} } \frac{ q_{1}q_{2} }{ r^{2} }\]
where, \(\varepsilon_{o}\) is the permittivity of free space.

\[\frac{1}{4\pi\varepsilon_{o}} = k = 9 \times 10^{9} \symup{\ N m^{2} C^{-2} }\]

This force acts along the line joining the two charges (central force)
and is conservative and thus we can de fine potential energy for it.

\pandocbounded{\includegraphics[keepaspectratio]{images/Pasted image 20231018214056.png}}

\subsubsection{Permittivity}\label{permittivity}

\(\varepsilon_{o}\) is the permittivity of free space/vacuum/air.
\[\varepsilon_{o} = \frac{ 1 }{ 4\pi k } = 8.85 \times 10^{-12} \symup{ \ \frac{C^{2}}{N-m^{2}} }\]

\(\varepsilon\) is the permittivity of a medium and it equal to or more
than \(\varepsilon_{o}\). The more \(\varepsilon\) is the less is the
force applied.

This is because when a charge is placed in a medium, the molecules
around the charge are polarised which decrease the effective charge and
thus the force between the charged particles.

\pandocbounded{\includegraphics[keepaspectratio]{images/Pasted image 20231018215301.png}}

The media which are easily polarised have high value of \(\varepsilon\).

\paragraph{Relative Permittivity}\label{relative-permittivity}

\(\varepsilon_{r}\) is the relative permittivity of medium wrt free
space and is a unitless and dimensionless quantity.

\[\varepsilon_{r} = \frac{ \varepsilon  }{ \varepsilon_{o} }\]

Relative permittivity is always greater than or equal to 1.

For air, \(\varepsilon_{r} = 1\) For water, \(\varepsilon_{r} = 81\) For
metal, \(\varepsilon_{r} \to \infty\)

The value of k for any other medium,
\[k_{m} = \frac{ k }{ \varepsilon_{r} }\] And thus the force in any
other medium becomes \(1 /\varepsilon_{r}\) times its value in vacuum.
\[F_{m} = \frac{ F }{ \varepsilon_{r} }\]

\subsection{Vector Form}\label{vector-form}

\[\vec{f} = \frac{ kq_{1}q_{2} }{ r^{2} }\ \hat{r}\]
\[\vec{f} = \frac{ kq_{1}q_{2} }{ r^{3} }\ \vec{r}\] Here
\(q_{1},q_{2}\) are with sign and \(\vec{r}\) is vector whose tail is at
the charge due to which force is to be found.

\pandocbounded{\includegraphics[keepaspectratio]{images/Pasted image 20231019185547.png}}

\subsection{Principle of
Superposition}\label{principle-of-superposition}

The net force on a charge due to multiple charges is the vector sum of
all the individual forces.

\[\vec{f}_{net} = \vec{f}_{1} + \vec{f}_{2}\]

\pandocbounded{\includegraphics[keepaspectratio]{images/Pasted image 20231019190621.png}}

\subsection{Charge Densities}\label{charge-densities}

\begin{itemize}
\tightlist
\item
  Linear: \(\lambda = \displaystyle \frac{Q}{l}\)
\item
  Areal: \(\sigma = \displaystyle \frac{Q}{A}\)
\item
  Volumetric: \(\rho = \displaystyle \frac{Q}{V}\)
\end{itemize}

For extended object, - Make a small element and consider it a point
charge. - Find this charge in terms of l, A or V. - Find the force using
coulomb's law on this element. - Now integrate it over the whole body to
find the net force.

\section{Examples}\label{examples}

\pandocbounded{\includegraphics[keepaspectratio]{images/Pasted image 20231019191108.png}}

\pandocbounded{\includegraphics[keepaspectratio]{images/Pasted image 20231019191907.png}}

3 Charge Systems,
\hyperref[simple-harmonic-motion]{Simple Harmonic Motion}
\pandocbounded{\includegraphics[keepaspectratio]{images/Pasted image 20231019193226.png}}
\pandocbounded{\includegraphics[keepaspectratio]{images/Pasted image 20231019193236.png}}

\pandocbounded{\includegraphics[keepaspectratio]{images/Pasted image 20231019193734.png}}
\pandocbounded{\includegraphics[keepaspectratio]{images/Pasted image 20231019193930.png}}

\newpage

\chapter{Electric Field}\label{electric-field}

\etocsettocstyle{\textbf{Chapter Contents}\par\rule{\linewidth}{0.5pt}}{\par\rule{\linewidth}{0.5pt}}
\localtableofcontents

\noindent {} It is the region of space in which a charge particle or
system of charges have their electric effects.

\textbf{Electric Field Intensity (\(\vec{E}\))} represents just how much
the electric effects are. It quantifies the electric effect at a point
and is a vector quantity.

Popularly instead of ``electric field intensity'' we simply say
``Electric Field.''

E at a point P due to a system of charges is defined as electrostatic
force experienced by test charge placed at point P, per unit test
charge. Thus, E does not depend on test charge.

E is also defined as the amount of force experienced by unit +ve charge
(+1 C) at a point.

That is, \[\vec{E} = \frac{ \vec{f}_{e} }{ q_{o} }\] If \(q_{o}\) is +ve
then \(E \parallel f_{e}\). If \(q_{o}\) is -ve then
\(E \upharpoonleft \! \downharpoonright  f_{e}\).

E has Unit is N/C or V/m (Volt per meter).

To change the electric field, we can alter charge distribution or
medium.

The principle of superposition applied to electric field also.

\section{Electric Fields due to Various
Objects}\label{electric-fields-due-to-various-objects}

The density of field lines dictate the strength of electric field.

Constant electric field means that the direction and distance between
the field lines does not change.

\subsection{E due to Point Charge}\label{e-due-to-point-charge}

\[\vec{E} = \frac{ \vec{f}_{e} }{ q_{o} } = \frac{ kQ }{ r^{3} } \vec{r}\]
Here \(Q\) is with sign.

And magnitude is, \[E = \frac{ kQ }{ r^{2} }\] Due to +ve charge, E is
radially outwards and due to -ve charge, it is radially inwards.

E is inversely proportional to (distance from charge)\(^{2}\)
\pandocbounded{\includegraphics[keepaspectratio]{images/Pasted image 20231019204416.png}}

\paragraph{Examples}\label{examples-1}

\pandocbounded{\includegraphics[keepaspectratio]{images/Pasted image 20231019205224.png}}
\pandocbounded{\includegraphics[keepaspectratio]{images/Pasted image 20231019205234.png}}

\pandocbounded{\includegraphics[keepaspectratio]{images/Pasted image 20231019205824.png}}
\pandocbounded{\includegraphics[keepaspectratio]{images/Pasted image 20231019205830.png}}

\subsection{\texorpdfstring{\hyperref[ring-and-disk]{Field due to Ring
\& Disk}}{Field due to Ring \& Disk}}\label{field-due-to-ring-disk}

\subsection{\texorpdfstring{\hyperref[wire]{Field due to
Wire}}{Field due to Wire}}\label{field-due-to-wire}

\subsection{\texorpdfstring{\hyperref[sphere]{Field due to
Sphere}}{Field due to Sphere}}\label{field-due-to-sphere}

\section{Electric Lines of Force
(ELOF)}\label{electric-lines-of-force-elof}

They are imaginary lines or curves, the tangent to which at a point
provides direction of electric field at that point.

They gives pictorial representation of electric field.

\pandocbounded{\includegraphics[keepaspectratio]{images/Pasted image 20231205174142.png}}

\textbf{Line Density:} No.~of ELOF per unit cross sectional area.

\subsubsection{Properties of ELOF}\label{properties-of-elof}

\begin{enumerate}
\def\labelenumi{\arabic{enumi}.}
\item
  They are imaginary lines.
\item
  Electric field will be tangential to ELOF.
\item
  ELOF can only originate form +ve charge or infinity.
\item
  ELOF can only terminate at -ve charge or infinity.
\item
  Two ELOF will never intersect.
\end{enumerate}

\pandocbounded{\includegraphics[keepaspectratio]{images/Pasted image 20231205174636.png}}

\begin{enumerate}
\def\labelenumi{\arabic{enumi}.}
\setcounter{enumi}{5}
\item
  ELOF due to stationary charges will never make a closed loop.

  This is because the line originates from a +ve charge and it cannot
  terminate at it.

  This can also be explained by the fact that if it made loops, then to
  travel a +ve charge on the loop, the work done will be +ve. This
  cannot happen as electric force is conservative and the work done
  should be 0 for 0 displacement.
\end{enumerate}

\pandocbounded{\includegraphics[keepaspectratio]{images/Pasted image 20231205175309.png}}

\begin{enumerate}
\def\labelenumi{\arabic{enumi}.}
\setcounter{enumi}{6}
\item
  ELOF are always from high to low potential.
\item
  Line density at any point is proportional to electric field intensity.
  Thus it is proportional to magnitude of electric field intensity at
  that point.
\end{enumerate}

\subsubsection{ELOF due to Various
Objects}\label{elof-due-to-various-objects}

\textbf{ELOF due to +ve point charge is radially outwards.}

Here we can see that ratio of line density is the same as ratio of
electric field.

\pandocbounded{\includegraphics[keepaspectratio]{images/Pasted image 20231205180159.png}}

\textbf{ELOF due to -ve point charge is radially inwards.}

\pandocbounded{\includegraphics[keepaspectratio]{images/Pasted image 20231205180703.png}}

No of ELOF is directly proportional to charge. In the diagram below,
there are 8 ELOF for 5 \(\mu\)C and 16 for -10 \(\mu\)C.

\pandocbounded{\includegraphics[keepaspectratio]{images/Pasted image 20231205180814.png}}

\textbf{ELOF for dipole,}
\pandocbounded{\includegraphics[keepaspectratio]{images/Pasted image 20231205181514.png}}

\textbf{ELOF for two +ve point charges,}
\pandocbounded{\includegraphics[keepaspectratio]{images/Pasted image 20231205182037.png}}

\newpage

\chapter{Ring and Disk}\label{ring-and-disk}

\etocsettocstyle{\textbf{Chapter Contents}\par\rule{\linewidth}{0.5pt}}{\par\rule{\linewidth}{0.5pt}}
\localtableofcontents

\noindent {} Ring is uniform, i.e.~equal lengths have equal charge.

We define a \(\lambda\) here, \[\lambda = \frac{ Q }{ 2\pi R }\]

\subsubsection{At Centre}\label{at-centre}

We take a small element subtending an angle \(d\theta\) at the centre.

We can see that the field by one element is cancelled out by the element
diametrically opposite to it. Thus the E at centre will be zero.

\[E_{C} = 0\]

\pandocbounded{\includegraphics[keepaspectratio]{images/Pasted image 20231019210551.png}}

\subsubsection{On Axis}\label{on-axis}

We take a small charge \(dQ\), due to which we have \(dE\),
\[dE = \frac{ kdQ }{ (R^{2} + x^{2}) }\] Now this has a vertical and
horizontal component. The vertical is cancelled by the small charge
\(dQ\) diametrically opposite to the first \(dQ\).

Thus, E on axis is along it.

That is, \[
\begin{split}
E &= \int dE \cos \theta \\
&= \int \frac{ kdQ }{ (R^{2} + x^{2}) } . \frac{ x }{ \sqrt{ R^{2} + x^{2} } } \, dx \\
&= \frac{ kx }{ (R^{2} + x^{2})^{3/2} } \int_{0}^{Q} dQ \\
&= \frac{ kQx }{ (R^{2} + x^{2})^{3/2} } 
\end{split}
\]

Thus finally, \[E = \frac{ kQx }{ (R^{2} + x^{2})^{3/2} }\]

\pandocbounded{\includegraphics[keepaspectratio]{images/Pasted image 20231019211856.png}}

Now, if \(x \gg R\), i.e.~very far, \[E_{far} = \frac{kQ}{x^{2}}\]

Again, if \(x \ll R\), i.e.~very near but not at centre,
\[E_{near} = \frac{ kQx }{ R^{3} }\]

\subsubsection{Graph of E vs x for +vely charged
Ring}\label{graph-of-e-vs-x-for-vely-charged-ring}

E is max at \(\displaystyle \frac{R}{\sqrt{ 2 }}\) and is,
\[E = \frac{ 2kQ }{ 3\sqrt{ 3 }R^{2} }\]

\pandocbounded{\includegraphics[keepaspectratio]{images/Pasted image 20231019212400.png}}

\section{E due to Half Ring}\label{e-due-to-half-ring}

Here, \[\lambda = \frac{ Q }{ \pi R }\]

\subsubsection{At Centre}\label{at-centre-1}

We take an element subtending an angle \(d\theta\), \(\theta\) from the
diameter. This has charge, \[dQ = \frac{ Q }{ \pi }d\theta\]

Due to this electric field is, \[dE = \frac{ kQd\theta }{ \pi R^{2} }\]
Which has two components, \[dE_{x} = dE\cos \theta\]
\[dE_{y} = dE\sin \theta\]

Now the x components cancel each other out. Thus, \[
\begin{split}
E &= \int_{0}^{\pi} \frac{ kQ }{ \pi R^{2} } \sin \theta \, d\theta \\
&= \frac{ 2kQ }{ \pi R^{2} }
\end{split}
\]

In terms of \(\lambda\), \[E = \frac{ 2k\lambda }{ R }\]

\pandocbounded{\includegraphics[keepaspectratio]{images/Pasted image 20231020191012.png}}

\subsubsection{On Axis}\label{on-axis-1}

Here too we take elements as before. This element is at
\((R\cos \theta, R\sin \theta, 0)\)

We have to find E at point \((0,0,z)\), i.e.
\[\vec{r} = -R\cos \theta \hat{i} - R\sin \theta \hat{j} + z\hat{k}\]
\[r = \sqrt{ R^{2} + z^{2} }\]

Now, dE due to dQ is,
\[d\vec{E} = \frac{ k(\lambda Rd\theta)(-R\cos \theta \hat{i} - R\sin \theta \hat{j} + z\hat{k}) }{ (R^{2} + z^{2})^{3/2} }\]
The net E will be, \[
\begin{split}
d\vec{E} &= \int dE_{x} + \int dE_{y} + \int dE_{z} \\
\\
\int dE_{x} &= \frac{ -k\lambda R^{2} }{ (R^{2} + z^{2})^{3/2} } \int_{0}^{\pi} \cos \theta \, d\theta = 0 \\
\\
\int dE_{y} &= \frac{ -k\lambda R^{2} }{ (R^{2} + z^{2})^{3/2} } \int_{0}^{\pi} \sin \theta \, d\theta = \frac{ -2k\lambda R^{2} }{ (R^{2} + z^{2})^{3/2} } \\
\\
\int dE_{y} &= \frac{ -k\lambda Rz }{ (R^{2} + z^{2})^{3/2} } \int_{0}^{\pi} \, d\theta = \frac{ -k\lambda z R \pi }{ (R^{2} + z^{2})^{3/2} } \\
\end{split}
\] Thus the net E comes out to be,
\[\vec{E} = \frac{ -2k\lambda R^{2} }{ (R^{2} + z^{2})^{3/2} } \hat{j} + \frac{ k\lambda Rz\pi }{ (R^{2} + z^{2})^{3/2} } \hat{k}\]

\pandocbounded{\includegraphics[keepaspectratio]{images/Pasted image 20231020192619.png}}

Similarly we can find for quarter ring.

\subsection{E due to Arc}\label{e-due-to-arc}

\pandocbounded{\includegraphics[keepaspectratio]{images/Pasted image 20230109153240.png}}

1/4th: \[E = \frac{ k\lambda }{ r }\sqrt{ 2 }\]

Half: \[E = \frac{ 2k\lambda }{ r }\]

\section{E due to Disk}\label{e-due-to-disk}

Here we define \(\sigma\), \[\sigma = \frac{ Q }{ \pi R^{2} }\]

We will take elemental rings of width dr, r distance from the centre.
This elemental ring has charge, \[dQ = \frac{ 2Qrdr }{ R^{2} }\]

Due to this dE is, \[dE = \frac{ kdQx }{ (r^{2} + x^{2})^{3/2} }\] The
net E is, \[
\begin{split}
E &= \int_{0}^{R} \frac{ k_{2}Qrdr }{ R^{2}(r^{2} + x^{2})^{3/2} } \\
&= \frac{ 2kQx }{ R^{2} } \int_{0}^{R} \frac{ rdr }{ (r^{2} + x^{2})^{3/2} } \\
&= \frac{ 2kQ }{ R^{2} } \left( 1 - \frac{ x }{ \sqrt{ R^{2} + x^{2} } } \right)
\end{split}
\] Thus finally,
\[E = \frac{ 2kQ }{ R^{2} } \left( 1 - \frac{ x }{ \sqrt{ R^{2} + x^{2} } } \right)\]
\[E = \frac{ \sigma }{ 2\varepsilon_{o} } \left( 1 - \frac{ x }{ \sqrt{ R^{2} + x^{2} } } \right)\]

We can also write this as,
\[E = \frac{ 2kQ }{ R^{2} } (1 - \cos \theta)\]
\[E = \frac{ \sigma }{ 2\varepsilon_{o} } (1 - \cos \theta)\]

Now, if \(x \to 0\), \[E_{C} = \frac{\sigma}{2\varepsilon_{o}}\] And if
\(x \to \infty\), \[E_{\infty} \to 0\]

\pandocbounded{\includegraphics[keepaspectratio]{images/Pasted image 20231020194352.png}}

\subsubsection{Graph of E vs x}\label{graph-of-e-vs-x}

\pandocbounded{\includegraphics[keepaspectratio]{images/Pasted image 20231020194402.png}}

\subsection{Examples}\label{examples-2}

For cavity questions, we will use the same method as in
\hyperref[centre-of-masscom-of-objects-with-cavity]{Centre of Mass\#COM of Objects with Cavity}

\pandocbounded{\includegraphics[keepaspectratio]{images/Pasted image 20231020195105.png}}
\pandocbounded{\includegraphics[keepaspectratio]{images/Pasted image 20231020195114.png}}

\newpage

\chapter{Wire}\label{wire}

\etocsettocstyle{\textbf{Chapter Contents}\par\rule{\linewidth}{0.5pt}}{\par\rule{\linewidth}{0.5pt}}
\localtableofcontents

\noindent {} Here, \[\lambda = \frac{ Q }{ l }\]

\subsubsection{Finitely Long Wire}\label{finitely-long-wire}

The top end makes angle \(\theta_{1}\) and the bottom makes
\(\theta_{2}\).

\[
\begin{split}
E_{\perp} &= \frac{ k\lambda }{ r }(\sin \theta_{2} + \sin \theta_{2}) \\
E_{\parallel} &= \frac{ k\lambda }{ r }|\cos \theta_{2} - \cos \theta_{2}|
\end{split}
\] The direction of parallel one is towards the part which is smaller
because the other larger part will have more E than the smaller part.

\pandocbounded{\includegraphics[keepaspectratio]{images/Pasted image 20231020195654.png}}

\paragraph{Derivation}\label{derivation}

We take an element of length dx, x distance from the foot of
perpendicular. This is at an angle \(\theta\) from the perpendicular.
This has charge, \[dQ = \lambda dx\]

Also, \[
\begin{split}
x &= r \tan \theta \\
dx &= r\sec^{2}\theta d\theta 
\end{split}
\]

Now, due to dQ, the dE at P is,
\[dE = \frac{ k\lambda dx }{ (r\sec \theta)^{2} }\]

Along x axis, \[
\begin{split}
E_{x} &= \int dE\cos \theta \\
&= \int \frac{ k\lambda r\sec^{2}\theta d\theta \cos \theta }{ r^{2} \sec^{2} \theta } \\
&= \frac{ k\lambda }{ r } \int_{-\theta_{2}}^{\theta_{1}} \cos \theta \, d\theta \\
&= \frac{ k\lambda }{ r } (\sin \theta_{1} + \sin \theta_{2}) 
\end{split}
\]

Along y axis, similarly, \[
\begin{split}
E_{y} &= \int dE\sin \theta \\
&= \frac{ k\lambda }{ r } |\cos \theta_{1} - \cos \theta_{2}|
\end{split}
\]

\pandocbounded{\includegraphics[keepaspectratio]{images/Pasted image 20231020200516.png}}

\subsubsection{Semi Infinite Wire}\label{semi-infinite-wire}

When one end is at \(\infty\) but the other is not.

We will find E at perpendicular point from the finite end point.

Here, since the wire if infinite, \(\theta_{1} \to 90^{\circ}\) and
\(\theta_{2} = 0\). Thus, \[
\begin{split}
E_{\perp} &= \frac{ k\lambda }{ r } \\
E_{\parallel} &= \frac{ k\lambda }{ r } \\
E_{net} &= \frac{ \sqrt{ 2 }k\lambda }{ r } 
\end{split}
\]

\pandocbounded{\includegraphics[keepaspectratio]{images/Pasted image 20231020201354.png}}

\subsubsection{Infinitely Long Wire}\label{infinitely-long-wire}

Both the ends are at \(\infty\) and hence
\(\theta_{1}, \theta_{2} \to 90\)

Thus, \[E_{\perp} = \frac{ 2k \lambda }{ r } = E_{net}\]
\[E_{\parallel} = 0\]

E is directed perpendicularly outward from the wire.

\[\vec{E} = \frac{ 2k\lambda }{ r^{2} } (\overline{OP})\]

\pandocbounded{\includegraphics[keepaspectratio]{images/Pasted image 20231020201920.png}}

\subsubsection{Examples}\label{examples-3}

\pandocbounded{\includegraphics[keepaspectratio]{images/Pasted image 20231020203153.png}}

\#important
\pandocbounded{\includegraphics[keepaspectratio]{images/Pasted image 20231020210451.png}}
\pandocbounded{\includegraphics[keepaspectratio]{images/Pasted image 20231020210934.png}}

\subsubsection{Infinitely Large Sheet}\label{infinitely-large-sheet}

\(W \to \infty\) and \(l \to \infty\).

\[E = \frac{ \sigma }{ 2 \varepsilon_{o} }\] which is independent of x
and thus unirform.

\pandocbounded{\includegraphics[keepaspectratio]{images/Pasted image 20231020211247.png}}

\newpage

\chapter{Sphere}\label{sphere}

\etocsettocstyle{\textbf{Chapter Contents}\par\rule{\linewidth}{0.5pt}}{\par\rule{\linewidth}{0.5pt}}
\localtableofcontents

\noindent {} \#\#\# Uniformly Charged Hollow Sphere \[
\begin{split}
E_{in} &= 0 \\
E_\text{surface} &= \frac{ kQ }{ R^{2} } \\
E_{out} &= \frac{ kQ }{ r^{2} } \\
\end{split}
\]

\pandocbounded{\includegraphics[keepaspectratio]{images/Pasted image 20231020213101.png}}

Thus for a point outside the shell, it behaves like a point charge,
\pandocbounded{\includegraphics[keepaspectratio]{images/Pasted image 20231020214146.png}}

\paragraph{Graph}\label{graph}

\pandocbounded{\includegraphics[keepaspectratio]{images/Pasted image 20231020213112.png}}

\section{Uniformly Charged Solid
Sphere}\label{uniformly-charged-solid-sphere}

\[
\begin{split}
E_{in} &= \frac{ kQr }{ R^{3} } = \frac{ \rho r }{ 3\varepsilon_{o} } \\
E_\text{surface} &= \frac{ kQ }{ R^{2} } \\
E_{out} &= \frac{ kQ }{ r^{2} } \\
\end{split}
\]

\pandocbounded{\includegraphics[keepaspectratio]{images/Pasted image 20231020214608.png}}

Here too, for outside points, uniform solid sphere behaves like a point
charge placed at its centre.

\subsubsection{Derivation}\label{derivation-1}

Here, \[\rho = \frac{ 3Q }{ 4\pi R^{3} }\]

\paragraph{Outside}\label{outside}

Here we take elemental shells of dx thickness and of radius x. This
shell has charge, \[dQ = \frac{ 2Qx^{2}dx }{ R^{2} }\]

Now this elemental shell has field at P, \[dE = \frac{ kdQ }{ r^{2} }\]
The net field is,
\[E = \int \frac{ kdQ }{ r^{2} } = \frac{ kQ }{ r^{2} }\]

\pandocbounded{\includegraphics[keepaspectratio]{images/Pasted image 20231020215245.png}}

\paragraph{Inside}\label{inside}

We take an elemental shell of radius x and thickness dx.

Now we find that the E at P is only due to a solid sphere of radius r as
the outside shells don't contribute to the field.

\[E = \int_{0}^{R} dE = \int_{0}^{r} dE + \cancelto{ 0 }{ \int_{r}^{R} dE }\]

\pandocbounded{\includegraphics[keepaspectratio]{images/Pasted image 20231020220228.png}}

Now this inside sphere has charge, \[
\begin{split}
Q &= \rho V \\
&= \frac{ 3Q }{ 4\pi R^{3} } . \frac{ 4\pi r^{3} }{ 3 } \\
&= \frac{ Qr^{3} }{ R^{3} }
\end{split}
\]

Thus the field is,
\[E_{in} = \frac{ kQ_{in} }{ r^{2} } = \frac{ kQr }{ R^{3} } = \frac{ \rho r }{ 3\varepsilon_{o} }\]

\subsubsection{Graph}\label{graph-1}

\pandocbounded{\includegraphics[keepaspectratio]{images/Pasted image 20231020220449.png}}

\subsubsection{Inside a Cavity}\label{inside-a-cavity}

The E is constant inside a cavity if it is not concentric.

If the cavity is concentric then the field is zero.

\[E_{cav} = \frac{ \rho }{ 3\varepsilon_{o} } (\overline{OC})\] Since
this is independent of P, the field inside the cavity is uniform.

\pandocbounded{\includegraphics[keepaspectratio]{images/Pasted image 20231021164308.png}}
\pandocbounded{\includegraphics[keepaspectratio]{images/Pasted image 20231021164109.png}}

\newpage

\chapter{Electrostatic Potential}\label{electrostatic-potential}

\etocsettocstyle{\textbf{Chapter Contents}\par\rule{\linewidth}{0.5pt}}{\par\rule{\linewidth}{0.5pt}}
\localtableofcontents

\noindent {} Work done by external force in bringing a unit +ve charge
very slowly from infinity to a point P is called the Potential at point
P. \[or\] Negative of work done by electric field to bring unit +ve
charge from \(\infty\) to point P.

\[\frac{ (W_{ext})_{\infty \to P} }{ q_{o} } = \frac{ -(W_{ele})_{\infty \to P} }{ q_{o} } = V_{P}\]

\[V_{p} = \frac{ (W_{ele})_{P \to \infty} }{ q_{o} }\]

It is a scalar quantity and can be +ve, -ve or zero. Unit is J
C\(^{-1}\) = V.

We consider electric potential to be zero at infinity.

Now, \[
\begin{split}
V &= \frac{ -W_{ele} }{ q_{o} } \\
&= \frac{ - \int_{\infty}^{P} q_{o}\vec{E} \, d\vec{r} }{ q_{o} } \\
&= \int_{P}^{\infty} \vec{E} \, d\vec{r} \\
\end{split}
\] This integral is called \emph{line integral of electric field.}

\section{Electric Potentials due to Various
Objects}\label{electric-potentials-due-to-various-objects}

\subsection{V due to Point Charge}\label{v-due-to-point-charge}

\[
\begin{split}
V &= \frac{ (W_{ele})_{P \to \infty} }{ q_{o} } \\
&= \frac{ \int_{P}^{\infty} q_{o}E \, dr }{ q_{o} } \\
&= \int_{r}^{\infty} \frac{ kQ }{ x^{2} } \, dx \\
&= \frac{ kQ }{ r }
\end{split}
\] Thus, \[V = \frac{ kQ }{ r }\]

Here, Q should be put with sign. And thus for +ve Q, V will be +ve, -ve
for -ve and 0 for 0.

\subsubsection{Graph}\label{graph-2}

\pandocbounded{\includegraphics[keepaspectratio]{images/Pasted image 20231021171844.png}}

\subsection{Potential due to Ring}\label{potential-due-to-ring}

\hyperref[ring-and-disk]{E due to Ring and Disk}

\paragraph{At Centre}\label{at-centre-2}

We take a small charge dQ on the ring. Due to this we have potential,
\[dV = \frac{ kdQ }{ R }\]

Thus, \[V_{C} = \frac{ kQ }{ R }\] This is valid even if the ring has
non uniformly distributed charge.

This kQ/R is the potential for the centre of any arc as well.

\pandocbounded{\includegraphics[keepaspectratio]{images/Pasted image 20231021172427.png}}

\subsubsection{On axis}\label{on-axis-2}

We again take a small charge dQ on the ring. Due to this the potential
will be, \[dV = \frac{ kdQ }{ \sqrt{ R^{2} + x^{2} } }\]

Now every charge on the ring has equal distance from the point and thus,
\[V = \frac{ kQ }{ \sqrt{ R^{2} + x^{2} } }\]

This is valid even if the charge is non uniformly distributed.

\pandocbounded{\includegraphics[keepaspectratio]{images/Pasted image 20231021172959.png}}

\subsubsection{Graph}\label{graph-3}

\pandocbounded{\includegraphics[keepaspectratio]{images/Pasted image 20231021173206.png}}

\subsection{Potential due to Uniform
Disk}\label{potential-due-to-uniform-disk}

Here, \[\sigma = \frac{ Q }{ \pi R^{2} }\]

We take elemental rings of radius r and thickness dr. This ring has
charge, \[dQ = 2\sigma \pi r dr = \frac{ 2Qrdr }{ R^{2} }\]

The potential due to this ring will be, \[
\begin{split}
dV &= \frac{ kdQ }{ \sqrt{ r^{2} + x^{2} } } \\
&= \frac{ 2k Q }{ R^{2} }\frac{ rdr }{ \sqrt{ r^{2} + x^{2} } }
\end{split}
\]

The net potential is, \[
\begin{split}
V &= \frac{ 2kQ }{ R^{2} } \int_{0}^{R} \frac{ rdr }{ \sqrt{ r^{2} + x^{2} } } \\
&= \frac{ 2kQ }{ R^{2} } (\sqrt{ R^{2} + x^{2} } - x) 
\end{split}
\] After rationalisation,
\[V = \frac{ 2kQ }{ \sqrt{ R^{2} + x^{2} } + x }\]

In terms of \(\sigma\)
\[V_{P} = \frac{ \sigma }{ 2\varepsilon_{o} }(\sqrt{ R^{2}+x^{2} } - x)\]

For a very near point, \[V_{near} = \frac{ 2kQ }{ R }\] For a very far
point, \[V_{far} \to 0\]

\pandocbounded{\includegraphics[keepaspectratio]{images/Pasted image 20231021174252.png}}

\subsubsection{Graph}\label{graph-4}

\pandocbounded{\includegraphics[keepaspectratio]{images/Pasted image 20231021174415.png}}

\subsection{Potential Due to Sphere}\label{potential-due-to-sphere}

For an outside point, sphere behaves like a point charge with all the
charge at its centre.

\subsubsection{Hollow Sphere}\label{hollow-sphere}

Here, \[\sigma = \frac{ Q }{ 4\pi R^{2} }\]

Potentials will be, \[
\begin{split}
V_{in} &= \frac{ kQ }{ R } = V_\text{centre} \\
V_\text{surface} &= \frac{ kQ }{ R } \\
V_\text{out} &= \frac{ kQ }{ r } \\
\end{split}
\] Since there is no electric field inside,
\(V_\text{surface} = V_{in}\)

\pandocbounded{\includegraphics[keepaspectratio]{images/Pasted image 20231021180518.png}}

\(V_\text{centre} = kQ /R\) even if the charge is non uniformly
distributed.

\paragraph{Graph}\label{graph-5}

\pandocbounded{\includegraphics[keepaspectratio]{images/Pasted image 20231024200859.png}}

\subsubsection{Solid Sphere}\label{solid-sphere}

\[
\begin{split}
V_{in} &= \frac{ kQ }{ 2R^{3} } (3R^{2} - r^{2}) \\
V_\text{centre} &= \frac{3}{2} \frac{ KQ }{ R } \\
V_\text{surface} &= \frac{ kQ }{ R } \\
V_\text{out} &= \frac{ kQ }{ r } 
\end{split}
\]

\pandocbounded{\includegraphics[keepaspectratio]{images/Pasted image 20231024201204.png}}

\paragraph{Derivation for Inside}\label{derivation-for-inside}

We take two elemental shells of thickness dx and radius, one when x
\textless{} r and one when x \textgreater{} r.

These shells will have charge, \[dQ = \frac{ 3Qx^{2}dx }{ R^{3} }\]

Thus potential, \[
\begin{split}
V &= \int_{0}^{R} dV = \int_{0}^{r} dV + \int_{r}^{R} dV \\
&= \int \frac{ kdQ }{ r } + \int \frac{ kdQ }{ x } \\
&= \frac{ k 3Q }{ R^{3}r } \int_{0}^{r} x^{2} \, dx + \frac{ k 3Q }{ R^{2} } \int_{r}^{R} x \, dx \\
&= \frac{ kQr^{2} }{ R^{3} } + \frac{ 3kQ }{ 2R } - \frac{ 3kQr^{2} }{ 2R^{3} } 
\end{split}
\]

\pandocbounded{\includegraphics[keepaspectratio]{images/Pasted image 20231024201216.png}}

\paragraph{Graph}\label{graph-6}

\pandocbounded{\includegraphics[keepaspectratio]{images/Pasted image 20231024201927.png}}

\newpage

\chapter{Potential Difference}\label{potential-difference}

\etocsettocstyle{\textbf{Chapter Contents}\par\rule{\linewidth}{0.5pt}}{\par\rule{\linewidth}{0.5pt}}
\localtableofcontents

\noindent {} \[
\begin{split}
V_{BA} &= V_{B} - V_{A} \\
&= \frac{ (W_{ext})_{A \to B} }{q_{o}} \\
&= \frac{ -(W_{ele})_{A \to B} }{ q_{o} } \\
&= \frac{ (W_{ele})_{B \to A} }{ q_{o} }
\end{split}
\]

Potential difference between points B and A is defined as work done by
external agent against electrostatic forces to bring unit +ve charge
from A to B, slowly.

Thus, \[
\begin{split}
\Delta V &= - \int_{A}^{B} \vec{E} \, d\vec{r}\\
&= \int_{B}^{A} \vec{E} \, d\vec{r} \\
\end{split}
\]

Or, \[dV = -\vec{E}.d\vec{r}\]

\pandocbounded{\includegraphics[keepaspectratio]{images/Pasted image 20231024204046.png}}

\subsubsection{Relation between Field and
Potential}\label{relation-between-field-and-potential}

\hyperref[potential-energype-and-force-relation]{Potential Energy\#PE and Force Relation}

From the above result, we can write,
\[\Delta V = -\int_{x_{i}}^{x_{f}} E_{x} \, dx - \int_{y_{i}}^{y_{f}} E_{y} \, dy - \int_{z_{i}}^{z_{f}} E_{z} \, dz\]

And to find field,
\[E = -\frac{ \partial V }{ \partial x } \hat{i} - \frac{ \partial V }{ \partial y } \hat{j} - \frac{ \partial V }{ \partial z }\hat{k}\]
This can be written as,
\[E = - \left( \frac{ \partial  }{ \partial x } \hat{i} + \frac{ \partial  }{ \partial y } \hat{j} + \frac{ \partial  }{ \partial z } \hat{k} \right) V\]
\[E = - \nabla V\]

Similar to
\hyperref[energypotential-energy-from-force]{Energy\#Potential Energy from Force}

\subsection{PD in Uniform Field}\label{pd-in-uniform-field}

\paragraph{Perpendicular to the Field}\label{perpendicular-to-the-field}

If displacement is perpendicular to the field, then the PD will be zero.

In a perpendicular plane to the electric field, the potentials at every
point is the same and thus the PD is zero. This plane is thus called
\textbf{Equipotential Surface.} And there is no work done in moving a
charge on this plane.

\pandocbounded{\includegraphics[keepaspectratio]{images/Pasted image 20231024205453.png}}

\paragraph{Along the Field}\label{along-the-field}

If displacement is along the field, the potential decreases.
\[V_{B} - V_{A} = \frac{ q_{o}E d\cos 0 }{ q_{o} }\]
\[V_{B} - V_{A} = Ed\] Thus field is from lower to higher potential.

Thus, \[E = \frac{ \Delta V }{ d }\]

\pandocbounded{\includegraphics[keepaspectratio]{images/Pasted image 20231024205723.png}}

\paragraph{Any Direction}\label{any-direction}

\[
\begin{split}
V_{B} - V_{A} &= \frac{ q_{o}E d \cos \theta }{ q_{o} } \\
&= Ed\cos \theta \\
&= \vec{E} . \vec{d} \\
&= \vec{E} d_{\parallel}
\end{split}
\]

\pandocbounded{\includegraphics[keepaspectratio]{images/Pasted image 20231024205945.png}}

\section{PD due to Various Objects}\label{pd-due-to-various-objects}

\subsubsection{Point Charge}\label{point-charge}

\[V_{B} - V_{A} = kQ \left( \frac{ 1 }{ r_{B} } - \frac{ 1 }{ r_{A} } \right)\]

\pandocbounded{\includegraphics[keepaspectratio]{images/Pasted image 20231024210638.png}}

\subsubsection{Infinite Non Conducting
Sheet}\label{infinite-non-conducting-sheet}

\hyperref[wireinfinitely-large-sheet]{Wire\#Infinitely Large Sheet}

\[V_{A} - V_{C} = \frac{ \sigma d }{ 2 \varepsilon_{o} }\]

\pandocbounded{\includegraphics[keepaspectratio]{images/Pasted image 20231024210941.png}}

If the direction is not parallel to the field,
\[V_{A} - V_{D} = \frac{ \sigma x\cos \theta }{ 2 \varepsilon_{o} }\]

\pandocbounded{\includegraphics[keepaspectratio]{images/Pasted image 20231024211032.png}}

\subsubsection{Infinitely Long Wire (Line
Charge)}\label{infinitely-long-wire-line-charge}

If displacement is perpendicular to the field, i.e.~parallel to the
wire, the potential difference will be zero.

\pandocbounded{\includegraphics[keepaspectratio]{images/Pasted image 20231127190540.png}}

If charge is displaced perpendicular to the wire, \[
\begin{split}
dW &= \frac{ 2k\lambda q_{o} }{ r } dr \cos 0^{\circ} \\
W &= 2k\lambda q_{o} \int_{r_{A}}^{r_{B}} \frac{ 1 }{ r } \, dr
\end{split}
\] Thus giving,
\[V_{A} - V_{B} = 2k\lambda \log\frac{ r_{B} }{ r_{A} }\]

\pandocbounded{\includegraphics[keepaspectratio]{images/Pasted image 20231127191059.png}}

\section{Equipotential Surface}\label{equipotential-surface}

Surface, real or imaginary, on which all points have the same potential.

The potential difference between any two points will be zero. Thus, work
done by electric field to move charge on equipotential surface is zero.

If we move the charge slowly, the work done by external agent will also
be zero.

\pandocbounded{\includegraphics[keepaspectratio]{images/Pasted image 20231127191251.png}}

If work done is zero, there are two possibilities, 1. Electric field may
be zero at equipotential surface. 2. Electric field may be perpendicular
to the equipotential surface.

\pandocbounded{\includegraphics[keepaspectratio]{images/Pasted image 20231127191834.png}}

\emph{Two distinct equipotential surfaces can never intersect.} This is
because, 1. We cannot define two potentials at the same point. 2.
Electric field cannot have two directions at the same point.

\pandocbounded{\includegraphics[keepaspectratio]{images/Pasted image 20231127192035.png}}

Note that if the equipotential surface is the same, it can intersect
itself. At the point of intersection, the field is zero.

\pandocbounded{\includegraphics[keepaspectratio]{images/Pasted image 20231127192305.png}}

\subsubsection{Eq. Surface due to Point
Charge}\label{eq.-surface-due-to-point-charge}

For a point charge equipotential surface is a spherical shell centred at
the charge of radius r.

\pandocbounded{\includegraphics[keepaspectratio]{images/Pasted image 20231127193007.png}}

The equipotential surfaces are concentric spherical shells.

\pandocbounded{\includegraphics[keepaspectratio]{images/Pasted image 20231127193502.png}}

\subsubsection{Eq. Surface due to Infinitely Large
Sheet}\label{eq.-surface-due-to-infinitely-large-sheet}

Equipotential surface is plane parallel to the sheet.

If electric field is uniform, the equipotential surfaces are planar.

\pandocbounded{\includegraphics[keepaspectratio]{images/Pasted image 20231127193724.png}}

An example,
\pandocbounded{\includegraphics[keepaspectratio]{images/Pasted image 20231127194931.png}}

\subsubsection{Eq. Surface due to Infinitely Long
Wire}\label{eq.-surface-due-to-infinitely-long-wire}

Here the equipotential surface is hollow cylinder with the wire as axis.

\pandocbounded{\includegraphics[keepaspectratio]{images/Pasted image 20231127195959.png}}
\pandocbounded{\includegraphics[keepaspectratio]{images/Pasted image 20231127200027.png}}

\pandocbounded{\includegraphics[keepaspectratio]{images/Pasted image 20231127200053.png}}

Also, here, \[V_{2} - V_{1} = 2k\lambda \log \frac{ r_{2} }{ r_{1} } \]

\newpage

\chapter{Electric Potential Energy}\label{electric-potential-energy}

\etocsettocstyle{\textbf{Chapter Contents}\par\rule{\linewidth}{0.5pt}}{\par\rule{\linewidth}{0.5pt}}
\localtableofcontents

\noindent {} It is the work done by external agent, or -ve of work done
by electric field to bring the charges slowly from infinity to form the
system.

\[U = [W_{ext}(\infty \to P)]_{\Delta K = 0}\] \[U = V_{sys} . q\]

Multiplying potential with charge gives potential energy. Similar to
force and field.
\pandocbounded{\includegraphics[keepaspectratio]{images/Pasted image 20231127202309.png}}

It is a scalar and its SI unit is Joule. Another unit is eV called
\emph{Electron Volt,} which is, \[1\ eV = 1.6 \times 10^{-19}\ J\]

We consider potential energy at infinity to be zero.

Electric potential energy can be used by any of, or all of, the charges
in the system.

We can write V due to any of the charges and q of any of the charges.

\pandocbounded{\includegraphics[keepaspectratio]{images/Pasted image 20231127204038.png}}

\section{EPE of Various Systems}\label{epe-of-various-systems}

One interaction between two charges will have one potential energy.

\subsubsection{EPE of 2 Point Charges}\label{epe-of-2-point-charges}

\[U = \frac{ kQ_{1}Q_{2} }{ r }\] where \(Q_{1},Q_{2}\) are with signs.

\pandocbounded{\includegraphics[keepaspectratio]{images/Pasted image 20231127204330.png}}

\subsubsection{EPE due to System of
Charges}\label{epe-due-to-system-of-charges}

Charges here are considered point charges.

\[
\begin{split}
U_{sys} &= [W_{ext}(\infty \to \text{Here})]_{\Delta K = 0} \\
&= (U_{12} + U_{13} + U_{14} + \dots + U_{1n}) \\
&\ + (U_{23} + U_{24} + \dots + U_{2n}) \\
&\ + (U_{34} + \dots + U_{4n}) \\
&\ \ \ \vdots\\
&\ + U_{(n-1)n} 
\end{split}
\] This has \((n-1) + (n-2) + (n-3) + \dots + 1\) terms, i.e.~
\[\symup{ no. of terms } = \frac{ n(n-1) }{ 2 }\]

Each interaction between two charges contributes a term. Thus n charges
will have \({}^{n}C_{2}\) terms.

If there are 3 point charges, then there will be 3 terms.
\[U = U_{12} + U_{13} + U_{23}\]

\pandocbounded{\includegraphics[keepaspectratio]{images/Pasted image 20231127213632.png}}

\paragraph{3 Point Charge System}\label{point-charge-system}

EPE for an equilateral triangle made of equal point charges will be,
\[U = \frac{ 2kQ^{2} }{ l }\]

\pandocbounded{\includegraphics[keepaspectratio]{images/Pasted image 20231127213913.png}}

\subsubsection{Symmetric Systems}\label{symmetric-systems}

For a symmetric system,
\[U = \frac{ n }{ 2 } \times \text{PE of any one charge with all other charges}\]
where n is the number of charges.

Division by 2 is to remove repetitions.

\pandocbounded{\includegraphics[keepaspectratio]{images/Pasted image 20231203135519.png}}
\pandocbounded{\includegraphics[keepaspectratio]{images/Pasted image 20231203141010.png}}

\subsection{Examples}\label{examples-4}

\pandocbounded{\includegraphics[keepaspectratio]{images/Pasted image 20231127205444.png}}

\pandocbounded{\includegraphics[keepaspectratio]{images/Pasted image 20231127205948.png}}

\pandocbounded{\includegraphics[keepaspectratio]{images/Pasted image 20231127210745.png}}
\pandocbounded{\includegraphics[keepaspectratio]{images/Pasted image 20231127210754.png}}

\pandocbounded{\includegraphics[keepaspectratio]{images/Pasted image 20231127211455.png}}
\pandocbounded{\includegraphics[keepaspectratio]{images/Pasted image 20231127211559.png}}

\pandocbounded{\includegraphics[keepaspectratio]{images/Pasted image 20231127212316.png}}

\pandocbounded{\includegraphics[keepaspectratio]{images/Pasted image 20231203134820.png}}

\pandocbounded{\includegraphics[keepaspectratio]{images/Pasted image 20231203142546.png}}
\pandocbounded{\includegraphics[keepaspectratio]{images/Pasted image 20231203142919.png}}

\newpage

\chapter{Self Energy}\label{self-energy}

\etocsettocstyle{\textbf{Chapter Contents}\par\rule{\linewidth}{0.5pt}}{\par\rule{\linewidth}{0.5pt}}
\localtableofcontents

\noindent {} It is the potential energy stored in a body by virtue of
interaction between various charge particles inside the body.

It is a kind of electrostatic potential energy.

Total electrostatic potential energy of a system is given by,
\[U_\text{total} = U_\text{self energy} + U_\text{interaction}\]
\(U_\text{self}\) is to be written of all charges.

\subsubsection{Point Charge}\label{point-charge-1}

Since there is no other charge and hence no other interaction,
\[U_\text{self energy} = 0\]

\subsection{Hollow Sphere}\label{hollow-sphere-1}

Uniformly Charged

\[
\begin{split}
U_\text{system} &= W_{ext}(\infty \to \text{here}) \\
dW &= Vdq \\
&= \frac{ kq }{ R }dq \\
W &= \int_{0}^{Q} \frac{ kq }{ R } \, dq \\
&= \frac{ kQ^{2} }{ 2R }  
\end{split}
\] This is the self energy of hollow sphere.
\[U_\text{self} = \frac{ kQ^{2} }{ 2R }\] Thus increasing the radius
decreases self energy.

\pandocbounded{\includegraphics[keepaspectratio]{images/Pasted image 20231203144420.png}}

\subsection{Solid Sphere}\label{solid-sphere-1}

Uniformly charged.

We will bring charges from \(\infty\) to their place and form the
sphere.

Now, while making the big sphere, a smaller sphere of radius of r is
formed. This sphere has charge, \[q = \frac{ Q }{ R^{3} }r^{3}\] where Q
is the charge to be of the big sphere.

Now, a small charge dq will be, \[dq = \frac{ Q }{ R^{3} } 3r^{2}dr\]

To bring this small charge dq from \(\infty\) to the sphere and
spreading it uniformly, increasing the radius by dr, work done will be,
\[
\begin{split}
dW &= Vdq \\
&= \frac{ kq }{ r }dq \\
&= \frac{ kQr^{3} }{ R^{3}r }. \frac{ Q }{ R^{3} }. 3r^{2}dr \\
&= \frac{ 3kQ^{2} }{ R^{6} } r^{4}dr
\end{split}
\]

This gives, \[
\begin{split}
W_{ext} &= \int_{0}^{R}  \, dW \\
&= \frac{ 3kQ^{2} }{ R^{6} } \int_{0}^{R} r^{4} \, dr \\
&= \frac{ 3kQ^{2} }{ 5R } 
\end{split}
\]

Which is the self energy of solid sphere,
\[U_\text{self} = \frac{ 3kQ^{2} }{ 5R } = \frac{ 0.6kQ^{2} }{ R }\]

\pandocbounded{\includegraphics[keepaspectratio]{images/Pasted image 20231203151129.png}}

For example,
\pandocbounded{\includegraphics[keepaspectratio]{images/Pasted image 20231203151419.png}}

\section{Energy Density}\label{energy-density}

Wherever there is a conservative field, there will be energy
corresponding to that field. This energy per unit volume is Energy
Density.

We can define gravitational, electrical and magnetic ED.

\[u = \frac{ dU }{ dV }\] And its unit will be \(\symup{ J m^{-3} }\).

Electric energy density is, in any medium,
\[\frac{ dU }{ dV } = \frac{1}{2} \varepsilon E^{2}\] And in air/vacuum,
\[\frac{ dU }{ dV } = \frac{1}{2} \varepsilon_{o} E^{2}\]

Here we can see that, \[\frac{ dU }{ dV } \propto E^{2}\]

\pandocbounded{\includegraphics[keepaspectratio]{images/Pasted image 20231203152918.png}}

\pandocbounded{\includegraphics[keepaspectratio]{images/Pasted image 20231204162053.png}}

For a system,
\[\int \frac{1}{2} \varepsilon_{o}E^{2}  \, dV = U_\text{epe}\] This
energy due to electric field is the same as electric potential energy
(epe or self energy) of the system.

\subsubsection{Energy due to Field of Hollow
Sphere}\label{energy-due-to-field-of-hollow-sphere}

Since there is no field inside the sphere, there is no energy density.
Thus no energy is present inside the sphere.

For outside, we make an elemental shell of thickness dr and radius r.
This elemental shell has the same magnitude of field at every point.
Thus it will have same energy density,
\[\frac{ dU }{ dV } = \frac{ 1 }{ 2 }\varepsilon_{o} \frac{ k^{2}Q^{2} }{ r^{4} }\]
Giving, \[
\begin{split}
dU &= \frac{ 1 }{ 2 }\varepsilon_{o} \frac{ k^{2}Q^{2} }{ r^{4} } 4\pi r^{2} dr \\
dU &= \frac{ kQ^{2} }{ 2r^{2} } dr 
\end{split}
\]

Integrating this will give energy due to hollow sphere, \[
\begin{split}
U &= \frac{ kQ^{2} }{ 2 } \int_{R}^{\infty} \frac{ 1 }{ r^{2} } \, dr \\
&= \frac{ kQ^{2} }{ 2R } 
\end{split}
\] This is same as the self energy of hollow sphere.

This means that self energy is the energy which is stored in the
electric field due to an object. I.e. electric potential is stored in
the form of electric field.

\pandocbounded{\includegraphics[keepaspectratio]{images/Pasted image 20231204162952.png}}

\subsubsection{Energy due to Field of Solid
Sphere}\label{energy-due-to-field-of-solid-sphere}

Outside the sphere, since the field is the same, the energy will also be
the same as that of hollow sphere.
\[U_\text{out} = \frac{ kQ^{2} }{ 2R }\]

Now for inside, we take an elemental sphere of radius r and thickness
dr. This will have energy density,
\[\frac{ dU }{ dV } = \frac{ 1 }{ 2 }\frac{ \varepsilon_{o} k^{2} Q^{2} r^{2} }{ R^{6} }\]
Giving, \[
\begin{split}
dU &= \frac{ \varepsilon_{o} k^{2} Q^{2} r^{2} }{ 2R^{6} }. 4\pi r^{2} dr \\
U &= \frac{ kQ^{2} }{ 2R^{6} } \int_{0}^{R} r^{4}dr \\
&= \frac{ kQ^{2} }{ 10R }
\end{split}
\] Thus we get, \[U_{in} = \frac{ kQ^{2} }{ 10R }\]

The total energy will be,
\[U = \frac{ kQ^{2} }{ 10R } + \frac{ kQ^{2} }{ 2R } = \frac{ 3kQ^{2} }{ 5R }\]
Which is the same as the self energy of solid sphere.

\pandocbounded{\includegraphics[keepaspectratio]{images/Pasted image 20231204164221.png}}

\newpage

\chapter{Electric Dipole}\label{electric-dipole}

\etocsettocstyle{\textbf{Chapter Contents}\par\rule{\linewidth}{0.5pt}}{\par\rule{\linewidth}{0.5pt}}
\localtableofcontents

\noindent {} Electric dipole is an electrically neutral system
consisting of two equal an opposite charges placed at a very small
separation.

Separation between charges should be small in relation to the separation
between dipole and the point of interest.

\subsubsection{Electric Dipole Moment}\label{electric-dipole-moment}

The vector joining -q to +q is \(\vec{l}\).

Dipole moment is represented by \(\vec{p}\) and is defined as,
\[\vec{p} = q . \vec{l}\] Its unit is \(\symup{ C m }\) or debye and
dimensions are \(\symup{ [ITL] }\) Also,
\(1 \text{debye} = 3.3 \times 10^{-30} \symup{\ C m }\)

Its direction is opposite to that of electric field, i.e.~from the -ve
to the +ve charge.

\pandocbounded{\includegraphics[keepaspectratio]{images/Pasted image 20231204165635.png}}

We can divide a charge to form dipole.

\pandocbounded{\includegraphics[keepaspectratio]{images/Pasted image 20231204165852.png}}

We can use ``centre of charge'' similar to
\hyperref[centre-of-mass]{Centre of Mass} to club charges together.
\#important

\pandocbounded{\includegraphics[keepaspectratio]{images/Pasted image 20231204170217.png}}

But this clubbing, i.e.~``centre of charge'' can only be done in
calculation of dipole moment. This is because both dipole moment and
mass moment are defined in a similar way.

\subsubsection{Finding dipole moment}\label{finding-dipole-moment}

To find dipole moment of continuous bodies,
\pandocbounded{\includegraphics[keepaspectratio]{images/Pasted image 20231204170705.png}}

Using the concept of ``centre of charge'',
\pandocbounded{\includegraphics[keepaspectratio]{images/Pasted image 20231204170815.png}}

\pandocbounded{\includegraphics[keepaspectratio]{images/Pasted image 20231204171103.png}}

\section{Electric Field due to
Dipole}\label{electric-field-due-to-dipole}

The line along which dipole is placed is called axis of dipole.

The perpendicular line to axis is called equatorial line and it passes
through the middle of the dipole.

\subsubsection{Axis}\label{axis}

Electric field on axis is along dipole moment. I.e.
\[\vec{E}_\text{axial} \parallel \vec{p}\]

This electric field is, \[
\begin{split}
E_{net} &= E_{+} - E_{-} \\
&= \frac{ kq }{ (r-a)^{2} } - \frac{ kq }{ (r-a)^{2} }\\
&= \frac{ 4arkq }{ (r^{2} - a^{2})^{2} } \\
&= \frac{ 2kpr }{ (r^{2} - a^{2})^{2} } 
\end{split}
\] Using assumption, \(2a \ll r\),
\[E_\text{axis} = \frac{ 2kp }{ r^{3} }\]

In vector form, \[E_\text{axis} = \frac{ 2k\vec{p} }{ r^{3} }\]

\pandocbounded{\includegraphics[keepaspectratio]{images/Pasted image 20231204173722.png}}

\subsubsection{Equatorial Plane}\label{equatorial-plane}

From the diagram, we can see that the perpendicular components of field
cancel and thus the electric field is anti parallel to dipole moment.
I.e. \[\vec{E}_\text{equa} \upharpoonleft \! \downharpoonright \vec{p}\]

Now, \[E = \frac{ kq }{ (r^{2} + a^{2}) }\] Thus, \[
\begin{split}
E_{net} &= 2E\cos \theta \\
&= \frac{ 2kq }{ (r^{2} + a^{2}) } \frac{ q }{ \sqrt{ r^{2} + a^{2} } } \\
&= \frac{ kp }{ (r^{2} + a^{2})^{3/2} } 
\end{split}
\] Using assumption \(2a \ll r\),
\[E_\text{equa} = \frac{ kp }{ r^{3} }\]

In vector form, \[\vec{E}_\text{equa} = - \frac{ k\vec{p} }{ r^{3} }\]
The -ve indicates direction opposite to that of \(\vec{p}\).

\pandocbounded{\includegraphics[keepaspectratio]{images/Pasted image 20231204174509.png}}

Also,
\pandocbounded{\includegraphics[keepaspectratio]{images/Pasted image 20231204175343.png}}

\subsubsection{Any General Point}\label{any-general-point}

This point is r distance from the centre of the dipole and makes an
angle \(\theta\) with the direction of dipole moment.

We will break the dipole moment into two rectangular components. One
parallel and one perpendicular to the line joining the point and centre.

For \(p\cos \theta\), the electric field is,
\[E_{\parallel} = \frac{ 2kp\cos \theta }{ r^{3} }\] For
\(p\sin \theta\), the electric field is,
\[E_{\perp} = \frac{ kp\sin \theta }{ r^{3} }\]

Thus the net electric field is, \[
\begin{split}
E_{net} &= \sqrt{ E_{\parallel}^{2} + E_{\perp}^{2} } \\
&= \frac{ kp }{ r^{3} } \sqrt{ \sin ^{2}\theta + 4\cos ^{2}\theta } \\
&= \frac{ kp }{ r^{3} } \sqrt{ 1 + 3\cos ^{2}\theta } 
\end{split}
\]

Now, \(\phi\) is angle between \(\vec{r}\) and \(\vec{E}_{net}\). \[
\begin{split}
\tan \phi &= \frac{ E_{\perp} }{ E_{\parallel} } \\
&= \frac{ \tan \theta }{ 2 }
\end{split}
\]

And, the angle between \(\vec{p}\) and \(\vec{E}_{net}\) will be,
\[\angle = \theta + \phi = \theta + \tan ^{-1}\left( \frac{ \tan \theta }{ 2 } \right)\]

\pandocbounded{\includegraphics[keepaspectratio]{images/Pasted image 20231205150434.png}}

For example,
\pandocbounded{\includegraphics[keepaspectratio]{images/Pasted image 20231205151202.png}}

\section{Potential due to Small
Dipole}\label{potential-due-to-small-dipole}

Close to the +ve charge, or on the side of +ve charge, the potential
will be +ve. Similarly for the -ve charge.

On the equatorial plane, the potential will be zero as it is equidistant
from both charges. Thus the equatorial plane is an Equipotential
Surface.

\pandocbounded{\includegraphics[keepaspectratio]{images/Pasted image 20231205151710.png}}

\subsubsection{At axis}\label{at-axis}

\[
\begin{split}
V_{p} &= V_{+} + V_{-} \\
&= \frac{ kq }{ (r-a) } + \frac{ k(-q) }{ (r+a) } \\
&= \frac{ kq(2a) }{ r^{2} - a^{2} } \\
&= \frac{ kp }{ r^{2} - a^{2} }
\end{split}
\]

Using the assumption, \(a \ll r\), \[V = \frac{ kp }{ r^{2} }\]

\pandocbounded{\includegraphics[keepaspectratio]{images/Pasted image 20231205152015.png}}

\subsubsection{Any General Point}\label{any-general-point-1}

Dividing \(\vec{p}\) into two components like before, we see that point
P is equatorial for \(p\sin \theta\) and axial for \(p\cos \theta\).

Thus, \[V_{p} = \frac{ kp\cos \theta }{ r^{2} }\]

In vector form, \[V = \frac{ k\ \vec{p} . \vec{r} }{ r^{3} }\]

\pandocbounded{\includegraphics[keepaspectratio]{images/Pasted image 20231205152404.png}}

For example,
\pandocbounded{\includegraphics[keepaspectratio]{images/Pasted image 20231205152814.png}}

\newpage

\chapter{Dipole in Field}\label{dipole-in-field}

\etocsettocstyle{\textbf{Chapter Contents}\par\rule{\linewidth}{0.5pt}}{\par\rule{\linewidth}{0.5pt}}
\localtableofcontents

\noindent {} There is no external force on the dipole and thus
\(a_{com} = 0\).

But due to \hyperref[me-force-couple-and-poafforce-couple]{force
couple,} there will be torque on the dipole.

This torque will be, \[
\begin{split}
\uptau &= QE(l \sin \theta) \\
&= pE\sin \theta
\end{split}
\]

In vector form, \[
\begin{split}
\vec{\tau} &= \vec{p} \times \vec{E}
\end{split}
\]

For angular acceleration, \[
\begin{split}
\uptau &= I\alpha \\
&= I_\text{hinge} \alpha \\
&= I_\text{com} \alpha
\end{split}
\]

\pandocbounded{\includegraphics[keepaspectratio]{images/Pasted image 20231205160549.png}}

\subsubsection{Equilibrium Positions}\label{equilibrium-positions}

The one when \(\theta = 0^{\circ}\) is stable and the one when
\(\theta = 180^{\circ}\) is unstable.

This can be found by slightly rotating the dipole and then checking if
it rotates further or comes back to initial position.

Oscillatory motion of dipole is possible about \(\theta = 0\) because it
is stable equilibrium position.

\pandocbounded{\includegraphics[keepaspectratio]{images/Pasted image 20231205161203.png}}

\pandocbounded{\includegraphics[keepaspectratio]{images/Pasted image 20231205161404.png}}

\subsubsection{Angular SHM}\label{angular-shm}

\hyperref[angular-shm]{Angular SHM}

When dipole is rotated from the equilibrium position by an angle
\(\theta\), it experiences a restoring torque,
\[\uptau = pE\sin \theta\]

Now for small angles, \(\sin \theta \sim \theta\),
\[\uptau = pE \theta\] I.e. \(\uptau \propto \theta\) and thus it is
performing angular SHM.

For time period, \[
\begin{split}
\uptau &= pE \theta \\
I\alpha &= pE \\
\alpha &= \frac{ pE }{ I } \theta \\
\omega &= \sqrt{ \frac{ pE }{ I } } \\
T &= 2\pi \sqrt{ \frac{ I }{ pE } }
\end{split}
\]

If the dipole is hinged, \[T = 2\pi \sqrt{ \frac{ I_{H} }{ pE } }\] If
the dipole is not hinged, \[T = 2\pi \sqrt{ \frac{ I_{com} }{ pE } }\]

\pandocbounded{\includegraphics[keepaspectratio]{images/Pasted image 20231205162608.png}}

Take care to write the correct moment of inertia of the system.
\pandocbounded{\includegraphics[keepaspectratio]{images/Pasted image 20231205162827.png}}

\subsection{Potential Energy}\label{potential-energy}

of dipole placed in uniform electric field.

The PE of dipole will be the same as the PE of the two charges of the
dipole in the electric field.

\[
\begin{split}
U &= QV_{1} + (-Q)V_{2} \\
&= Q(V_{1} - V_{2}) \\
&= -QEl\cos \theta \\
&= -pE\cos \theta 
\end{split}
\]

In terms of vectors, \[U = -\vec{p} . \vec{E}\]

\pandocbounded{\includegraphics[keepaspectratio]{images/Pasted image 20231205163711.png}}

\subsubsection{Graph}\label{graph-7}

\pandocbounded{\includegraphics[keepaspectratio]{images/Pasted image 20231205163815.png}}

\subsubsection{Example}\label{example}

\pandocbounded{\includegraphics[keepaspectratio]{images/Pasted image 20231205164432.png}}

\section{Dipole in Non Uniform Field}\label{dipole-in-non-uniform-field}

Dipole is placed along electric field.

The dipole is of length dx.

Now, force on this dipole, \[
\begin{split}
f &= Q(E + dE) - QE \\
&= QdE \\
&= Qdx \frac{ dE }{ dx } \\
&= p \frac{ dE }{ dx } 
\end{split}
\]

Thus, \[f = p \frac{ dE }{ dx } \]

\pandocbounded{\includegraphics[keepaspectratio]{images/Pasted image 20231205172215.png}}

\subsubsection{Examples}\label{examples-5}

\pandocbounded{\includegraphics[keepaspectratio]{images/Pasted image 20231205172741.png}}

\pandocbounded{\includegraphics[keepaspectratio]{images/Pasted image 20231205173238.png}}

\pandocbounded{\includegraphics[keepaspectratio]{images/Pasted image 20231205173546.png}}
\pandocbounded{\includegraphics[keepaspectratio]{images/Pasted image 20231205173600.png}}

\newpage

\chapter{Electric Flux \& Gauss Law}\label{electric-flux--gauss-law}

\etocsettocstyle{\textbf{Chapter Contents}\par\rule{\linewidth}{0.5pt}}{\par\rule{\linewidth}{0.5pt}}
\localtableofcontents

\noindent {}
\hyperref[electric-fieldelectric-lines-of-force-elof]{Electric Field\#Electric Lines of Force (ELOF)}

It is a physical quantity which represents no. of ELOF passing through a
surface area.

The more ELOF are passing though a surface, the more flux there is.

\pandocbounded{\includegraphics[keepaspectratio]{images/Pasted image 20231205182340.png}}

Consider a small area, whose area vector is \(d\vec{s}\). An electric
field which makes an angle \(\theta\) with the area vector.

Now, the component of this E which is perpendicular to the area vector
is tangential to the surface and thus does not pass through it.

The component of E along ds is contributing to the no. of ELOF and thus
the flux.

\pandocbounded{\includegraphics[keepaspectratio]{images/Pasted image 20231205182735.png}}

Thus we get, small flux through this small surface, \[
\begin{split}
d\phi &= E\cos \theta. ds \\
&= \vec{E} . d\vec{s} \\
&= \vec{E}_{\parallel} ds
\end{split}
\]

For an extended surface, \[\phi = \int \vec{E} . d\vec{s} \] This is
called \textbf{Surface integral of Electric Field.}

If E is uniform, \[\phi = \vec{E}.\vec{S} = ES\cos \theta\] where
\(\vec{ S }\) is the area vector.

Unit of flux is \(\symup{ N C^{-1} m^{2} }\) or \(\symup{ Vm }\) and it
has dimensions \(\symup{ [ML^{3}T^{-3}I^{-1}] }\).

Since it is a scalar product of two vectors, flux is a scalar quantity.

\pandocbounded{\includegraphics[keepaspectratio]{images/Pasted image 20231211173345.png}}

\pandocbounded{\includegraphics[keepaspectratio]{images/Pasted image 20231211174221.png}}
\pandocbounded{\includegraphics[keepaspectratio]{images/Pasted image 20231211174232.png}}

For a 3D curved surface, we will use projection of the surface (cross
sectional area),
\pandocbounded{\includegraphics[keepaspectratio]{images/Pasted image 20231212150952.png}}

\subsection{Flux due to Point Charge}\label{flux-due-to-point-charge}

\(\oint\) represents closed surface, i.e.~closed integral.

\[
\begin{split}
\phi &= \oint \vec{E} . \, d\vec{s} \\
&= \oint Eds \cos \theta \\
&= \oint Eds \\
&= E \oint ds in\\
&= E 4\pi R^{2} \\
&= \frac{ kQ }{ R^{2} } 4\pi R^{2} \\
&= 4\pi kQ 
\end{split}
\]

Thus, for a point charge, \[\phi = \frac{ Q }{ \varepsilon_{o} }\] where
Q is with sign. This is independent of radius of sphere, R.

Flux going out of a surface is +ve and going into it is -ve.

\pandocbounded{\includegraphics[keepaspectratio]{images/Pasted image 20231212151504.png}}

\paragraph{Some PTR}\label{some-ptr}

The ELOF which pass through smaller sphere, also pass through the bigger
sphere and thus the amount of flux through them is the same.

\pandocbounded{\includegraphics[keepaspectratio]{images/Pasted image 20231212152631.png}}

This also works for non concentric spheres as no ELOF disappears as we
move from the smaller to the bigger sphere.

\pandocbounded{\includegraphics[keepaspectratio]{images/Pasted image 20231212152809.png}}

Thus, for any closed surface, if there is a charge inside the surface,
the flux passing through it will be \(\phi = Q /\varepsilon_{o}\).

\pandocbounded{\includegraphics[keepaspectratio]{images/Pasted image 20231212153227.png}}

If charge is not inside the closed surface, the flux through the surface
will be zero as the lines which enter will also leave.

\pandocbounded{\includegraphics[keepaspectratio]{images/Pasted image 20231212153703.png}}

\section{Gauss Law}\label{gauss-law}

aka \textbf{Gauss Theorem}

Flux emitted by a charge Q is equal to
\(\displaystyle \frac{Q}{\varepsilon_{o}}\).

Closed surface is called \textbf{Gaussian Surface.}

If there is no charge inside a gaussian surface, the flux through it is
zero.

\emph{Surface integral of electric field on a closed surface is always
equal to charge enclosed per unit permittivity of free space.}

That is, the total flux through a gaussian surface is equal to the total
charge inside divided by \(\varepsilon_{o}\).

\[\phi = \frac{ Q_{in} }{ \varepsilon_{o} }\] where \(Q_{in}\) is with
sign.

\pandocbounded{\includegraphics[keepaspectratio]{images/Pasted image 20231212154419.png}}

If flux is zero, it does not mean that electric field is zero. However,
if E is zero, flux must be zero.

The total flux is equally distributed through all the faces of a 3D
shape if charge is placed at the centre.

\pandocbounded{\includegraphics[keepaspectratio]{images/Pasted image 20231212154808.png}}

To use symmetry, the shapes we are dividing the flux among must be the
same for the charge (i.e.~same E through them, same distance from
charge).

\pandocbounded{\includegraphics[keepaspectratio]{images/Pasted image 20231212155133.png}}

If charge is placed at surface or vertex or edge, we will consider the
point charge as a small sphere.

\pandocbounded{\includegraphics[keepaspectratio]{images/Pasted image 20231212155453.png}}
\pandocbounded{\includegraphics[keepaspectratio]{images/Pasted image 20231212155504.png}}
\pandocbounded{\includegraphics[keepaspectratio]{images/Pasted image 20231212155711.png}}

\newpage

\chapter{Finding Field Using Gauss
Law}\label{finding-field-using-gauss-law}

\etocsettocstyle{\textbf{Chapter Contents}\par\rule{\linewidth}{0.5pt}}{\par\rule{\linewidth}{0.5pt}}
\localtableofcontents

\noindent {} \#\#\#\# Uniform Hollow Sphere
\hyperref[sphereuniformly-charged-hollow-sphere]{Sphere\#Uniformly Charged Hollow Sphere}

\paragraph{Outside}\label{outside-1}

We consider a gaussian surface of radius r.

Now, flux is \[
\begin{split}
\phi &= \oint \vec{E}. d\vec{s} \\
&= \oint Eds.\cos 0\\
&= E \oint ds \\
&= E. 4\pi r^{2} 
\end{split}
\] Using gauss law, \[
\begin{split}
\phi &= \frac{ Q }{ \varepsilon_{o} } \\
E. 4 \pi r^{2} &= \frac{ Q }{ \varepsilon_{o} } \\
E &= \frac{ 1 }{ 4\pi\varepsilon_{o} } \frac{ Q }{ r^{2} } \\
&= \frac{ kQ }{ r^{2} }
\end{split}
\]

\pandocbounded{\includegraphics[keepaspectratio]{images/Pasted image 20231213183327.png}}

\paragraph{Inside}\label{inside-1}

Like before, we make a gaussian surface of radius r.

Now, flux can be written as, \[\phi = E 4\pi r^{2}\] Using gauss law,
\[\phi = 0\]

Thus giving, \[E_{in} = 0\]

\pandocbounded{\includegraphics[keepaspectratio]{images/Pasted image 20231213183525.png}}

\subsubsection{Uniform Solid Sphere}\label{uniform-solid-sphere}

\hyperref[sphereuniformly-charged-hollow-sphere]{Sphere\#Uniformly Charged Hollow Sphere}

Outside will be the same as that of hollow sphere.

Here, charge density, \[\rho = \frac{ 3Q }{ 4\pi R^{3} }\]

\paragraph{Inside}\label{inside-2}

We make a gaussian surface of radius r.

Now, flux through this surface, \[
\begin{split}
\phi &= \oint \vec{E}. d\vec{s} \\
&= E \oint ds \\
&= E. 4 \pi r^{2}
\end{split}
\]

Charge inside the gaussian surface will be, \[
\begin{split}
Q_{in} &= \rho \frac{ 4 }{ 3 } \pi r^{3} \\
&= \frac{ Qr^{3} }{ R^{3} }
\end{split}
\]

Using gauss law, \[
\begin{split}
\phi &= \frac{ Q_{in} }{ \varepsilon_{o} } \\
&= \frac{ \rho 4\pi r^{3} }{ 3\varepsilon_{o} } \\
&= \frac{ Qr^{3} }{ R^{3} \varepsilon_{o} } \\
E. 4\pi r^{2} &= \frac{ \rho 4\pi r^{3} }{ 3 } \\
E. 4\pi r^{2} &= \frac{ Qr^{3} }{ R^{3} \varepsilon_{o} }
\end{split} 
\]

Thus giving,
\[E = \frac{ \rho r }{ 3 \varepsilon_{o} } = \frac{ 1 }{ 4\pi\varepsilon_{o} } \frac{ Qr }{ R^{3} }\]

\pandocbounded{\includegraphics[keepaspectratio]{images/Pasted image 20231213184313.png}}

\subsubsection{Large Non Conducting
Sheet}\label{large-non-conducting-sheet}

\hyperref[wireinfinitely-large-sheet]{Wire\#Infinitely Large Sheet}

We make a cylindrical and symmetrical gaussian surface.

Flux can be given as, \[
\begin{split}
\phi &= \oint \vec{E} . d\vec{s}\\
&= \phi_{1} + \phi_{2} + \phi_{3} \\
&= ES \cos 0 + ES \cos 0 + 0 \\
&= 2ES
\end{split}
\]

Using gauss law, \[
\begin{split}
\phi &= \frac{ \sigma S }{ \varepsilon_{o} } \\
2ES &= \frac{ \sigma S }{ \varepsilon_{o} } \\
E &= \frac{ \sigma }{ 2 \varepsilon_{o} }
\end{split}
\]

Which is uniform E.

\pandocbounded{\includegraphics[keepaspectratio]{images/Pasted image 20231213184807.png}}

\subsubsection{Infinitely Long Wire}\label{infinitely-long-wire-1}

\hyperref[wireinfinitely-long-wire]{Wire\#Infinitely Long Wire}

We will make a cylindrical gaussian surface, axis of which is the wire.

Flux is given as, \[
\begin{split}
\phi &= \phi_{1} + \phi_{2} + \phi_{3} \\
&= 0 + 0 + \int \vec{E} \, d\vec{s} \\
&= E \int ds \\
&= E. 2\pi rl   
\end{split}
\]

Using gauss law, \[
\begin{split}
\phi &= \frac{ \lambda l }{ \varepsilon_{o} } \\
E. 2\pi rl &= \frac{ \lambda l }{ \varepsilon_{o} } \\
E &= \frac{ \lambda }{ 2\pi \varepsilon_{o} r } \\
&= \frac{ 2k\lambda }{ r }
\end{split}
\]

\pandocbounded{\includegraphics[keepaspectratio]{images/Pasted image 20231213185331.png}}

\subsubsection{Long Hollow Cylindrical}\label{long-hollow-cylindrical}

Infinitely long and having radius R.

Here, \[
\begin{split}
Q &= \lambda l \\
&= \sigma 2 \pi R l \\
\lambda &= \sigma 2\pi R
\end{split}
\] Where \(\lambda, \sigma\) mean charge per unit length and area.

\paragraph{Inside}\label{inside-3}

We make a smaller cylindrical gaussian surface of radius r.

Flux is, \[
\begin{split}
\phi &= \phi_{1} + \phi_{2} + \phi_{3} \\
&= 0 + 0 + E. 2\pi rl
\end{split}
\]

Using gauss law, \[\phi = 0\] Thus giving, \[E_{in} = 0\]

There will be no E due to infinitely long uniformly charged cylinder at
inside point.

\paragraph{Outside}\label{outside-2}

We make a larger cylindrical gaussian surface of radius r.

Flux will be, \[
\begin{split}
\phi &= \phi_{1} + \phi_{2} + \phi_{3} \\
&= 0 + 0 + E 2\pi rl\\
&= E. 2\pi rl
\end{split}
\]

Using gauss law, \[
\begin{split}
\phi &= \frac{ \lambda l }{ \varepsilon_{o} } \\
&= \frac{ \sigma. 2\pi rl }{ \varepsilon_{o} } \\
E &= \frac{ \lambda }{ 2\pi \varepsilon_{o} r } \\
&= \frac{ 2k\lambda }{ r }
\end{split}
\] Which is the same that of infinitely long wire.

Thus long hollow cylinder acts like an infinite wire placed at its axis
for outside point.

In terms of \(\sigma\), \[E = \frac{ \sigma R }{ \varepsilon_{o} r }\]

\pandocbounded{\includegraphics[keepaspectratio]{images/Pasted image 20231213190857.png}}

\paragraph{Graph}\label{graph-8}

\pandocbounded{\includegraphics[keepaspectratio]{images/Pasted image 20231213190906.png}}

\subsubsection{Long Uniform Solid
Sphere}\label{long-uniform-solid-sphere}

Infinitely long and having radius R.

The charge is in the volume, i.e.~\(\rho\).

\paragraph{Outside}\label{outside-3}

We make a larger cylindrical gaussian surface of radius r.

Flux through this surface, \[
\begin{split}
\phi &= \phi_{1} + \phi_{2} + \phi_{3} \\
&= 0 + 0 + E.2\pi rl
\end{split}
\]

Using gauss law, \[
\begin{split}
\phi &= \frac{ \rho \pi R^{2}l }{ \varepsilon_{o} } \\
E_{out} &= \frac{ \rho R^{2} }{ 2\varepsilon_{o} r }
\end{split}
\]

\pandocbounded{\includegraphics[keepaspectratio]{images/Pasted image 20231213191718.png}}

\paragraph{Inside}\label{inside-4}

We make a smaller cylindrical gaussian surface of radius r.

Flux through this is, \[
\begin{split}
\phi &= \phi_{1} + \phi_{2} + \phi_{3} \\
&= 0 + 0 + E.2\pi rl \\
&= E. 2\pi rl
\end{split}
\]

Using gauss law, \[
\begin{split}
\phi &= \frac{ \rho \pi r^{2} l }{ \varepsilon_{o} } \\
E_{in} &= \frac{ \rho r }{ 2\varepsilon_{o} }
\end{split}
\]

\pandocbounded{\includegraphics[keepaspectratio]{images/Pasted image 20231213192140.png}}

\paragraph{Graph}\label{graph-9}

\pandocbounded{\includegraphics[keepaspectratio]{images/Pasted image 20231213192705.png}}

\newpage

\chapter{Electrostatics of
Conductors}\label{electrostatics-of-conductors}

\etocsettocstyle{\textbf{Chapter Contents}\par\rule{\linewidth}{0.5pt}}{\par\rule{\linewidth}{0.5pt}}
\localtableofcontents

\noindent {} in Electrostatic Conditions i.e.~when the charge is at
rest.

\hyperref[electrostaticsconductor]{Electrostatics\#Conductor}

Here, we will consider a conductor to be a metal.

When field is applied on a conductor, the free e are the only ones that
move.

\pandocbounded{\includegraphics[keepaspectratio]{images/Pasted image 20231213193457.png}}

\subsection{Properties of Conductors}\label{properties-of-conductors}

\begin{enumerate}
\def\labelenumi{\arabic{enumi}.}
\item
  If a metal has no charge in cavity, then any charge given will reside
  on the outer surface.

  In non metals, charge can stay in bulk (i.e.~inside).
\end{enumerate}

\pandocbounded{\includegraphics[keepaspectratio]{images/Pasted image 20231213193733.png}}

\begin{verbatim}
Metals behave like hollow object in terms of electrostatic properties.
\end{verbatim}

\pandocbounded{\includegraphics[keepaspectratio]{images/Pasted image 20231213194333.png}}
\(\\\)

\begin{enumerate}
\def\labelenumi{\arabic{enumi}.}
\setcounter{enumi}{1}
\item
  Net electric field in the bulk of conductor is always zero in
  electrostatic conditions. \[or\] ELOF will not enter inside a
  conductor.

  This is because in electrostatic conditions, in the conductor,
  external field and induced fields are equal and opposite and cancel
  each other.
\end{enumerate}

\pandocbounded{\includegraphics[keepaspectratio]{images/Pasted image 20231213194813.png}}
\(\\\)

\begin{enumerate}
\def\labelenumi{\arabic{enumi}.}
\setcounter{enumi}{2}
\tightlist
\item
  Field at the surface of conductor is always normal to it. \[or\] ELOF
  always strike normally at the conductor.
\end{enumerate}

\pandocbounded{\includegraphics[keepaspectratio]{images/Pasted image 20231213200121.png}}
\(\\\)

\begin{enumerate}
\def\labelenumi{\arabic{enumi}.}
\setcounter{enumi}{3}
\item
  Conductor has equal potential at each point of it. \[or\] Conductor is
  an equipotential surface.

  Since there is no field inside, there is no work done in moving charge
  from point A to B. Work done in moving a charge on the surface is also
  zero as the field is perpendicular to displacement. Thus
  \(\Delta V = 0\) and conductor becomes equipotential surface.
\end{enumerate}

\pandocbounded{\includegraphics[keepaspectratio]{images/Pasted image 20231213205219.png}}

\subsubsection{Examples}\label{examples-6}

\pandocbounded{\includegraphics[keepaspectratio]{images/Pasted image 20231213200451.png}}

\pandocbounded{\includegraphics[keepaspectratio]{images/Pasted image 20231213205857.png}}

\pandocbounded{\includegraphics[keepaspectratio]{images/Pasted image 20231213210642.png}}

\subsection{Field at the Surface of
Conductor}\label{field-at-the-surface-of-conductor}

Here we have \(\sigma\) which is \emph{local charge density.}

We will take a cylindrical gaussian surface at the surface.

The flux can be written as, \[
\begin{split}
\phi &= \phi_{1} + \phi_{2} + \phi_{3} \\
&= 0 + 0 + ES
\end{split}
\]

By gauss law, \[
\begin{split}
\phi &= \frac{ \sigma S }{ \varepsilon_{o} } \\
ES &= \frac{ \sigma S }{ \varepsilon_{o} } \\
E &= \frac{ \sigma }{ \varepsilon_{o} }
\end{split}
\]

\pandocbounded{\includegraphics[keepaspectratio]{images/Pasted image 20231213201136.png}}

\paragraph{Local Charge Density}\label{local-charge-density}

\[\sigma \propto \frac{ 1 }{ \text{Radius of Curvature} }\] This means
that at the nooks and crannies, the charge density will be high. I.e.
the shape edges will produce more electric field.

\pandocbounded{\includegraphics[keepaspectratio]{images/Pasted image 20231213201459.png}}

\subsection{Electrostatic Pressure}\label{electrostatic-pressure}

Any charge on the surface of the conductor experiences a repulsive force
due to the other charges. This force per unit area is called
electrostatic pressure.

\[P = \frac{ \sigma^{2} }{ 2\varepsilon_{o} }\] where \(\sigma\) is
local charge density.

\pandocbounded{\includegraphics[keepaspectratio]{images/Pasted image 20231213201903.png}}

\begin{quote}
Proof: (A and B are very very close)
\end{quote}

\pandocbounded{\includegraphics[keepaspectratio]{images/Pasted image 20231213203015.png}}
\textgreater{}

\pandocbounded{\includegraphics[keepaspectratio]{images/Pasted image 20231213203034.png}}
\textgreater{}

\pandocbounded{\includegraphics[keepaspectratio]{images/Pasted image 20231213203051.png}}

Force on part of body, \[f = PA\] where \(A\) is the projected area of
the surface perpendicular to force.

\pandocbounded{\includegraphics[keepaspectratio]{images/Pasted image 20231213204938.png}}

\subsubsection{Force on Part of Solid conducting
Sphere}\label{force-on-part-of-solid-conducting-sphere}

\pandocbounded{\includegraphics[keepaspectratio]{images/Pasted image 20231213204248.png}}
\pandocbounded{\includegraphics[keepaspectratio]{images/Pasted image 20231213204726.png}}

\subsection{\texorpdfstring{\hyperref[sharing-of-charge]{Sharing of Charge}}{Sharing of Charge}}\label{sharing-of-charge}

\newpage

\chapter{Sharing of Charge}\label{sharing-of-charge}

\etocsettocstyle{\textbf{Chapter Contents}\par\rule{\linewidth}{0.5pt}}{\par\rule{\linewidth}{0.5pt}}
\localtableofcontents

\noindent {} Charges will flow until the potentials at both the
conductors is equal.

The net charge of the system is conserved.

\pandocbounded{\includegraphics[keepaspectratio]{images/Pasted image 20231213211212.png}}

\paragraph{Conducting Spheres Connected by
Wire}\label{conducting-spheres-connected-by-wire}

Condition in which there is no flow of charge in the wire, \[
\begin{split}
\frac{ kQ_{1} }{ R_{1} } &= \frac{ kQ_{2} }{ R_{2} } \\
\frac{ Q_{1} }{ R_{1} } &= \frac{ Q_{2} }{ R_{2} }
\end{split}
\]

If \(Q_{1} /R_{1} > Q_{2} /R_{2}\), then final charge on the spheres,
\(q_{1},q_{2}\)

Equating final voltages, \[
\begin{split}
V = \frac{ kq_{1} }{ R_{1} } &= \frac{ kq_{2} }{ R_{2} } \\
\frac{ q_{1} }{ q_{2} } &= \frac{ R_{1} }{ R_{2} }
\end{split}
\] Using conservation of charge, \[Q_{1} + Q_{2} = q_{1} + q_{2}\]

Thus giving, \[
\begin{split}
q_{1} &= \frac{ R_{1} }{ R_{1} + R_{2} } (Q_{1} + Q_{2}) \\
q_{2} &= \frac{ R_{2} }{ R_{1} + R_{2} } (Q_{1} + Q_{2}) \\
\end{split}
\] i.e.~shells \emph{share charge proportional to their radii.}

\pandocbounded{\includegraphics[keepaspectratio]{images/Pasted image 20231214195123.png}}

Now, \[
\begin{split}
\frac{ q_{1} }{ q_{2} } &= \frac{ R_{1} }{ R_{2} } \\
\frac{ \sigma_{1} 4\pi R_{1}^{2} }{ \sigma_{2} 4\pi R_{2}^{2} } &= \frac{ R_{1} }{ R_{2} } \\
\frac{ \sigma_{1} }{ \sigma_{2} } &= \frac{ R_{2} }{ R_{1} }
\end{split}
\] Thus, areal \emph{charge density is inversely proportional to
radius.}

Heat dissipated from the conducting wire can be given as, \[
\begin{split}
\text{Heat} &= U_{i} - U_{f} \\
&= \text{self energy}_{i} - \text{self energy}_{f} \\
&= \frac{ kQ_{1}^{2} }{ 2R_{1} } + \frac{ kQ_{2}^{2} }{ 2R_{2} } - \left( \frac{ kq_{1}^{2} }{ 2R_{1} } + \frac{ kq_{2}^{2} }{ 2R_{2} } \right) 
\end{split}
\]

\pandocbounded{\includegraphics[keepaspectratio]{images/Pasted image 20231214200020.png}}

\paragraph{Concentric Spheres Connected by
Wire}\label{concentric-spheres-connected-by-wire}

Final charges on the shells, \[
\begin{split}
V_{A} &= V_{B} \\
\frac{ kx^{2} }{ R_{1} } + \frac{ k(Q_{1}+Q_{2} - x)^{2} }{ R_{2} } &= \frac{ kx^{2} }{ R_{2} } + \frac{ k(Q_{1}+Q_{2} - x)^{2} }{ R_{2} } \\
\frac{ kx^{2} }{ R_{1} } &= \frac{ kx^{2} }{ R_{2} } \\
x &= 0
\end{split}
\] This means that all the charge goes to the outer shell.

\pandocbounded{\includegraphics[keepaspectratio]{images/Pasted image 20231214200553.png}}

Example,
\pandocbounded{\includegraphics[keepaspectratio]{images/Pasted image 20231214201201.png}}

\subsection{Grounding/Earthing}\label{groundingearthing}

Earth is considered a large conducting sphere.

Earth's charge and potential is fixed and is thus considered zero.

Any appliance connected with Earth has its potential dropped to zero.

If a conducting object is connected to earth through a conducting wire,
then its electric potential will become the same as earth, i.e.~zero.

\pandocbounded{\includegraphics[keepaspectratio]{images/Pasted image 20231214204639.png}}
\pandocbounded{\includegraphics[keepaspectratio]{images/Pasted image 20231214204650.png}}

\pandocbounded{\includegraphics[keepaspectratio]{images/Pasted image 20231214204952.png}}

\subsection{Charge dist. and Induction in
Cavity}\label{charge-dist.-and-induction-in-cavity}

There is a point charge q inside the cavity.

We take a gaussian surface enclosing the cavity.

Flux through this surface, \[\phi = \oint \vec{E}.d\vec{s} = 0\] Since
there is no field inside conductor.

Using gauss law, \[\phi = \frac{ q + q_{ind} }{ \varepsilon_{o} } = 0\]
Thus giving, \[q_{ind} = -q\]

\pandocbounded{\includegraphics[keepaspectratio]{images/Pasted image 20231214205910.png}}

\pandocbounded{\includegraphics[keepaspectratio]{images/Pasted image 20231214210831.png}}

\pandocbounded{\includegraphics[keepaspectratio]{images/Pasted image 20231214211107.png}}

\subsection{Electrostatic Shielding}\label{electrostatic-shielding}

Distribution of induced charge in the cavity is decided by shaped of it
and positions of the charges inside it.

Distribution of charge on outer surface depends on shape of it and the
positions of other charges.

Charges inside cavity and charge induced on it will have no \emph{net}
electric effect outside the cavity.

Charges outside outer surface and charge on it will have no \emph{net}
electric effect inside the cavity.

This is known as Electrostatic Shielding,

\pandocbounded{\includegraphics[keepaspectratio]{images/Pasted image 20231214211957.png}}

\pandocbounded{\includegraphics[keepaspectratio]{images/Pasted image 20231214212917.png}}

\subsection{Large Conducting Plate}\label{large-conducting-plate}

Between non metallic and metallic plates, charge distribution is
different. In metallic plate, the charge is distributed in two layers,
and thus the charge density is half for metallic plate.

\pandocbounded{\includegraphics[keepaspectratio]{images/Pasted image 20231214213601.png}}

However, if charge given is the same, then field is also the same. And
if \(\sigma\) is the same, the charge and thus the field for the
metallic plate will be double.

Proof that the charge is distributed evenly,
\pandocbounded{\includegraphics[keepaspectratio]{images/Pasted image 20231214214909.png}}

\subsubsection{Field due to large conducting
plate}\label{field-due-to-large-conducting-plate}

By gauss law.

We make a cylindrical gaussian surface on one of the sides of the plate.
One of the circles of the cylinder is inside the metal plate.

The flux through this is, \[
\begin{split}
\phi &= \phi_{1} + \phi_{2} + \phi_{3} \\
&= EA + 0 + 0 
\end{split}
\]

Using gauss law, \[
\begin{split}
\phi &= \frac{ \sigma A }{ \varepsilon_{o} } \\
EA &= \frac{ \sigma A }{ \varepsilon_{o} } \\
E &= \frac{ \sigma }{ \varepsilon_{o} }
\end{split}
\]

\pandocbounded{\includegraphics[keepaspectratio]{images/Pasted image 20231214214707.png}}

\subsubsection{Induction of Charge}\label{induction-of-charge}

If two large metallic plates are placed parallelly near each other, then
charging on their opposing faces will be equal and opposite.

\pandocbounded{\includegraphics[keepaspectratio]{images/Pasted image 20231214215830.png}}

To find the induced charge, we always do \(E_{in} = 0\) for any point
inside any of the conductors.

Short cut if there is no external electric field,
\pandocbounded{\includegraphics[keepaspectratio]{images/Pasted image 20231214221427.png}}

\paragraph{Examples}\label{examples-7}

We are not writing the elrctric field due to Q-x and x-Q because they
will cancel each other.
\pandocbounded{\includegraphics[keepaspectratio]{images/Pasted image 20231214220323.png}}

\pandocbounded{\includegraphics[keepaspectratio]{images/Pasted image 20231214220749.png}}

\pandocbounded{\includegraphics[keepaspectratio]{images/Pasted image 20231214221126.png}}

\pandocbounded{\includegraphics[keepaspectratio]{images/Pasted image 20231214221940.png}}

\subsubsection{Earthing of Plate}\label{earthing-of-plate}

Final charge on isolated metallic plate becomes zero.

\pandocbounded{\includegraphics[keepaspectratio]{images/Pasted image 20231215172026.png}}

However if there is some other charged plate near it, the charge on one
side of the plate will increase while the other side will decrease.

The left most and the right most charge will become zero.
\pandocbounded{\includegraphics[keepaspectratio]{images/Pasted image 20231215172332.png}}

\pandocbounded{\includegraphics[keepaspectratio]{images/Pasted image 20231215173026.png}}
\pandocbounded{\includegraphics[keepaspectratio]{images/Pasted image 20231215173141.png}}

\newpage

\chapter{Dielectrics}\label{dielectrics}

\etocsettocstyle{\textbf{Chapter Contents}\par\rule{\linewidth}{0.5pt}}{\par\rule{\linewidth}{0.5pt}}
\localtableofcontents

\noindent {} They are non-conductors and they have no free electrons.

They are divided into two parts, \#\#\#\# Polar Molecules Where the +ve
and -ve charge do not coincide. I.e. nucleus and COM of e doesn't
coincide.

Centres of +ve charge and -ve charge do not coincide.

Polar molecules have some dipole moment. \[P = qd\]

On applying electric field, dipole moment changes. In general dipole
moment increases.

First the dipole aligns and then the dipole moment increases.

\pandocbounded{\includegraphics[keepaspectratio]{images/Pasted image 20231219122610.png}}

\subsubsection{Non Polar Molecules}\label{non-polar-molecules}

Centres of +ve charge and -ve charge coincide and thus the dipole moment
is zero in absence of external electric field.

On applying external electric field, the molecules get polarized and a
dipole moment is created along the field.

On removing the external electric field, dipole moment becomes zero.

\pandocbounded{\includegraphics[keepaspectratio]{images/Pasted image 20231219123059.png}}

\subsection{Polarization}\label{polarization}

The aligning of all the dipole moments along the electric field is
called Polarization.

The ease with which the dielectric aligns with the field is a property
of the material.

\pandocbounded{\includegraphics[keepaspectratio]{images/Pasted image 20231219123647.png}}

After the application of external field, there will be a layer of +ve
and a layer of -ve charges.

These charges are bounded and has charge density \(\sigma_{d}\).

Due to these bounded charges, there will be an electric field,
\(E_{b}\). The intensity of this field is a property of the material.

Thus the net electric field inside will be, \[E_{net} = E_{o} - E_{b}\]
This field is, \[E_{net} = \frac{ E_{o} }{ \varepsilon_{r} }\]
\(\varepsilon_{r}\) is dielectric constant and is property of the
material.

Material which are easily polarized have higher \(\varepsilon_{r}\).

The field due to bounded charge is, \[
\begin{split}
\frac{ E_{o} }{ \varepsilon_{r} } &= E_{o} - E_{b} \\
E_{b} &= E_{o} \left( 1 - \frac{ 1 }{ \varepsilon_{r} } \right)
\end{split}
\]

\pandocbounded{\includegraphics[keepaspectratio]{images/Pasted image 20231219124502.png}}

\pandocbounded{\includegraphics[keepaspectratio]{images/Pasted image 20231219125224.png}}

\subsection{Polarization Vector}\label{polarization-vector}

It is dipole moment per unit volume.

It is always along dipole moment (from -ve to +ve charge).

Its unit is \(\symup{ C m^{-2} }\). Which is the same as \(\sigma\).

\pandocbounded{\includegraphics[keepaspectratio]{images/Pasted image 20231219125551.png}}

\newpage

\backmatter
\end{document}
